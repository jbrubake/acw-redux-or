% TODO: Improve title page styling
% TODO: Index footer is wrong
% TODO: search for \oobrecap as some needed commented out to compile

\documentclass[twoside,12pt]{book}
\usepackage{or}

\begin{document} % {{{1

\setcounter{chapter}{16}
% I'd rather use \part but then that ends up in the ToC
\def\thevolume{VII}

% Prefer empty space at the bottom of the page instead of spreading things out
\raggedbottom

\pagestyle{empty}
\frontmatter

\begin{titlepage} % {{{1
    \centering
    \MakeUppercase{
        \vspace{1.00cm}{\large      The\par}
        \vspace{1.00cm}{\Huge       War of the Rebellion:\par}
        \vspace{1.00cm}{\large      A Compilation of the\par}
        \vspace{1.00cm}{\Large      Official Records\par}
        \vspace{1.00cm}{\normalsize of the\par}
        \vspace{1.00cm}{\Large      Union Armies.\par}
        \vspace{1.00cm}\myrule\par
        \vspace{1.50cm}{\small      Prepared Under the Direction of the
            Secretary of War, by \theauthorrank\ \theauthor, Third U. S.
            Artillery,\par}
        \vspace{0.50cm}{\small      and\par}
        \vspace{0.50cm}{\small      Published Pursuant to Act of Congress
            Approved June 16, 1880.\par}
        \vspace{1.00cm}\myrule\par
        \vspace{0.15cm}
        \vspace{0.25cm}{\large      Series I--Volume \thevolume.\par}
        \vspace{0.25cm}\myrule\par
        \vfill
                       {\large      Washington:\par}
        \vspace{0.10cm}{\normalsize Government Printing Office.\par}
        \vspace{0.10cm}{\normalsize 1882.}
    }
\end{titlepage}
\newpage

% Copyright {{{1
%
\null\vfill % Place at bottom of page
\begin{wrapfigure}[2]{L}{0.1\textwidth} % 2 => set float to 2 lines so it
                                        % doesn't push into the following page
    \vspace{-1\intextsep} % Push image back to top of paragraph
    \includesvg[width=0.1\textwidth]{by-nc-nd}
\end{wrapfigure}
\noindent{\footnotesize\thelicense.}

\section*{Preface}{} % {{{1
\vspace{-1em} % cheat to get the dinkus close enough
\secdinkus

\pagestyle{frontmatter}

By an act approved June 23, 1874, Congress made an appropriation ``to enable the
Secretary of War to begin the publication of the Official Records of the War of
the Rebellion,'' and directed him ``to have copied for the Public Printer all
reports, letters, telegrams and general orders not heretofore copied or printed,
and properly arranged in chronological order.''

Appropriations for continuing such preparation have been made from time to time,
and the act approved June 16, 1880, has provided ``for the printing and binding,
under direction of the Secretary of War, of 10,000 copies of a compilation of
the Official Records of the War of the Rebellion, so far as the same may be
ready for publication, during the fiscal year''; and that ``of said number,
7,000 copies shall be for the use of the House of Representatives, 2,000 copies
for the use of the Senate, and 1,000 copies for the use of the Executive
Departments.''
\footnote[1]{
Volume I to V distributed under act approved June 16, 1880. The act approved
August 7, 1882, provides that---

``The volumes of the official records of the war of the rebellion shall be
distributed as follows: One thousand copies to the executive departments, as now
provided by law. One thousand copies for distribution by the Secretary of War
among officers of the Army and contributors to the work. Eight thousand three
hundred copies shall be sent by the Secretary of War to such libraries,
organizations, and individuals as may be designated by the Senators,
Representatives, and Delegates of the Forty-seventh Congress. Each Senator shall
designate note exceeding twenty-one of such addresses, and the volumes shall be
sent thereto from time to time as they are published, until the publication is
completed. Senators, Representatives, and Delegates shall inform the Secretary of
War in each case how many volumes of those heretofore published they have
forwarded to such addresses. The remaining copies of the eleven thousand to be
published, and all sets that may not be ordered to be distributed as provide
herein, shall be sold by the Secretary of War for cost of publication with ten
per cent. added thereto, and the proceeds of such sale shall be covered into the
Treasury. If two or more sets of said volumes are ordered to the same address
the Secretary of War shall inform the Senators, Representatives or Delegates,
who have designated the same, who thereupon may designate other libraries,
organizations, or individuals. The Secretary of War shall report to the first
session of the Forty-eighth Congress what volumes of the series heretofore
published have not been furnished to such libraries, organizations, and
individuals. He shall also inform distributees at whose instance the volumes are
sent.''
}

This compilation will be the first general publication of the military records
of the war, and will embrace all official documents that can be obtained by the
compiler, and that appear to be of any historical value.

The publication will present the records in the following order of arrangement:

The \textbf{1st Series} will embrace the formal reports of the first seizures of
United States property in the Southern States, and of all military operations in
the field, with the correspondence, orders, and returns relating specially
thereto, and, as proposed, is to be accompanied by an Atlas.

In this series the reports will be arranged according to the campaigns and
several theaters of operations (in the chronological order of the events). The
correspondence, \&c., not embraced in the ``reports'' proper will follow in
chronological order.

The \textbf{2d Series} will contain the correspondence, orders, reports, and
returns, relating to prisoners of war, and (so far as the military authorities
were concerned) to State or political prisoners.

The \textbf{3d Series} will contain the correspondence, orders, reports, and
returns of the Union authorities (embracing their correspondence with the
Confederate officials) not relating specially to the subjects of the
\textit{first} and \textit{second} series. It will set forth the annual and
special reports of the Secretary of War, of the General-in-Chief, and of
the chiefs of the several staff corps and departments; the calls for troops, and
the correspondence between the National and the several State authorities.

\gramClosing{}
    {\MakeUppercase{\theauthor}}
    {Major, Third Art., and \theauthorrank}
\nopagebreak

\textsc{War Department}, \textit{August 23, 1880}.
\\ % Add a blank line
\nopagebreak

Approved:
\nopagebreak

\gramClosing{}
    {Alex Ramsey}
    {Secretary of War}

\newpage

% Table of Contents {{{1

{ % Override hyperref's link styling setting
    \tocloftpagestyle{frontmatter}
    \hypersetup{hidelinks}
    \tableofcontents
}
\newpage

\mainmatter
\pagestyle{mainmatter}

\chapter{Operations in Missouri, Kentucky and Tennessee} % {{{1
    {January 1, 1862--MM DD, 18YY}
% Required for TOC entry. Doubles as optional arg to \chapter
% for the header label
    \chaptermark{Operations in Mo., Ky., and Tenn.}

% Summary of Events {{{2
%
\begin{toc}[
    caption = {Summary of the Principal Events.*},
]{}
Jan. & 3,  1862. & ---Confederate occupation of Paducah, Ky. \\
     & 5,  1862. & --Feb. 25, 1862.---The Fort Donelson, Tenn., Campaign. \\
     & 24, 1862. & ---First Battle of New Madrid, Mo. \\
     & 27, 1862. & ---Confederate gunboat raid on Cairo, Ill. \\
     & 28, 1862. & ---Battle of Sikeston, Mo. \\
     & 30, 1862. & ---Paducah, Ky., evacuated by the Confederates and occupied by the Union forces. \\
Feb. & 3,  1862. & ---Occupation of Elizabethtown, Ky. \\
     & 10, 1862. & ---Brig. Gen. Thomas Caldwell, U.S. Army, relieves Maj. Gen. Thomas Smith as commander of the Army of the Tennessee \\
     & 10, 1862. & --Mar. 17, 1862.---Siege of Fort Columbus, Ky. \\
     & 12, 1862. & ---Battle of Black Bayou, Mo. \\
     & 17, 1862. & ---Arrival of Maj. Gen. Richard Steele's Army of the Kanawha at Cairo, Ill. \\
     & 25, 1862. & --Mar. 3, 1862---Engagements around Munfordville, Ky. \\
Mar. & 8,  1862. & ---Battle of Munfordville, Ky. \\
     & 10, 1862. & ---Battle of Prewitt's Knob, Ky. \\
     & 14, 1862. & ---Brig. Gen. James Howard, relieves Maj. Gen. Christopher Stoeffler as commander of the Army of the Kentucky \\
     & 17, 1862. & ---Surrender of Fort Columbus, Ky. \\
     & 18, 1862. & --May 16, 1862.---The Nashville, Tenn., Campaign. \\
     & 20, 1862. & ---Second Battle of New Madrid, Mo. \\
Apr. & 1, 1862.  & --M. D., 1862---The Memphis Campaign. \\
     & 21, 1862. & ---Capture of Bowling Green, Ky. \\
May  & 20, 1862. & --M. D., 1862.---The Corinth, Tenn., Campaign. \\
\end{toc}
\footnotetext[1]{
    Of some of the minor conflicts noted in this ``Summary'' no circumstantial
    reports are on file, the only record of such events being references to them
    on muster rolls and returns.
}

\section[The Fort Donelson Campaign]{January 5, 1862--February 25, 1862} % {{{2
    {The Fort Donelson, Tenn. Campaign}

\begin{toc}[ % Summary of Events {{{3
    caption = {Summary of the Principal Events.},
]{}
    Jan. & 5, 1862.  & ---Skirmish at Golden Pond, Ky. \\
         & 15, 1862. & ---Gunboat reconnaissance to Forts Henry and Donelson, Tenn. \\
         & 25, 1862. & ---Battle of Fort Donelson. \\
         & 28, 1862. & ---Engagement at Bufford's Hill. \\
    Feb. &  7, 1862. & ---Confederate cavalry sighted at Rayburn's Creek. \\
         &  8, 1862. & ---Engagement at Bufford's Hill \& the Sawmill. \\
         & 21, 1862. & ---Battle of Fort Henry \& engagement at Hopewell Church. \\
         & 25, 1862. & ---Destruction of Fort Donelson by the Confederates and Dover, Tenn. occupied by Union forces. \\
\end{toc}

\begin{toc}[ % Reports {{{3
    caption = {Reports, Etc.},
]{}
No. & 1.  & ---Organization of the Army of the Cumberland \\
No. & 2.  & ---Organization of the Army of the Kanawha \\
No. & 3.  & ---Return of casualties in the Union forces after the skirmish at Golden Pond, Ky., January 5, 1862 \\
No. & 4.  & ---Return of casualties in the Union forces after the battle of Fort Donelson, January 25, 1862 \\
No. & 5.  & ---Return of casualties in the Union forces after the engagement at Bufford's Hill, January 28, 1862 \\
No. & 6.  & ---Return of casualties in the Union forces after the engagement at Bufford's Hill \& the Sawmill, February 8, 1862 \\
No. & 7.  & ---Return of casualties in the Union forces after the Battle of Fort Henry \& engagement at Hopewell Church, February 21, 1862 \\
No. & 8.  & ---Maj. Gen. James Blake, U.S. Army, commanding Army of the Cumberland \\
No. & 9.  & ---Col. Walter Chekov, Adjutant General \\
No. & 10. & ---Lieut. Col. Tyler Remington, Assistant Adjutant General \\
No. & 11. & ---Surg. Charles Keeney, U.S. Army, Medical Director \\
No. & 12. & ---Brig. Gen. Ptolemy Smith, commanding VIIIth Corps \\
No. & 13. & ---Brig. Gen. William Sherman, U.S. Army, commanding 3d Brigade, 2d Division \\
No. & 14. & ---Brig. Gen. Thomas Wood, U.S. Army, commanding 1st Division \\
No. & 15. & ---Brig. Gen. Horatio Van Cleve, commanding 2d Division XIIth Corps \\
No. & 16. & ---Maj. Gen. John McClernand, commanding XIVth Corps \\
No. & 17. & ---Brig. Gen. George Thomas, U.S. Army, commanding 1st Division \\
No. & 18. & ---Brig. Gen. Lawrence Graham, U.S. Army, commanding Cavalry Division \\
No. & 19. & ---Lieut. Col. James W. Herrick, Executive Officer \\
No. & 20. & ---Maj. Gen. Richard Steele, commanding Army of the Kanawha \\
No. & 21. & ---Cdre. Daniel Lewis, U.S.N., commanding Mississippi Squadron \\
\end{toc}

\subsection{Organization of the Army of the Cumberland, Maj. Gen. James Blake, % {{{3
U.S. Army, commanding, January 25, 1862--February, 25, 18662}
\footnotetext[1]{
    Arranged according to the numerical designation of the corps, divisions and
    brigades as prescribed in General Orders, No.~5, Headquarters, Army of the
    Cumberland, December~XX, 1861.
}

\begin{fulloob}
    \staff{Chief of Staff}{Brig. Gen.}{Harold Fawcett, III}
    \staff{Chief of Artillery}{Col.}{James Cotter}
    \staff{Adjutant General}{Col.}{Walter Chekov}
    
    \corps{Eighth Corps}{Brig. Gen.}{Ptolemy Smith} % {{{4

    \division{First Division}{Brig. Gen.}{John Wool} % {{{5
    \begin{leftBde}
        \bde{First Brigade}{Col.}{Ralph Buckland}
        \rgt{6th}{Ohio}{Col. William Bosley}
        \rgt{24th}{Ohio}{Col. Frederick Jones}
        \rgt{36th}{Indiana}{Col. William Grose}
        \rgt{3d}{Kentucky}{Col. Thomas Bramlette}
    \end{leftBde}
    \begin{rightBde}
        \bde{Second Brigade}{Col.}{William Hazen}
        \rgt{9th}{United States}{Lieut. Col. Stephen Carpenter}
        \rgt{9th}{Indiana}{Col. Gideon Moody}
        \rgt{17th}{Indiana}{Col. Milo Hascall}
        \rgt{39th}{Illinois}{Col. William Morrison}
    \end{rightBde}
    \begin{middleBde}
        \bde{Third Brigade}{Col.}{Sanders Bruce}
        \rgt{78th}{Pennsylvania}{Col. William Sirwell}
        \rgt{10th}{Ohio}{Col. William Lytle}
        \rgt{13th}{Ohio}{Lieut. Col. Joseph Hawkins}
        \rgt{7th}{Kentucky}{Col. Reuben May}
    \end{middleBde}

    \division{Second Division}{Brig. Gen.}{Thomas Crittenden} % {{{5
    \begin{leftBde}
        \bde{First Brigade}{Brig. Gen.}{Jeremiah Boyle}
        \rgt{19th}{Ohio}{Col. Charles Manderson}
        \rgt{59th}{Ohio}{Col. James Fyffe}
        \rgt{8th}{Kentucky}{Col. Sidney Barnes}
        \rgt{12th}{Kentucky}{Col. William Hoskins}
    \end{leftBde}
    \begin{rightBde}
        \bde{Second Brigade}{Col.}{William Smith}
        \rgt{31s}{Ohio}{Col. Moses Walker}
        \rgt{33d}{Ohio}{Col. Joshua Sill}
        \rgt{65th}{Ohio}{Col. Charles Harker}
        \rgt{11th}{Indiana}{Col. Lew Wallace}
    \end{rightBde}
    \begin{middleBde}
        \bdeCdrs{Third Brigade}{
            \divOneCdr{Brig. Gen.}{William Sherman}
            \divOneCdr{Col.}{John Pope Cook}
        }
        \rgt{3d}{Ohio}{Col. Warren Keifer}
        \rgt{21st}{Ohio}{Lieut. Col. Dwella Stoughton}
        \rgt{25th}{Indiana}{Col. James Veatch}
        \rgt{31st}{Indiana}{Col. Charles Cruft}
    \end{middleBde}

    \bde{Artillery}{Lieut. Col.}{Peter Simonson} % {{{5
    \begin{leftBde}
        \otherbde{First Division}
        \rgt{1st}{Kentucky, Battery A}{Capt. David Stone}
        \rgt{1st}{Ohio, Battery F}{Capt. Daniel Cockerill}
    \end{leftBde}
    \begin{rightBde}
        \otherbde{Second Division}
        \rgt{4th}{Ohio Battery}{Capt. Louis Hoffman}
        \rgt{1st}{Kentucky, Battery B}{Capt. John Hewitt}
    \end{rightBde}
    \begin{middleBde}
        \otherbde{Reserve Artillery}
        \rgt{5th}{Ohio Battery}{Capt. Andrew Hickenlooper}
        \rgt{9th}{Ohio Battery}{Capt. Henry Wetmore}
        \rgt{5th}{Indiana Battery}{Capt. Daniel Chandler}
    \end{middleBde}

    \corps{Twelfth}{Brig. Gen.}{Charles Smith} % {{{4

    \divisionCdrs{First Division}{ % {{{5
        \divOneCdr{Brig. Gen.}{Thomas Wood}
        \divOneCdr{Brig. Gen.}{William Sherman}
    }
    \begin{leftBde}
        \bdeCdrs{First Brigade}{
            \bdeOneCdr{Brig. Gen.}{James Garfield}
            \bdeOneCdr{Col.}{Benjamin Smith}
        }
        \rgtCdrs{1st}{Ohio}{
            \rgtOneCdr{Col. Benjamin Smith}
            \rgtOneCdr{Maj. Joab Stafford}
        }
        \rgtCdrs{8th}{Illinois}{
            \rgtOneCdr{Col. Richard Oglesby}
            \rgtOneCdr{Lieut. Col. Richard Rowett}
        }
        \rgt{24th}{Illinois}{Col. Friedrich Hecker}
        \rgt{35th}{Illinois}{Col. Gustavus Smith}
        \rgt{40th}{Ohio\footnotemark[2]}{Col. Edwin Bradley}
    \end{leftBde}
    \begin{rightBde}
        \bde{Second Brigade}{Col.}{George Wagner}
        \rgt{15th}{Indiana}{Col. Gustavus Wood}
        \rgtCdrs{11th}{Illinois}{
            \rgtOneCdr{Col. William Wallace}
            \rgtOneCdr{Lieut. Col. Thomas Ransom}
        }
        \rgt{15th}{Illinois}{Col. Thomas Turner}
        \rgtCdrs{21st}{Illinois}{
            \rgtOneCdr{Col. Ulysses Grant}
            \rgtOneCdr{Lieut. Col. John Alexander}
        }
        \rgt{38th}{Indiana\footnotemark[3]}{Col. Benjamin Scribner}
    \end{rightBde}
    \begin{middleBde}
        \bde{Third Brigade\footnotemark[1]}{Col.}{John Pope Cook}
        \rgt{40th}{Ohio\footnotemark[2]}{Col. Edwin Bradley}
        \rgt{37th}{Indiana\footnotemark[1]}{Col. George Hazzard}
        \rgt{38th}{Indiana\footnotemark[3]}{Col. Benjamin Scribner}
        \rgt{7th}{Illinois\footnotemark[1]}{Lieut. Col. Richard Rowett}
    \end{middleBde}
    \footnotetext[1]{Deactivated, Feb. 3, 1862}
    \footnotetext[2]{Reassigned to 1st Brigade, Feb. 3, 1862}
    \footnotetext[3]{Reassigned to 2d Brigade, Feb. 3, 1862}

    \division{Second Division}{Brig. Gen.}{Horatio Van Cleve} % {{{5
    \begin{leftBde}
        \bde{First Brigade}{Col.}{Samuel Beatty}
        \rgt{6th}{Michigan}{Col. Frederick Curtenius}
        \rgt{42d}{Indiana}{Col. James G. Jones}
        \rgt{38th}{Illinois}{Col. William Carlin}
        \rgt{2d}{Minnesota}{Col. James George}
    \end{leftBde}
    \begin{rightBde}
        \bde{Second Brigade}{Col.}{William Stoughton}
        \rgt{9th}{Michigan}{Col. William Duffeld}
        \rgt{11th}{Michigan}{Lieut. Col. Melvin Mudge}
        \rgt{15th}{Michigan}{Col. John Oliver}
        \rgt{1st}{Wisconsin}{Col. John Starkweather}
    \end{rightBde}
    \begin{middleBde}
        \bde{Third Brigade}{Col.}{Jefferson Davis}
        \rgt{8th}{Wisconsin}{Col. George Robbins}
        \rgt{10th}{Wisconsin}{Col. Alfred Chapin}
        \rgt{13th}{Wisconsin}{Col. Maurice Maloney}
        \rgt{35th}{Indiana}{Col. Bernard Mullen}
    \end{middleBde}

    \bde{Artillery}{Lieut. Col.}{Charles Humphrey}% % {{{5
    \begin{leftBde}
        \otherbde{First Division}
        \rgt{1st}{Illinois, Battery D}{Capt. Henry Rogers}
        \rgt{2d}{Illinois, Battery E}{Capt. Adolphus Schwartz}
    \end{leftBde}
    \begin{rightBde}
        \otherbde{Second Division}
        \rgt{1st}{Michigan, Battery B}{Capt. William Ross}
        \rgt{3d}{Wisconsin Battery}{Capt. Lu Drury}
    \end{rightBde}
    \begin{middleBde}
        \otherbde{Reserve Artillery}
        \rgt{5th}{Wisconsin Battery}{Capt. George Gardner}
        \rgt{1st}{Michigan, Battery C}{Capt. Alexander Dees}
        \rgt{1st}{Illinois, Battery A}{Capt. Charles Willard}
    \end{middleBde}

    \corps{Fourteenth}{Brig. Gen.}{John McClernand} % {{{4

    \division{First Division}{Brig. Gen.}{George Thomas} % {{{5
    \begin{leftBde}
        \bde{First Brigade}{Col.}{Samuel Carter}
        \rgt{1st}{Kentucky}{Col. David Enyart}
        \rgt{2d}{Kentucky}{Col. Thomas Sedgewick}
        \rgt{1st}{Tennessee}{Col. Robert Byrd}
        \rgt{2d}{Tennessee\footnotemark[1]}{Col. James Carter}
    \end{leftBde}
    \footnotetext[1]{Destroyed Feb. 8, 1862}
    \begin{rightBde}
        \bde{Second Brigade}{Col.}{Madison Miller}
        \rgtCdrs{10th}{Indiana}{
            \rgtOneCdr{Lieut. Col. William Kise}
            \rgtOneCdr{Lieut. Col. William Carroll}
        }
        \rgt{4th}{Kentucky}{Col. John Croxton}
        \rgt{14th}{Ohio}{Col. James Steedman}
        \rgt{17th}{Ohio}{Col. John Connell}
    \end{rightBde}
    \begin{middleBde}
        \bde{Third Brigade}{Col.}{Thomas Kilby Smith}
        \rgt{44th}{Illinois}{Col. Charles Knobelsdorff}
        \rgt{9th}{Ohio}{Lieut. Col. Karl Sonderson}
        \rgt{35th}{Ohio}{Col. Ferdinand Van Derveer}
        \rgt{38th}{Ohio}{Lieut. Col. William Choate}
    \end{middleBde}

    \division{Second Division}{Brig. Gen.}{William Rosecrans} % {{{5
    \begin{leftBde}
        \bde{First Brigade}{Brig. Gen.}{Benjamin Prentiss}
        \rgtCdrs{15th}{United States}{
            \rgtOneCdr{Col. Oliver Shepherd}
            \rgtOneCdr{Lieut. Col. John Kung}
        }
        \rgt{6th}{Indiana}{Col. Philemon Baldwin}
        \rgt{77th}{Pennsylvania}{Col. Frederick Stumbaugh}
        \rgt{79th}{Pennsylvania}{Col. Henry Hambright}
    \end{leftBde}
    \begin{rightBde}
        \bde{Second Brigade}{Col.}{Edward Kirk}
        \rgt{16th}{United States}{Lieut. Col. Edmund Schriver}
        \rgt{29th}{Indiana}{Lieut. Col. David Dunn}
        \rgt{30th}{Indiana}{Col. Sion Bass}
        \rgt{34th}{Illinois}{Lieut. Col. Charles Levanway}
    \end{rightBde}
    \begin{middleBde}
        \bde{Third Brigade}{Col.}{William Gibson}
        \rgt{15th}{Ohio}{Col. Moses Dickey}
        \rgt{5th}{Kentucky}{Col. Harvey Buckley}
        \rgt{32d}{Indiana}{Lieut. Col. Henry von Trebra}
        \rgt{39th}{Indiana}{Col. Thomas Harrison}
    \end{middleBde}

    \bde{Artillery}{Lieut. Col.}{Charles Muehler}% % {{{5
    \begin{leftBde}
        \otherbde{First Division}
        \rgt{---}{Kentucky, Simmond's Battery}{Capt. Seth Simmonds}
        \rgt{1st}{Ohio, battery D}{Capt. Andrew Konkle}
    \end{leftBde}
    \begin{rightBde}
        \otherbde{Second Division}
        \rgt{4th}{United States}{Lieut. S. Canby}
        \rgt{4th}{Indiana Battery}{Capt. Asahel Bush}
    \end{rightBde}
    \begin{middleBde}
        \otherbde{Reserve Artillery}
        \rgt{4th}{United States, Battery M}{Capt. John Brannan}
        \rgt{1st}{Michigan, Battery A}{Capt. Cyrus Loomis}
        \rgt{---}{Pennsylvania, Battery B}{Capt. Alanson Stevens}
    \end{middleBde}

    \corps{Cavalry Division}{Brig. Gen.}{Lawrence Graham} % {{{4

    \begin{leftBde}
        \bde{First Brigade}{Col.}{John Bridgeland}
        \rgtCdrs{1st}{Kentucky}{
            \rgtOneCdr{Col. Frank Wolford}
            \rgtOneCdr{Col. Silas Adams}
        }
        \rgt{3d}{Kentucky}{Col. James Jackson}
        \rgt{2d}{Indiana}{Lieut. Col. Edward McCook}
        \rgt{4th}{United States}{Col. James Oakes}
    \end{leftBde}
    \begin{rightBde}
        \bde{Second Brigade}{Col.}{Charles Doubleday}
        \rgt{2d}{Ohio}{Col. August Kautz}
        \rgtCdrs{3d}{Ohio}{
            \rgtOneCdr{Col. Lewis Zahm}
            \rgtOneCdr{Lieut. Col. Horace Howland}
        }
        \rgt{4th}{Ohio}{Col. Eli Long}
        \rgt{2d}{Wisconsin}{Col. Cadwallader Washburn}
    \end{rightBde}
    \begin{middleBde}
        \bde{Third Brigade}{Col.}{Theophilus Dickey}
        \rgt{2d}{Illinois}{Col. Silas Noble}
        \rgt{4th}{Illinois}{Col. Martin Wallace}
        \rgt{13th}{Illinois}{Col. Joseph Bell}
        \rgt{1st}{U.S. Lancers (Michigan)}{Col. Arthur Rankin}
    \end{middleBde}
    \begin{middleBde}
        \otherbde{Artillery}
        \rgt{1st}{Ohio, Battery E}{Capt. Warren Edgarton}
        \rgt{5th}{United States, Battery H}{Capt. William Terrill}
    \end{middleBde}

    \corps{Artillery Reserve}{Col.}{Charles Cotter} % {{{4
    \begin{leftBde}
        \bde{First Brigade}{Lieut. Col.}{William Standart}
        \rgt{1st}{Ohio, Battery A}{Capt. Wilbur Goodspeed}
        \rgt{1st}{Ohio, Battery B}{Capt. J. Hale Snyder}
        \rgt{1st}{Ohio, Battery C}{Capt. Dennis Kenny, Jr}
    \end{leftBde}
    \begin{rightBde}
        \bde{Second Brigade}{Lieut. Col.}{Alonzo Bidwell}
        \rgt{1st}{Michigan, Battery D}{Capt. Josiah Church}
        \rgt{1st}{Illinois, Battery C}{Capt. Charles Houghtaling}
        \rgt{2d}{Illinois, Battery C}{Capt. Caleb Hopkins}
    \end{rightBde}
\end{fulloob}
% TODO: Remove requirement for this blank line

\subsecdinkus

\subsection{Organization of the Army of the Kanawha, Maj. Gen. Richard Steele, % {{{3
    U.S. Army, commanding, February 17, 1862--February 25, 1862}

\begin{fulloob}
    \corps{First Division}{Brig. Gen.}{Benjamin Kelley} % {{{4

    \begin{leftBde}
        \bde{First Brigade}{}{}
    \end{leftBde}
    \begin{rightBde}
        \bde{Second Brigade}{}{}
    \end{rightBde}
    \begin{middleBde}
        \bde{Third Brigade}{}{}
    \end{middleBde}

    \corps{Second Division}{Brig. Gen.}{Robert Schenck} % {{{4

    \begin{leftBde}
        \bde{First Brigade}{}{}
    \end{leftBde}
    \begin{rightBde}
        \bde{Second Brigade}{}{}
    \end{rightBde}
    \begin{middleBde}
        \bde{Third Brigade}{}{}
    \end{middleBde}

    \corps{Third Division}{Brig. Gen.}{Robert Milroy} % {{{4

    \begin{leftBde}
        \bde{First Brigade}{}{}
    \end{leftBde}
    \begin{rightBde}
        \bde{Second Brigade}{}{}
    \end{rightBde}
    \begin{middleBde}
        \bde{Third Brigade}{}{}
    \end{middleBde}

    \corps{Cavalry Division}{Brig. Gen.}{John Hall} % {{{4

    \begin{leftBde}
        \bde{First Brigade}{}{}
    \end{leftBde}
    \begin{rightBde}
        \bde{Second Brigade}{}{}
    \end{rightBde}

\end{fulloob}
% TODO: Remove requirement for this blank line

\subsecdinkus

\subsection{Return of casualties in the Union forces % {{{3
    after the skirmish at Golden Pond, Ky., January 5, 1862}

\begin{oob}{}
\oobhdr

\oobtop{Cavalry Division}{Brig. Gen.}{Lawrence Graham}

\oobbde{First Brigade}{Col.}{John Bridgeland} % {{{4
\oobrgt{1st} {Kentucky} {Col. Frank Wolford} {300} {3} {1.00}
\oobrgt{3d} {Kentucky} {Col. James Jackson} {300} {2} {0.67}
\oobrgt{2d} {Indiana} {Lieut. Col. Edward McCook} {300} {3} {1.00}
\oobrgt{4th} {United States} {Col. James Oakes} {300} {4} {1.33}
\oobtot{Total First Brigade}{1,200}{12}{1.00}

\oobbde{Second Brigade}{Col.}{Charles Doubleday} % {{{4
\oobrgt{2d} {Ohio} {Col. August Kautz} {300} {5} {1.67}
\oobrgt{3d} {Ohio} {Col. Lewis Zahm} {300} {5} {1.67}
\oobrgt{4th} {Ohio} {Col. Eli Long} {300} {5} {1.67}
\oobrgt{2d} {Wisconsin} {Col. Cadwallader Washburn} {300} {2} {0.67}
\oobtot{Total Second Brigade}{1,200}{17}{1.42}

\oobbde{Third Brigade}{Col.}{Theophilus Dickey} % {{{4
\oobrgt{2d} {Illinois} {Col. Silas Noble} {300} {\oobnone} {\oobnone}
\oobrgt{4th} {Illinois} {Col. Martin Wallace} {300} {\oobnone} {\oobnone}
\oobrgt{13th} {Illinois} {Col. Joseph Bell} {300} {\oobnone} {\oobnone}
\oobrgt{1st} {U.S. Lancers (Michigan)} {Col. Arthur Rankin} {300} {\oobnone} {\oobnone}
\oobtot{Total Third Brigade}{1,200}{\oobnone}{\oobnone}

\oobart{Artillery} % {{{4
\oobrgt{1st} {Ohio Light Artillery, Battery E} {Capt. Warren Edgarton} {6} {\oobnone} {\oobnone}
\oobrgt{5th} {United States Artillery, Battery H}{Capt. William Terrill} {6} {\oobnone} {\oobnone}
\oobtot{Total Artillery}{1,200}{\oobnone}{\oobnone}

\oobsum{Total Cavalry Division}{3,600}{29}{0.81} % {{{4

\bottomrule
\end{oob}
\subsection{Return of casualties in the Union forces % {{{3
    after the battle of Fort Donelson, January 25, 1862}

\begin{oob}{}
\oobhdr

\oobtop{Army of the Cumberland}{Maj. Gen.}{James Blake} 

\oobtop{Eighth Corps}{Brig. Gen.}{Ptolemy Smith} % {{{4

\oobdiv{First Division}{Brig. Gen.}{John Wool} % {{{5
\oobbde{First Brigade}{Col.}{Ralph Buckland} % {{{6
\oobrgt{6th}{Ohio}{Col. William Bosley}{\textsuperscript{*}}{\textsuperscript{*}}{\textsuperscript{*}}
\oobrgt{24th}{Ohio}{Col. Frederick Jones}{600}{\oobnone}{\oobnone}
\oobrgt{36th}{Indiana}{Col. William Grose}{600}{\oobnone}{\oobnone}
\oobrgt{3d}{Kentucky}{Col. Thomas Bramlette}{600}{\oobnone}{\oobnone}
\SetCell[c=2]{c}\footnotesize{\textsuperscript{*} The 6th Ohio was guarding depot at
    Collies Mills, Ky. and not present at the engagement} \\
\oobtot{Total First Brigade}{1,800}{\oobnone}{\oobnone}

\oobbde{Second Brigade}{Col.}{William Hazen} % {{{6
\oobrgt{9th}{United States}{Lieut. Col. Stephen Carpenter}{600}{\oobnone}{\oobnone}
\oobrgt{9th}{Indiana}{Col. Gideon Moody}{600}{\oobnone}{\oobnone}
\oobrgt{17th}{Indiana}{Col. Milo Hascall}{600}{\oobnone}{\oobnone}
\oobrgt{39th}{Illinois}{Col. William Morrison}{600}{\oobnone}{\oobnone}
\oobtot{Total Second Brigade}{2,400}{\oobnone}{\oobnone}

\oobbde{Third Brigade}{Col.}{Sanders Bruce} % {{{6
\oobrgt{78th}{Pennsylvania}{Col. William Sirwell}{600}{\oobnone}{\oobnone}
\oobrgt{10th}{Ohio}{Col. William Lytle}{600}{\oobnone}{\oobnone}
\oobrgt{13th}{Ohio}{Lieut. Col. Joseph Hawkins}{600}{\oobnone}{\oobnone}
\oobrgt{7th}{Kentucky}{Col. Reuben May}{600}{\oobnone}{\oobnone}
\oobtot{Total Third Brigade}{2,400}{\oobnone}{\oobnone}

\oobsum{Total First Division}{6,600}{\oobnone}{\oobnone} % {{{6
\oobdblrule

\oobdiv{Second Division}{Brig. Gen.}{Thomas Crittenden} % {{{5
\oobbde{First Brigade}{Brig. Gen.}{Jeremiah Boyle} % {{{6
\oobrgt{19th}{Ohio}{Col. Charles Manderson}{600}{20}{3.33}
\oobrgt{59th}{Ohio}{Col. James Fyffe}{600}{5}{0.83}
\oobrgt{8th}{Kentucky}{Col. Sidney Barnes}{600}{9}{1.50}
\oobrgt{12th}{Kentucky}{Col. William Hoskins}{600}{8}{1.33}
\oobtot{Total First Brigade}{2,400}{42}{1.75}

\oobbde{Second Brigade}{Col.}{William Smith} % {{{6
\oobrgt{31st}{Ohio}{Col. Moses Walker}{600}{110}{18.33}
\oobrgt{33d}{Ohio}{Col. Joshua Sill}{600}{220}{36.67}
\oobrgt{65th}{Ohio}{Col. Charles Harker}{600}{130}{21.67}
\oobrgt{11th}{Indiana}{Col. Lew Wallace}{600}{140}{23.33}
\oobtot{Total Second Brigade}{2,400}{600}{25.00}

\oobbde{3d Brigade}{Brig. Gen.}{William Sherman} % {{{6
\oobrgt{3d}{Ohio}{Col. Warren Keifer}{600}{\oobnone}{\oobnone}
\oobrgt{21st}{Ohio}{Lieut. Col. Dwella Stoughton}{600}{\oobnone}{\oobnone}
\oobrgt{25th}{Indiana}{Col. James Veatch}{600}{\oobnone}{\oobnone}
\oobrgt{31st}{Indiana}{Col. Charles Cruft}{600}{\oobnone}{\oobnone}
\oobtot{Total Third Brigade}{2,400}{\oobnone}{\oobnone}

\oobsum{Total Second Division}{7,200}{642}{8.92} % {{{6
\oobdblrule

\oobdiv{Artillery}{Lieut. Col.}{Peter Simonson} % {{{5

\oobart{First Division Artillery} % {{{6
\oobrgt{1st}{Kentucky Light Artillery, Battery A}{Capt. David Stone}{6}{\oobnone}{\oobnone}
\oobrgt{1st}{Ohio Light Artillery, Battery F}{Capt. Daniel Cockerill}{6}{\oobnone}{\oobnone}
\oobtot{Total First Division Artillery}{12}{\oobnone}{\oobnone}

\oobart{Second Division Artillery} % {{{6
\oobrgt{4th}{Ohio Independent Battery}{Capt. Louis Hoffman}{6}{\oobnone}{\oobnone}
\oobrgt{1st}{Kentucky Light Artillery, Battery B}{Capt. John Hewitt}{6}{\oobnone}{\oobnone}
\oobtot{Total Second Division Artillery}{12}{\oobnone}{\oobnone}

\oobart{Reserve Artillery} % {{{6
\oobrgt{5th}{Ohio Independent Battery}{Capt. Andrew Hickenlooper}{6}{\oobnone}{\oobnone}
\oobrgt{9th}{Ohio Independent Battery}{Capt. Henry Wetmore}{6}{\oobnone}{\oobnone}
\oobrgt{5th}{Indiana Battery}{Capt. Daniel Chandler}{6}{\oobnone}{\oobnone}
\oobtot{Total Reserve Artillery}{18}{\oobnone}{\oobnone}

\oobsum{Total Artillery}{42}{\oobnone}{\oobnone}
\oobdblrule

% Overall Totals % {{{5
\oobsum{Eighth Corps}{13,800}{642}{4.65}
\oobdblrule

\oobtop{Twelfth Corps}{Brig. Gen.}{Charles Smith} % {{{4

\oobdiv{First Division}{Brig. Gen.}{Thomas Wood} % {{{5

\oobbde{First Brigade}{Brig. Gen.}{James Garfield} % {{{6
\oobrgt{1st}{Ohio}{Col. Benjamin Smith}{600}{110}{18.33}
\oobrgt{8th}{Illinois}{Col. Richard Oglesby}{600}{190}{31.67}
\oobrgt{24th}{Illinois}{Col. Friedrich Hecker}{600}{118}{19.67}
\oobrgt{35th}{Illinois}{Col. Gustavus Smith}{600}{182}{30.33}
\oobtot{Total First Brigade}{2,400}{600}{25.00}

\oobbde{Second Brigade}{Col.}{George Wagner} % {{{6
\oobrgt{15th}{Indiana}{Col. Gustavus Wood}{600}{105}{17.50}
\oobrgt{11th}{Illinois}{Col. William Wallace}{600}{190}{31.67}
\oobrgt{15th}{Illinois}{Col. Thomas Turner}{600}{110}{18.33}
\oobrgt{21st}{Illinois}{Col. Ulysses Grant}{600}{195}{32.50}
\oobtot{Total Second Brigade}{2,400}{600}{25.00}

\oobbde{Third Brigade}{Col.}{John Pope Cook} % {{{6
\oobrgt{40th}{Ohio}{Col. Edwin Bradley}{600}{275}{45.83}
\oobrgt{37th}{Indiana}{Col. George Hazzard}{600}{325}{54.17}
\oobrgt{38th}{Indiana}{Col. Benjamin Scribner}{600}{281}{46.83}
\oobrgt{7th}{Illinois}{Col. Andrew Babcock}{600}{319}{53.17}
\oobtot{Total Second Brigade}{2,400}{1,200}{50.00}

% Overall Totals % {{{6
\oobsum{First Division}{7,200}{2,400}{33.33}
\oobdblrule

\oobdiv{Second Division}{Brig. Gen.}{Horatio Van Cleve} % {{{5

\oobbde{First Brigade}{Col.}{Samuel Beatty} % {{{6
\oobrgt{6th}{Michigan}{Col. Frederick Curtenius}{600}{\oobnone}{\oobnone}
\oobrgt{42d}{Indiana}{Col. James G. Jones}{600}{\oobnone}{\oobnone}
\oobrgt{38th}{Illinois}{Col. William Carlin}{600}{\oobnone}{\oobnone}
\oobrgt{2d}{Minnesota}{Col. James George}{600}{\oobnone}{\oobnone}
\oobtot{Total First Brigade}{2,400}{\oobnone}{\oobnone}

\oobbde{Second Brigade}{Col.}{William Stoughton} % {{{6
\oobrgt{9th}{Michigan}{Col. William Duffeld}{600}{\oobnone}{\oobnone}
\oobrgt{11th}{Michigan}{Lieut. Col. Melvin Mudge}{600}{\oobnone}{\oobnone}
\oobrgt{15th}{Michigan}{Col. John Oliver}{600}{\oobnone}{\oobnone}
\oobrgt{1st}{Wisconsin}{Col. John Starkweather}{600}{\oobnone}{\oobnone}
\oobtot{Total Second Brigade}{2,400}{\oobnone}{\oobnone}

\oobbde{Third Brigade}{Col.}{Jefferson Davis} % {{{6
\oobrgt{8th}{Wisconsin}{Col. George Robbins}{600}{\oobnone}{\oobnone}
\oobrgt{10th}{Wisconsin}{Col. Alfred Chapin}{600}{\oobnone}{\oobnone}
\oobrgt{13th}{Wisconsin}{Col. Maurice Maloney}{600}{\oobnone}{\oobnone}
\oobrgt{35th}{Indiana}{Col. Bernard Mullen}{600}{\oobnone}{\oobnone}
\oobtot{Total Third Brigade}{2,400}{\oobnone}{\oobnone}

% Overall Totals % {{{6
\oobsum{Second Division}{7,200}{\oobnone}{\oobnone}
\oobdblrule

\oobdiv{Artillery}{Lieut. Col.}{Charles Humphrey} % {{{5

\oobart{First Division Artillery} % {{{6
\oobrgt{1st}{Illinois Light Artillery, Battery D}{Capt. Henry Rogers}{6}{\oobnone}{\oobnone}
\oobrgt{2d}{Illinois Light Artillery, Battery E}{Capt. Adolphus Schwartz}{6}{\oobnone}{\oobnone}
\oobtot{Total First Division Artillery}{12}{\oobnone}{\oobnone}

\oobart{Second Division Artillery} % {{{6
\oobrgt{1st}{Michigan Light Artillery, Battery B}{Capt. William Ross}{6}{\oobnone}{\oobnone}
\oobrgt{3d}{Wisconsin Independent Battery}{Capt. Lu Drury}{6}{\oobnone}{\oobnone}
\oobtot{Total Second Division Artillery}{12}{\oobnone}{\oobnone}

\oobart{Reserve Artillery} % {{{6
\oobrgt{5th}{Wisconsin Independent Battery}{Capt. George Gardner}{6}{\oobnone}{\oobnone}
\oobrgt{1st}{Michigan Light Artillery, Battery C}{Capt. Alexander Dees}{6}{\oobnone}{\oobnone}
\oobrgt{1st}{Illinois Light Artillery, Battery A}{Capt. Charles Willard}{6}{\oobnone}{\oobnone}
\oobtot{Total Reserve Artillery}{18}{\oobnone}{\oobnone}

\oobsum{Total Artillery}{42}{\oobnone}{\oobnone}
\oobdblrule

% Overall Totals % {{{5
\oobsum{Twelfth Corps}{14,400}{2,400}{16.67}
\oobdblrule

\oobtop{Fourteenth Corps}{Brig. Gen.}{Charles Smith} % {{{4
& \SetCell{c}The XIVth corps remained in Smithland Ky. and was not present for the engagement.\\
\\ % blank line

\oobtop{Cavalry Division}{Brig. Gen.}{Lawrence Graham} % {{{4

\oobbde{First Brigade}{Col.}{John Bridgeland} % {{{5

\oobrgt{1st}{Kentucky}{Col. Frank Wolford}{297}{77}{25.93}
\oobrgt{3d}{Kentucky}{Col. James Jackson}{298}{79}{26.51}
\oobrgt{2d}{Indiana}{Lieut. Col. Edward McCook}{297}{36}{12.12}
\oobrgt{4th}{United States}{Col. James Oakes}{296}{96}{32.43}
\oobtot{Total First Brigade}{1,188}{288}{24.24}

\oobbde{Second Brigade}{Col.}{Charles Doubleday} % {{{5
\oobrgt{2d}{Ohio}{Col. August Kautz}{295}{2}{0.68}
\oobrgt{3d}{Ohio}{Col. Lewis Zahm}{295}{\oobnone}{0.00}
\oobrgt{4th}{Ohio}{Col. Eli Long}{295}{\oobnone}{0.00}
\oobrgt{2d}{Wisconsin}{Col. Cadwallader Washburn}{298}{2}{0.67}
\oobtot{Total Second Brigade}{1,183}{4}{0.34}

\oobbde{Third Brigade}{Col.}{Theophilus Dickey} % {{{5
\oobrgt{2d}{Illinois}{Col. Silas Noble}{300}{3}{1.00}
\oobrgt{4th}{Illinois}{Col. Martin Wallace}{300}{1}{0.33}
\oobrgt{13th}{Illinois}{Col. Joseph Bell}{300}{1}{0.33}
\oobrgt{1st}{U.S. Lancers (Michigan)}{Col. Arthur Rankin}{300}{3}{1.00}
\oobtot{Total Third Brigade}{1,200}{8}{0.67}

\oobart{Division Artillery} % {{{5
\oobrgt{1st}{Ohio Light Artillery, Battery E}{Capt. Warren Edgarton}{6}{\oobnone}{\oobnone}
\oobrgt{5th}{United States Artillery, Battery H}{Capt. William Terrill}{6}{\oobnone}{\oobnone}
\oobtot{Total Division Artillery}{12}{\oobnone}{\oobnone}

% Overall Totals % {{{5
\oobsum{Cavalry Division}{3,271}{300}{8.40}
\oobdblrule

\oobtop{Artillery Reserve}{Col.}{Charles Cotter} % {{{4
\oobbde{First Brigade}{Lieut. Col.}{William Standart} % {{{5
\oobrgt{1st}{Ohio Light Artillery, Battery A}{Capt. Wilbur Goodspeed}{6}{\oobnone}{\oobnone}
\oobrgt{1st}{Ohio Light Artillery, Battery B}{Capt. J. Hale Sypher}{6}{\oobnone}{\oobnone}
\oobrgt{1st}{Ohio Light Artillery, Battery F}{Capt. Dennis Kenny, Jr.}{6}{\oobnone}{\oobnone}
\oobtot{Total First Brigade Artillery}{18}{\oobnone}{\oobnone}

\oobbde{Second Brigade}{Lieut. Col.}{Alonzo Bidwell} % {{{5
\oobrgt{1st}{Michigan Light Artillery, Battery D}{Capt. Josiah Church}{6}{\oobnone}{\oobnone}
\oobrgt{1st}{Illinois Light Artillery, Battery C}{Capt. Charles Houghtaling}{6}{\oobnone}{\oobnone}
\oobrgt{2d}{Illinois Light Artillery, Battery C}{Capt. Caleb Hopkins}{6}{\oobnone}{\oobnone}
\oobtot{Total Second Brigade Artillery}{18}{\oobnone}{\oobnone}

\oobsum{Total Artillery Reserve}{36}{\oobnone}{\oobnone} % {{{5

% Overall Totals % {{{4
\oobrecap

\oobsub{Eighth Corps}{13,800}{642}{4.65}
\oobsub{Twelfth Corps}{14,400}{2,400}{16.67}

\oobdblrule
\oobsum{Grand total}{28,200}{3,042}{10.79}
\oobdblrule

\oobsub{Cavalry Division}{3,571}{300}{8.40}

\oobdblrule
\oobsum{Grand total}{3,271}{300}{8.40}
\oobdblrule

\oobsub{Eighth Corps Artillery}{42}{\oobnone}{\oobnone}
\oobsub{Twelfth Corps Artillery}{42}{\oobnone}{\oobnone}
\oobsub{Cavalry Division Artillery}{12}{\oobnone}{\oobnone}
\oobsub{Artillery Reserve}{36}{\oobnone}{\oobnone}

\oobdblrule
\oobsum{Grand total}{132}{\oobnone}{\oobnone}

\bottomrule
\end{oob}

\begin{casualties} % {{{4
    \kiawia{Officers Killed}

    \kiawiaState{Illinois}
    Col. Andrew Babcock, 7th Infantry. \\
    Col. William Wallace, 11th Infantry. \\

    \kiawiaState{Indiana}
    Col. George Hazzard, 37th Infantry. \\

    \kiawiaState{Kentucky}
    Col. Frank Wolford, 1st Cavalry. \\

    \vspace{10pt}\kiawia{Officers Wounded}\vspace{5pt}

    Brig. Gen. James Garfield \\
    Brig. Gen. Thomas Wood \\

    \kiawiaState{Illinois}
    Col. Silas Noble, 2d Cavalry. \\
    Col. Richard Oglesby, 8th Infantry. \\
    Col. Ulysses Grant, 21st infantry. \\

    \kiawiaState{Kentucky}
    Col. William Hoskins, 12th Infantry. \\

    \kiawiaState{Ohio}
    Col. Lewis Zahm, 3d Cavalry. \\
    Col. Joshua Sill, 33d Infantry. \\
\end{casualties}
\subsecdinkus

\subsection{Return of casualties in the Union forces % {{{3
    after the engagement at Bufford's Hill, January 28, 1862}

\begin{oob}{}
\oobhdr

\oobtop{Army of the Cumberland}{Maj. Gen.}{James Blake} 

\oobtop{Eighth Corps}{Brig. Gen.}{Ptolemy Smith} % {{{4

\oobdiv{First Division}{Brig. Gen.}{John Wool} % {{{5
\oobbde{First Brigade}{Col.}{Ralph Buckland} % {{{6
\oobrgt{6th}{Ohio}{Col. William Bosley}{\textsuperscript{*}}{\textsuperscript{*}}{\textsuperscript{*}}
\oobrgt{24th}{Ohio}{Col. Frederick Jones}{600}{\oobnone}{\oobnone}
\oobrgt{36th}{Indiana}{Col. William Grose}{600}{\oobnone}{\oobnone}
\oobrgt{3d}{Kentucky}{Col. Thomas Bramlette}{600}{\oobnone}{\oobnone}
\SetCell[c=2]{c}\footnotesize{\textsuperscript{*} The 6th Ohio was guarding the depot at
    Collies Mills, Ky. and not present at the engagement} \\
\oobtot{Total First Brigade}{1,800}{\oobnone}{\oobnone}

\oobbde{Second Brigade}{Col.}{William Hazen} % {{{6
\oobrgt{9th}{US}{Lieut. Col.Stephen Carpenter}{600}{\oobnone}{\oobnone}
\oobrgt{9th}{Indiana}{Col. Gideon Moody}{600}{\oobnone}{\oobnone}
\oobrgt{17th}{Indiana}{Col. Milo Hascall}{600}{\oobnone}{\oobnone}
\oobrgt{39th}{Illinois}{Col. William Morrison}{600}{\oobnone}{\oobnone}
\oobtot{Total Second Brigade}{2,400}{\oobnone}{\oobnone}

\oobbde{Third Brigade}{Col.}{Sanders Bruce} % {{{6
\oobrgt{78th}{Pennsylvania}{Col. William Sirwell}{600}{\oobnone}{\oobnone}
\oobrgt{10th}{Ohio}{Col. William Lytle}{600}{\oobnone}{\oobnone}
\oobrgt{13th}{Ohio}{Lieut. Col. Joseph Hawkins}{600}{\oobnone}{\oobnone}
\oobrgt{7th}{Kentucky}{Col. Reuben May}{600}{\oobnone}{\oobnone}
\oobtot{Total Third Brigade}{2,400}{\oobnone}{\oobnone}

\oobsum{Total First Division}{6,600}{\oobnone}{\oobnone} % {{{6
\oobdblrule

\oobdiv{Second Division}{Brig. Gen.}{Thomas Crittenden} % {{{5

\oobbde{First Brigade}{Brig. Gen.}{Jeremiah Boyle} % {{{6
\oobrgt{19th}{Ohio}{Col. Charles Manderson}{580}{\oobnone}{\oobnone}
\oobrgt{59th}{Ohio}{Col. James Fyffe}{595}{\oobnone}{\oobnone}
\oobrgt{8th}{Kentucky}{Col. Sidney Barnes}{591}{\oobnone}{\oobnone}
\oobrgt{12th}{Kentucky}{Lieut. Col. Laurence Rousseau\textsuperscript{*}}{592}{\oobnone}{\oobnone}
\SetCell[c=2]{c}\footnotesize{\textsuperscript{*} Col. Hoskins was recovering from wounds.} \\
\oobtot{Total First Brigade}{2,358}{\oobnone}{\oobnone}

\oobbde{Second Brigade}{Col.}{William Smith} % {{{6
\oobrgt{31st}{Ohio}{Col. Moses Walker}{490}{160}{32.65}
\oobrgt{33d}{Ohio}{Lieut. Col. Oscar Moore\textsuperscript{*}}{380}{180}{47.37}
\oobrgt{65th}{Ohio}{Col. Charles Harker}{470}{140}{29.79}
\oobrgt{11th}{Indiana}{Col. Lew Wallace}{460}{130}{38.26}
\SetCell[c=2]{c}\footnotesize{\textsuperscript{*} Col. Sill was recovering from wounds.} \\
\oobtot{Total Second Brigade}{1,800}{610}{33.89}

\oobbde{Third Brigade}{Col.}{John Pope Cook} % {{{6
\oobrgt{3d}{Ohio}{Col. Warren Keifer}{600}{\oobnone}{\oobnone}
\oobrgt{21st}{Ohio}{Lieut. Col. Dwella Stoughton}{600}{\oobnone}{\oobnone}
\oobrgt{25th}{Indiana}{Col. James Veatch}{600}{\oobnone}{\oobnone}
\oobrgt{31st}{Indiana}{Col. Charles Cruft}{600}{\oobnone}{\oobnone}
\oobtot{Total Third Brigade}{2,400}{\oobnone}{\oobnone}

\oobsum{Total Second Division}{6,558}{610}{9.30} % {{{6

\oobdblrule

\oobdiv{Artillery}{Lieut. Col.}{Peter Simonson} % {{{5

\oobart{First Division Artillery} % {{{6
\oobrgt{1st}{Kentucky Light Artillery, Battery A}{Capt. David Stone}{6}{\oobnone}{\oobnone}
\oobrgt{1st}{Ohio Light Artillery, Battery F}{Capt. Daniel Cockerill}{6}{\oobnone}{\oobnone}
\oobtot{Total First Division Artillery}{12}{\oobnone}{\oobnone}

\oobart{Second Division Artillery} % {{{6
\oobrgt{4th}{Ohio Independent Battery}{Capt. Louis Hoffman}{6}{\oobnone}{\oobnone}
\oobrgt{1st}{Kentucky Light Artillery, Battery B}{Capt. John Hewitt}{6}{\oobnone}{\oobnone}
\oobtot{Total Second Division Artillery}{12}{\oobnone}{\oobnone}

\oobart{Reserve Artillery} % {{{6
\oobrgt{5th}{Ohio Independent}{Capt. Andrew Hickenlooper}{6}{\oobnone}{\oobnone}
\oobrgt{9th}{Ohio Independent}{Capt. Henry Wetmore}{6}{\oobnone}{\oobnone}
\oobrgt{5th}{Indiana}{Capt. Daniel Chandler}{6}{\oobnone}{\oobnone}
\oobtot{Total Reserve Artillery}{18}{\oobnone}{\oobnone}

\oobsum{Total Artillery}{42}{\oobnone}{\oobnone}
\oobdblrule

\oobsum{Eighth Corps}{13,158}{610}{4.64} % {{{5
\oobdblrule

\oobtop{Twelfth Corps}{Brig. Gen.}{Charles Smith} % {{{4

\oobdiv{First Division}{Brig. Gen.}{William Sherman} % {{{5

\oobbde{First Brigade}{Col.}{Benjamin Smith} % {{{6
\oobrgt{1st}{Ohio}{Maj. Joab Stafford}{490}{\oobnone}{\oobnone}
\oobrgt{8th}{Illinois}{Maj. Edward Johnson}{410}{\oobnone}{\oobnone}
\oobrgt{24th}{Illinois}{Col. Friedrich Hecker}{482}{\oobnone}{\oobnone}
\oobrgt{35th}{Illinois}{Col. Gustavus Smith}{418}{\oobnone}{\oobnone}
\oobtot{Total First Brigade}{1,800}{\oobnone}{\oobnone}

\oobbde{Second Brigade}{Col.}{George Wagner} % {{{6
\oobrgt{15th}{Indiana}{Col. Gustavus Wood}{495}{\oobnone}{\oobnone}
\oobrgt{11th}{Illinois}{Lieut. Col. Thomas Ransom}{410}{\oobnone}{\oobnone}
\oobrgt{15th}{Illinois}{Col. Thomas Turner}{490}{\oobnone}{\oobnone}
\oobrgt{21st}{Illinois}{Lieut. Col. John Alexander}{405}{\oobnone}{\oobnone}
\oobtot{Total Second Brigade}{1,800}{\oobnone}{\oobnone}

\oobbde{Third Brigade}{Col.}{Edwin Bradley} % {{{6
\oobrgt{40th}{Ohio}{Lieut. Col. Jonathon Cranor}{325}{\oobnone}{\oobnone}
\oobrgt{37th}{Indiana}{Lieut. Col. William Ward}{275}{\oobnone}{\oobnone}
\oobrgt{38th}{Indiana}{Col. Benjamin Scribner}{319}{\oobnone}{\oobnone}
\oobrgt{7th}{Illinois}{Lieut. Col. Richard Rowett}{281}{\oobnone}{\oobnone}
\oobtot{Total Second Brigade}{1,200}{\oobnone}{\oobnone}

% Overall Totals % {{{6
\oobsum{First Division}{4,800}{\oobnone}{\oobnone}
\oobdblrule

\oobdiv{Second Division}{Brig. Gen.}{Horatio Van Cleve} % {{{5

\oobbde{First Brigade}{Col.}{Samuel Beatty} % {{{6
\oobrgt{6th}{Michigan}{Col. Frederick Curtenius}{600}{\oobnone}{\oobnone}
\oobrgt{42d}{Indiana}{Col. James G. Jones}{600}{\oobnone}{\oobnone}
\oobrgt{38th}{Illinois}{Col. William Carlin}{600}{\oobnone}{\oobnone}
\oobrgt{2d}{Minnesota}{Col. James George}{600}{\oobnone}{\oobnone}
\oobtot{Total First Brigade}{2,400}{\oobnone}{\oobnone}

\oobbde{Second Brigade}{Col.}{William Stoughton} % {{{6
\oobrgt{9th}{Michigan}{Col. William Duffeld}{600}{\oobnone}{\oobnone}
\oobrgt{11th}{Michigan}{Lieut. Col. Melvin Mudge}{600}{\oobnone}{\oobnone}
\oobrgt{15th}{Michigan}{Col. John Oliver}{600}{\oobnone}{\oobnone}
\oobrgt{1st}{Wisconsin}{Col. John Starkweather}{600}{\oobnone}{\oobnone}
\oobtot{Total Second Brigade}{2,400}{\oobnone}{\oobnone}

\oobbde{Third Brigade}{Col.}{Jefferson Davis} % {{{6
\oobrgt{8th}{Wisconsin}{Col. George Robbins}{600}{\oobnone}{\oobnone}
\oobrgt{10th}{Wisconsin}{Col. Alfred Chapin}{600}{\oobnone}{\oobnone}
\oobrgt{13th}{Wisconsin}{Col. Maurice Maloney}{600}{\oobnone}{\oobnone}
\oobrgt{35th}{Indiana}{Col. Bernard Mullen}{600}{\oobnone}{\oobnone}
\oobtot{Total Third Brigade}{2,400}{\oobnone}{\oobnone}

% Overall Totals % {{{6
\oobsum{Second Division}{7,200}{\oobnone}{\oobnone}
\oobdblrule


\oobdiv{Artillery}{Lieut. Col.}{Charles Humphrey} % {{{5

\oobart{First Division Artillery} % {{{6
\oobrgt{1st}{Illinois Light Artillery, Battery D}{Capt. Henry Rogers}{6}{\oobnone}{\oobnone}
\oobrgt{2d}{Illinois Light Artillery, Battery E}{Capt. Adolphus Schwartz}{6}{\oobnone}{\oobnone}
\oobtot{Total First Division Artillery}{12}{\oobnone}{\oobnone}

\oobart{Second Division Artillery} % {{{6
\oobrgt{1st}{Michigan Light Artillery, Battery B}{Capt. William Ross}{6}{\oobnone}{\oobnone}
\oobrgt{3d}{Wisconsin Independent Battery}{Capt. Lu Drury}{6}{\oobnone}{\oobnone}
\oobtot{Total Second Division Artillery}{12}{\oobnone}{\oobnone}

\oobart{Reserve Artillery} % {{{6
\oobrgt{5th}{Wisconsin Independent Battery}{Capt. George Gardner}{6}{\oobnone}{\oobnone}
\oobrgt{1st}{Michigan Light Artillery, Battery C}{Capt. Alexander Dees}{6}{\oobnone}{\oobnone}
\oobrgt{1st}{Illinois Light Artillery, Battery A}{Capt. Charles Willard}{6}{\oobnone}{\oobnone}
\oobtot{Total Reserve Artillery}{18}{\oobnone}{\oobnone}

\oobsum{Total Artillery}{42}{\oobnone}{\oobnone}
\oobdblrule

\oobsum{Twelfth Corps}{12,000}{\oobnone}{\oobnone} % {{{5
\oobdblrule

\oobtop{Fourteenth Corps}{Brig. Gen.}{John McClernand} % {{{4

\oobdiv{First Division}{Brig. Gen.}{George Thomas} % {{{5

\oobbde{First Brigade}{Col.}{Samuel Carter} % {{{6
\oobrgt{1st}{Kentucky}{Col. David Enyart}{600}{\oobnone}{\oobnone}
\oobrgt{2d}{Kentucky}{Col. Thomas Sedgewick}{600}{\oobnone}{\oobnone}
\oobrgt{1st}{Tennessee}{Col. Robert Byrd}{600}{\oobnone}{\oobnone}
\oobrgt{2d}{Tennessee}{Col. James Carter}{600}{\oobnone}{\oobnone}
\oobtot{Total First Brigade}{2,400}{\oobnone}{\oobnone}

\oobbde{Second Brigade}{Col.}{Madison Miller} % {{{6
\oobrgt{10th}{Indiana}{Lieut. Col. William Kise}{600}{\oobnone}{\oobnone}
\oobrgt{4th}{Kentucky}{Col. John Croxton}{600}{\oobnone}{\oobnone}
\oobrgt{14th}{Ohio}{Col. James Steedman}{600}{\oobnone}{\oobnone}
\oobrgt{17th}{Ohio}{Col. John Connell}{600}{\oobnone}{\oobnone}
\oobtot{Total Second Brigade}{2,400}{\oobnone}{\oobnone}

\oobbde{Third Brigade}{Col.}{Thomas Kilby Smith} % {{{6
\oobrgt{44th}{Illinois}{Col. Charles Knobelsdorff}{600}{\oobnone}{\oobnone}
\oobrgt{9th}{Ohio}{Lieut. Col. Karl Sonderson}{600}{\oobnone}{\oobnone}
\oobrgt{35th}{Ohio}{Col. Ferdinand Van Derveer}{600}{\oobnone}{\oobnone}
\oobrgt{38th}{Ohio}{Lieut. Col. William Choate}{600}{\oobnone}{\oobnone}
\oobtot{Total Third Brigade}{2,400}{\oobnone}{\oobnone}

\oobsum{Total First Division}{7,200}{\oobnone}{\oobnone} % {{{6
\oobdblrule

\oobdiv{Second Division}{Brig. Gen.}{William Rosecrans} % {{{5

\oobbde{First Brigade}{Brig. Gen.}{Benjamin Prentiss} % {{{6
\oobrgt{15th}{United States}{Col. Oliver Shepherd}{600}{\oobnone}{\oobnone}
\oobrgt{6th}{Indiana}{Col. Philemon Baldwin}{600}{\oobnone}{\oobnone}
\oobrgt{77th}{Pennsylvania}{Col. Frederick Stumbaugh}{600}{\oobnone}{\oobnone}
\oobrgt{79th}{Pennsylvania}{Col. Henry Hambright}{600}{\oobnone}{\oobnone}
\oobtot{Total First Brigade}{2,400}{\oobnone}{\oobnone}

\oobbde{Second Brigade}{Col.}{Edward Kirk} % {{{6
\oobrgt{16th}{United States}{Lieut. Col. Edmund Schriver}{600}{\oobnone}{\oobnone}
\oobrgt{29th}{Indiana}{Lieut. Col. David Dunn}{600}{\oobnone}{\oobnone}
\oobrgt{30th}{Indiana}{Col. Sion Bass}{600}{\oobnone}{\oobnone}
\oobrgt{34th}{Illinois}{Lieut. Col. Charles Levanway}{600}{\oobnone}{\oobnone}
\oobtot{Total Second Brigade}{2,400}{\oobnone}{\oobnone}

\oobbde{Third Brigade}{Col.}{William Gibson} % {{{6
\oobrgt{15th}{Ohio}{Col. Moses Dickey}{\textsuperscript{*}}{\textsuperscript{*}}{\textsuperscript{*}}
\oobrgt{5th}{Kentucky}{Col. Harvey Buckley}{\textdagger}{\textdagger}{\textdagger}
\oobrgt{32d}{Indiana}{Lieut. Col. Henry von Trebra}{\textdagger}{\textdagger}{\textdagger}
\oobrgt{39th}{Indiana}{Col. Thomas Harrison}{\textdagger}{\textdagger}{\textdagger}
% TODO: make this a single footnote
\SetCell[c=2]{c}\footnotesize{\textsuperscript{*} The 15th Ohio was detached to
    garrison Evansville, Ind. and was not present for the engagement} \\
\SetCell[c=2]{c}\footnotesize{\textdagger\ The remainder was detached to garrison
    Smithland, Ky. and was not present for the engagement} \\
\oobtot{Total Third Brigade}{\oobnone}{\oobnone}{\oobnone}

\oobsum{Total Second Division}{4,800}{\oobnone}{\oobnone} % {{{6
\oobdblrule

\oobdiv{Artillery}{Lieut. Col.}{Charles Muehler} % {{{5

\oobart{First Division Artillery} % {{{6
\oobrgt{---}{Kentucky Light Artillery, Simmond's Battery}{Capt. Seth Simmonds}{6}{\oobnone}{\oobnone}
\oobrgt{1st}{Ohio Light Artillery, Battery D}{Capt. Andrew Konkle}{6}{\oobnone}{\oobnone}
\oobtot{Total First Division Artillery}{12}{\oobnone}{\oobnone}

\oobart{Second Division Artillery} % {{{6
\oobrgt{4th}{United States Artillery, Battery H}{Lieut. S.  Canby}{6}{\oobnone}{\oobnone}
\oobrgt{4th}{Indiana Battery}{Capt. Asahel Bush}{6}{\oobnone}{\oobnone}
\oobtot{Total Second Division Artillery}{12}{\oobnone}{\oobnone}

\oobart{Reserve Artillery} % {{{6
\oobrgt{4th}{United States, Battery M}{Capt. John Brannan}{6}{\oobnone}{\oobnone}
\oobrgt{1st}{Michigan Light Artillery, Battery A}{Capt. Cyrus Loomis}{6}{3}{50.00}
\oobrgt{---}{Pennsylvania Light Artillery, Battery B}{Capt. Alanson Stevens}{6}{3}{50.00}
\oobtot{Total Reserve Artillery}{18}{6}{33.33}

\oobsum{Total Artillery}{42}{6}{14.29}
\oobdblrule

\oobsum{Fourteenth Corps}{12,000}{\oobnone}{\oobnone} % {{{5
\oobdblrule

\oobtop{Cavalry Division}{Brig. Gen.}{Lawrence Graham} % {{{4

\oobbde{First Brigade}{Col.}{John Bridgeland} % {{{5
\oobrgt{1st}{Kentucky}{Lieut. Col. Silas Adams}{220}{\oobnone}{\oobnone}
\oobrgt{3d}{Kentucky}{Col. James Jackson}{219}{\oobnone}{\oobnone}
\oobrgt{2d}{Indiana}{Lieut. Col. Edward McCook}{261}{\oobnone}{\oobnone}
\oobrgt{4th}{United States}{Col. James Oakes}{200}{\oobnone}{\oobnone}
\oobtot{Total First Brigade}{900}{\oobnone}{\oobnone}

\oobbde{Second Brigade}{Col.}{Charles Doubleday} % {{{5
\oobrgt{2d}{Ohio}{Col. August Kautz}{293}{\oobnone}{\oobnone}
\oobrgt{3d}{Ohio}{Lieut. Col. Horace Howland}{295}{\oobnone}{\oobnone}
\oobrgt{4th}{Ohio}{Col. Eli Long}{295}{\oobnone}{\oobnone}
\oobrgt{2d}{Wisconsin}{Col. Cadwallader Washburn}{296}{\oobnone}{\oobnone}
\oobtot{Total Second Brigade}{1,179}{\oobnone}{\oobnone}

\oobbde{Third Brigade}{Col.}{Theophilus Dickey} % {{{5
\oobrgt{2d}{Illinois}{Lieut. Col. John Mudd\textsuperscript{*}}{297}{\oobnone}{\oobnone}
\oobrgt{4th}{Illinois}{Col. Martin Wallace}{299}{\oobnone}{\oobnone}
\oobrgt{13th}{Illinois}{Col. Joseph Bell}{299}{\oobnone}{\oobnone}
\oobrgt{1st}{U.S. Lancers (Michigan)}{Col. Arthur Rankin}{297}{\oobnone}{\oobnone}
\SetCell[c=2]{c}\footnotesize{\textsuperscript{*} Col. Noble was recovering from wounds.} \\
\oobtot{Total Third Brigade}{1,192}{\oobnone}{\oobnone}

\oobart{Division Artillery} % {{{5
\oobrgt{1st}{Ohio Light Artillery, Battery E}{Capt. Warren Edgarton}{6}{\oobnone}{\oobnone}
\oobrgt{5th}{United States Artillery, Battery H}{Capt. William Terrill}{6}{\oobnone}{\oobnone}
\oobtot{Total Division Artillery}{12}{\oobnone}{\oobnone}

% Overall Totals % {{{5
\oobsum{Cavalry Division}{3,271}{\oobnone}{\oobnone}
\oobdblrule

\oobtop{Artillery Reserve}{Col.}{Charles Cotter} % {{{4
\oobbde{First Brigade}{Lieut. Col.}{William Standart} % {{{5
\oobrgt{1st}{Ohio Light Artillery, Battery A}{Capt. Wilbur Goodspeed}{6}{\oobnone}{\oobnone}
\oobrgt{1st}{Ohio Light Artillery, Battery B}{Capt. J. Hale Sypher}{6}{\oobnone}{\oobnone}
\oobrgt{1st}{Ohio Light Artillery, Battery F}{Capt. Dennis Kenny, Jr.}{6}{\oobnone}{\oobnone}
\oobtot{Total First Brigade Artillery}{18}{\oobnone}{\oobnone}

\oobbde{Second Brigade}{Lieut. Col.}{Alonzo Bidwell} % {{{5
\oobrgt{1st}{Michigan Light Artillery, Battery D}{Capt. Josiah Church}{6}{\oobnone}{\oobnone}
\oobrgt{1st}{Illinois Light Artillery, Battery C}{Capt. Charles Houghtaling}{6}{\oobnone}{\oobnone}
\oobrgt{2d}{Illinois Light Artillery, Battery C}{Capt. Caleb Hopkins}{6}{\oobnone}{\oobnone}
\oobtot{Total Second Brigade Artillery}{18}{\oobnone}{\oobnone}

\oobsum{Total Artillery Reserve}{36}{\oobnone}{\oobnone} % {{{5

% Overall Totals % {{{4
\oobrecap

\oobsub{Eighth Corps}{13,158}{610}{4.64}
\oobsub{Twelfth Corps}{12,000}{\oobnone}{\oobnone}
\oobsub{Fourteenth Corps}{12,000}{\oobnone}{\oobnone}

\oobdblrule
\oobsum{Grand total}{37,148}{610}{1.64}
\oobdblrule

\oobsub{Cavalry Division}{3,271}{\oobnone}{\oobnone}

\oobdblrule
\oobsum{Grand total}{3,571}{\oobnone}{\oobnone}
\oobdblrule

\oobsub{Eighth Corps Artillery}{42}{\oobnone}{\oobnone}
\oobsub{Twelfth Corps Artillery}{42}{\oobnone}{\oobnone}
\oobsub{Fourteenth Corps Artillery}{42}{6}{14.29}
\oobsub{Cavalry Division Artillery}{12}{\oobnone}{\oobnone}
\oobsub{Artillery Reserve}{36}{\oobnone}{\oobnone}

\oobdblrule
\oobsum{Grand total}{174}{6}{3.45}

\bottomrule
\end{oob}
\subsecdinkus

\subsection{Return of casualties in the Union forces % {{{3
    after the engagement at Bufford's Hill \& the Sawmill, February 8, 1862}

\begin{oob}{}
\oobhdr

\oobtop{Army of the Cumberland}{Maj. Gen.}{James Blake} 

\oobtop{Eighth Corps}{Brig. Gen.}{Ptolemy Smith} % {{{4

\oobdiv{First Division}{Brig. Gen.}{John Wool} % {{{5
\oobbde{First Brigade}{Col.}{Ralph Buckland} % {{{6
\oobrgt{6th}{Ohio}{Col. William Bosley}{600}{\oobnone}{\oobnone}
\oobrgt{24th}{Ohio}{Col. Frederick Jones}{600}{\oobnone}{\oobnone}
\oobrgt{36th}{Indiana}{Col. William Grose}{600}{\oobnone}{\oobnone}
\oobrgt{3d}{Kentucky}{Col. Thomas Bramlette}{600}{\oobnone}{\oobnone}
\oobtot{Total First Brigade}{2,400}{\oobnone}{\oobnone}

\oobbde{Second Brigade}{Col.}{William Hazen} % {{{6
\oobrgt{9th}{US}{Col. Stephen Carpenter}{600}{\oobnone}{\oobnone}
\oobrgt{9th}{Indiana}{Col. Gideon Moody}{600}{\oobnone}{\oobnone}
\oobrgt{17th}{Indiana}{Col. Milo Hascall}{600}{\oobnone}{\oobnone}
\oobrgt{39th}{Illinois}{Col. William Morrison}{600}{\oobnone}{\oobnone}
\oobtot{Total Second Brigade}{2,400}{\oobnone}{\oobnone}

\oobbde{Third Brigade}{Col.}{Sanders Bruce} % {{{6
\oobrgt{78th}{Pennsylvania}{Col. William Sirwell}{600}{2}{0.33}
\oobrgt{10th}{Ohio}{Col. William Lytle}{600}{1}{0.17}
\oobrgt{13th}{Ohio}{Col. Joseph Hawkins}{600}{\oobnone}{\oobnone}
\oobrgt{7th}{Kentucky}{Col. Reuben May}{600}{\oobnone}{\oobnone}
\oobtot{Total Third Brigade}{2,400}{3}{0.13}

\oobsum{Total First Division}{7,200}{3}{0.04} % {{{6
\oobdblrule

\oobdiv{Second Division}{Brig. Gen.}{Thomas Crittenden} % {{{5

\oobbde{First Brigade}{Brig. Gen.}{Jeremiah Boyle} % {{{6
\oobrgt{19th}{Ohio}{Col. Charles Manderson}{580}{\oobnone}{\oobnone}
\oobrgt{59th}{Ohio}{Col. James Fyffe}{595}{\oobnone}{\oobnone}
\oobrgt{8th}{Kentucky}{Col. Sidney Barnes}{591}{\oobnone}{\oobnone}
\oobrgt{12th}{Kentucky}{Col. William Hoskins}{592}{\oobnone}{\oobnone}
\oobtot{Total First Brigade}{2,358}{\oobnone}{\oobnone}

\oobbde{Second Brigade}{Col.}{William Smith} % {{{6
\oobrgt{31st}{Ohio}{Col. Moses Walker}{330}{\oobnone}{\oobnone}
\oobrgt{33d}{Ohio}{Lieut. Col. Oscar Moore\textsuperscript{*}}{200}{\oobnone}{\oobnone}
\oobrgt{65th}{Ohio}{Col. Charles Harker}{330}{\oobnone}{\oobnone}
\oobrgt{11th}{Indiana}{Lieut. Col. George McGinnis}{330}{\oobnone}{\oobnone}
\SetCell[c=2]{c}\footnotesize{\textsuperscript{*} Col. Sill was recovering from wounds.} \\
\oobtot{Total Second Brigade}{1,190}{\oobnone}{\oobnone}

\oobbde{Third Brigade}{Col.}{John Pope Cook} % {{{6
\oobrgt{3d}{Ohio}{Col. Warren Keifer}{600}{\oobnone}{\oobnone}
\oobrgt{21st}{Ohio}{Col. Dwella Stoughton}{600}{\oobnone}{\oobnone}
\oobrgt{25th}{Indiana}{Col. James Veatch}{600}{\oobnone}{\oobnone}
\oobrgt{31st}{Indiana}{Col. Charles Cruft}{600}{\oobnone}{\oobnone}
\oobtot{Total Third Brigade}{2,400}{\oobnone}{\oobnone}

\oobsum{Total Second Division}{5,948}{\oobnone}{\oobnone} % {{{6

\oobdblrule

\oobdiv{Artillery}{Lieut. Col.}{Peter Simonson} % {{{5

\oobart{First Division Artillery} % {{{6
\oobrgt{1st}{Kentucky Light Artillery, Battery A}{Capt. David Stone}{6}{\oobnone}{\oobnone}
\oobrgt{1st}{Ohio Light Artillery, Battery F}{Capt. Daniel Cockerill}{6}{\oobnone}{\oobnone}
\oobtot{Total First Division Artillery}{12}{\oobnone}{\oobnone}

\oobart{Second Division Artillery} % {{{6
\oobrgt{4th}{Ohio Independent Battery}{Capt. Louis Hoffman}{6}{\oobnone}{\oobnone}
\oobrgt{1st}{Kentucky Light Artillery, Battery B}{Capt. John Hewitt}{6}{\oobnone}{\oobnone}
\oobtot{Total Second Division Artillery}{12}{\oobnone}{\oobnone}

\oobart{Reserve Artillery} % {{{6
\oobrgt{5th}{Ohio Independent}{Capt. Andrew Hickenlooper}{6}{\oobnone}{\oobnone}
\oobrgt{9th}{Ohio Independent}{Capt. Henry Wetmore}{6}{\oobnone}{\oobnone}
\oobrgt{5th}{Indiana}{Capt. Daniel Chandler}{6}{\oobnone}{\oobnone}
\oobtot{Total Reserve Artillery}{18}{\oobnone}{\oobnone}

\oobsum{Total Artillery}{42}{\oobnone}{\oobnone}
\oobdblrule

\oobsum{Eighth Corps}{13,148}{3}{0.02} % {{{5
\oobdblrule

\oobtop{Twelfth Corps}{Brig. Gen.}{Charles Smith} % {{{4

\oobdiv{First Division}{Brig. Gen.}{William Sherman} % {{{5

\oobbde{First Brigade}{Col.}{Benjamin Smith} % {{{6
\oobrgt{1st}{Ohio}{Maj. Joab Stafford}{490}{\oobnone}{\oobnone}
\oobrgt{40th}{Ohio}{Col. Edwin Bradley}{325}{\oobnone}{\oobnone}
\oobrgt{8th}{Illinois}{Lieut. Col. Richard Rowett}{500}{\oobnone}{\oobnone}
\oobrgt{24th}{Illinois}{Col. Friedrich Hecker}{585}{\oobnone}{\oobnone}
\oobrgt{35th}{Illinois}{Col. Gustavus Smith}{500}{\oobnone}{\oobnone}
\oobtot{Total First Brigade}{2,400}{\oobnone}{\oobnone}

\oobbde{Second Brigade}{Col.}{George Wagner} % {{{6
\oobrgt{15th}{Indiana}{Col. Gustavus Wood}{495}{\oobnone}{\oobnone}
\oobrgt{11th}{Illinois}{Lieut. Col. Thomas Ransom}{410}{\oobnone}{\oobnone}
\oobrgt{15th}{Illinois}{Col. Thomas Turner}{490}{\oobnone}{\oobnone}
\oobrgt{21st}{Illinois}{Lieut. Col. John Alexander}{411}{\oobnone}{\oobnone}
\oobrgt{38th}{Indiana}{Col. Benjamin Scribner}{594}{\oobnone}{\oobnone}
\oobtot{Total Second Brigade}{2,400}{\oobnone}{\oobnone}

% Overall Totals % {{{6
\oobsum{First Division}{4,800}{\oobnone}{\oobnone}
\oobdblrule

\oobdiv{Second Division}{Brig. Gen.}{Horatio Van Cleve} % {{{5

\oobbde{First Brigade}{Col.}{Samuel Beatty} % {{{6
\oobrgt{6th}{Michigan}{Col. Frederick Curtenius}{600}{\oobnone}{\oobnone}
\oobrgt{42d}{Indiana}{Col. James G. Jones}{600}{\oobnone}{\oobnone}
\oobrgt{38th}{Illinois}{Col. William Carlin}{600}{\oobnone}{\oobnone}
\oobrgt{2d}{Minnesota}{Col. James George}{600}{\oobnone}{\oobnone}
\oobtot{Total First Brigade}{2,400}{\oobnone}{\oobnone}

\oobbde{Second Brigade}{Col.}{William Stoughton} % {{{6
\oobrgt{9th}{Michigan}{Col. William Duffeld}{600}{\oobnone}{\oobnone}
\oobrgt{11th}{Michigan}{Col. Melvin Mudge}{600}{\oobnone}{\oobnone}
\oobrgt{15th}{Michigan}{Col. John Oliver}{600}{\oobnone}{\oobnone}
\oobrgt{1st}{Wisconsin}{Col. John Starkweather}{600}{\oobnone}{\oobnone}
\oobtot{Total Second Brigade}{2,400}{\oobnone}{\oobnone}

\oobbde{Third Brigade}{Col.}{Jefferson Davis} % {{{6
\oobrgt{8th}{Wisconsin}{Col. George Robbins}{600}{\oobnone}{\oobnone}
\oobrgt{10th}{Wisconsin}{Col. Alfred Chapin}{600}{\oobnone}{\oobnone}
\oobrgt{13th}{Wisconsin}{Col. Maurice Maloney}{600}{\oobnone}{\oobnone}
\oobrgt{35th}{Indiana}{Col. Bernard Mullen}{600}{\oobnone}{\oobnone}
\oobtot{Total Third Brigade}{2,400}{\oobnone}{\oobnone}

% Overall Totals % {{{6
\oobsum{Second Division}{7,200}{\oobnone}{\oobnone}
\oobdblrule

\oobdiv{Artillery}{Lieut. Col.}{Charles Humphrey} % {{{5

\oobart{First Division Artillery} % {{{6
\oobrgt{1st}{Illinois Light Artillery, Battery D}{Capt. Henry Rogers}{6}{\oobnone}{\oobnone}
\oobrgt{2d}{Illinois Light Artillery, Battery E}{Capt. Adolphus Schwartz}{6}{\oobnone}{\oobnone}
\oobtot{Total First Division Artillery}{12}{\oobnone}{\oobnone}

\oobart{Second Division Artillery} % {{{6
\oobrgt{1st}{Michigan Light Artillery, Battery B}{Capt. William Ross}{6}{\oobnone}{\oobnone}
\oobrgt{3d}{Wisconsin Independent Battery}{Capt. Lu Drury}{6}{\oobnone}{\oobnone}
\oobtot{Total Second Division Artillery}{12}{\oobnone}{\oobnone}

\oobart{Reserve Artillery} % {{{6
\oobrgt{5th}{Wisconsin Independent Battery}{Capt. George Gardner}{6}{\oobnone}{\oobnone}
\oobrgt{1st}{Michigan Light Artillery, Battery C}{Capt. Alexander Dees}{6}{\oobnone}{\oobnone}
\oobrgt{1st}{Illinois Light Artillery, Battery A}{Capt. Charles Willard}{6}{\oobnone}{\oobnone}
\oobtot{Total Reserve Artillery}{18}{\oobnone}{\oobnone}

\oobsum{Total Artillery}{42}{\oobnone}{\oobnone}
\oobdblrule

\oobsum{Twelfth Corps}{12,000}{\oobnone}{\oobnone} % {{{5
\oobdblrule

\oobtop{Fourteenth Corps}{Brig. Gen.}{John McClernand} % {{{4

\oobdiv{First Division}{Brig. Gen.}{George Thomas} % {{{5

\oobbde{First Brigade}{Col.}{Samuel Carter} % {{{6
\oobrgt{1st}{Kentucky}{Col. David Enyart}{600}{6}{1.00}
\oobrgt{2d}{Kentucky}{Col. Thomas Sedgewick}{600}{5}{0.83}
\oobrgt{1st}{Tennessee}{Col. Robert Byrd}{600}{\oobnone}{\oobnone}
\oobrgt{2d}{Tennessee}{Col. James Carter\textsuperscript{*}}{600}{590\textsuperscript{*}}{98.33}
\SetCell[c=2]{c}\footnotesize{\textsuperscript{*} Missing.} \\
\oobtot{Total First Brigade}{2,400}{601}{25.04}

\oobbde{Second Brigade}{Col.}{Madison Miller} % {{{6
\oobrgt{10th}{Indiana}{Col. William Kise}{600}{163}{27.17}
\oobrgt{4th}{Kentucky}{Col. John Croxton}{600}{172}{28.67}
\oobrgt{14th}{Ohio}{Col. James Steedman}{600}{160}{26.67}
\oobrgt{17th}{Ohio}{Col. John Connell}{600}{105}{17.50}
\oobtot{Total Second Brigade}{2,400}{600}{25.00}

\oobbde{Third Brigade}{Col.}{Thomas Kilby Smith} % {{{6
\oobrgt{44th}{Illinois}{Col. Charles Knobelsdorff}{600}{1}{0.17}
\oobrgt{9th}{Ohio}{Col. Karl Sonderson}{600}{\oobnone}{\oobnone}
\oobrgt{35th}{Ohio}{Col. Ferdinand Van Derveer}{600}{1}{0.17}
\oobrgt{38th}{Ohio}{Col. William Choate}{600}{\oobnone}{\oobnone}
\oobtot{Total Third Brigade}{2,400}{2}{0.08}

\oobsum{Total First Division}{7,200}{1,203}{16.71} % {{{6
\oobdblrule

\oobdiv{Second Division}{Brig. Gen.}{William Rosecrans} % {{{5

\oobbde{First Brigade}{Brig. Gen.}{Benjamin Prentiss} % {{{6
\oobrgt{15th}{United States}{Col. Oliver Shepherd}{600}{105}{17.50}
\oobrgt{6th}{Indiana}{Col. Philemon Baldwin}{600}{179}{29.83}
\oobrgt{77th}{Pennsylvania}{Col. Frederick Stumbaugh}{600}{143}{23.83}
\oobrgt{79th}{Pennsylvania}{Col. Henry Hambright}{600}{173}{28.83}
\oobtot{Total First Brigade}{2,400}{600}{25.00}

\oobbde{Second Brigade}{Col.}{Edward Kirk} % {{{6
\oobrgt{16th}{United States}{Col. Edmund Schriver}{600}{\oobnone}{\oobnone}
\oobrgt{29th}{Indiana}{Col. David Dunn}{600}{\oobnone}{\oobnone}
\oobrgt{30th}{Indiana}{Col. Sion Bass}{600}{\oobnone}{\oobnone}
\oobrgt{34th}{Illinois}{Col. Charles Levanway}{600}{\oobnone}{\oobnone}
\oobtot{Total Second Brigade}{2,400}{\oobnone}{\oobnone}

\oobbde{Third Brigade}{Col.}{William Gibson} % {{{6
\oobrgt{15th}{Ohio}{Col. Moses Dickey}{600}{\oobnone}{\oobnone}
\oobrgt{5th}{Kentucky}{Col. Harvey Buckley}{600}{\oobnone}{\oobnone}
\oobrgt{32d}{Indiana}{Col. Henry von Trebra}{600}{\oobnone}{\oobnone}
\oobrgt{39th}{Indiana}{Col. Thomas Harrison}{600}{\oobnone}{\oobnone}
\oobtot{Total Third Brigade}{\oobnone}{\oobnone}{\oobnone}

\oobsum{Total Second Division}{7,200}{600}{8.33} % {{{6
\oobdblrule

\oobdiv{Artillery}{Lieut. Col.}{Charles Muehler} % {{{5

\oobart{First Division Artillery} % {{{6
\oobrgt{---}{Kentucky Light Artillery, Simmond's Battery}{Capt. Seth Simmonds}{6}{3}{50.00}
\oobrgt{1st}{Ohio Light Artillery, Battery D}{Capt. Andrew Konkle}{6}{3}{50.00}
\oobtot{Total First Division Artillery}{12}{6}{50.00}

\oobart{Second Division Artillery} % {{{6
\oobrgt{4th}{United States Artillery, Battery H}{Capt. S.  Canby}{6}{\oobnone}{\oobnone}
\oobrgt{4th}{Indiana Battery}{Capt. Asahel Bush}{6}{\oobnone}{\oobnone}
\oobtot{Total Second Division Artillery}{12}{\oobnone}{\oobnone}

\oobart{Reserve Artillery} % {{{6
\oobrgt{4th}{United States, Battery M}{Capt. John Brannan}{6}{\oobnone}{\oobnone}
\oobrgt{1st}{Michigan Light Artillery, Battery A}{Capt. Cyrus Loomis}{3}{\oobnone}{\oobnone}
\oobrgt{---}{Pennsylvania Light Artillery, Battery B}{Capt. Alanson Stevens}{3}{\oobnone}{\oobnone}
\oobtot{Total Reserve Artillery}{12}{\oobnone}{\oobnone}

\oobsum{Total Artillery}{36}{6}{16.67}
\oobdblrule

\oobsum{Fourteenth Corps}{14,400}{1,803}{12.52} % {{{5
\oobdblrule

\oobtop{Cavalry Division}{Brig. Gen.}{Lawrence Graham} % {{{4

\oobbde{First Brigade}{Col.}{John Bridgeland} % {{{5
\oobrgt{1st}{Kentucky}{Lieut. Col. Silas Adams}{220}{\oobnone}{\oobnone}
\oobrgt{3d}{Kentucky}{Col. James Jackson}{219}{\oobnone}{\oobnone}
\oobrgt{2d}{Indiana}{Col. Edward McCook}{261}{\oobnone}{\oobnone}
\oobrgt{4th}{United States}{Col. James Oakes}{200}{\oobnone}{\oobnone}
\oobtot{Total First Brigade}{900}{\oobnone}{\oobnone}

\oobbde{Second Brigade}{Col.}{Charles Doubleday} % {{{5
\oobrgt{2d}{Ohio}{Col. August Kautz}{293}{\oobnone}{\oobnone}
\oobrgt{3d}{Ohio}{Lieut. Col. Horace Howland}{295}{\oobnone}{\oobnone}
\oobrgt{4th}{Ohio}{Col. Eli Long}{295}{\oobnone}{\oobnone}
\oobrgt{2d}{Wisconsin}{Col. Cadwallader Washburn}{296}{\oobnone}{\oobnone}
\oobtot{Total Second Brigade}{1,179}{\oobnone}{\oobnone}

\oobbde{Third Brigade}{Col.}{Theophilus Dickey} % {{{5
\oobrgt{2d}{Illinois}{Col. Silas Noble}{297}{\oobnone}{\oobnone}
\oobrgt{4th}{Illinois}{Col. Martin Wallace}{299}{\oobnone}{\oobnone}
\oobrgt{13th}{Illinois}{Col. Joseph Bell}{299}{\oobnone}{\oobnone}
\oobrgt{1st}{U.S. Lancers (Michigan)}{Col. Arthur Rankin}{297}{\oobnone}{\oobnone}
\oobtot{Total Third Brigade}{1,192}{\oobnone}{\oobnone}

\oobart{Division Artillery} % {{{5
\oobrgt{1st}{Ohio Light Artillery, Battery E}{Capt. Warren Edgarton}{6}{\oobnone}{\oobnone}
\oobrgt{5th}{United States Artillery, Battery H}{Capt. William Terrill}{6}{\oobnone}{\oobnone}
\oobtot{Total Division Artillery}{12}{\oobnone}{\oobnone}

% Overall Totals % {{{5
\oobsum{Cavalry Division}{3,271}{\oobnone}{\oobnone}
\oobdblrule

\oobtop{Artillery Reserve}{Col.}{Charles Cotter} % {{{4
\oobbde{First Brigade}{Lieut. Col.}{William Standart} % {{{5
\oobrgt{1st}{Ohio Light Artillery, Battery A}{Capt. Wilbur Goodspeed}{6}{\oobnone}{\oobnone}
\oobrgt{1st}{Ohio Light Artillery, Battery B}{Capt. J. Hale Sypher}{6}{\oobnone}{\oobnone}
\oobrgt{1st}{Ohio Light Artillery, Battery F}{Capt. Dennis Kenny, Jr.}{6}{\oobnone}{\oobnone}
\oobtot{Total First Brigade Artillery}{18}{\oobnone}{\oobnone}

\oobbde{Second Brigade}{Lieut. Col.}{Alonzo Bidwell} % {{{5
\oobrgt{1st}{Michigan Light Artillery, Battery D}{Capt. Josiah Church}{6}{\oobnone}{\oobnone}
\oobrgt{1st}{Illinois Light Artillery, Battery C}{Capt. Charles Houghtaling}{6}{\oobnone}{\oobnone}
\oobrgt{2d}{Illinois Light Artillery, Battery C}{Capt. Caleb Hopkins}{6}{\oobnone}{\oobnone}
\oobtot{Total Second Brigade Artillery}{18}{\oobnone}{\oobnone}

\oobsum{Total Artillery Reserve}{36}{\oobnone}{\oobnone} % {{{5

% Overall Totals % {{{4
\oobrecap

\oobsub{Eighth Corps}{13,148}{3}{0.02}
\oobsub{Twelfth Corps}{12,000}{\oobnone}{\oobnone}
\oobsub{Fourteenth Corps}{14,400}{1,803}{12.52}

\oobdblrule
\oobsum{Grand total}{39,548}{1,806}{4.56}
\oobdblrule

\oobsub{Cavalry Division}{3,271}{\oobnone}{\oobnone}

\oobdblrule
\oobsum{Grand total}{3,271}{\oobnone}{\oobnone}
\oobdblrule

\oobsub{Eighth Corps Artillery}{42}{\oobnone}{\oobnone}
\oobsub{Twelfth Corps Artillery}{42}{\oobnone}{\oobnone}
\oobsub{Fourteenth Corps Artillery}{36}{6}{16.67}
\oobsub{Cavalry Division Artillery}{12}{\oobnone}{\oobnone}
\oobsub{Artillery Reserve}{36}{\oobnone}{\oobnone}

\oobdblrule
\oobsum{Grand total}{168}{6}{3.57}

\bottomrule
\end{oob}

\begin{casualties} % {{{4
    \kiawia{Officers Killed}

    \kiawiaState{Indiana}
    Col. William Kise, 10th Infantry. \\

    \vspace{10pt}\kiawia{Officers Wounded}

    \kiawiaState{Kentucky}
    Col. Thomas Sedgewick, 2d Infantry. \\

    \kiawiaState{Ohio}
    Maj. Joab Stafford, 1st Infantry. \\
    Col. James Steedman, 14th Infantry. \\

    \kiawiaState{Regulars}
    Col. Oliver Shepherd, 15th Infantry. \\

    \vspace{10pt}\kiawia{Officers Missing}

    \kiawiaState{Tennessee}
    Col. James Carter, 2d Infantry. \\
    Lieut. Col. Daniel Trewhitt, 2d Infantry. \\
\end{casualties}
\subsecdinkus

\subsection{Return of casualties in the Union forces % {{{3
    after the Battle of Fort Henry \& engagement at Hopewell Church, February 21, 1862}

\begin{oob}{}
\oobhdr

\oobtop{Army of the Cumberland}{Maj. Gen.}{James Blake} 

\oobtop{Eighth Corps}{Brig. Gen.}{Ptolemy Smith} % {{{4

\oobdiv{First Division}{Brig. Gen.}{John Wool} % {{{5
\oobbde{First Brigade}{Col.}{Ralph Buckland} % {{{6
\oobrgt{6th}{Ohio}{Col. William Bosley}{600}{\oobnone}{\oobnone}
\oobrgt{24th}{Ohio}{Col. Frederick Jones}{600}{\oobnone}{\oobnone}
\oobrgt{36th}{Indiana}{Col. William Grose}{600}{\oobnone}{\oobnone}
\oobrgt{3d}{Kentucky}{Col. Thomas Bramlette}{600}{\oobnone}{\oobnone}
\oobtot{Total First Brigade}{2,400}{\oobnone}{\oobnone}

\oobbde{Second Brigade}{Col.}{William Hazen} % {{{6
\oobrgt{9th}{US}{Col. Stephen Carpenter}{600}{\oobnone}{\oobnone}
\oobrgt{9th}{Indiana}{Col. Gideon Moody}{600}{\oobnone}{\oobnone}
\oobrgt{17th}{Indiana}{Col. Milo Hascall}{600}{\oobnone}{\oobnone}
\oobrgt{39th}{Illinois}{Col. William Morrison}{600}{\oobnone}{\oobnone}
\oobtot{Total Second Brigade}{2,400}{\oobnone}{\oobnone}

\oobbde{Third Brigade}{Col.}{Sanders Bruce} % {{{6
\oobrgt{78th}{Pennsylvania}{Col. William Sirwell}{598}{\oobnone}{\oobnone}
\oobrgt{10th}{Ohio}{Col. William Lytle}{599}{\oobnone}{\oobnone}
\oobrgt{13th}{Ohio}{Col. Joseph Hawkins}{600}{\oobnone}{\oobnone}
\oobrgt{7th}{Kentucky}{Col. Reuben May}{600}{\oobnone}{\oobnone}
\oobtot{Total Third Brigade}{2,400}{\oobnone}{\oobnone}

\oobsum{Total First Division}{7,200}{\oobnone}{\oobnone} % {{{6
\oobdblrule

\oobdiv{Second Division}{Brig. Gen.}{Thomas Crittenden} % {{{5

\oobbde{First Brigade}{Brig. Gen.}{Jeremiah Boyle} % {{{6
\oobrgt{19th}{Ohio}{Col. Charles Manderson}{580}{\oobnone}{\oobnone}
\oobrgt{59th}{Ohio}{Col. James Fyffe}{595}{\oobnone}{\oobnone}
\oobrgt{8th}{Kentucky}{Col. Sidney Barnes}{591}{\oobnone}{\oobnone}
\oobrgt{12th}{Kentucky}{Col. William Hoskins}{592}{\oobnone}{\oobnone}
\oobtot{Total First Brigade}{2,358}{\oobnone}{\oobnone}

\oobbde{Second Brigade}{Col.}{William Smith} % {{{6
\oobrgt{31st}{Ohio}{Col. Moses Walker}{330}{\oobnone}{\oobnone}
\oobrgt{33d}{Ohio}{Col. Joshua Sill}{200}{\oobnone}{\oobnone}
\oobrgt{65th}{Ohio}{Col. Charles Harker}{330}{\oobnone}{\oobnone}
\oobrgt{11th}{Indiana}{Lieut. Col. George McGinnis}{330}{\oobnone}{\oobnone}
\oobtot{Total Second Brigade}{1,190}{\oobnone}{\oobnone}

\oobbde{Third Brigade}{Col.}{John Pope Cook} % {{{6
\oobrgt{3d}{Ohio}{Col. Warren Keifer}{600}{\oobnone}{\oobnone}
\oobrgt{21st}{Ohio}{Col. Dwella Stoughton}{600}{\oobnone}{\oobnone}
\oobrgt{25th}{Indiana}{Col. James Veatch}{600}{\oobnone}{\oobnone}
\oobrgt{31st}{Indiana}{Col. Charles Cruft}{600}{\oobnone}{\oobnone}
\oobtot{Total Third Brigade}{2,400}{\oobnone}{\oobnone}

\oobsum{Total Second Division}{5,948}{\oobnone}{\oobnone} % {{{6

\oobdblrule

\oobdiv{Artillery}{Lieut. Col.}{Peter Simonson} % {{{5

\oobart{First Division Artillery} % {{{6
\oobrgt{1st}{Kentucky Light Artillery, Battery A}{Capt. David Stone}{6}{\oobnone}{\oobnone}
\oobrgt{1st}{Ohio Light Artillery, Battery F}{Capt. Daniel Cockerill}{6}{\oobnone}{\oobnone}
\oobtot{Total First Division Artillery}{12}{\oobnone}{\oobnone}

\oobart{Second Division Artillery} % {{{6
\oobrgt{4th}{Ohio Independent Battery}{Capt. Louis Hoffman}{6}{\oobnone}{\oobnone}
\oobrgt{1st}{Kentucky Light Artillery, Battery B}{Capt. John Hewitt}{6}{\oobnone}{\oobnone}
\oobtot{Total Second Division Artillery}{12}{\oobnone}{\oobnone}

\oobart{Reserve Artillery} % {{{6
\oobrgt{5th}{Ohio Independent}{Capt. Andrew Hickenlooper}{6}{\oobnone}{\oobnone}
\oobrgt{9th}{Ohio Independent}{Capt. Henry Wetmore}{6}{\oobnone}{\oobnone}
\oobrgt{5th}{Indiana}{Capt. Daniel Chandler}{6}{\oobnone}{\oobnone}
\oobtot{Total Reserve Artillery}{18}{\oobnone}{\oobnone}

\oobsum{Total Artillery}{42}{\oobnone}{\oobnone}
\oobdblrule

\oobsum{Eighth Corps}{13,148}{\oobnone}{\oobnone} % {{{5
\oobdblrule

\oobtop{Twelfth Corps}{Brig. Gen.}{Charles Smith} % {{{4

\oobdiv{First Division}{Brig. Gen.}{William Sherman} % {{{5

\oobbde{First Brigade}{Col.}{Benjamin Smith} % {{{6
\oobrgt{1st}{Ohio}{Maj. Joab Stafford}{490}{\oobnone}{\oobnone}
\oobrgt{40th}{Ohio}{Col. Edwin Bradley}{325}{\oobnone}{\oobnone}
\oobrgt{8th}{Illinois}{Lieut. Col. Richard Rowett}{500}{\oobnone}{\oobnone}
\oobrgt{24th}{Illinois}{Col. Friedrich Hecker}{585}{\oobnone}{\oobnone}
\oobrgt{35th}{Illinois}{Col. Gustavus Smith}{500}{\oobnone}{\oobnone}
\oobtot{Total First Brigade}{2,400}{\oobnone}{\oobnone}

\oobbde{Second Brigade}{Col.}{George Wagner} % {{{6
\oobrgt{15th}{Indiana}{Col. Gustavus Wood}{495}{\oobnone}{\oobnone}
\oobrgt{11th}{Illinois}{Lieut. Col. Thomas Ransom}{410}{\oobnone}{\oobnone}
\oobrgt{15th}{Illinois}{Col. Thomas Turner}{490}{\oobnone}{\oobnone}
\oobrgt{21st}{Illinois}{Lieut. Col. John Alexander}{411}{\oobnone}{\oobnone}
\oobrgt{38th}{Indiana}{Col. Benjamin Scribner}{594}{\oobnone}{\oobnone}
\oobtot{Total Second Brigade}{2,400}{\oobnone}{\oobnone}

% Overall Totals % {{{6
\oobsum{First Division}{4,800}{\oobnone}{\oobnone}
\oobdblrule

\oobdiv{Second Division}{Brig. Gen.}{Horatio Van Cleve} % {{{5

\oobbde{First Brigade}{Col.}{Samuel Beatty} % {{{6
\oobrgt{6th}{Michigan}{Col. Frederick Curtenius}{600}{\oobnone}{\oobnone}
\oobrgt{42d}{Indiana}{Col. James G. Jones}{600}{\oobnone}{\oobnone}
\oobrgt{38th}{Illinois}{Col. William Carlin}{600}{\oobnone}{\oobnone}
\oobrgt{2d}{Minnesota}{Col. James George}{600}{\oobnone}{\oobnone}
\oobtot{Total First Brigade}{2,400}{\oobnone}{\oobnone}

\oobbde{Second Brigade}{Col.}{William Stoughton} % {{{6
\oobrgt{9th}{Michigan}{Col. William Duffeld}{600}{\oobnone}{\oobnone}
\oobrgt{11th}{Michigan}{Col. Melvin Mudge}{600}{\oobnone}{\oobnone}
\oobrgt{15th}{Michigan}{Col. John Oliver}{600}{\oobnone}{\oobnone}
\oobrgt{1st}{Wisconsin}{Col. John Starkweather}{600}{\oobnone}{\oobnone}
\oobtot{Total Second Brigade}{2,400}{\oobnone}{\oobnone}

\oobbde{Third Brigade}{Col.}{Jefferson Davis} % {{{6
\oobrgt{8th}{Wisconsin}{Col. George Robbins}{600}{\oobnone}{\oobnone}
\oobrgt{10th}{Wisconsin}{Col. Alfred Chapin}{600}{\oobnone}{\oobnone}
\oobrgt{13th}{Wisconsin}{Col. Maurice Maloney}{600}{\oobnone}{\oobnone}
\oobrgt{35th}{Indiana}{Col. Bernard Mullen}{600}{\oobnone}{\oobnone}
\oobtot{Total Third Brigade}{2,400}{\oobnone}{\oobnone}

% Overall Totals % {{{6
\oobsum{Second Division}{7,200}{\oobnone}{\oobnone}
\oobdblrule

\oobdiv{Artillery}{Lieut. Col.}{Charles Humphrey} % {{{5

\oobart{First Division Artillery} % {{{6
\oobrgt{1st}{Illinois Light Artillery, Battery D}{Capt. Henry Rogers}{6}{\oobnone}{\oobnone}
\oobrgt{2d}{Illinois Light Artillery, Battery E}{Capt. Adolphus Schwartz}{6}{\oobnone}{\oobnone}
\oobtot{Total First Division Artillery}{12}{\oobnone}{\oobnone}

\oobart{Second Division Artillery} % {{{6
\oobrgt{1st}{Michigan Light Artillery, Battery B}{Capt. William Ross}{6}{\oobnone}{\oobnone}
\oobrgt{3d}{Wisconsin Independent Battery}{Capt. Lu Drury}{6}{\oobnone}{\oobnone}
\oobtot{Total Second Division Artillery}{12}{\oobnone}{\oobnone}

\oobart{Reserve Artillery} % {{{6
\oobrgt{5th}{Wisconsin Independent Battery}{Capt. George Gardner}{6}{\oobnone}{\oobnone}
\oobrgt{1st}{Michigan Light Artillery, Battery C}{Capt. Alexander Dees}{6}{\oobnone}{\oobnone}
\oobrgt{1st}{Illinois Light Artillery, Battery A}{Capt. Charles Willard}{6}{\oobnone}{\oobnone}
\oobtot{Total Reserve Artillery}{18}{\oobnone}{\oobnone}

\oobsum{Total Artillery}{42}{\oobnone}{\oobnone}
\oobdblrule

\oobsum{Twelfth Corps}{12,000}{\oobnone}{\oobnone} % {{{5
\oobdblrule

\oobtop{Fourteenth Corps}{Brig. Gen.}{John McClernand} % {{{4

\oobdiv{First Division}{Brig. Gen.}{George Thomas} % {{{5

\oobbde{First Brigade}{Col.}{Samuel Carter} % {{{6
\oobrgt{1st}{Kentucky}{Col. David Enyart}{594}{\oobnone}{\oobnone}
\oobrgt{2d}{Kentucky}{Lieut. Col. Laurence Rousseau\textsuperscript{*}}{595}{\oobnone}{\oobnone}
\oobrgt{1st}{Tennessee}{Col. Robert Byrd}{600}{\oobnone}{\oobnone}
\SetCell[c=2]{c}\footnotesize{\textsuperscript{*} Col. Sedgewick was recovering from wounds.} \\
\oobtot{Total First Brigade}{2,400}{\oobnone}{\oobnone}

\oobbde{Second Brigade}{Col.}{Madison Miller} % {{{6
\oobrgt{10th}{Indiana}{Col. William Kise}{437}{\oobnone}{\oobnone}
\oobrgt{4th}{Kentucky}{Col. John Croxton}{428}{\oobnone}{\oobnone}
\oobrgt{14th}{Ohio}{Col. James Steedman}{440}{\oobnone}{\oobnone}
\oobrgt{17th}{Ohio}{Col. John Connell}{495}{\oobnone}{\oobnone}
\oobtot{Total Second Brigade}{2,400}{\oobnone}{\oobnone}

\oobbde{Third Brigade}{Col.}{Thomas Kilby Smith} % {{{6
\oobrgt{44th}{Illinois}{Col. Charles Knobelsdorff}{599}{\oobnone}{\oobnone}
\oobrgt{9th}{Ohio}{Col. Karl Sonderson}{600}{\oobnone}{\oobnone}
\oobrgt{35th}{Ohio}{Col. Ferdinand Van Derveer}{599}{\oobnone}{\oobnone}
\oobrgt{38th}{Ohio}{Col. William Choate}{600}{\oobnone}{\oobnone}
\oobtot{Total Third Brigade}{2,400}{\oobnone}{\oobnone}

\oobsum{Total First Division}{7,200}{\oobnone}{\oobnone} % {{{6
\oobdblrule

\oobdiv{Second Division}{Brig. Gen.}{William Rosecrans} % {{{5

\oobbde{First Brigade}{Brig. Gen.}{Benjamin Prentiss} % {{{6
\oobrgt{15th}{United States}{Lieut. Col. John Kung}{495}{\oobnone}{\oobnone}
\oobrgt{6th}{Indiana}{Col. Philemon Baldwin}{421}{\oobnone}{\oobnone}
\oobrgt{77th}{Pennsylvania}{Col. Frederick Stumbaugh}{457}{\oobnone}{\oobnone}
\oobrgt{79th}{Pennsylvania}{Col. Henry Hambright}{427}{\oobnone}{\oobnone}
\oobtot{Total First Brigade}{2,400}{\oobnone}{\oobnone}

\oobbde{Second Brigade}{Col.}{Edward Kirk} % {{{6
\oobrgt{16th}{United States}{Col. Edmund Schriver}{600}{\oobnone}{\oobnone}
\oobrgt{29th}{Indiana}{Col. David Dunn}{600}{\oobnone}{\oobnone}
\oobrgt{30th}{Indiana}{Col. Sion Bass}{600}{\oobnone}{\oobnone}
\oobrgt{34th}{Illinois}{Col. Charles Levanway}{600}{\oobnone}{\oobnone}
\oobtot{Total Second Brigade}{2,400}{\oobnone}{\oobnone}

\oobbde{Third Brigade}{Col.}{William Gibson} % {{{6
\oobrgt{15th}{Ohio}{Col. Moses Dickey}{600}{\oobnone}{\oobnone}
\oobrgt{5th}{Kentucky}{Col. Harvey Buckley}{600}{\oobnone}{\oobnone}
\oobrgt{32d}{Indiana}{Col. Henry von Trebra}{600}{\oobnone}{\oobnone}
\oobrgt{39th}{Indiana}{Col. Thomas Harrison}{600}{\oobnone}{\oobnone}
\oobtot{Total Third Brigade}{\oobnone}{\oobnone}{\oobnone}

\oobsum{Total Second Division}{7,200}{\oobnone}{\oobnone} % {{{6
\oobdblrule

\oobdiv{Artillery}{Lieut. Col.}{Charles Muehler} % {{{5

\oobart{First Division Artillery} % {{{6
\oobrgt{---}{Kentucky Light Artillery, Simmond's Battery}{Capt. Seth Simmonds}{6}{\oobnone}{\oobnone}
\oobrgt{1st}{Ohio Light Artillery, Battery D}{Capt. Andrew Konkle}{6}{\oobnone}{\oobnone}
\oobtot{Total First Division Artillery}{12}{\oobnone}{\oobnone}

\oobart{Second Division Artillery} % {{{6
\oobrgt{4th}{United States Artillery, Battery H}{Capt. S.  Canby}{6}{\oobnone}{\oobnone}
\oobrgt{4th}{Indiana Battery}{Capt. Asahel Bush}{6}{\oobnone}{\oobnone}
\oobtot{Total Second Division Artillery}{12}{\oobnone}{\oobnone}

\oobart{Reserve Artillery} % {{{6
\oobrgt{4th}{United States, Battery M}{Capt. John Brannan}{6}{3}{50.00}
\oobrgt{1st}{Michigan Light Artillery, Battery A}{Capt. Cyrus Loomis}{3}{3}{100.00}
\oobrgt{---}{Pennsylvania Light Artillery, Battery B}{Capt. Alanson Stevens}{3}{\oobnone}{\oobnone}
\oobtot{Total Reserve Artillery}{12}{6}{50.00}

\oobsum{Total Artillery}{36}{6}{16.67}
\oobdblrule

\oobsum{Fourteenth Corps}{14,400}{\oobnone}{\oobnone} % {{{5
\oobdblrule

\oobtop{Cavalry Division}{Brig. Gen.}{Lawrence Graham} % {{{4

\oobbde{First Brigade}{Col.}{John Bridgeland} % {{{5
\oobrgt{1st}{Kentucky}{Lieut. Col. Silas Adams}{220}{1}{0.45}
\oobrgt{3d}{Kentucky}{Col. James Jackson}{219}{1}{0.46}
\oobrgt{2d}{Indiana}{Col. Edward McCook}{261}{4}{1.53}
\oobrgt{4th}{United States}{Col. James Oakes}{200}{2}{1.00}
\oobtot{Total First Brigade}{900}{8}{0.89}

\oobbde{Second Brigade}{Col.}{Charles Doubleday} % {{{5
\oobrgt{2d}{Ohio}{Col. August Kautz}{293}{63}{21.50}
\oobrgt{3d}{Ohio}{Lieut. Col. Horace Howland}{295}{80}{27.12}
\oobrgt{4th}{Ohio}{Col. Eli Long}{295}{70}{23.73}
\oobrgt{2d}{Wisconsin}{Col. Cadwallader Washburn}{296}{68}{22.97}
\oobtot{Total Second Brigade}{1,179}{281}{23.83}

\oobbde{Third Brigade}{Col.}{Theophilus Dickey} % {{{5
\oobrgt{2d}{Illinois}{Col. Silas Noble}{297}{2}{0.67}
\oobrgt{4th}{Illinois}{Col. Martin Wallace}{299}{5}{1.67}
\oobrgt{13th}{Illinois}{Col. Joseph Bell}{299}{1}{0.33}
\oobrgt{1st}{U.S. Lancers (Michigan)}{Col. Arthur Rankin}{297}{1}{0.34}
\oobtot{Total Third Brigade}{1,192}{9}{0.76}

\oobart{Division Artillery} % {{{5
\oobrgt{1st}{Ohio Light Artillery, Battery E}{Capt. Warren Edgarton}{6}{\oobnone}{\oobnone}
\oobrgt{5th}{United States Artillery, Battery H}{Capt. William Terrill}{6}{\oobnone}{\oobnone}
\oobtot{Total Division Artillery}{12}{\oobnone}{\oobnone}

% Overall Totals % {{{5
\oobsum{Cavalry Division}{3,271}{298}{9.11}
\oobdblrule

\oobtop{Artillery Reserve}{Col.}{Charles Cotter} % {{{4
\oobbde{First Brigade}{Lieut. Col.}{William Standart} % {{{5
\oobrgt{1st}{Ohio Light Artillery, Battery A}{Capt. Wilbur Goodspeed}{6}{\oobnone}{\oobnone}
\oobrgt{1st}{Ohio Light Artillery, Battery B}{Capt. J. Hale Sypher}{6}{\oobnone}{\oobnone}
\oobrgt{1st}{Ohio Light Artillery, Battery F}{Capt. Dennis Kenny, Jr.}{6}{\oobnone}{\oobnone}
\oobtot{Total First Brigade Artillery}{18}{\oobnone}{\oobnone}

\oobbde{Second Brigade}{Lieut. Col.}{Alonzo Bidwell} % {{{5
\oobrgt{1st}{Michigan Light Artillery, Battery D}{Capt. Josiah Church}{6}{\oobnone}{\oobnone}
\oobrgt{1st}{Illinois Light Artillery, Battery C}{Capt. Charles Houghtaling}{6}{\oobnone}{\oobnone}
\oobrgt{2d}{Illinois Light Artillery, Battery C}{Capt. Caleb Hopkins}{6}{\oobnone}{\oobnone}
\oobtot{Total Second Brigade Artillery}{18}{\oobnone}{\oobnone}

\oobsum{Total Artillery Reserve}{36}{\oobnone}{\oobnone} % {{{5

% Overall Totals % {{{4
\oobrecap

\oobsub{Eighth Corps}{13,145}{\oobnone}{\oobnone}
\oobsub{Twelfth Corps}{12,000}{\oobnone}{\oobnone}
\oobsub{Fourteenth Corps}{12,597}{\oobnone}{\oobnone}

\oobdblrule
\oobsum{Grand total}{39,548}{\oobnone}{\oobnone}
\oobdblrule

\oobsub{Cavalry Division}{3,271}{298}{9.11}

\oobdblrule
\oobsum{Grand total}{3,271}{298}{9.11}
\oobdblrule

\oobsub{Eighth Corps Artillery}{42}{\oobnone}{\oobnone}
\oobsub{Twelfth Corps Artillery}{42}{\oobnone}{\oobnone}
\oobsub{Fourteenth Corps Artillery}{36}{\oobnone}{\oobnone}
\oobsub{Cavalry Division Artillery}{12}{\oobnone}{\oobnone}
\oobsub{Artillery Reserve}{36}{\oobnone}{\oobnone}

\oobdblrule
\oobsum{Grand total}{168}{\oobnone}{\oobnone}

\oobtop{Army of the Kanawha}{Maj. Gen.}{Richard Steele} % {{{4

\oobtop{First Division}{Brig. Gen.}{Benjamin Kelley} % {{{4

\oobrgt{}{First Brigade}{}{2,400}{\oobnone}{\oobnone}
\oobrgt{}{Second Brigade}{}{2,400}{\oobnone}{\oobnone}
\oobrgt{}{Third Brigade}{}{2,400}{600}{25.00}
\oobsum{First Division}{7,200}{600}{8.33}
\oobdblrule

\oobtop{Second Division}{Brig. Gen.}{Robert Schenck} % {{{4
\oobrgt{}{First Brigade}{}{2,400}{600}{25.00}
\oobrgt{}{Second Brigade}{}{2,400}{600}{25.00}
\oobrgt{}{Third Brigade}{}{2,400}{600}{25.00}
\oobsum{Second Division}{7,200}{1,800}{25.00}
\oobdblrule

\oobtop{Third Division}{Brig. Gen.}{Robert Milroy} % {{{4
\oobrgt{}{First Brigade}{}{2,400}{\oobnone}{\oobnone}
\oobrgt{}{Second Brigade}{}{2,400}{\oobnone}{\oobnone}
\oobrgt{}{Third Brigade}{}{2,400}{600}{25.00}
\oobsum{Third Division}{7,200}{600}{8.33}
\oobdblrule

\oobtop{Cavalry Division}{Brig. Gen.}{Hall} % {{{4
\oobrgt{}{First Brigade}{}{1,200}{300}{25.00}
\oobrgt{}{Second Brigade}{}{1,200}{300}{25.00}
\oobsum{Cavalry Division}{2,400}{600}{25.00}
\oobdblrule

\oobtop{Artillery}{Col.}{???} % {{{4
\oobrgt{}{First Division Artillery}{}{12}{\oobnone}{\oobnone}
\oobrgt{}{Second Division Artillery}{}{12}{\oobnone}{\oobnone}
\oobrgt{}{Third Division Artillery}{}{12}{\oobnone}{\oobnone}
\oobrgt{}{Cavalry Division Artillery}{}{12}{\oobnone}{\oobnone}
\oobrgt{}{Reserve Artillery}{}{18}{6}{33.33}
\oobsum{Total Artillery}{66}{6}{9.09}

% Overall Totals % {{{4
\oobrecap

\oobsub{First Division}{7,200}{600}{8.33}
\oobsub{Second Division}{7,200}{1,800}{25.00}
\oobsub{Third Division}{7,200}{600}{8.33}

\oobdblrule
\oobsum{Grand total}{21,600}{3,000}{13.89}
\oobdblrule

\oobsub{Cavalry Division}{2,400}{600}{25.00}

\oobdblrule
\oobsum{Grand total}{2,400}{600}{25.00}
\oobdblrule

\oobsub{First Division Artillery}{12}{\oobnone}{\oobnone}
\oobsub{Second Division Artillery}{12}{\oobnone}{\oobnone}
\oobsub{Third Division Artillery}{12}{\oobnone}{\oobnone}
\oobsub{Cavalry Division Artillery}{12}{\oobnone}{\oobnone}
\oobsub{Artillery Reserve}{18}{6}{33.33}

\oobdblrule
\oobsum{Grand total}{66}{6}{9.09}
\bottomrule
\end{oob}

\subsecdinkus

\subsection{Reports of Maj. Gen. James Blake, U.S. Army, commanding Army of the Cumberland} % {{{3
\gramCircular{Headquarters, Army of the Cumberland} % {{{4
    {In the field north of Fort Donelson, Tenn., January}{24, 1862}

This Army's goal is to open the Tennessee and Cumberland Rivers by clearing Fort
Henry on the Tennessee River and Fort Donelson on the Cumberland River.  Earlier
in the month our cavalry skirmished with enemy cavalry of approximately division
size at Golden Pond, Kentucky, about 20 miles to the north. VIIIth and XIIth
Corps have since landed at Golden Pond and marched south towards Fort Donelson.

Yesterday the cavalry spotted enemy positions east of Hickman's Creek. They
estimate 10,000--20,000 men, including the garrison at Fort Donelson. The enemy
force is believed to be under the command of Maj. Gen. Whisper although Maj.
Gen. Clarke may also be in the area. The navy has previously reported around a
dozen cannon defending the river approaches to the fort.

We will begin movement tomorrow to isolate Fort Donelson and the town of Dover,
intending to force the enemy to withdraw or be surrounded. I have ordered XIVth
Corps to move here by boat from Smithland but they will take about three days to
arrive. The navy has sent a small gunboat flotilla that can provide supporting
fire against the fort for about three hours before they will need to retire to
rearm.

I intend to be aggressive but I do not want to risk unnecessary casualties.
Under no circumstance will the fort itself be assaulted without direct orders
from me. Attacks against defensive works outside the fort should be conducted
only if absolutely necessary.

I intend to be with the XIIth Corps commander for the majority of the battle. He
will take my place in the event I am killed or wounded or otherwise unable to
provide guidance.

\gramClosingBy{Maj. Gen. J. W. Blake}
    {Walter Chekov}
    {Col., Adjutant General}
(To Generals McClernand, Smith, Smith and Graham)

\reportdinkus

\gramHeader{Headquarters, Army of the Cumberland} % {{{4
    {7.15~am January}{25, 1862}
\gramTo{Brig. Gen.}{Charles Smith}
    {Commander, XIIth Corps}

\gramHi{General} I need 2d Division to move up behind Gen'l Wood's division.
Gen'l Wood will determine if the direction of his attack needs to be altered.

\gramClosing{Respectfully}
    {J. W. Blake}
    {Maj. Gen., Commanding}
\reportdinkus

\gramHeader{Headquarters, Army of the Cumberland} % {{{4
    {7.45~am January}{25, 1862}
\gramTo{Brig. Gen.}{Lawrence Graham}
    {Commander, Cavalry Division}

\gramHi{General} Enemy dug in on the hills north and south of the ford.  What is
the state of enemy defenses in the south? Is the forest south of the ford
occupied?

\gramClosing{Respectfully}
    {J. W. Blake}
    {Maj. Gen., Commanding}
\reportdinkus

\gramHeader{Headquarters, Army of the Cumberland} % {{{4
    {8.00~am January}{25, 1862}
\gramTo{Brig. Gen.}{Lawrence Graham}
    {Commander, Cavalry Division}

\gramHi{General} We are attempting to force the northern ford. At this time I am
not going to commit the reserves. Has your 1st Brigade arrived yet?

Do your best to keep the enemy occupied. If he advances on you he is no longer
holding the ford

\gramClosing{Respectfully}
    {J. W. Blake}
    {Maj. Gen., Commanding}
\reportdinkus

\gramHeader{Headquarters, Army of the Cumberland} % {{{4
    {8.15~am January}{25, 1862}
\gramTo{Brig. Gen.}{Lawrence Graham}
    {Commander, Cavalry Division}

\gramHi{General} Can you confirm that you see over 15 regiments of infantry? If
it is that many I am going to send 1st Division, VIIIth Corps to support you.

Regardless, do your best to preserve your valuable cavalry but please keep us
informed if an attack is developing from the south.

\gramClosing{Respectfully}
    {J. W. Blake}
    {Maj. Gen., Commanding}
\reportdinkus

\gramHeader{Headquarters, Army of the Cumberland} % {{{4
    {8.15~am January}{25, 1862}
\gramTo{Brig. Gen.}{Lawrence Graham}
    {Commander, Cavalry Division}

\gramHi{General} You are authorized to withdraw to the T-intersection to your
north. 2d Division, XII Corps is going to move to cover our right flank

\gramClosing{Respectfully}
    {J. W. Blake}
    {Maj. Gen., Commanding}
\reportdinkus

\gramHeader{Headquarters, Army of the Cumberland} % {{{4
    {8.15~am January}{25, 1862}
\gramTo{Brig. Gen.}{John Wool}
    {Commander, 1st Division, VIIIth Corps}

\gramHi{General} Move up on the double and report to Brig. Gen. van Cleve's 2d
Division, XIIth Corps who you will assist in defense of our right flank

\gramClosing{Respectfully}
    {J. W. Blake}
    {Maj. Gen., Commanding}
\reportdinkus

\gramHeader{Headquarters, Army of the Cumberland} % {{{4
    {---~am January}{25, 1862}
\gramTo{Brig. Gen.}{Lawrence Graham}
    {Commander, Cavalry Division}

\gramHi{General} Can you fall back north of the T-intersection and reform? We
have taken the northern redoubt and I am also forming two divisions along the
east-west road to cover our flank.

\gramClosing{Respectfully}
    {J. W. Blake}
    {Maj. Gen., Commanding}
\reportdinkus

\gramCircular{Headquarters, Army of the Cumberland} % {{{4
{9.45 am, Field Headquarters north of Fort Donelson, Tenn., January}{25, 1862}

XIIth Corps' attack on the enemy works across Hickman' Creek has been halted
with Wood's Division having suffered heavy loss. Crittenden's Division of VIIIth
Corps has also suffered loss and needs time to reorganize. Gen'l Graham's
cavalry has done yeoman's work slowing the enemy advance from the south but is
also in need of a rest. Multiple batteries of XIIth Corps have less than 30
minutes of ammunition remaining.

Granbury's Division of the Army of Mississippi has been identified holding the
redoubts across Hickman's Creek while two divisions of the Army of East
Tennessee have been pressing our cavalry from the south. One or two more
divisions are likely present but have not yet been identified.

The urgent task now is to pull shattered units back to reorganize and establish
a good defense around the northern ford across Hickman's Creek.

XIIth Corps' artillery is to pull back to a reserve position where it will only
fire if absolutely necessary. Wood's Division, XIIth Corps and Boyle's and
Smith's Brigades of VIIIth Corps will also pull back north to reorganize.

Gen'l Graham's cavalry is to move to the west flank to cover against any approach
from Fort Henry.

The Artillery Reserve will withdraw to the northern hill, protected by Sherman's
Brigade, VIIIth Corps to defend the northern ford. Van Cleve's Division, XIIth
Corps will shift its lines to assist.

If the enemy attacks in the south, I believe we can hold. If the enemy attacks
in the north we may be able to hold if given time to establish the defense.
Retreat from the field is acceptable if pushed too hard. We can return in a few
days when XIVth Corps arrives.

If the enemy does not attack by noon, skirmishers are to push onto the large
hill to the south and to determine if the enemy has withdrawn. If they have,
push forward and secure the west side of southern ford. Do not launch a full
attack as we cannot afford anymore losses today.

\gramClosingBy{Maj. Gen. James Blake}
    {Walter Chekov}
    {Col., Adjutant General}
(To Generals McClernand, Smith, Smith and Graham)

\reportdinkus

\gramHeader{Headquarters, Army of the Cumberland} % {{{4
    {Field Headquarters, Near Fort Donelson, Tenn., January}{26, 1862}
\gramTo{Maj. Gen.}{Cornelius Van Royne}
    {United States Army, Commanding}

\gramHi{General} Two corps of my army engaged the enemy near Fort Donelson,
Tennessee yesterday. We attempted to gain a foothold in their fortifications but
were repulsed. My losses are approximately 3,000 infantrymen and
300 cavalry troopers. Enemy losses are unknown but I estimate them to be around
    1,000 infantry.

We have identified three enemy divisions: Whisper's Division of the Army of East
Tennessee and another division of the same army as well as Granbury's Division
of the Army of Mississippi. I believe Gen'ls. Whisper and Clarke to have been
present at the battle. I believe it likely the enemy has a fourth division in
reserve, as yet unseen but presumably from the Army of Mississippi.

I await the arrival of my XIV Corps tomorrow and hope to resume operations on
the 28th. If the enemy is not similarly reinforced I believe I can carry the
position although it may take some time as the enemy seems intent on holding
this ground.

\gramClosing{Respectfully}
    {J. W. Blake}
    {Maj. Gen., Commanding}
\reportdinkus

\gramHeader{Headquarters, Army of the Cumberland} % {{{4
    {Field  Headquarters, Near Fort Donelson, Tenn., January}{27, 1862}
\gramTo{Maj. Gen.}{Cornelius Van Royne}
    {United States Army, Commanding}

\gramHi{General} I regret that my previous report on estimated enemy strength
was incorrect.  I believe Gen'l Whisper's Army of Tennessee to number 7,000 to
9,000 while each wing of Gen'l Clarke's Army of Mississippi is 10,000 to 12,000.
This puts the total enemy strength at 27,000 to 33,000.

\gramClosing{Respectfully}
    {J. W Blake}
    {Maj. Gen., Commanding}
\reportdinkus

\gramOrdersHeader{Headquarters, Army of the Cumberland} % {{{4
{10.30 am, Field Headquarters near Fort Donelson, Tenn., January}{28, 1862}
{Special Field Orders}{10}

This Army's goal continues to be to open the Tennessee and Cumberland Rivers by
clearing the Fort Henry on the Tennessee River and Fort Donelson on the
Cumberland River.  Three days ago (the 25th instant) a total of five regiments
of infantry were lost attempting to create a breach in the enemy defenses along
Hickman's Creek. At that time the enemy had approximately four divisions of
infantry on the field: two from Gen'l Whisper's Army of East Tennessee and two
from the Right Wing of Gen'l Clarke's Army of Mississippi.

Since then at least one division from the Left Wing of the Army of Mississippi
has been reported to have arrived here. Gen'l McClernand's XIVth Corps has also
arrived on the field and, we believe, has not been identified by the enemy prior
to sometime this morning.

As of two days ago, the enemy had two divisions of the Army of East Tennessee on
what is now known as Whisper's Hill and three to four divisions defending the
direct approaches to Fort Donelson.

We expected the enemy to take advantage of our perceived weakness and attempt to
attack our western flank and it seems the enemy has obliged.

At 7.30 this morning, cavalry patrols spotted two to three divisions moving
north along Beaver Creek. XIVth Corps was ordered to move to defend the direct
approach to our lines and the road leading to our depot along the river. XIVth
Corps has now been engaged for only a few minutes and we are still uncertain as
to what enemy troops they are engaging and how many.

However, pickets seem to indicate that Whisper's Hill and the approaches to Fort
Donelson are now held by the Right Wing of the Army of Mississippi which would
mean the Army of East Tennessee and Left Wing of the Army of Mississippi
are moving on XIVth Corps with three to four divisions.

Our objective is preservation of our own force while causing maximum damage to
the enemy. Holding or seizing ground is, in general, not as important.

\centerline{\textbf{Plan of Battle}}
\vspace{5pt}

XIVth Corps will defend, falling back to the east slightly if needed, although
they may advance if the situation is advantageous. Gen'l McClernand must prevent
the enemy from either driving on the river landing or cutting the Happy Hollow
Road as the rest of the army will need this road for any retreat to the north.
The 6th Ohio constructing the depot to the east is to move west to support XIVth
Corps.

Van Cleve's Division, XIIth Corps may withdraw from their positions if necessary
as holding that ground is not critical.

Bridgeland's Cavalry Brigade in the north should scout the enemy left to
determine the number of troops and their identity. Avoid a general engagement.

The remaining cavalry in the south is to scout both sides of Whisper's Hill to
both identify what enemy forces are present and to make the enemy think we are
intending to attack them in the south. Avoid a general engagement.

The reserve brigade of Cook's Division, VIIIth Corps should be sent to support
XIVth Corps. The remainder of VIIIth Corps and Sherman's Division, XIIth Corps
should attempt to hold if possible but must be ready to support XIVth Corps if
required. If the cavalry identifies that Whisper's Hill is lightly held, those
troops should consider a probing attack to seize the hill but must not become so
engaged that they cannot assist XIVth Corps.

I intend to be with Gen'l McClernand for now. He will take my place in the event
I am killed or wounded or otherwise unable to command. It is important to note
that Sherman's Division, XIIth Corps is quite weak---its brigades are three,
three and two infantry regiments respectively.

\gramClosingBy{Maj. Gen. James Blake}
{Walter Chekov}
{Col., Adjutant General}
\reportdinkus

\gramHeader{Headquarters, Army of the Cumberland} % {{{4
    {Field Headquarters, Near Fort Donelson, Tenn., January}{28, 1862}
\gramTo{Maj. Gen.}{Cornelius Van Royne}
    {United States Army, Commanding}

\gramHi{Sir} As expected, the enemy launched an attack on our right flank. I
redeployed XIVth Corps to engage the enemy and then spent the morning driving
back three divisions of the enemy; Whisper's Division of the Army of East
Tennessee and two divisions of the Left Wing of the Army of Mississippi were
spotted.

Our losses are a mere 600 infantry and a battery of artillery. I expect enemy
casualties to be significantly higher due to our excellent positions.

I believe it may be wise for me to withdraw slightly, given the preponderance of
enemy force arrayed against me. This may even induce the enemy to attack me
again. The terrain here is much more favorable to defense than offense.

I will also consider ways to strike Fort Henry as I believe it to be defended
only by its garrison.

\gramClosing{Respectfully}
    {J. W. Blake}
    {Maj. Gen., Commanding}
\reportdinkus

\gramHeader{Headquarters, Army of the Cumberland} % {{{4
    {Near Fort Donelson, February 3,}{1862}
\gramTo{Maj. Gen.}{Cornelius Van Royne}
    {United States Army, Commanding}

\gramHi{General} I have the honor to submit the following report of the
engagements near Fort Donelson, Tenn., on Saturday and Tuesday, the 25th and
28th, respectively, of January, 1862.

In response to the enemy seizure of Paducah on the 5th of January, my Cavalry
Division moved south from Smithland three weeks prior to the planned start of
our campaign to clear the Cumberland and Tennessee Rivers. After the cavalry was
repulsed at Golden Pond  by enemy cavalry, I resolved to begin the campaign as
soon as I could deploy my forces from Evansville, Ind. to Smithland,
Ky.

On the 23d of January, all three corps having arrived in Kentucky, XIVth Corps
remained in Smithland while VIIIth and XIIth Corps, along with the Army
Artillery Reserve moved up the Cumberland river by barge to join the cavalry
which had, by then, occupied Golden Pond, it having been abandoned by the enemy
in the interim.

Upon the arrival of the infantry on the 24th, the cavalry was sent towards Fort
Donelson to reconnoiter the enemy positions previously identified by the Navy;
Brig. Gen. Graham reported that two divisions of the enemy were entrenched on
the eastern bank of Hickman's Creek, just to the west of Fort Donelson and
stated he was certain he had seen the flag of Maj. Gen. Clarke's Army of
Mississippi, and thus assumed that general to be present at Fort Donelson.

Believing the enemy to be outnumbered, I resolved to move into position just
west of the enemy positions and launch an attack at dawn.  Cdre. Daniel Lewis
was requested to provide a diversionary bombardment of Fort Donelson to which he
readily acquiesced. XIVth Corps was also ordered to depart Smithland (leaving
three regiments behind as garrison, one already having been detached in
Evansville) en route to our position.

All units present, save the 6th Ohio guarding our supply train at Collies Mills,
Ky., were ordered to break camp a few hours prior to dawn and begin marching
south towards the enemy.  Brig. Gen. Charles Smith's XIIth Corps was selected to
lead the attack across the creek, Brig. Gen. Thomas Wood's 1st Division in the
van, with Brig. Gen. Thomas Crittenden's 2d Division of Brig. Gen.  Ptolemy
Smith's VIIIth Corps, along with Col. Cotter's Army Artillery Reserve providing
support from the hill on the north side of the road leading to the ford. Brig.
Gen. John Wool's 1st Division of VIIIth Corps was to be in reserve. At the same
time, Brig.  Gen. Graham's cavalry was to demonstrate in front of the southern
ford of the creek in order to keep the enemy in the entrenchments facing him in
their positions, uncertain as to our intentions.

The attack proper began at around 8 o'clock with the 1st and 2d Brigades of 1st
Division, XIIth Corps, commanded by Brig. Gen. Garfield and Col. Wagner,
respectively, attacking together, with Col. Cook's 3d Brigade held in reserve.
Col.  Cotter's batteries, deployed to the north, performed quite well in their
duty but the enemy position was more strongly held than previously reported.

Brig. Gen. Woods was wounded during the assault as was the 1st Brigade
commander, Brig. Gen. James Garfield. All told, losses among the division were
severe, to include three regimental commanders killed by the murderous enemy
fire. Seeing that the attacking regiments had, however, reached the enemy
positions, Brig. Gen.  Ptolemy Smith quickly sent two brigades of his 2d
Division in to support the attack. The enemy redoubt north of the crossing was
seized, albeit temporarily, as a vicious enemy counterattack soon drove our
forces back across the creek at about 9.20~am.

Meanwhile, at around 8.15~am the cavalry had reported an enemy attack by
two divisions from the entrenchments facing them. Brig. Gen. Graham's troopers
performance was exemplary, more than redeeming their lackluster performance at
Golden Pond. However, weight of numbers drove them back northwards, but not
before 1st Division, VIIIth Corps and 2d Division, XIIth Corps, commanded by
Brig. Gen'ls. John Wool and Horatio van Cleve, respectively, were able to form a
line to halt the enemy advance.

As I believed these divisions to be, as yet, unseen by the enemy, I ordered the
cavalry to feign a rout when approached by the advancing infantry, hoping to
draw the enemy in against my fresh divisions. This they did, but the enemy,
likely spent by over an hour of being harassed by the cavalry, rapidly retreated
back up the hill after a volley from Brig. Gen. Graham's dismounted cavalrymen.

After a period of reorganization, skirmishers were sent forward to ascertain the
enemy position and reported the large hill to the south was strongly held by two
divisions of the enemy. Given the grievous losses sustained by the forces
attacking the redoubts in the north and that some of our artillery batteries
were beginning to run short of ammunition, the decision was made to halt for the
day.

As stated previously, our cavalry performed quite well although 300 troopers
were lost in their retreat north, including Col. Wolford, commander of the 1st
Kentucky Cavalry Regiment. Loss to the infantry was 3,042 men, including Col's.
Andrew Babcock (7th Illinois), William Wallace (11th Illinois) and George
Hazzard (37th Indiana). In addition, Col's. Lewis Zahm (3d Ohio) and Ulysses
Grant (21st Illinois) were severely wounded and sent back to Smithland for
further care.  Lieut. Col. John Alexander, who replaced Col. Grant, personally
informed me that he saw his commander in front of the regiment, inspiring them
forward by his own disregard for the danger. I cannot spare officers such as
Col. Grant; such brave men have shown a devotion to our cause and willingness to
fight that cannot be matched.

Despite our grievous loss, my army still held its position just west of Fort
Donelson although the loss of the large hill to our south (now referred to as
``Whisper's Hill'' as it was held by Whisper's Division of the Army of East
Tennessee) complicated our situation as enemy troops on said hill could deny any
approach to Fort Donelson from the west. At this point we estimated the enemy to
number three or four divisions, the previously mentioned Whisper's Division,
Granbury's Division of Brig. Gen. Vexley's Right Wing of the Army of Mississippi
and one or two additional that had not been identified. I did not believe that
my army had the strength to drive the enemy away from the fort, at least until
the arrival of Brig. Gen. John McClernand's XIVth Corps', expected late on the
27th.

This corps did arrive as expected and, its movements being successfully screened
by my cavalry (of which arm the enemy, most surprisingly, had none), we remained
quite confident that the enemy was not aware that our army had been thus
reinforced. Unfortunately my cavalry reported that same day that some portion of
Brig. Gen. Pickford's Left Wing of the Army of Mississippi had also arrived on
the field. With this information in hand my Chief of Staff, Brig. Gen. Harold
Fawcett submitted that the enemy most likely felt he had us outnumbered and
would attempt to drive us from the field. This seemed most likely and, given the
rough nature of the ground, I resolved to tempt the enemy to attack before he
could learn of the arrival of XIVth Corps; VIIIth and XIIth Corps began
construction of field fortifications (which have now been completed) while the
cavalry continued to screen my right flank and XIVth Corps. At the same time I
ordered the 6th Kentucky Regiment to move south from Collies Mills to construct
a supply depot closer to my lines, about two miles to the north along the
Cumberland River.

At 7.30~am on January 28th, our cavalry reported two to three enemy
divisions moving north along Bear Creek, approximately two and a half miles to
the west of my positions. In response I began deploying XIVth Corps to my right
with Brig. Gen. George Thomas' 1st Division on the north slope of Bufford's Hill
and Brig. Gen. William Rosecrans' 2d Division just east of the same hill,
supported by Lieut. Col. Alonzo Bidwell's 2d Brigade of Artillery. Col. John
Bridgeland's cavalry brigade continued to shadow the enemy's slow movement along
the creek; the remaining cavalry maintained position screening and scouting the
western flank of Whisper's Hill.

At 10.00~am, Brig. Gen. Rosecrans' division spotted the head of the enemy
column cresting Bufford's Hill at Bufford Cemetery, headed directly towards his
position. With a preponderance of artillery at his disposal---not less
than eight batteries could be directly brought to bear---Brig.  Gen.
Rosecrans held fire, waiting for more enemy troops to present themselves.
Shortly following, 1st Division observed the enemy approaching from their
southwest around Hopewell Church and from the south across Bufford's hill; both
divisions soon opened fire.

Col. William Smith's 2d Brigade of 2d Division, VIIIth Corps was ordered to move
to support Brig. Gen. Rosecrans position east of Bufford's Hill; shortly after,
Brig. Gen.  Jeremiah Boyle's 1st Brigade was sent to cover XIVth Corps' right
flank while Brig. Gen. William Sherman's 3d Brigade occupied the hill just
south of Trinity Church in order to secure the right flank of the line facing
Whisper's Hill. This proved to be fortuitous as cavalry scouting south towards
Fitzhugh Cemetery soon reported the enemy extending the line southwest from
Whisper's Hill; no attack was in the offing but without Brig. Gen. Sherman's
movement to the right, the enemy may have been able to drive between our western
and southern positions.

The engagement in the west began in earnest around 10.30~am with 1st
Division driving down the road towards Hopewell Church with heavy fighting
around the Sawmill to the north of Bufford's Hill. Less fighting occurred on the
east slope of that hill as the enemy concentrated his attack against 1st
Division; therefore, at around 12.30~am, Brig. Gen. Rosecrans began sending
his division, supported by the brigade from VIIIth Corps, against the hill.
Movement of both divisions onto and across the hill was slowed significantly by
the steep slopes and dense woods, but the enemy continued to give ground. Our
advance halted about 1.00~am in the afternoon when the enemy determined to
retreat no further and successfully engaged Col. Smith's Brigade with enfilading
fire.

All told our losses were 610 infantry and six cannon; enemy infantry losses are
expected to have been at least double our own. Once prisoners were collected and
questioned it was determined that XIVth Corps had been engaged by three
divisions: Whisper's Division of the Army of East Tennessee and Breckinridge's
and Cleburne's Divisions, both of the newly arrived Left Wing of the Army of
Mississippi. These latter divisions continue to occupy positions just to the
west and northwest of Bufford's Hill, Whisper's Division being reported to have
moved to Fort Henry on the 30th ultimo. My army, reinforced by the regiments
previously detached for garrison duties, numbers about 39,600 infantry, 3,300
cavalry and 204 guns; it currently occupies a position from the west bank of the
Cumberland River extending west towards Trinity Church and then northwest
towards Bufford's hill and the Sawmill with the portion from the river to
Trinity Church now reinforced with field fortifications. The enemy is also
constructing defenses but these, as of this date, have not yet been completed.
No enemy reinforcements have been spotted save a force of cavalry of unknown
size spotted around Fort Henry on the 2d instant.

With the aforementioned withdrawal of Whisper's Division, I believe the enemy
force to my front to be no more than five divisions although I am somewhat
confident it is only four; in either case the enemy has suffered loss although,
as many of the losses of the action of the 25th ult. were concentrated in
Whisper's Division, I expect the enemy divisions still present here here to be
closer to full strength.  However, my army occupies good positions and, as has
been made clear over two days of hard fighting, the enemy will have a difficult
time if he chooses to attack us again. I await the return of better weather when
I hope to resume operations here.

\gramClosing{Respectfully}
    {J. W. Blake}
    {Maj. Gen. Commanding}
\reportdinkus

\gramHeader{Headquarters, Army of the Cumberland} % {{{4
    {-- am, February}{7, 1862}
\gramTo{Brig. Gen.}{Lawrence Graham}
    {Commanding, Cavalry Division}

\gramHi{General} Your brigades in the south are to move as quickly as possible
to support the action in the north. One brigade and the remaining artillery
should head directly there while the other heads east towards the river and
skirts the hills to approach from around Stone Cemetery.

If attacked, your 1st Brigade is to conduct a fighting retreat south to new
positions behind Brandon Spring Branch. Be ready to withdraw rapidly if the
enemy heads east down Rayburn Creek towards the river.

\gramClosing{With urgency}
    {J. W. Blake}
    {Maj. Gen., Commanding}
\reportdinkus

\gramHeader{Headquarters, Army of the Cumberland} % {{{4
    {-- am, February}{7, 1862}
\gramTo{Brig. Gen.}{Ptolemy Smith}
    {Commanding VIIIth Corps}

\gramHi{General} Send two reserve regiments and three batteries north to defend
the depot. If reinforcing cavalry approaches from the south, they are to have
right of way along the road.

\gramClosing{With urgency}
    {J. W. Blake}
    {Maj. Gen., Commanding}
\reportdinkus

\gramHeader{Headquarters, Army of the Cumberland} % {{{4
    {6.00~am, February}{8, 1862}
\gramTo{Brig. Gen.}{Charles Smith}
    {Commanding XIIth Corps}

Send two brigades north: one to Stones Gap and one directly toward the action with as
much artillery as you can spare.

\gramClosing{Respectfully,}
{James W. Blake}
{Maj. Gen., Commanding}
\reportdinkus

\gramCircular{Headquarters, Army of the Cumberland} % {{{4
{7.30 am, February}{8, 1862}

Whisper's Division and a division of cavalry arrived in the north yesterday
morning and are holding, screened by our cavalry. Earlier today XIVth Corps was
attacked by three divisions and forced to fall back. Two regiments of Thomas'
Division are now surrounded near the Sawmill.

One cavalry brigade has moved south and is in contact with a regiment of the
enemy north of the Sawmill. Two brigades of Sherman's Division, XIIth Corps are
in column headed west to support XIVth Corps. The rest of the army is not yet
engaged.

Although the situation is dire, I have decided to remain on this ground and
defend the river landing as attempting a river evacuation could result in loss
of the entire army.

In the south, VIIIth Corps and Van Cleve's Division, XIIth Corps have field
fortifications as does the enemy facing them, which I believe to be only two
divisions at this point.  Additional reserves can likely be drawn from this area
without seriously threatening our position.

It is absolutely critical that we defend the heights to the northwest of the
river landing  as enemy artillery placed there would threaten our supplies.

If possible we need to take positions east of the Sawmill to keep the northern
perimeter away from the river landing.

\gramClosingBy{Maj. Gen. James Blake}
{Walter Chekov}
{Col., Adjutant General}
(To Generals McClernand, Smith, Smith and Graham)

\reportdinkus

\gramHeader{Headquarters, Army of the Cumberland} % {{{4
    {7.30~am}{Fair View School}
\gramTo{Brig. Gen.}{Charles Smith}
    {Commander, XIIth Corps}

Send your entire corps except for a small force to watch the ford north in
between Bear and Happy Hollow, there is possibly a confederate corps marching
east along the road hitting the very right of our line. Once your corps is on
position and the line is stable we will launch an attack with your men toward
hill south of bear.

Move as many of your men off the line and leave any artillery currently firing,
move as quick as possible.

\gramClosing{}
    {James W. Blake}
    {Maj. Gen., Commanding}
\reportdinkus

\gramHeader{Headquarters, Army of the Cumberland} % {{{4
    {7.30~am}{Fair View School}
\gramTo{Brig. Gen.}{Lawrence Graham}
    {Commander, Cavalry Division}

If no enemy is discovered north of the Brandon Spring Branch, move south quickly
and attempt to link up with the northern flank of our army, an enemy corps is
here you should attempt to hit their side when moving south to join.

\gramClosing{}
    {James W. Blake}
    {Maj. Gen., Commanding}
\reportdinkus

\gramCircular{Headquarters, Army of the Cumberland} % {{{4
{Morning, February}{9, 1862}

The following enemy forces are arrayed against us:

% TODO: Make a general table environment
\begin{center}
    \begin{dispatch}[
    ]{
        % rowsep=0pt,
        colspec = {X[l]r},
        % cells = {font=\footnotesize},
    }

\MakeUppercase{Army of Mississippi} \\

\textsc{Brig. Gen. Vexley's Right Wing} \\
\hspace*{1em}Cleburne's Division\dotfill                & 3--4 brigades \\
\hspace*{1em}Breckinridge's Division\dotfill            & 3 brigades \\
\hspace*{1em}Unknown Army of Tennessee Division\dotfill & 3 brigades \\

\textsc{Brig. Gen. Pickford's Left Wing} \\
\hspace*{1em}Clayton's Division\dotfill                 & 3 brigades \\
\hspace*{1em}Granbury's Division\dotfill                & 2--3 brigades \\

\MakeUppercase{Army of East Tennessee} \\
\hspace*{1em}Whisper's Division\dotfill                 & 4 brigades \\
\hspace*{1em}Cavalry Division\dotfill                   & 3 brigades \\

    \end{dispatch}
\end{center}

The enemy's current strength is estimated to roughly equal our own. The Army of
East Tennessee may have withdrawn north during the night.

\gramClosingBy{Maj. Gen. J. W. Blake}
{Walter Chekov}
{Col., Adjutant General}
(To Generals McClernand, Smith, Smith and Graham)

\reportdinkus

\gramHeader{Headquarters, Army of the Cumberland} % {{{4
    {Near Fort Donelson, Tenn., February}{8, 1862}
\gramTo{Maj. Gen.}{Cornelius Van Royne}
    {Commander, United States Army}

\gramHi{General} I submit to you my preliminary report on the action of today's
date near Fort Donelson.

Yesterday, the 7th instant, my cavalry reported a division of enemy cavalry
approaching from the north near Rayburn's Creek. I immediately sent the rest of
my cavalry north as this area was undefended as I did not expect the enemy to
make the long march to reach that position.

This morning Brig. Gen. Graham reported that the enemy cavalry had been joined
by a division of infantry. I expect this to be Whisper's Division, spotted near
Fort Henry on the 30th ult. Shortly after, at 6.00 am, two, possibly three,
divisions attacked my western flank. I was forced to give some ground, falling
back to roughly the positions held before the action of the 28th ult. The enemy
in the north did not attempt to attack. Our own counterattack in the south
managed to push the enemy back a bit but was slowed considerably by the mud and
rain. Fighting ended around 9 o'clock.

Our casualties were about 1,800 infantry with 300 of those missing and assumed
captured plus a battery of artillery. Enemy losses are expected to be between
three and four thousand. My situation is currently precarious as the enemy has
cut the road north and is close to being able to threaten my river depot. I do
not believe I can evacuate via the river without significant loss and am
resolved to hold this position.

However, as the enemy seems to have the bulk of his force between the rivers
facing my army directly, I have a strong feeling that Fort Henry is lightly
defended, perhaps by only a brigade or two. If Gen'l Smith could land a force
near that fort, he may be able to take it and threaten the enemy position here
near Fort Donelson.

\gramClosing{Respectfully}
    {J. W. Blake}
    {Maj. Gen., Commanding}
\reportdinkus

\gramHeader{Headquarters, Army of the Cumberland} % {{{4
    {Next Fort Donelson, Tenn., February}{10, 1862}

\gramTo{Maj. Gen.}{Cornelius Van Royne}
    {Commanding, United States Army}

\gramHi{Sir} Whisper's Division and his cavalry are withdrawing toward Fort
Henry. I have identified the following forces arrayed against me:

\begin{center}
    \begin{dispatch}[
    ]{
        colspec = {l|l|l},
    }

    \MakeUppercase{Army of Mississippi}: & \MakeUppercase{Army of East Tennessee}: & \MakeUppercase{Army of Tennessee}: \\
    \textsc{Pickford's Left Wing}:       & Whisper's Division                      & Unknown division \\

    \hspace*{1em}Breckindridge's Division \\
    \hspace*{1em}Cleburne's Division \\
                
    \textsc{Vexley's Right Wing}: \\
    \hspace*{1em}Clayton's Division \\
    \hspace*{1em}Granbury's Division \\

    \end{dispatch}
\end{center}

I have drawn in my lines slightly but am in a good position for now.

\gramClosing{Respectfully}
    {J. W. Blake}
    {Maj. Gen., Commanding}
\reportdinkus

\gramOrdersHeader{Headquarters, Army of the Cumberland} % {{{4
    {Near Fort Donelson, Tenn., February}{10, 1862}
    {Special Field Orders}{11}

This army will maintain its present position defending the river landing.

First Division, XIVth Corps will refuse its right flank while leaving pickets in
place to maintain contact with the enemy.

All brigades, including reserve brigades, of VIIIth, XIIth and XIVth Corps will
construct field fortifications in their current positions.

Brig. Gen. Graham will pursue Whisper's Division westward as far as the northern
approach to Fort Henry. The cavalry will maintain contact with the enemy but
avoid a general engagement. If the enemy continues past Fort Henry, the cavalry
will continue to patrol the roads west of the army to ensure no enemy force is
able to advance to the north without warning.

\gramClosingBy{Maj. Gen. J. W. Blake}
{Walter Chekov}
{Col., Adjutant General}
\subsecdinkus

\gramOrdersHeader{Headquarters, Army of the Cumberland} % {{{4
    {Happy Hollow, Tenn., February}{17, 1862}
    {Special Field Orders}{12}

I. This army will continue to defend our position near Fort Donelson and the
river landing southeast of Shemwell Cemetery.

II. XIVth Corps, supported by 2d Brigade of the Artillery Reserve, will maintain
positions in the west. 1st Division will refuse the right flank while keeping
strong pickets in front of the enemy. 2d Division will, likewise, maintain
strong pickets on the hill between Pickford's and Vexley's Wings. The two
batteries of the corps' artillery will be situated to cover the Bear Creek
valley.

III. XIIth Corps, supported by 1st Brigade of the Artillery Reserve, will
maintain positions in the south. The three batteries of the corps' artillery
should be placed to defend against any push through Fitzhugh's Landing.

IV. VIIIth Corps will be the army reserve with 1st Division in the center and 2d
Division and the corps' artillery in the south.

V. The Cavalry Division will maintain a screen north of Fort Henry to identify
and delay any enemy movement towards our northern flank. Once the Army of the
Kanawha lands near Fort Henry, Brig. Gen. Graham is to place his cavalry at
the disposal of Maj. Gen. Steele while still keeping this headquarters apprised
of his movements and any intelligence gained.

VI. On the night of the 18th, one regiment of 2d Division, VIIIth Corps will
light torches and move north to Stone Cemetery and then west a mile or so in the
direction of Fort Henry. The regiment will repeat this maneuver throughout the
night so as to give the enemy the impression a significant force is moving west
towards Fort Henry.

VII. As the enemy has already made to attempts two turn our right flank, the next
blow is expected to fall in the south. 1st Division of VIIIth Corps should be
positioned so as to rapidly respond to any attack against XIIth Corps.

\gramClosingBy{Maj. Gen. J. Blake}
    {Walter Chekov}
    {Col., Adjutant General}
\reportdinkus

\gramHeader{Headquarters, Army of the Cumberland} % {{{4
    {Happy Hollow, Tenn., February}{17, 1862}
\gramTo{Maj. Gen.}{Cornelius Van Royne}
    {Commanding the United States Army}

\gramHi{Sir} Now that I have received all reports from my subordinates I have
the honor to present you with my report of the action of the 3d instant around
Bufford's Hill and the Sawmill.

Early on the morning of the 7th instant, Brig. Gen. Graham's cavalry spotted
enemy cavalry to our north around Rayburn's Creek. This cavalry had previously
been spotted near Fort Henry on the 2d instant but had, unnoticed to Gen'l
Graham's scouts, marched around to our north, cutting the road to Golden Pond,
Ky. The remainder of the cavalry was sent north to Rayburn's Creek with orders
to delay any advance by the enemy while falling back south to Brandon Spring
Branch. On the morning of the 8th, the cavalry spotted Whisper's Division of the
Army of the East Tennessee moving up behind the cavalry. However, during the
action which followed, the enemy declined to advance south, likely exhausted
from the long march through the mud. Thankfully, this allowed Brig. Gen. Graham,
to pull back to Brandon Spring Branch without difficulty and, later in the
morning, to send two of his brigades south to assist in the fighting that would
occur around XIVth Corps' right flank.

The enemy opened his attack on Brig. Gen. John McClernand's XIVth Corps at
around 6.15~am when four batteries of artillery engaged those of Thomas'
Division, XIVth Corps deployed in the Bear Creek Valley near the right flank of
this army. Col. Samuel Carter's brigade, holding the right flank of this army
around the Sawmill was hard pressed by a brigade of Pickford's Left Wing of the
Army of Mississippi.  Gen'l Thomas rushed additional regiments to his right but
the enemy pressure was too great and the loss to his division was severe.

Despite this, the enemy was unable to turn Gen'l McClernand's flank due to the
valiant efforts of three regiments of Thomas' Division. Col. James Carter's 2d
Tennessee fought a valiant last stand that delayed the enemy long enough for the
rest of the division to form a defense on the hills south of Bear Creek. Of this
regiment, only 10 men were able to return to friendly lines but, to a man, they
state that Col. Carter held as long as he could before finally raising the white
flag of surrender.

Col. Sedgwick's 2d Kentucky and Col. Steedman's 14th Ohio regiments were also
engaged around the Sawmill and, although they initially broke, they were able to
reform.  They were soon joined by Col. Theophilus Dickey's brigade of cavalry
and held the line there, preventing the enemy from turning the flank.

The enemy began to widen his attack, advancing against the north slope of
Bufford's hill around 6.45~am and soon drove the remainder of Thomas' Division
east across the main road. There, however, Gen'l Thomas rallied his troops and
was able to halt the enemy.

During this attack, reserve brigades of VIIIth and XIIth Corps were ordered to
move to support the Brig. Gen. William Rosecrans' division of XIVth Corps which
was defending the southern portion of Bufford's Hill. These brigades, Col.
Bruce's and Col. Cook's of VIIIth Corps and Col. Wagner's of XIIth Corps,
arrived in time to blunt an enemy attack that advanced north along the east side
of the hill. At least one enemy regiment was thrown back in disarray.

Elements of the reserve brigades attempted to advance through the gap previously
identified between the Left and Right Wings of the Army of Mississippi. Their
advance was not resisted and they were able to move a significant distance
towards Hickman Creek Hill astride the main road leading east towards Fort
Donelson but, the mud and steep slopes soon halted their advance.

During the attack in the west, artillery duels persisted between XIIth Corps'
artillery in the south and the enemy on Whisper's Hill and near Fort Donelson
but no attack came from that direction. By 8.45~am the enemy attack had ceased
along the entire line. 

Although XIVth Corps was pushed off Bufford's Hill, Gen'l McClernand maintained
a strong position on the hill across the road to the east and the hill just to
the south, across B Branch.  The sawmill position was lost and the enemy was
able to there occupy a portion of the road north to Golden Pond.

Late in the evening, Gen'l Graham reported that torches were spotted in the
enemy positions to the north of Brandon Spring Branch and, when scouts were sent
out in the morning, the enemy was found to have vacated that position, headed
back west to Fort Henry.

The following list of casualties shows a loss in the army as follows:

\vspace{5pt}
\begin{dispatch}[
]{
    colspec = {X[l]r},
}

Commissioned officers: \\
\hspace{2em}Killed\dotfill & 12\\
\hspace{2em}Wounded\dotfill & 25 \\
\hspace{2em}Missing\dotfill & 36 \\

Enlisted men: \\
\hspace{2em}Killed\dotfill & 269 \\
\hspace{2em}Wounded\dotfill & 841 \\
\SetRow{belowsep+=2pt}
\hspace{2em}Missing\dotfill & 620 \\

\cmidrule{2}\SetRow{abovesep+=4pt}
\hspace{3em}Total killed, wounded and missing\dotfill & 1,803 \\
\end{dispatch}
\vspace{5pt}

The army also lost six pieces of artillery. In the list of officers killed is
Col. William Kise, 10th Indiana.  In the list of officers missing are Colonel
James Carter and Lieutenant Colonel Daniel Trewhitt, 2d Tennessee.  These two
officers and the men who fought beside them as well as the officers and men of
the 2d Kentucky and 14th Ohio deserve the nations highest praise for their
service in holding this army's flank. Without their dedication to duty, the day,
and the army, may have been lost.

Based on questioning of prisoners and the reports of my officers, I believe that
XIVth Corps was engaged by Breckinridge's and Cleburne's divisions of the Left
Wing of the Army of Mississippi as well as an unknown division of the Army of
Tennessee. This puts the total number of enemy division defending the positions
around Fort Donelson as six infantry, as Whisper's Division has since returned,
and one cavalry division of the Army of East Tennessee. Since the engagement,
the enemy has been joined by additional cavalry from the Army of Mississippi
although the amount has not yet been determined.

I have since drawn in my lines, holding the salient between the enemy's wings
only with pickets and have been able, thanks to the lull in fighting over the
previous week, to construct field fortifications along the entire line. With
these and the benefit of good interior lines, I believe I can weather the
assault I fear is coming against my southern position, opposite Whisper's Hill.

\gramClosing{Respectfully}
    {J. W. Blake}
    {Maj. Gen., Commanding}
\reportdinkus

\subsecdinkus

\gramHeader{Headquarters, Army of the Cumberland} % {{{4
    {9.00 am, February}{21, 1862}
\gramTo{Brig. Gen.}{Lawrence Graham}
    {Commanding, Cavalry Division:}

\gramHi{General} At this hour, artillery can be heard from the west so I presume
you or Gen'l Steele are engaged. Please keep me informed of the situation. I am
prepared to probe the enemy here if Gen'l Steele feels it appropriate.

\gramClosing{Respectfully}
    {J. W. Blake}
    {Maj. Gen., Commanding}
(Same to Maj. Gen. Steele.)

\reportdinkus

\gramHeader{Headquarters, Army of the Cumberland} % {{{4
{Happy Hollow, Tenn., 2.00 pm, February}{21, 1862}
\gramTo{Maj. Gen.}{Steele}
{Commanding Army of the Kanawha}

\gramHi{General} I am sending Brig. Gen.  John Wool's division to assist you.
Hopefully they arrive in time to do some good. Have you identified which enemy
units you are facing and their strength?

\gramClosing{Respectfully}
{J. W. Blake}
{Maj. Gen., Commanding}
\reportdinkus

\gramHeader{Headquarters, Army of the Cumberland} % {{{4
{Happy Hollow, Tenn., 2.00 pm, February}{21, 1862}
\gramTo{Brig. Gen.}{Ptolemy Smith}
{Commanding, VIIIth Corps}

\gramHi{General} You are to send your First Division  west to assist Gen'l
Steele immediately. They will report to him until further notice. 

\gramClosing{}
{J. W. Blake}
{Maj. Gen., Commanding}
\reportdinkus

\gramHeader{Headquarters, Army of the Cumberland} % {{{4
{Happy Hollow, Tenn., 2.00 pm, February}{21, 1862}
\gramTo{Brig. Gen.}{John McClernand}
{Commanding, XIVth Corps}

\gramHi{Sir} You are to advance Thomas' Division to occupy the Rebel positions
around the Sawmill, striking enemy forces that are attempting to withdraw if they
remain there and otherwise turning Breckinridge's flank.

\gramClosingBy{Maj. Gen. James Blake}
{Harold Fawcett, III}
{Brig. Gen., Chief of Staff}
\reportdinkus

\gramHeader{Headquarters, Army of the Cumberland} % {{{4
{Happy Hollow, Tenn., 2.45 pm, February}{21, 1862}
\gramTo{Maj. Gen.}{Steele}
{Commanding, Army of the Kanawha}

\gramHi{General} Rebel forces have just marched away from my position. Their
size and destination is unknown but I would surmise around a division is headed
west to support the engagement against you. I have given orders to seize the
ground vacated by the enemy so, perhaps, your maneuvers are beginning to break
the stalemate here.

\gramClosing{Respectfully}
{J. W. Blake}
{Maj. Gen., Commanding}
\reportdinkus

\gramHeader{Headquarters, Army of the Kanawha} % {{{4
{Fort Henry, 4.00 pm, February}{21, 1862}
\gramTo{Maj. Gen.}{ James Blake}
{Commanding Army of the Cumberland}

\gramHi{General} I have just received your message of 2.00~pm.

The enemy cavalry attacking my eastern flank has now been replaced by infantry.
I believe these newcomers are from the enemy 5th Division. We will do our best
to hold them, but if the worst should occur, we will spike the guns at Henry and
blow the magazine to render the fort useless. Hopefully it does not come to
that.

\gramClosing{Respectfully}
{Richard Steele}
{Maj. Gen., Commanding}
\reportdinkus

\gramHeader{Headquarters, Army of the Cumberland} % {{{4
{4.15 pm, February}{21, 1862}
\gramTo{Brig. Gen.}{John McClernand}
{Commander, XIVth Corps}

\gramHi{General} I am content to have you hold the positions around the Sawmill
and allow the enemy to again dash himself against our positions. Unless you deem
it prudent, hold there and do not attempt to drive the enemy off Bufford's Hill.

Do you need additional troops from VIIIth Corps to enable you to hold or do you
deem your own forces to be sufficient?

\gramClosing{}
{J. W. Blake}
{Maj. Gen., Commanding}
\reportdinkus

\gramOrdersHeader{Headquarters, Army of the Cumberland} % {{{4
{Happy Hollow, Tenn., February}{22, 1862}
{Special Field Orders}{13}

I. XIVth Corps will occupy the line from around Hopewell Church to the large
hill west of Bee Branch. Corps artillery will be positioned around Hopewell
Church while 2d Brigade, Artillery Reserve will support the Bufford's Hill
position.

II. XII Corps will occupy the remainder of the line with corps artillery
covering the eastern flank and 1st Brigade, Artillery Reserve positioned to
support the areas around Trinity Church and Whisper's Hill.

III. 2d Division, VIIIth Corps will be in reserve with 1st Brigade and the corps
artillery at the intersection northwest of Trinity Church and 2d and 3d Brigades
and division artillery at the intersection east of Trinity Church.

IV. 1st Division, VIIIth Corps and the cavalry will remain to support Gen'l
Steele.

\gramClosingBy{Maj. Gen. J. Blake}
{Walter Chekov}
{Col., Adjutant General}
\reportdinkus

\gramHeader{Headquarters, Army of the Cumberland} % {{{4
{Happy Hollow, Tenn., February}{23, 1862}
\gramTo{Maj. Gen.}{Richard Steele}
{Commanding, Army of the Kanawha}

\gramHi{General} I must congratulate your army on the success yesterday. You
have barely joined us here in the west and are already enjoying success. Your
engagement forced the enemy to withdraw troops from my lines, allowing my army
to seize key ground without significant loss. Those enemy troops have since
returned but do not enjoy the same advantages they did yesterday morning. This
maneuvering has now forced the enemy to withdraw even farther. 

My scouts have this morning reported the enemy has abandoned his positions from
Whisper's Hill  extending west. Unless those units have simply pulled back out
of sight I believe the enemy may have retreated back to the original positions
around Fort Donelson and Dover.

Not including the forces I have attached to your army, I currently field five
divisions of around 30,600 men  and 138 guns. We are holding a line from
Hopewell Church towards Trinity Church and ending against the river at Fitzhughs
Landing.  What is your current strength and disposition?

The enemy had six divisions here yesterday morning but I do not know what is
currently here near Fort Donelson.  How much force do you believe the enemy has
around Fort Henry and where are they positioned?

I think we should take advantage of our success and continue to press the enemy
before he is able to create a strong defense. However, there is certainly the
possibility the enemy chooses to counterattack one or both of our armies so
until both our positions are firm I recommend against being too aggressive.

I intend to advance quickly to seize the ground vacated by the enemy and attempt
to cut the main road between Dover and the Tennessee River . If you are able, I
think it would be prudent if you could send troops southward to gain control of
the rail line between Dover and Stewart Crossing. I recommend that you maintain
control of Brig. Gen. Graham's cavalry for now so that our combined cavalry
force can counter the enemy's. I leave it to your judgement if you need to
maintain control of Wool's Division.

Please let me know as soon as possible if additional enemy forces are spotted
moving in your direction so that I may send additional force to assist you. Our
combined force outnumbers the enemy; we need to coordinate our actions so that
we can use that strength effectively and not allow the enemy any advantage.

\gramClosing{Respectfully}
{J. W. Blake}
{Maj. Gen., Commanding}
\reportdinkus

\gramHeader{Headquarters, Army of the Cumberland} % {{{4
{Happy Hollow, Tenn., February}{23, 1862}
\gramTo{Maj. Gen.}{Richard Steele}
{Commanding, Army of the Kanawha}

\gramHi{General} My army now occupies Hopewell Church, preventing the enemy from
advancing on you down the Panther Creek Road so there should be no need to
continue to defend that position. If the enemy forces that withdrew from in
front of my positions are moving to attack you they would almost certainly be
moving along the Austin Peay Road and then attacking along the trails leading
west from Wynn, moving down Lost Creek or continuing on towards Scott Fitzhugh
Bridge and marching north along the river.

Although the rain will slow my movements, I intend to occupy the ground vacated
by the enemy and regain contact. I expect to encounter a new defensive line
either along the Austin Peay Road from around Hickman Creek hill extending east
or to find the enemy has refused his position and is holding around Bubbling
Spring, east of Hinson Creek.

I ask that you send our combined cavalry force to regain contact with the enemy
cavalry and keep such from interfering with our operations. As to your remaining
force, the first priority is finding the enemy and blunting any possible
counterattack at Fort Henry. Sending Wool's Division, as well as any additional
forces you deem prudent, along Austin Peay Road to rejoin my army should suffice
to identify any counterattack moving in that direction and provide the force
necessary to maneuver around the enemy here near Fort Donelson.

\gramClosing{Respectfully}
{J. W. Blake}
{Maj. Gen., Commanding}
\reportdinkus

\gramHeader{Headquarters, Army of the Cumberland} % {{{4
{Happy Hollow, Tenn., February}{23, 1862}

Orders of the day for the Army of the Cumberland for to-day, February 23, 1862:
The enemy has withdrawn from Whisper's and Bufford's Hills and is either sending
a significant force to counterattack Gen'l Steele around Fort Henry or is
consolidating to defend against an attack on Fort Donelson and Dover from the
west.

I. The army is to conduct a ``wheel'' movement, anchored on 2d Division, XIIth
Corps. This division will remain in roughly its current position while 1st
Division, XIIth Corps will push forward across Whisper's Hill and establish
positions on the southeast slope, facing the known enemy works across Hickman's
Creek. 1st Brigade, Artillery Reserve will support XIIth Corps.

II. 2d Division, VIIIth Corps, supported by the corps artillery and 2d Brigade,
Artillery Reserve, will move across Whisper's Hill and establish positions
facing Hickman Creek. When 1st Division arrives from Fort Henry, it will
establish positions on the hill between Bear Creek and Hickman Creek Hill to
prevent any movement down Bear Creek.

III. XIVth Corps is to advance along Bear Creek and Bee Branch towards the
Austin Peay Road and then turn east, moving to occupy the east slope of the hill
west of Hinson Creek. 2d Division will be in the north and 1st Division,
supported by the corps artillery in the south. If the hills to the east of
Hinson Creek are not occupied by the enemy, XIVth Corps will continue to advance
but is not to enter a general engagement.

IV. The cavalry will remain attached to Gen'l Steele but is expected to provide
reports to this headquarters on any engagements and any intelligence gained on
the enemy.

V. The enemy may have established defenses along the Austin Peay Road or on the
hills between Andrews Hollow and Hinson Creek. All movements are to be
cautiously aggressive to avoid entering an ambush and are to avoid a general
engagement.

\gramClosingBy{Maj. Gen. J. Blake}
{Walter Chekov}
{Col., Adjutant General}
\subsecdinkus

\pagebreak\gramHeader{Headquarters, Army of the Cumberland} % {{{4
{Trinity Church, Tenn., February}{24, 1862}
\gramTo{Maj. Gen.}{Cornelius Van Royne}
{Commander, United States Army}

\gramHi{Sir} Gen'l Steele's attack on and seizure of Fort Henry has not only
opened the Tennessee River but disrupted the enemy position around Fort Donelson
as expected. Minor skirmishing around the northern end of this army's position
took place on the 21st ult. during the battle of Fort Henry, and the enemy
pulled in his left flank allowing XIVth Corps to advance and gain control of the
shortest road between Fort Henry and the Fort Donelson area and deny that route
to the enemy.

During the 22d of February, the enemy abandoned the entirety of his works west
of Hickman's Creek. My army has since advanced and is in position overlooking
the enemy works across Hickman's Creek. However, the army surgeon reports this
army cannot remain much longer in the field before it wastes away from sickness
and disease. He prefers a full withdrawal to Golden Pond, Ky. but states that
manning siege lines for a few weeks would be acceptable.

Gen'l Steele and I are currently discussing our options for bringing this
campaign to a quick end before this army is forced to retire in total.

\gramClosing{Respectfully}
{J. W. Blake}
{Maj. Gen., Commanding}
\reportdinkus

\gramOrdersHeader{Headquarters, Army of the Cumberland} % {{{4
{Trinity Church, Tenn., February}{24, 1862}
{Special Field Orders}{14}

I. It is imperative that this campaign soon be brought to a close so this army
can establish a proper camp and see to the reorganization and refitting of its
troops. The arrival of Gen'l Steele's Army of the Kanawha has provided an
opportunity to do so.

II. XIIth and XIVth Corps along with 1st Brigade, Artillery Reserve will
maintain their positions facing the enemy works across Hickman's Creek and
around Bubbling Spring.

III. 1st Brigade of 1st Division, VIIIth Corps is to march to Fort Henry to
relieve the garrison there. 2d and 3d Brigades will maintain position around
Hickman Creek Hill as the army reserve and to prevent any movement north along
Bear Creek or eastward on Austin Peay Road. 2d Division, VIIIth Corps will
replace the cavalry in Andrews Hollow and prevent the enemy advancing north from
Taylor's Chapel.

IV. 2d Brigade, Artillery Reserve will remain near Sykes Cemetery as a reserve.
The cavalry will remain attached to Gen'l Steele.

V. Commanders will rotate regiments from front line positions as much as
possible to provide time for the men to rest.

\gramClosingBy{Maj. Gen. James Blake}
{Walter Chekov}
{Col., Adjutant General}
\reportdinkus

\gramHeader{Headquarters, Army of the Cumberland} % {{{4
{Trinity Church, Tenn., 9.00 am, February}{25, 1862}

\gramTo{Brig. Gen.}{Ptolemy Smith}
{Commanding, VIIIth Corps}

\gramHi{General} You are to occupy the Taylor Chapel area with Crittenden's
Division. Scout east through Hurricane and Spring Hollows to determine if the
route to the road along West Fork is clear. Once Taylor Chapel is occupied, you
may consolidate your entire corps, save the Fort Henry garrison, at Taylor
Church.

\gramClosing{Respectfully}
{J. W. Blake}
{Maj. Gen., Commanding}

\gramHeader{Headquarters, Army of the Cumberland} % {{{4
{Trinity Church, Tenn., 9.00 am, February}{25, 1862}

\gramTo{Maj. Gen.}{Richard Steele}
{Commanding, Army of the Kanawha}

\gramHi{General} Taylor Chapel seems to have been abandoned by the enemy. I will
occupy it with 2d Division, VIIIth Corps and scout the route east to the West
Fork. If that appears clear you or I may be able to outflank the enemy from that
direction.

\gramClosing{Respectfully}
{J. W. Blake}
{Maj. Gen., Commanding}
\reportdinkus

\gramHeader{Headquarters, Army of the Cumberland} % {{{4
{Trinity Church, Tenn., 11.00 am, February}{25, 1862}

\gramTo{Maj. Gen.}{Richard Steele}
{Commanding, Army of the Kanawha}

\gramHi{General} I have occupied Taylor Chapel with no sign of the enemy.
Pickets report increased movement around Fort Donelson and believe the fort's
guns are being moved. I suspect that the enemy is preparing to withdraw
entirely. If so, the most likely route of retreat is along the main road leading
east from Dover.

If practicable, send Gen'l Graham's cavalry eastward to Long Creek and then
north to block the road near Lower Long Creek School. If the cavalry can delay
the enemy long enough your infantry and my VIIIth Corps may be able to cut the
enemy's line of retreat.

I intend to send VIIIth Corps from Taylor Chapel to the West Fork and then north
towards the entrance to Bufford Hollow.

\gramClosing{Respectfully}
{J. W. Blake}
{Maj. Gen., Commanding}
P.S.---Please keep me updated as to the location of your army.

\reportdinkus

\gramHeader{Headquarters, Army of the Cumberland} % {{{4
{Trinity Church, Tenn., 11.00 am, February}{25, 1862}

\gramTo{Brig. Gen.}{Ptolemy Smith}
{Commanding, VIIIth Corps}

\gramHi{General} Send your corps, save the Fort Henry garrison, east to the West
Fork and then north down Lick Creek towards where it crosses the road leading
west into Dover. Take whatever route to West Fork seems feasible if the direct
path eastward is not practical.

Give right of way to Graham's cavalry if they attempt to pass.

\gramClosing{Respectfully}
{J. W. Blake}
{Maj. Gen., Commanding}
\reportdinkus

\gramHeader{Headquarters, Army of the Cumberland} % {{{4
{Trinity Church, Tenn., 11.00 am, February}{25, 1862}

\gramTo{Brig. Gen.}{John McClernand}
{Commanding, XIVth Corps}

\gramHi{General} Please scout routes from your position to the road leading
north from Lindsey Hollow to Bufford Hollow.

If the enemy abandons the positions east of Bubbling Spring you are to occupy
them immediately, pushing east as much as possible without entering a general
engagement.

\gramClosing{Respectfully}
{J. W. Blake}
{Maj. Gen., Commanding}
\reportdinkus

\gramHeader{Headquarters, Army of the Cumberland} % {{{4
{Trinity Church, Tenn., 11.00 am, February}{25, 1862}

\gramTo{Brig. Gen.}{Hall}
{Commanding, Army of the Kanawha Cavalry}

\gramHi{General} You are to move rapidly to rejoin Gen'l Steele. The Clarksville
operation is canceled for the time being.

\gramClosing{Respectfully}
{J. W. Blake}
{Maj. Gen., Commanding}
\reportdinkus

\gramHeader{Headquarters, Army of the Cumberland} % {{{4
{Trinity Church, Tenn., 12.00 am, February}{25, 1862}

\gramTo{Maj. Gen.}{Richard Steele}
{Commanding, Army of the Kanawha}

\gramHi{General} With the destruction of Fort Donelson, my army is moving to
secure the remains of the fort as well as the town of Dover. You are to continue
your maneuver to outflank the enemy and prevent his retreat. My army will
support as much as possible.

I have called off the Clarksville operation and returned your cavalry to you. As
soon as you no longer have need of my own cavalry, please send them to Dover so
they may begin a period of recovery.

You are to continue to maneuver to capture the rail line and rail bridge near
Stewart Crossing and, if the enemy makes good his escape, this is to be your
primary goal. However, do not destroy the bridge or rail line unless you are
forced to retreat away from it.

\gramClosing{Respectfully}
{J. W. Blake}
{Maj. Gen., Commanding}
\reportdinkus

\gramOrdersHeader{Headquarters, Army of the Cumberland} % {{{4
{Trinity Church, Tenn., February}{25, 1862}
{Special Field Orders}{15}

I. With Fort Donelson captured and the enemy having made good his escape, this
army will occupy Dover, Tenn. and establish a proper camp and hospital for
recovery and refit. This camp is to be named Camp Carter in honor of Col. Carter
late of the 2d Tennessee.

II. A river landing for the depositing of supplies and reinforcements is to be
established at a suitable location along he river. The landings north of Fort
Donelson and at Collies Mills are to be dismantled as soon as practical.

III. All standard procedures for maintaining the security of the army in camp are to
be followed, to include cavalry patrols, pickets and forward positions on key
terrain.

\gramClosingBy{Maj. Gen. J. Blake}
{George Campbell}
{Capt., Aide de Camp}
(To Generals McClernand, Smith, Smith and Graham)

\reportdinkus

\gramHeader{Headquarters, Army of the Cumberland} % {{{4
{Trinity Church, Tenn., February}{25, 1862}

\gramTo{Maj. Gen.}{Cornelius Van Royne}
{Commanding, United States Army}

\gramHi{Sir} I have the honor to report that the enemy has abandoned Fort
Donelson today. Gen'l Steele's army is still in pursuit of the enemy but I doubt
he will able to accomplish much besides rounding up stragglers. He has been
directed to make cutting the rail road across the Tennessee River his priority
once it is clear the enemy has made good his escape.

I will provide a full report on the actions here since Gen'l Steele's landing on
the 21st instant once I have collected all reports from my subordinates.

Orders have been given for my army to establish a proper field camp and river
landing at Dover and to begin a period of recuperation. I estimate my army will
be fit for action by mid-March. I will continue to direct operations in this
area but Gen'l Steele's army will be the only force capable of action.

\gramClosing{Respectfully}
{J. W. Blake}
{Maj. Gen., Commanding}
\reportdinkus

\gramHeader{Headquarters, Army of the Cumberland} % {{{4
{Camp Carter, Tenn., March}{3, 1862}
\gramTo{Maj. Gen.}{Cornelius Van Royne}
{Commander, United States Army}

\gramHi{Sir} I submit to you my report of the actions of this headquarters in
support of the battle at Fort Henry as well as of the engagement around Hopewell
Church, Tenn. on the 21st ult. and following.  The report of Maj. Gen. Steele
will describe the specific actions of units of this Army during that actions
around Fort Henry itself.

Brig. Gen. Graham's division of cavalry, having been in position north of Fort
Henry since the 12th ult. was ordered to report to Gen'l Steele upon his landing
above Fort Henry. Gen'l Graham did so and, in conjunction with the cavalry of
the Army of the Kanawha, was instrumental in the success at Fort Henry.

At 2.00~pm, a dispatch from two hours previous was received from Gen'l Steele
stating he had begun the assault on Fort Henry. Wool's Division of VIIIth Corps,
being a portion of the army's reserve, was sent westward to assist Gen'l Steele.
This division was not expected to arrive at its destination until late in the
evening but it was hoped fresh troops would be useful to consolidate Gen'l
Steele's position. This, too, proved fortuitous, as Brig. Gen. Wool's arrival on
the field occurred when the enemy resolve was wavering; this, by all accounts,
convinced the enemy to break off the attack, leaving Gen'l Steele in sole
position of Fort Henry and the immediate area.

Less than one hour after the order had been given to Gen'l Wool, Brig. Gen.
McClernand, commanding XIVth Corps, reported the enemy had withdrawn from the
position around the Sawmill and had also thinned his defenses along Bufford's
Hill. Gen'l McClernand was ordered to advance Thomas' Division toward the
Sawmill so as to re-open the road north to Golden Pond, Ky. This was done and,
despite some skirmishing with enemy troops on Bufford's Hill, Gen'l Thomas was
able to advance through the Sawmill and reach Hopewell Church by 4.30~pm. Taking
this position was critical as it cut the enemy's quickest route west to Fort
Henry and opened that same route for our own forces.

Skirmishing between Thomas' and Rosecrans' Divisions and the enemy on Bufford's
Hill continued until the enemy withdrew from the northern portion of that hill
by 5~o'clock. Gen'l McClernand was ordered to halt his advance for the day and
reorganize his corps for the morning. Before nightfall reports had been received
that Gen'l Steele had seized Fort Henry and defeated the enemy's counterattack.
Gen'l McClernand reported significant movement of enemy troops behind the rebel
lines atop Bufford's hill, likely those defeated by Gen'l Steele returning to
their original positions.

The next day, February 22d, was again rainy, precluding any aggressive action by
this army. The next morning, however, the enemy was found to have abandoned his
works on Bufford's and Whisper's Hill and the ground in between. The only enemy
positions that remained occupied were the redoubts behind Hickman's Creek
directly defending Fort Donelson.

Orders were given for the army to move through the abandoned positions in order
to regain contact with the enemy. Messages were sent between this headquarters
and Gen'l Steele to coordinate actions and he ordered the bulk of his army,
along with Wool's Division to move east along the Austin Peay Road towards Fort
Donelson. By the end of the 23d, this army had occupied Whisper's Hill and
Bubbling Springs southwest of Fort Donelson and Dover. The bulk of the Army of
the Kanawha arrived the next morning and Graham's cavalry moved south up Andrews
Hollow, regaining contact with an enemy force at Taylor's Chapel.

In the interest of bringing the Donelson campaign to a quick close so that this
army could begin a much-needed period of rest, Gen'l Steele was ordered to take
his entire force south to both cut the rail road across the Tennessee River near
Stewart, Tenn. Buckland's Brigade of Wool's Division was ordered to relieve
Gen'l Steele's troops garrisoning Fort Henry; Graham's cavalry was to continue
supporting Gen'l Steele. Meanwhile, Hall's cavalry of the Army of the Kanawha
was to cross the Cumberland River and move to Clarksville, Tenn. in order to cut
the rail line there. The overall intent was to prevent the enemy from quickly
moving forces away from Fort Donelson in the hopes that we could trap a bulk of
the enemy there.

In the event however, the enemy had already begun withdrawing. Smith's VIIIth
Corps was sent to occupy Taylor Chapel and, finding it abandoned, began scouting
routes east to get behind Dover. Gen'l Steele was ordered to send Graham's
cavalry ahead to delay the enemy retreat, hopefully allowing the infantry to
catch up and engage. Gen'l Hall's raid on Clarksville was called off as I deemed
his troopers of more use in the pursuit of the enemy.

Increased movement was noted from within the walls of Fort Henry and pickets
reported they believed cannon were being moved. At about noon an explosion
erupted from within the fort as the enemy detonated what supplies and equipment
could not be carried off. My army quickly occupied the fort and town of Dover
and, once it was clear the enemy had decamped from the immediate area, began
establishing Camp Carter (named for Col. James Carter of the 2d Tenn.) for the
recuperation of the army. 

Gen'l Steele was ordered to continue the pursuit of the enemy of which he has
already sent a report.

\gramClosing{Respectfully}
{J. W. Blake}
{Maj. Gen., Commanding}
Send copy to Maj. Gen. Steele.

\subsecdinkus

\subsection{Reports of Col. Walter Chekov, Adjutant General} % {{{3
\gramHeader{Headquarters, Army of the Cumberland} % {{{4
    {Field Headquarters, January}{9.30~am 25th, 1862}
\gramTo{Major General}{James W. Blake,}
    {Commanding}

\gramHi{General} I have the honor to submit the following report of the
operations of this command upon the present date.

At 7 o'clock this morning our forces advanced upon the enemy's works across
Hickman's Creek. The First Division, XII Corps was ordered to storm the heights
opposite, and though at the outset they carried the enemy's breastworks with
gallant determination, they were presently met with such a murderous fire of
musketry and grape as to be thrown back in disorder. The Second Division, VIII
Corps advanced in support and maintained its ground with firmness, occupying the
hill across the creek and sustaining its position by the vigorous co-operation
of the Army Reserve and such artillery as could be brought to bear. The fire of
our batteries proved effective in silencing certain of the rebel cannon to the
south, though strong breastworks upon the enemy's left still poured in a heavy
enfilade as our infantry approached. Within this quarter were observed the
standards of General Granbury's Division, Army of Mississippi, actively engaged
as well as General Clarke himself.

Simultaneously, the naval flotilla directed a concentrated cannonade upon the
works of Fort Donelson, which engagement still proceeds at the hour of this
writing.

Upon our right, the Cavalry Division encountered a full division of the Army of
East Tennessee under General Whisper. The contest in the thickets was obstinate
and severe, but the weight of numbers compelled our troopers to retire in good
order and now hold in front of 1st Division, VIII Corps, where they have
re-formed and now present an effective line, their losses estimated at
one-quarter strength. In this quarter the enemy obtained possession of the
southern hill fronting Hickman's Creek, but our dispositions now secure its
approaches. The Second Division, XII Corps, having as yet been but lightly
engaged, has extended its lines to cover our southern flank, whilst one of its
brigades is held in reserve.

As to the condition of the several commands: the First Division, VIII Corps
remains unshaken and in position behind the cavalry; the Second Division, VIII
Corps is much worn, the brigades having been reorganized under fire, though one
still maintains its ground on our side of Hickman's Creek. The First Division,
XII Corps is greatly broken, and of its regiments only four remain in any
semblance of order, these numbering less than one-quarter effective strength.
The Second Division, XII Corps reports negligible losses and occupies ground to
the left of the cavalry. The artillery of the line is actively engaged, though
four batteries of the XII Corps are reduced to but thirty minutes' supply of
ammunition, and for this cause have been withdrawn and replaced by the Reserve.

The enemy's force upon our front is judged to consist of not less than 7,000
infantry with four batteries of the Army of East Tennessee, and a like number of
infantry supported by four to eight batteries of the Army of Mississippi,
directed by Major Generals Whisper and Clarke, respectively.

Our last reports, received at 9.15~am, confirm that the line is held in
good order, While 1st Division, XII Corps having taken the brunt of the losses,
is on the verge of breaking, and 2d Division of VIII Corps is nervous, the rest
of the army is in good order and in stout spirits. Besides those units which
have taken severe casualties, the army remains ready and willing to fight.

\gramClosing{I am, General, very respectfully, Your obedient servant}
    {Col. Walter Chekov}
    {Adjutant General, Army of the Cumberland}
\reportdinkus

\gramHeader{Headquarters, Army of the Cumberland} % {{{4
    {Field Headquarters, North of Fort Donelson, Night, January}{27th, 1862}
\gramTo{Major General}{James W. Blake}
    {Commanding General, Army of the Cumberland}

\gramHi{General} I have the honor to lay before you the following report
touching upon the condition of this command and its late operations before Fort
Donelson.

The forces under our command are as follows: the VIII Corps, two divisions,
numbering 13,200 men with forty-two guns; the XII Corps, two divisions, 12,000
men with forty-two guns; the XIV Corps, 12,000 men with forty-two guns, posted
in support to the rear; and the Cavalry Division, 3,300 sabers with twelve
pieces of horse artillery. To secure our communications and depots we have
detailed: Evansville, Ind., one regiment (600 men); Smithland, Ky.,
three regiments (1,800 men); and Collies Mills, Ky., one regiment (600
men).

Opposing us are the Army of East Tennessee, commanded by General Whisper,
estimated at a strong division or small corps of 7,000 to 9,000 men; and the
Army of Mississippi, under General Clarke, its left and right wings reckoned at
10,000 to 12,000 infantry each. From reports gathered, the left wing of the Army
of Mississippi came up subsequent to the engagement of the 25th. Enemy cavalry
have not appeared in our immediate front, but are said to operate upon the
eastern bank of the Cumberland River and upon the western bank of the Tennessee,
where their activity has thus far been curtailed by the diligence of our
gunboats.

The engagement of the 25th closed after our assault upon the enemy's fortified
line across Hickman's Creek was repulsed. The 1st Division of the XII Corps bore
the severest loss, while the 2d Division of the VIII Corps met with moderate
casualties. The Cavalry Division, engaged upon the hill south of the creek, was
pressed back by elements of the Army of East Tennessee and suffered moderate
loss, but thereafter handsomely repulsed a determined effort of the enemy to
force a passage into our lines, inflicting in turn a measure of loss upon them.
With that, the action subsided into an artillery duel and sporadic skirmishing.

From the river Commodore Daniel Davyson Lewis reports that the flotilla engaged
Fort Donelson with a sustained bombardment. He cannot, as of yet, affirm the
fort's reduction, but states confidently that the fleet sustained no heavy
casualties in the operation. Beyond the field of battle, we have received word
that the locks on the Green River at Rochester and Calhoun have been destroyed
by Confederate raiders. The extent of the damage cannot presently be
ascertained. Until our forces reoccupy the ground and engineers may be
dispatched, no reliable estimate for repair or renewed use can be offered.

I may also note that the weather is likely to deteriorate. Reports for the
coming week (28th of January through the 3d of February) foretell much rain,
with occasional respite, though temperatures are expected to rise somewhat.

I shall continue to provide such reports as the situation may require.

\gramClosing{I am, General, very respectfully, Your obedient servant}
    {Walter Chekov}
    {Col., Adjutant General, Army of the Cumberland}
\reportdinkus

\gramHeader{Headquarters, Army of the Cumberland} % {{{4
    {North of Fort Donelson, Tenn., Evening, February}{8, 1862}
\gramTo{Major General}{James W. Blake}
    {Commanding}

\gramHi{General} I have the honor to submit the following report of the
engagement of this morning.

At dawn, 6 o'clock, the enemy under Major General Alexander Clark, commanding
the Army of Mississippi with elements of the Army of Tennessee, advanced in
heavy force, estimated at thirty thousand with an additional seven to eight
thousand in support. Their onset fell chiefly upon the 1st Division, XIV Corps,
positioned upon Buford Hill and the neighboring crossroads. After a severe
contest, this division was dislodged, yielding both hill and crossroads, and
sustained grievous loss.

The remainder of the XIV Corps, though pressed, held firm. A counterstroke was
delivered by a brigade, supported by artillery, which threw the foe into
disorder and compelled many to retire precipitately. At this juncture the 1st
Division, XII Corps, arrived upon the field and gave timely succor to both XIV
Corps and the southern elements of the 2d Division, VIII Corps. The latter
division, seizing the opportunity, broke and routed an enemy brigade along Bee
Creek, driving southward into the gap which yawned between the hostile wings.
Their pursuit was checked only by the severity of the weather and the rugged
character of the ground, and by 9 o'clock the movement had spent its vigor.

Upon the front of the XII Corps, the enemy attempted a probing advance, but was
easily repelled. The opposing batteries then engaged in a duel of several hours'
duration, wherein both sides lost guns to silencing fire, but without decisive
result.

The cavalry, which had fallen back from its northern outpost during the night,
formed anew south of Brandon Spring Branch. From this quarter no enemy was
observed throughout the engagement. Two brigades of horse, dispatched thence,
encountered and frustrated a Confederate flanking column, striking it about a
mile south of Brandon Spring Branch and near the crossroads northwest of Hickory
Grove Church.

By the ninth hour the Confederate attack had lost all momentum. Their columns
recoiled and their assault ceased. We not only maintained the greater part of
our line but extended our possession in the south.

Our casualties are severe: 1,800 men (three infantry blocks) killed and wounded,
300 captured, and the loss of one artillery battery of six pieces. The enemy's
losses are judged to be between three and four thousand.

\gramClosing{I have the honor to be, General, Very respectfully, your obedient servant}
    {Walter Chekov}
    {Colonel, Adjutant General, Army of the Cumberland}
\subsecdinkus

\subsection{Report of Lieut. Col. Tyler Remington, Assistant Adjutant General} % {{{3
\gramHeader{Headquarters, Army of the Cumberland} % {{{4
    {Near Fort Donelson, Evening of January}{28th, 1862}
\gramTo{Major General}{James W. Blake}
    {Commanding, Army of the Cumberland}

\gramHi{Sir} I respectfully submit the following report on the action of January
28th, fought upon the hillsides west of Fort Donelson, north-west of Dover.

The principal struggle occurred on Bufford's Hill, where repeated assaults and
counter-assaults were exchanged. After prolonged contest, the enemy gave ground
and fell back beyond the crest. Our regiments, though considerably reduced,
gained and held the height at day's end, the position now secured to our
possession.

In the course of the battle, our command sustained approximately 600 infantry
killed and wounded, in addition to the loss of one artillery battery, disabled
and left behind in the action. The remainder of the guns were well-served
throughout and contributed materially to checking the enemy's advance.

On the flanks, engagements developed with equal determination. To the north of
Bufford's Hill, the enemy was seen deploying infantry in wooded ground, while to
the south they were observed occupying the grounds about Hopewell Church. These
maneuvers, though active, did not alter the outcome, for the crest of Bufford's
Hill remained in our hands.

I deem it proper to note a new and unfamiliar formation employed by the enemy at
one point during the day. Arranged at a sharp angle, it resembled the form of
the letter ``L,'' and permitted them to pour fire from an unexpected quarter
upon advancing infantry in the low ground. Though strange to our experience,
some officers believe it may possess utility should circumstances of terrain
permit its employment in future engagements.

The conduct of the men was steady and determined under difficult circumstances,
and the day has ended with ground gained and the enemy driven from the top of
the hill.

\gramClosing{I remain, General, very respectfully, Your obedient servant}
    {Tyler Remington}
    {Lieut. Colonel, Assistant Adjutant General}
\subsecdinkus

\subsection{Report of Surg. Charles Keeney, U. S. Army, Medical Director} % {{{3

\gramHeader{Headquarters, Army of the Cumberland} % {{{4
{North of Fort Donelson, Tenn., February}{24, 1862}
\gramTo{Maj. Gen.}{James Blake}
{Commanding, Army of the Cumberland}

\gramHi{General} I have the honor to submit, for your consideration, certain
observations touching upon the health and condition of this Army, drawn from
reports received from the various divisions and hospitals now in the field.

The constant operations of the past month, conducted in all manner of weather
and through regions of deep mire and inundated ground, have borne heavily upon
the constitution of the men. The prolonged rains and unremitting marches,
coupled with the labors incident to camp and campaign, have produced a manifest
increase in sickness and physical exhaustion. Fevers, rheumatisms, affections of
the lungs, and disorders of the bowels are notably on the rise.

The hospitals are fast filling, and the convalescents returning to duty are few.
The strain upon the troops, if continued under the present conditions, must
inevitably result in a marked deterioration of the Army's strength and
efficiency. The spirit of the men remains willing, yet their bodies yield daily
to the combined assaults of fatigue, exposure, and disease.

It is therefore my duty, as Surgeon of the Army, to recommend in the strongest
terms that the troops be allowed a period of rest and recuperation of not less
than two or three weeks at the earliest practicable moment. Such respite would
serve to restore vigor, arrest the spread of sickness, and ensure the
preservation of that health upon which the success of future operations must
depend.

\gramClosing{I am, General, very respectfully, Your obedient servant}
{Charles C. Keeney}
{Surgeon, U.S. Army, Medical Director, Army of the Cumberland}
\subsecdinkus

\subsection{Reports of Brig. Gen. Ptolemy Smith, commanding VIIIth Corps} % {{{3
\gramHeader{Headquarters, VIIIth Corps} % {{{4
    {8.15~am January}{25, 1862}
\gramTo{Maj. Gen.}{James Blake}
    {Commanding the Army of the Cumberland}

\gramHi{Sir} XII Corps' lead brigades heavily engaged in the attack, some
losses. Enemy seems to be cycling one or two infantry regiments to the works.
Southern brigade taking heavy losses. Unsure if my bombardment has been of much
use. Will cease fire as they cross paths of my guns. I will make determination
at 8.30~am if I should commit my division.

\gramClosing{Your obedient servant}
    {P. Smith}
    {Brig. Gen., Commanding}
\reportdinkus

\gramHeader{Headquarters, VIIIth Corps} % {{{4
    {9.00~am January}{25, 1862}
\gramTo{Maj. Gen.}{James Blake}
    {Commanding the Army of the Cumberland}

\gramHi{Sir} Committing my 2d brigade, 2d Division. Confederate guns to
southeast are hitting the ford and my regt's are taking casualties as they
cross. Northern redoubt still in our hands. I am sending 4 btry's from the Army
reserve south to assist Gen. Wood \& the rest there.

\gramClosing{Your obedient servant}
    {P. Smith}
    {Brig. Gen., Commanding}
\reportdinkus

\gramHeader{Headquarters, VIIIth Corps} % {{{4
    {6.45~am}{Fair View School}
\gramTo{Maj. Gen.}{James Blake}
    {Commander, Army of the Cumberland}

No infantry engaged. Just a lively artillery duel on my front. Other than a minor
Confederate probe, should it be necessary, I can most likely direct infantry to
support other areas.

\gramClosing{}
    {Ptolemy Smith}
    {Brig. Gen., Commanding}
\subsecdinkus

\subsection{Report of Brig. Gen. William Sherman, U.S. Army, commanding 3d Brigade, 2d Division} % {{{3
\gramHeader{3d Brigade, 2d Division, VIIIth Corps} % {{{4
    {Field Headquarters, Near Fort Donelson, Tenn., January}{27th, 1862}
\gramTo{Maj. Gen.}{James W. Blake}
    {Commanding Army of the Cumberland}

\gramHi{General} During yesterday's battle, there was no movement from the
Confederates opposite the crossing. Five regiments were observed across from me
on their right wing of the barricade. Three of the units were of the Orphan
Brigade and two were from Hardee's Brigade; all five are of Granbury's division.
General Granbury himself was present, along with another senior commander who I
could not identify, although this may have been General Clarke himself.

My batteries and regiments remain in position, ready to repel any attack.

\gramClosing{I am your most obedient servant}
    {William T. Sherman}
    {Brig. Gen., Commander, 3d Brigade, 2d Division, VIII Corps}
\subsecdinkus

\subsection{Reports of Brig. Gen. Thomas Wood, U.S. Army, commanding 1st Division} % {{{3
\gramHeader{1st Division, XIIth Corps} % {{{4
    {7.30~am January}{25, 1862}
\gramTo{Brig. Gen.}{Lawrence Graham}
    {Commander, Cavalry Division}

\gramHi{Sir} can you send a very small skirmishing force Southwest of you, then
cross the river into the thick woods to see if there is an enemy present there?

\gramClosing{Your obedient servant}
    {Thomas Wood}
    {Brig. Gen., Commanding}
\reportdinkus

\gramHeader{1st Division, XIIth Corps} % {{{4
    {8.15~am January}{25, 1862}
\gramTo{Gen'ls.}{James Blake, Ptolemy Smith, \& Horatio Van Cleve}
    {Commanding Army of the Cumberland, VIIIth Corps, \& 2d Division, XIIth Corps, respectively}

\gramHi{Sirs} The enemy has at least two brigades defending the North crossing,
one per redoubt.  They also have infantry and artillery defending the center.
The whole thing looks like a ``U'' with the top facing the river.

We've pushed back a regiment in the North.  I'm going to concentrate two
brigades in a charge over the breastworks into the North redoubt to overwhelm
the enemy with pure numbers.  They are the weakest there.

I likely need a division behind me to follow up once we take the redoubt if this
works.

\gramClosing{Your obedient servant}
    {Thomas Wood}
    {Brig. Gen., Commanding}
\reportdinkus

\gramHeader{1st Division, XIIth Corps} % {{{4
    {9.15~am January}{25, 1862}
\gramTo{Gen'ls}{James Blake \& Ptolemy Smith}
    {Commanding Army of the Cumberland \& VIIIth Corps, respectively}

\gramHi{Sirs} The breastworks are just obliterated at this point.  Both sides
are fighting hand to hand atop it.  I see 5 Confederate regiments desperately
trying to hold on against the oncoming tide.

My guns are continuing to fire upon the enemy, but I have about 30 minutes of
ammunition left.  I see the reserve guns coming up to take their place.

\gramClosing{Your obedient servant}
    {Thomas Wood}
    {Brig. Gen., Commanding}
\reportdinkus

\gramHeader{1st Division, XIIth Corps} % {{{4
    {Fort Donelson, Tenn., January}{27, 1862}
\gramTo{Maj. Gen.}{James W. Blake,}
    {Commanding, Department of the Cumberland}

\gramHi{General} As requested, here is the After Action Report for the 1st Day
of the Battle of Fort Donelson.

My command arrived to the field early on the morning of the battle.  As ordered,
we moved East to the Northern crossing across the river, the crossing closest to
Fort Donelson, taking the place of the cavalry, which began moving South.

My men spotted two encampments guarding this crossing, one directly on the road
and one further South.  Both were built to the East of the creek upon high
ground.  The Artillery Reserve to our North began shelling the North Encampment
while I directed my artillery to do the same to the Southern one.

As discussed with you, it was believed that a test of the enemy's defenses was
necessary.  A reconnaissance in strength, if you will. First Brigade deployed
forward and crossed.  Immediately, it was hit by significant enemy fire.  It
turns out, the two encampments were part of a U-shaped defense with the opening
facing the crossing.  Two batteries sat in the South Encampment and one battery
was placed at the bottom of the U.  Across the whole enemy front was about two
brigades.

Despite the gunfire, my division was able to force back the enemy from the
Northern encampment and briefly held it for thirty minutes despite heavy
casualties.  The Confederates were forced to deploy a reserve brigade to push my
men out of the fortifications.  Despite reinforcement from VIII Corps, the men
were unable to hold onto the breastworks and fell back to the West side of the
creek.

The three brigades of my division suffered heavy casualties.  I believe that we
fought three Confederate brigades, breaking one of theirs.  They had a fourth
brigade behind the three we fought that did not engage.  We spotted no batteries
beyond the three mentioned earlier.  I suspect we faced an entire Confederate
division, which was part of the Right Wing of of the Army of Mississippi.

It is my belief that the enemy's position on the Northern Crossing is far too
strong to attack directly.  We must move around them to the South.

\gramClosing{Respectfully}
    {Thomas Wood}
    {Brigadier General, 1st Division, XII Corps, Army of the Cumberland}

\subsecdinkus

\subsection{Reports of Brig. Gen. Horatio van Cleve, commanding 2d Division, XIIth Corps} % {{{3
\gramHeader{2d Division, XIIth Corps} % {{{4
    {7.00~am January}{25, 1862}
\gramTo{Brig. Gen.}{Charles Smith}
    {Commander, XIIth Corps}

\gramHi{Sir} I Intend to halt at the Y-junction to be in the best position to
deploy if our main attack needs to come by the southern approach, unless in the
meantime intelligence shows we are ready to cross as planned. 

\gramClosing{Your servant}
    {Horatio Van Cleve}
    {Brig. Gen., Commanding}
\reportdinkus

\gramHeader{2d Division, XIIth Corps} % {{{4
    {7.15~am January}{25, 1862}
\gramTo{Brig. Gen.}{Lawrence Graham}
    {Commander, Cavalry Division}

\gramHi{Sir} I am reserve for the main attack. I am planning to halt near the
Y-junction in case I need to divert southwest for the main attack.  Could you
kindly advise me of your intelligence so if I need to march on your position I
know what I will be walking into. 

\gramClosing{Your servant}
    {Horatio Van Cleve}
    {Brig. Gen., Commanding}
\reportdinkus

\gramHeader{2d Division, XIIth Corps} % {{{4
    {7.15~am January}{25, 1862}
\gramTo{Brig. Gen.}{John Wool}
    {Commander, 1st Division, VIIIth Corps}

\gramHi{Sir} I am reserve for the main attack. I am planning to halt near the Y
junction in case I need to divert southwest towards Indian Creek for the main
attack. If you can ensure the roads remain clear I would appreciate it.

\gramClosing{Your servant}
    {Horatio Van Cleve}
    {Brig. Gen., Commanding}
\reportdinkus

\gramHeader{2d Division, XIIth Corps} % {{{4
    {7.45~am January}{25, 1862}
\gramTo{Maj. Gen.}{James Blake}
    {Commander, Army of the Cumberland}

\gramHi{Sir} I assume you likely received this report I received from the
cavalry but am forwarding to you just in case:

\begin{gramQuote}
    I've begun probing the crossing, defenses here are stiff. I can see 6
    brigades of infantry with 4 artillery batteries towards where Hickman Creek
    branches west. I assume there to be two entire infantry divisions plus
    artillery along Indian Creek. I do not advise joining me here as I doubt we
    could make a reasonable attempt at crossing for the moment.
\end{gramQuote}

\gramClosing{Your servant}
    {Horatio Van Cleve}
    {Brig. Gen., Commanding}
\reportdinkus

\gramHeader{2d Division, XIIth Corps} % {{{4
    {8.30~am January}{25, 1862}
\gramTo{Maj. Gen.}{James Blake}
    {Commanding the Army of the Cumberland}

\gramHi{Sir} The Cavalry Division reports they are under attack by two rebel
divisions coming across Hickman Creek. With this force headed towards our rear,
I do not see that we can make the planned attack under these conditions,
therefore If I do not receive other immediate instructions I will move my
division southwest on my own initiative to prepare to repel the rebel attack.
 
\gramClosing{Your servant}
    {Horatio Van Cleve}
    {Brig. Gen., Commanding}
\reportdinkus

\gramHeader{2d Division, XIIth Corps} % {{{4
    {8.30~am January}{25, 1862}
\gramTo{Brig. Gen.}{Lawrence Graham}
    {Commander, Cavalry Division}

\gramHi{Sir} I am forming up two brigades to prepare the follow up attack on the
north redoubt east of Hickman creek and stationing one with my artillery to
guard south in case of rebel attacks.
 
\gramClosing{Your servant}
    {Horatio Van Cleve}
    {Brig. Gen., Commanding}
\reportdinkus

\gramHeader{2d Division, XIIth Corps} % {{{4
    {8.30~am January}{25, 1862}
\gramTo{Brig. Gen'ls.}{Lawrence Graham \& Thomas Wood}
    {Commanding the Cavalry Division \& 1st Division, XIIth Corps, respectively}

\gramHi{Sir} I am forming up two brigades to prepare the follow up attack on the
north redoubt east of Hickman creek and stationing one with my artillery to
guard south in case of rebel attacks. 

\gramClosing{Your servant}
    {Horatio Van Cleve}
    {Brig. Gen., Commanding}
\reportdinkus

\gramHeader{2d Division, XIIth Corps} % {{{4
    {9.15~am January}{25, 1862}
\gramTo{Gen'ls}{James Blake \& Lawrence Graham}
    {Commanding the Army of the Cumberland \& Cavalry Division, resp'y}

\gramHi{Sir} There is no rebel activity near my defensive line by the woods. We
have good position and reserves here.  I am probing to check what the rebels are
doing and make sure they do not forget we are here.

\gramClosing{Your servant}
    {Horatio Van Cleve}
    {Brig. Gen., Commanding}
\subsecdinkus

\subsection{Reports of Brig. Gen. John McClernand, commanding XIVth Corps} % {{{3
\gramHeader{Headquarters, XIVth Corps} % {{{4
    {Near Bufford's Hill, 12.30~am, January}{28, 1862}
\gramTo{Brig. Gen.}{George Thomas}
    {Commander, 1st Division, XIVth Corps}

\gramHi{General} Do you have any idea how many divisions you are facing?  Is
there any chance the enemy is moving around to our north?

\gramClosing{Your obedient servant}
    {John McClernand}
    {Brig. Gen. Commanding}
\reportdinkus

\gramHeader{Headquarters, XIVth Corps} % {{{4
    {Near Bufford's Hill, 12.45~am, January}{28, 1862}
\gramTo{Brig. Gen.}{George Thomas}
    {Commander, 1st Division, XIVth Corps}

\gramHi{General} Continue to advance if the enemy is withdrawing. Once it is
clear the enemy is going to stand and fight, find good ground and hold.

\gramClosing{Your obedient servant}
    {John McClernand}
    {Brig. Gen. Commanding}
\reportdinkus

\gramHeader{Headquarters, XIVth Corps} % {{{4
    {6.00~am}{}
\gramTo{Maj. Gen.}{James Blake}
    {Commanding, Army of the Cumberland}

\gramHi{General} My men are positioned at the crossroads at the valley, just
Southwest of the Bear "river" word on the map.  They are positioned on good
terrain West to East, though their is a clear gap in the middle where the valley
is.

The gap is necessary as the enemy has established a grand battery in front of me
down the valley.  At least four batteries, possible more.  They have already
routed my own artillery and a regiment due to this bombardment.  The remaining
regiments in the valley I am repositioning behind a hill to protect them from
slaughter.  The rebel batteries cannot reach my regiments positioned on the
ridge.

I'm being pushed on my right by a brigade, which I am attempting to refuse.  I
do face rebels on the left, Northwest of the Bear writing, but the rebels seem
content to not engage me  there for some reason.

\gramClosing{Regards}
    {John McClernand}
    {Brig. Gen., Commanding}
\reportdinkus

\gramHeader{Headquarters, XIVth Corps} % {{{4
{At the Crossroads South of Bear Creek, 6.30 am, February}{3, 1862}
\gramTo{Maj. Gen.}{James Blake}
{Commanding, Army of the Cumberland}

\gramHi{General} Apologies for my last message.  In the heat of battle, I must
have gotten my orientation wrong.  My lines runs south to north.  Just west of
the crossroads of the main road running north to south and the road parallel to
Bear Creek.

Intense fighting on my right.  I suspect two rebel brigades pushing there.

Deploying my reserves towards that location, but will not commit them unless
needed.

Still no pressure on my left.  The artillery duel in the valley is pretty
finished with a rebel victory.

If the rebels should push me off this mountain, where do you want to withdraw my
men?

\gramClosing{Regards}
    {John McClernand}
    {Brig. Gen., Commanding}
\reportdinkus

\gramHeader{Headquarters, XIVth Corps} % {{{4
    {Hickory Grove Church, Tenn., February}{15, 1862}
\gramTo{Col.}{Walter Chekov}
    {Adjt. Gen., Army of the Cumberland}

\gramHi{Sir} I have the honor to submit to you my report of the action on
February 8 around Bufford's Hill.

Around 6.15~am, four batteries of enemy artillery moved down the valley from
Hopewell Church and engaged the artillery of Thomas' Division in a furious duel.
After about 30 minutes of ferocious bombardment, Capt's. Simmonds and Konkle
were forced to withdraw their batteries, suffering a total loss of six guns in
the engagement.  At the same time, Brig. Gen. Thomas reported Carter's Brigade,
deployed around the sawmill at the far right of the line, was being pressed by a
brigade of Pickford's Left Wing.

Miller's Brigade rapidly moved to reinforce Col. Carter but the enemy assault
was relentless and the right flank gradually gave way. Col. James Carter's 2d
Tennessee Regiment, however, bravely held as long as they could to slow the
enemy advance and give the remainder of Thomas' Division time to prepare to meet
the enemy.

Only 10 men of the 2d Tennessee have returned to my lines but the report of
their regiment's valor is unlike anything I have yet seen in this war. The units
involved in the action around the sawmill, including the 2d Tennessee,
eventually broke in front of the enemy onslaught. However, Col. James Carter
managed to rally the large portion of his regiment and, though surrounded,
encouraged them to fight as long as they could.  It is unknown how many of the
regiment survived the engagement and were taken prisoner but I fear the number
must be small.

The performance of the 2d Kentucky and 14th Ohio, commanded by Col's. Sedgewick
and Steedman, respectively, must also be mentioned. Those two regiments also
rallied and held their ground, joined eventually by troopers of Col. Dickey's
cavalry brigade. Their stiff defense was vital in preventing the enemy from
turning my, and the army's, right flank north of the sawmill.

The enemy began pressing the north slope of Bufford's Hill at about 6.45~am and
Smith's Brigade was shortly driven east across the road. There they, and the
remainder of the division, managed to hold, although, once Col. Carter's
Tennesseans around the sawmill ceased fighting, additional enemy forces were
brought to bear from the north.  Eventually, the superiority of our positions
won the day with the enemy ending the assault around 8.45~am.

At about 8.30~am, Rosecrans' Division began to engage enemy forces moving north
along the east side of Bufford's Hill. The counterattack south of Bufford's Hill
by VIIIth and XIIth Corps helped relieve much of the pressure on Brig. Gen.
Rosecrans and he maintained his general position by the end of the engagement.

My corps is now in much the same position as it was prior to the action of the
28th ult., although I was able to retain a foothold on the hill to the south of
Bufford's Hill, just south of the B Branch. The following list of casualties
shows a loss in the corps as follows:

\vspace{5pt}
\begin{dispatch}[
]{
    colspec = {X[l]r},
}

Commissioned officers: \\
\hspace{2em}Killed\dotfill & 12\\
\hspace{2em}Wounded\dotfill & 25 \\
\hspace{2em}Missing\dotfill & 36 \\

Enlisted men: \\
\hspace{2em}Killed\dotfill & 269 \\
\hspace{2em}Wounded\dotfill & 841 \\
\SetRow{belowsep+=2pt}
\hspace{2em}Missing\dotfill & 620 \\

\cmidrule{2}\SetRow{abovesep+=4pt}
\hspace{3em}Total killed, wounded and missing\dotfill & 1,803 \\
\end{dispatch}
\vspace{5pt}

The corps also lost six pieces of artillery. In the list of officers killed is
Col. William Kise, 10th Indiana.  In the list of officers missing are Colonel
James Carter and Lieutenant Colonel Daniel Trewhitt, 2d Tennessee. The history
of the war will record no brighter names than these two and the men who fought
beside them.

Among the wounded are Col. Thomas Sedgewick, 2d Kentucky and Col. James
Steedman, 14th Ohio, although, gratefully, their wounds are minor and I expect
them to return to duty quickly.

\gramClosing{Your obedient servant}
    {John McClernand}
    {Brig. Gen., Commanding}
\reportdinkus

\gramHeader{Headquarters, XIVth Corps} % {{{4
{Hopewell Church, 4.30 pm, February}{21, 1862}
\gramTo{Maj. Gen.}{James Blake}
{}

\gramHi{General} I am pleased to report that my men have occupied Hopewell
Church and the enemy has completely pulled back from the northern half of
Bufford's Hill though they still hold the area west of Bufford's Cemetery. I
have a brigade at Hopewell's Church and am bringing up my two batteries of my
First Division's artillery.

\gramClosing{Signed}
{McClernand}
{}
\subsecdinkus

\subsection{Reports of Brig. Gen. George Thomas, U.S. Army, commanding 1st Division} % {{{3
\gramHeader{1st Division, XIVth Corps} % {{{4
    {Near the Sawmill, 11.15~am, January}{28, 1862}
\gramTo{Brig. Gen.}{John McClernand}
    {Commander, XIVth Corps}

\gramHi{Sir} Nothing but an artillery duel here so far. I occupy good ground and
I am scouting to my north. I will send information when I get it.

\gramClosing{Your obedient servant}
    {George Thomas}
    {Brig. Gen. Commanding}
\reportdinkus

\gramHeader{1st Division, XIVth Corps} % {{{4
    {Near the Sawmill, 11.45~am, January}{28, 1862}
\gramTo{Brig. Gen.}{John McClernand}
    {Commander, XIVth Corps}

\gramHi{Sir} Enemy formations of infantry forming to my front at some distance.
I am not sure what they are doing exactly. I will send an update if they move on
me. 

\gramClosing{Your obedient servant}
    {George Thomas}
    {Brig. Gen. Commanding}
\reportdinkus

\gramHeader{1st Division, XIVth Corps} % {{{4
    {Near the Sawmill, 11.45~am, January}{28, 1862}
\gramTo{Brig. Gen.}{John McClernand}
    {Commander, XIVth Corps}

I am making progress on my north as well. I have engaged at least one brigade
here. My attack on the southern hill (Bear Hill) should be developing in my
favor as well.  Your artillery has just arrived.

\gramClosing{Your obedient servant}
    {George Thomas}
    {Brig. Gen. Commanding}
\subsecdinkus

\subsection{Reports of Brig. Gen. Lawrence Graham, U.S. Army, commanding Cavalry Division} % {{{3

\gramHeader{Cavalry Division} % {{{4
    {Camp near Golden Pond, Ky., Morning, January}{15, 1862}
\gramTo{Maj. Gen.}{James W. Blake}
    {Commanding General, Army of the Cumberland}

\gramHi{Sir} On the morning of the 5th instant, pursuant to orders to
reconnoiter the country and disperse any small parties of the enemy reported
therein, I advanced with my division upon the road leading southward from
Smithland. Despite my most scrupulous inquiries, I was furnished by the local
inhabitants with no precise account of the enemy's numbers or dispositions.
Thus, believing the opposition to be of minor character---mere
guerrillas or detached scouts---I pressed forward with celerity.
That's my style, sir. Attack, attack, attack!

At approximately ten o'clock, my advance guard engaged what was first thought to
be a trifling force. The enemy, however, proved unexpectedly numerous, being a
body of Confederate cavalry and cannon that were well placed.

Notwithstanding these obstacles, our troopers made repeated and spirited
charges. Unfortunately, the terrain afforded little room for maneuver, and the
enemy, profiting by his superior knowledge of the ground, enveloped both of my
flanks. The zeal of my officers and men, and especially myself, was beyond all
praise, yet the fatigue of the horses, the suddenness of the attack, and the
want of timely intelligence rendered it impossible to maintain our ground.

After nearly two hours of stubborn contest, and perceiving the day lost, I
directed a retirement upon the northerly road. The withdrawal was conducted with
as much order as the broken nature of the country would allow, and the command
is now refitting in its former camp. Our loss, though not inconsiderable, is
chiefly in prisoners, the number of killed and wounded being comparatively
small.

I cannot refrain from observing that the lamentable result of this affair was
due less to any fault of my command than to the extraordinary advantages enjoyed
by the enemy in position, numbers, and previous preparation. With adequate
intelligence and timely cooperation, I am confident the outcome must have been
far different.

\gramClosing{Very respectfully}
    {Lawrence Graham}
    {Brig. Gen., Commanding}
\reportdinkus

\gramHeader{Cavalry Division} % {{{4
    {7.30~am January}{25, 1862}
\gramTo{Maj. Gen.}{Blake}
    {Commander, Army of the Cumberland}

\gramHi{Sir} I've begun probing the crossing, defenses here are stiff. I can see
6 brigades of infantry with 4 artillery batteries towards where Hickman Creek
branches west. I assume there to be two entire infantry divisions plus artillery
along Indian Creek.

\gramClosing{Your humble servant}
    {Lawrence Graham}
    {Brig. Gen., Cavalry Division}
\reportdinkus

\gramHeader{Cavalry Division} % {{{4
    {7.30~am January}{25, 1862}
\gramTo{Brig. Gen.}{Horatio Van Cleve}
    {Commander, 2d Division, XIIth Corps}

\gramHi{Sir} I've begun probing the crossing, defenses here are stiff. I can see
6 brigades of infantry with 4 artillery batteries towards where Hickman Creek
branches west. I assume there to be two entire infantry divisions plus artillery
along Indian Creek. I do not advise joining me here as I doubt we could make a
reasonable attempt at crossing for the moment.

\gramClosing{Your humble servant}
    {Lawrence Graham}
    {Brig. Gen., Cavalry Division}
\reportdinkus

\gramHeader{Cavalry Division} % {{{4
    {7.45~am January}{25, 1862}
\gramTo{Maj. Gen.}{Blake}
    {Commander, Army of the Cumberland}

\gramHi{Sir} The two divisions of rebel infantry have decided to charge across
the creek and up a hill at my cavalry, exposing their flank. I am going to
attempt to exploit this by charging their flank, which is also under fire from
my artillery. I previously advised the reserves against deploying to my
position, however now I believe they may be necessary.

\gramClosing{Your humble servant}
    {Lawrence Graham}
    {Brig. Gen., Cavalry Division}
\reportdinkus

\gramHeader{Cavalry Division} % {{{4
    {7.45~am January}{25, 1862}
\gramTo{Brig. Gen.}{van Cleve}
    {Commander, 2d Division, XIIth Corps}

\gramHi{Sir} The two divisions of rebel infantry have decided to charge across
the creek and up a hill at my cavalry, exposing their flank. I am going to
attempt to exploit this by charging their flank, which is also under fire from
my artillery. I previously advised the reserves against deploying to my
position, however now I believe they may be necessary.  If my charge goes well,
we may be able to capitalize on the situation. If it does not go well we may
need a stronger force here to keep these aggressive rebels from flanking the
rest of our forces.

\gramClosing{Your humble servant}
    {Lawrence Graham}
    {Brig. Gen., Cavalry Division}
\reportdinkus

\gramHeader{Cavalry Division} % {{{4
    {8.15~am January}{25, 1862}
\gramTo{Maj. Gen.}{Blake}
    {Commander, Army of the Cumberland}

\gramHi{Sir} I now estimate there are 3 infantry divisions and with accompanying
artillery at Indian Creek. My force will be attempting to evacuate the area, I
now strongly recommend sending the reserves to my location.

\gramClosing{Your humble servant}
    {Lawrence Graham}
    {Brig. Gen., Cavalry Division}
\reportdinkus

\gramHeader{Cavalry Division} % {{{4
    {8.15~am January}{25, 1862}
\gramTo{Brig. Gen.}{van Cleve}
    {Commander, 2d Division, XIIth Corps}

\gramHi{Sir} I now estimate there are 3 infantry divisions and with accompanying
artillery at Indian Creek. My force will be attempting to evacuate the area, I
now strongly recommend sending the reserves to my location.

We have capitalized well on our initial advantage, but it is clear to me that we
cannot maintain this position. I will be attempting to pull back. I might
recommend you attempt to approach us not from the west but the north to prevent
the rebels from driving a wedge between us and the rest of the army.

\gramClosing{Your humble servant}
    {Lawrence Graham}
    {Brig. Gen., Cavalry Division}
\reportdinkus

\gramHeader{Cavalry Division} % {{{4
    {8.15~am January}{25, 1862}
\gramTo{Maj. Gen.}{Blake}
    {Commander, Army of the Cumberland}

\gramHi{Sir} My forces have come under full attack from these rebels, I can no
longer maintain my position. I have been all but forced to withdraw north. I
would strongly encourage you to deploy the reserves to cover the southern flank
of your attack.

Again, I am facing upwards of 3 infantry divisions with 4 artillery brigades.

\gramClosing{Your humble servant}
    {Lawrence Graham}
    {Brig. Gen., Cavalry Division}
\reportdinkus

\gramHeader{Cavalry Division} % {{{4
    {8.30~am January}{25, 1862}
\gramTo{Maj. Gen.}{Blake}
    {Commander, Army of the Cumberland}

\gramHi{Sir} There are at least 11 regiments of infantry I can see, as well as 4
batteries of artillery. I also know there are more beyond my vision.

I fully believe an attack is developing in the south as they are continuing to
march north after my retreating forces. My force is battered, but only one
brigade has broken and our artillery is intact.

\gramClosing{Your humble servant}
    {Lawrence Graham}
    {Brig. Gen., Cavalry Division}
\reportdinkus

\gramHeader{Cavalry Division} % {{{4
    {8.45~am January}{25, 1862}
\gramTo{Maj. Gen.}{Blake}
    {Commander, Army of the Cumberland}

\gramHi{Sir} I have arrived at the T-intersection and am going to attempt to
form a delaying action in the forest due south.

\gramClosing{Your humble servant}
    {Lawrence Graham}
    {Brig. Gen., Cavalry Division}
\reportdinkus

\gramHeader{Cavalry Division} % {{{4
    {8.30~am, February}{8, 1862}
\gramTo{Maj. Gen.}{James Blake}
    {Commander, Army of the Cumberland}

I sent two regiments of cavalry south on the Trinity Church Road as requested
and they ran smack into rebel infantry north of the intersection with Hickory
Grove Church Road: at least three regiments. I am bringing the rest of my
brigade in support and I have ordered another brigade to come south along the
river and then westward to flank them. 

\gramClosing{}
    {Lawrence Graham}
    {Brig. Gen., Commanding}
\reportdinkus

\gramHeader{Cavalry Division} % {{{4
{East of Fort Henry, on the Panther Creek Road, 9.00 pm, February}{21, 1862}
\gramTo{Maj. Gen.}{James Blake}
{Commanding, Army of the Cumberland}

\gramHi{General} I have the honor to report the operations of my division on
this date, east of Fort Henry.

In obedience to orders, my command advanced at 8.30 in the morning, following
the landing of General Steele's Army of the Kanawha the previous day. The
morning was cool and overcast, and the roads, though soft in places, offered
fair passage. Near the crossing at Panther Creek, our advance squadrons
encountered the enemy's cavalry belonging to the Army of Mississippi under
General Whisper. They were promptly engaged and, after a sharp contest of sabers
and carbines, were driven back in disorder, retreating westward toward the main
body.

The enemy's force was found to consist of one infantry division holding Fort
Henry and one cavalry division positioned westward of that point. About 2~
o'clock in the afternoon, a second infantry division was observed arriving from
the lower roads and uniting with the cavalry for an attack upon our left and
center. The assault was pressed with vigor, but my regiments held firm,
repelling each charge with steady volleys and determined resistance.

At 3.45~pm the enemy renewed the attack in greater strength, their infantry
advancing in line while their cavalry attempted to turn our right. The fighting
was close and severe until near 4.45~pm. My artillery, brought forward at the
height of the action, unlimbered within four hundred yards of the enemy and
opened with double canister, shattering their charge and driving them from the
field in confusion.

Our losses are estimated at three hundred in killed, wounded, and missing.
Reinforcements from General Steele's Cavalry Division---numbering some 2,400
men---reached us during the latter phase of the action and aided materially in
securing our ground. By nightfall the enemy had fallen back westward, and we now
occupy our former position, maintaining our screen and preventing further
reinforcement to Fort Henry.

The men have borne themselves admirably throughout the day's engagement. Special
mention is due to Capt. Edgarton's Battery, 1st Ohio Light Artillery and Capt.
Terrill's Battery, 5th U.S. Artillery, for their gallant conduct under fire, and
to the troopers whose sabers won us the early success.

The capture of Fort Henry by General Steele and his gallant command crowns the
day's operations with victory, yet I take pride in the steadfast discipline and
courage displayed by my Division in every phase of the battle.

\gramClosing{I am, General, very respectfully, Your obedient servant}
{Lawrence Graham}
{Brig. Gen., Commanding Cavalry Division}
\reportdinkus

\gramHeader{Cavalry Division} % {{{4
    {North of Fort Henry, Tenn., 11.00 pm, February}{20th, 1862}
\gramTo{Maj. Gen.}{James Blake}
    {Commanding}

\gramHi{Sir} I have the honor to report that I have this evening conferred with
General Steele, and am acting in concert with his command for the execution of
operations now in progress between the rivers. Pursuant to his instructions, I
shall move at first light to-morrow, the 21st instant, southeastward, with the
design of cutting the enemy's communications between Forts Henry and Donelson.

Our intention is to sever the line of passage by which the garrison of Fort
Henry might seek support or retreat upon Donelson, thereby isolating the post
and constraining its defense. My cavalry will take the advance, striking toward
the interior roads leading from the Henry position to the eastward.

Reports from my scouts indicate Confederate infantry in moderate strength along
the lower approaches, evidently employed in watching our movements. Their
presence will not delay the march, and I shall endeavor to sweep them aside or
disperse them in passing.

\gramClosing{I am, General, Very respectfully, Your obedient servant}
{Lawrence Graham}
{Brig. Gen., Commanding Cavalry Division}
\reportdinkus

\gramHeader{Headquarters, Cavalry Division} % {{{4
{South of Bubbling Springs, Tenn., along the Andrews Hollow Road February}{24, 1862}
\gramTo{Maj. Gen.}{James Blake}
{Commanding, Army of the Cumberland}

\gramHi{General} I have the honor to submit the following report regarding the
present condition of my command.

The Cavalry Division now numbers ten regiments in the field, organized as
follows: the First Brigade, three regiments; the Second Brigade, three
regiments; the Third Brigade, four regiments; with two Division batteries of
horse artillery attached and in effective service.

After a month of constant operations---embracing near-daily skirmishes, long
marches, and frequent exposure to inclement weather---the Division has suffered
a total loss of about six hundred men, killed, wounded, and missing. The
casualties have been made good as far as practicable by the consolidation of
depleted units, and the command, though reduced in numbers, remains cohesive and
well-disciplined.

The men are worn by the rigors of the campaign and by the hardships incident to
active service in this rough and broken country, which is most unfavorable to
cavalry operations. The roads are narrow, the hollows deep, and the ground so
intersected by creeks and wooded ridges that mounted movements are often
impossible except by single file. Nevertheless, the spirit of the troops
continues high. Their conduct has been exemplary under every trial, and I am
confident that, if ordered to advance, they would do so with their accustomed
steadiness and vigor.

That said, I cannot but acknowledge that a short period of rest would prove of
great benefit to both men and horses, who have borne the strain of constant
movement with admirable endurance but are now much in need of repose.

\gramClosing{I remain, General, very respectfully, Your obedient servant}
{Lawrence Graham}
{Brig. Gen., Commanding Cavalry Division}

\subsecdinkus

\subsection{Report of Lieut. Col. James W. Herrick, Executive Officer} % {{{3
\gramHeader{Headquarters, Cavalry Division} % {{{4
    {Camp near Golden Pond, Ky., Morning, January}{15, 1862}
\gramTo{Maj. Gen.}{James W. Blake}
    {Commanding General, Army of the Cumberland}

\gramHi{Sir} I feel compelled to submit a candid account of the action at Golden
Pond and of Brig. Gen. Lawrence Graham's conduct therein. On the 5th instant,
the Colonel advanced the division southward from Smithland with little heed for
reconnaissance. Though advised by his officers to proceed with caution, he
pressed forward upon the most direct road, declaring the enemy to be but
scattered guerrillas.

At ten o'clock our advance collided with a Confederate force in far greater
strength, supported by artillery and well posted upon ground unsuited to cavalry
maneuver. Still believing the foe trifling, Col. Graham ordered repeated charges
without reserves, exhausting men and horses alike. The enemy soon enveloped both
flanks, while no preparations had been made for withdrawal or reinforcement.
Messages for support were sent too late and too imprecisely to be of use.

The rank and file bore themselves gallantly, charging as ordered until endurance
failed. Their bravery cannot be faulted; their losses stem from the Colonel's
want of foresight. A more prudent reconnaissance, timely summons of support, or
choice of ground would likely have secured success.

It is my sober judgment that the reverse at Golden Pond lies not in the enemy's
advantages alone, but chiefly in the imprudence and rashness of Col. Graham,
whose preference for ``attack, attack, attack'' blinded him to the realities
before him.

\gramClosing{Respectfully}
    {James W. Herrick}
    {Lieut. Col., Executive Officer}
\subsecdinkus

\subsection{Reports of Maj. Gen. Richard Steele, commanding Army of the Kanawha} % {{{3

\gramHeader{Headquarters, Army of the Kanawha} % {{{4
{Near Fort Henry, Tenn., 12.00 pm, February}{21, 1862}
\gramTo{Maj. Gen.}{James Blake}
{Commanding, Army of the Cumberland}

\gramHi{General} I have begun my attack on Fort Henry. It is a tough fight but we
are driving the enemy back into their works. I am being attacked on my eastern
flank along the Panther Creek Road by Rebel cavalry. If there is any assistance
you could provide it would be appreciated.

\gramClosing{}
{Richard Steele}
{Maj. Gen., commanding}
\reportdinkus

\gramHeader{Headquarters, Army of the Kanawha} % {{{4
{Fort Henry, 4.45 pm, February}{21, 1862}
\gramTo{Maj. Gen.}{ James Blake}
{Commanding Army of the Cumberland}

\gramHi{General} I have just received your message of 2.45~pm.

I was able to throw enough men into the line to drive off the enemy. Have just
received word that General Wool will arrive shortly. I believe this position is
now secured. The Army of the Kanawha is grateful for the support. It was
instrumental in our victory this day.

\gramClosing{Respectfully}
{Richard Steele}
{Maj. Gen., Commanding}
\reportdinkus

\gramHeader{Headquarters, Army of the Kanawha} % {{{4
{Fort Henry, Tenn., February}{23, 1862}
\gramTo{Maj. Gen.}{J. W. Blake}
{Commanding, Army of the Cumberland}

\gramHi{General} It is certainly good to be here, doing what we can to put down
this perfidious rebellion. We were expecting a sharp fight and the enemy
certainly gave us one. I wanted to extend my personal gratitude for the use of
Gen'l Graham's cavalry as they proved crucial to our success. Without their
performance our victory here would have been far from certain.

The timely arrival of Gen'l Wool may have been enough to scare the enemy away
for good, and his welcome presence allowed us to secure this position with fresh
troops and allowed my badly worn men to recuperate.

All told, we lost 3,000 men and one battery out of 24,000 engaged so it was a
dearly bought victory. We lost three naval gunboats and one of the ironclads had
to return to Cairo for repairs. However, we were able to inflict a lot of pain
on the Confederates who opposed us. Of the army I arrived with, I can report
21,400 effectives consisting of 18,000 infantry, 1,800 cavalry, and 10
batteries.

Rough estimates suggest 700 defenders were lost within the fort, along with a
dozen heavy guns, blown to kingdom come by our gallant Navy. Outside the
ramparts we counted approximately 2,400--3,200 more enemy casualties to bring
estimated enemy losses to 3--4,000. I know we faced off against Cleburne's
Division in and around Fort Henry, and intercepted Clark's Division east of the
fort at the Panther Creek Road while attempting to relieve the garrison.

The fort was badly damaged in the assault and now hardly worthy of the name.
There are four guns still operational within it's perimeter. I might suggest to
the War Department that they might be better employed on the west bank of the
river on ground better situated to sight upriver to the south.

At the moment I am not aware of any enemy in my vicinity. I have sent out
cavalry patrols to locate them but they have found none within the limited area
the weather and terrain have allowed them to search.

I have kept General Wool's Division at the Panther Creek Road, with Graham's
cavalry. My own forces are in and around the fort's environs. The fort honestly
seems better placed to defend against an attack from the south than from the
north. It was quite poorly placed in my opinion.

Combined with the remaining flotilla I am quite confident the Rebels would
suffer greatly if they attacked me along the river, and if they were to come at
us from the east again they would face the same bottleneck to their front and
expose their flank and rear to your own forces. I have established a new supply
base safely behind the fort and we are prepared to resume operations at your
suggestion.

Though I must say I am unable to locate a railroad or even a Stewart Crossing on
my map. Do you perhaps mean Scott Fitzhugh Bridge and the Austin Peay Memorial
Highway? There is a road that follows the river more or less, so I could proceed
south and then inland, backed by the Navy. But that would put us out of
supporting range of each other. I would feel more confident moving inland and
linking up with you and then together investing Dover. That way I can leave a
few brigades behind at Fort Henry with Naval backup in case they try to sneak up
the river bank after I depart.

As you are the senior commander in these parts I will of course defer to your
orders.

\gramClosing{Respectfully}
{Richard Steele}
{Maj. Gen., Commanding}
\reportdinkus

\gramHeader{Headquarters, Army of the Kanawha} % {{{4
{Fort Henry, Tenn., February}{23, 1862}
\gramTo{Maj. Gen.}{J. W. Blake}
{Commanding, Army of the Cumberland}

\gramHi{General} With the Panther Creek Road secured, I shall proceed south
along the river before turning inland at the Lost Creek Road and advance along
the Austin Peay Highway toward Dover.

With your reinforcements I can field 30,000 men and leave 2,000 men to secure
the fort. This would still likely leave me outnumbered by some amount if the
enemy were to turn their entire force upon me, but I trust that to do so they
would have to leave their flanks and rear open to attack from your army.

I will advance behind a cavalry screen to prevent ambush, and see if we can't
catch the enemy in the middle.

\gramClosing{Respectfully}
{Richard Steele}
{Maj. Gen., Commanding}
\reportdinkus

\gramHeader{Headquarters, Army of the Kanawha} % {{{4
{Fort Henry, Tenn., February}{23d, 1862}
\gramTo{Maj. Gen.}{Cornelius Van Royne}
{Commanding General, United States Army}

\gramHi{General} After the last day and a half I am able to provide some
preliminary returns on the battle at Fort Henry. We arrived with the 24,500 men
I brought from West Virginia, and we were reinforced by Graham's division of
cavalry from the Army of the Cumberland, numbering 3,300 troopers with two
batteries. All told that gave us 28,000 combat effective. We were assisted by
two ironclads of the U.S. Navy, and a squadron of gunboats.

We found a division of enemy infantry reinforcing the garrison of the fort. I
attacked with 1st Division directly as 2d \& 3d Divisions swung east to attack
from that direction. 2d Division largely encountered and drove off Cleburne's
Division of the enemy and it was 1st \& 3d Divisions that were ultimately able
to overwhelm the garrison.

I deployed my cavalry to the east to protect my flank and prevent enemy
reinforcements. This eventually arrived, with General Clark's division under the
direct command of General Whisper. Graham's cavalry, supported by 2d Division
and later 3d Division, was able to repulse their attacks. General Wool's
Division from the Army of the Cumberland arrived at the end of the day to help
my exhausted army secure the position. The enemy army has since pulled back from
Fort Henry but is still positioned to defend Fort Donelson. I am coordinating with
Maj. Gen. Blake on further operations to secure that location as well.

Of 24,500 men engaged from the Army of the Kanawha, we suffered 3,000 killed,
wounded, and missing from the infantry. 100 men and six guns lost from the
reserve artillery.

Of Graham's Cavalry division, 300 men killed, wounded, and missing among the
troopers and another 100 men and six guns lost from their artillery.

The Navy suffered one ironclad damaged by enemy fire and has been sent it back
to Cairo for repairs. I do not have an assessment of the damage. Three gunboats
from the squadron were lost to enemy fire and sunk.

The enemy lost 700 men in the garrison of Fort Henry and 12 disabled artillery
due to naval action. Four operational guns were recovered. The ramparts of the
fort were heavily reduced due to naval shelling and are rendered mostly useless
on the side facing the river.

We estimate Cleburne's Division suffered 1,500 to 2,000 casualties around Fort
Henry. We counted another 900-1,200 enemy casualties suffered by Cleburne and
Clark's divisions east of the fort around Panther Creek.

As operations are still ongoing, once I have received the full reports from my
division commanders I will submit a full after action report for the records of
the War Department.

\gramClosing{Respectfully}
{Richard Steele}
{Maj. Gen., Commanding, Army of the Kanawha}
\reportdinkus

\gramOrdersHeader{Headquarters, Army of the Kanawha} % {{{4
{Fort Henry, Tenn., February}{24, 1862}
{Special Field Orders}{4}

I. It is imperative that operations in this theater be brought to a close as
soon as possible. The Army of the Cumberland will be forced to fall back upon
their supply base unless the Army of the Kanawha is able to provide a decisive
operational success that allows for the capture of Fort Donelson.

II. 4th Infantry Division and 2d Cavalry Division, currently located in
Charleston, western Virginia, will immediately transfer to Army of the Kanawha
via rail and river transport. If they unable to do so they will disembark at
Fort Henry to defend its environs.

III. 1st Cavalry Division will take on an extra week's worth of rations, take
river transport or march behind friendly lines (whichever is faster) and cross
the Cumberland River to Line Port, where it will conduct a raid to destroy the
enemy rail line at Clarksville, Tenn., before returning to Line Port and
crossing back to friendly lines. If Clarksville is too heavily defended, or the
task otherwise proves impractical, their secondary target will be the rail
bridge over the Red River located several miles to the northeast of Clarksville.
1st Cavalry Division will avoid an engagement with enemy forces unless decisive
victory is immediately assured. Preservation of your force is the highest
priority, secondary priority is successful completion of your mission.

IV. 1st and 3d Infantry Divisions, with 2d Division in support, will take an
extra week of rations and conduct a turning movement of the enemy flank
to unhinge their defense of Fort Donelson and Dover. The objective is to avoid
direct combat with the enemy, if possible, and instead attempt to find a route
south and then east that will take this army past the enemy and to threaten
their rear, compelling them to retreat. The ultimate goal is to capture the
enemy supply base at Dover.

V. During this movement, 2d Division will detach a brigade that will march to
Stewart Crossing and destroy the enemy railroad bridge over the Tennessee River.
When this is accomplished, the brigade will rejoin 2d Division.

VI. Graham's Cavalry Division will protect the line of communication between
Army of the Kanawha and the Army of the Cumberland.

\gramClosing{Respectfully}
{Richard Steele}
{Maj. Gen., Commanding, Army of the Kanawha}
\reportdinkus

\gramHeader{Headquarters, Army of the Kanawha} % {{{4
{Fort Henry, Tenn., 9.30 am, February}{24, 1862}
\gramTo{Maj. Gen.}{J. W. Blake}
{Commanding, Army of the Cumberland}

\gramHi{General} Yours of 9~o'clock has been received. My cavalry has begun to
relocate in preparation of their mission. I will keep you apprised of any
developments.

\gramClosing{Respectfully}
{Richard Steele}
{Maj. Gen., Commanding}
\reportdinkus

\gramHeader{Headquarters, Army of the Kanawha} % {{{4
{Fort Henry, Tenn., 9.35 am, February}{24, 1862}
\gramTo{Maj. Gen.}{J. W. Blake}
{Commanding, Army of the Cumberland}

\gramHi{General} I have ordered the cavalry to proceed with their orders as
given. In the slight off chance that the enemy has chosen to abandon their
position and we need to call off the cavalry, you might be in a better position
to amend their orders more quickly. So if you feel the need to do so, they will
obey your orders.

\gramClosing{Respectfully}
{Richard Steele}
{Maj. Gen., Commanding}

\subsecdinkus

\subsection{Report of Cdre. Daniel Lewis, U.S.N., commanding Mississippi Squadron} % {{{3
\gramHeader{U.S. Naval Flotilla, Mississippi Squadron} % {{{4
    {Aboard Flagship Resolute, Off Fort Donelson, Tenn. January}{15th, 1862}
\gramTo{Major General}{Blake}
    {Commanding, Western Theater of Operations}

\gramHi{Sir} I have the honor to submit for your consideration the following
report respecting the present situation on the river and the operations
conducted under my direction:

Despite the enemy capture of Paducah, Ky., traffic along the Ohio River is only
receiving intermittent and mild interruptions from rebel guns. However, if the
enemy were to emplace larger guns, or bring in a more significant quantity of
artillery, it could indeed pose a serious risk to river traffic.

The Mississippi Squadron's first foray up the Tennessee and Cumberland rivers
with a single gunboat each resulted in the both gunboats being engaged by enemy
cannon at Forts Henry and Donelson, respectively, although with only minor
damage.  The batteries of both forts number about 10--12 guns
including some significant pieces that were able to cause damage to the
gunboats. The gunboats were unable to determine the size of each fort's
garrison.

\gramClosing{I remain, sir, Your obedient servant}
    {Daniel Davyson Lewis}
    {Commodore, Commanding Officer, Mississippi Squadron, United States Navy}
\secdinkus

\section[First Battle of New Madrid, Mo.]{January 24, 1862} %{{{2
    {Battle of New Madrid, Mo}

\subsection*{Reports of Maj. Gen. Thomas Smith, commanding Army of the Tennessee} % {{{3
\gramHeader{Army Of The Tennessee} % {{{4
    {City of Charleston, January}{26, 1862}
\gramTo{Maj. Gen.}{C. N. Van Royne}
    {Commander, US Army}

\gramHi{Sir} The rebels are defending New Madrid with more tenacity and troops
than was expected. Four regiments lost, enemies losses are hard to be estimated
but are assumed to be about equal. No great alarm but we may have to move more
cautiously in the future.

\gramClosing{I remain vy rspy yours}
    {Thomas Smith}
    {Commander, Army Of The Tennessee}
\secdinkus

\gramHeader{Army Of The Tennessee} % {{{4
    {City of Charleston, January}{26, 1862}
\gramTo{Maj. Gen's.}{C. N. Van Royne and Karl Meyer}
    {Commander, US Army and Commander Army of the Arkansas}

\gramHi{Sirs} I have ascertained the units I engaged were 2d Division, Army of
the Trans-Mississippi, and Cavalry Division, Army of the Trans-Mississippi.
Their divisions seem quite a bit larger than ours. I intend to return to Number
5 and begin construction post haste.

\gramClosing{I remain vy rspy yours}
    {Thomas Smith}
    {Commander, Army Of The Tennessee}

\section[Confederate Gunboat Raid on Cairo, Ill.]{January 27, 1862} % {{{2
    {Confederate gunboat raid on Cairo, Ill}
\subsection*{Report of Cdre. Daniel Lewis, U.S.N., commanding Mississippi Squadron} % {{{3
\gramHeader{U.S. Naval Flotilla, Mississippi Squadron} % {{{4
    {Aboard Flagship Resolute, Off Fort Donelson, Tenn. January}{27th, 1862}
\gramTo{Major General}{Blake}
    {Commanding, Western Theater of Operations}

\gramHi{Sir} I have the honor to submit for your consideration the following
report respecting the present situation on the river and the operations
conducted under my direction:

In compliance with our coordinated movement against the enemy position at Fort
Donelson, the squadron did proceed to engage the enemy batteries on the morning
of the 25th instant. The bombardment was opened at close range, and maintained
with vigor for several hours. While considerable damage has been inflicted upon
the rebel fortifications, I regret to report that the enemy's guns have not been
wholly silenced. Their fire, though sporadic and less disciplined than at the
outset, remains of sufficient weight and accuracy to preclude any premature
assumption of naval superiority in the immediate vicinity.

Our vessels sustained moderate damage during the action, chiefly to upper works,
though casualties have been comparatively light. Repairs are underway, and we
remain in position, ready to renew the assault at your signal or as
circumstances may dictate. I await with confidence the combined efforts of your
gallant troops to effect the reduction of the fort.

I must further report a recent incursion by Confederate naval forces, which
steamed northward from Columbus in a bold but ultimately fruitless attempt to
disrupt our operations along the upper Mississippi. A rebel gunboat detachment,
under unknown command, succeeded in ascending the river as far as Cairo. The
expedition created a measure of panic amongst the civilian population. Though a
few shells were lobbed at the works under construct at Cairo, it resulted in no
significant destruction to Federal property or shipping.

Our picket boats were momentarily outmaneuvered by the swiftness of the enemy's
movement and their unexpected appearance, but the alarm having been sounded,
appropriate defensive preparations were made. The Confederate force withdrew
before our heavier vessels could come to bear. Measures have since been taken to
forestall any repetition of such an enterprise.

\gramClosing{I remain, sir, Your obedient servant}
    {Daniel Davyson Lewis}
    {Commodore, Commanding Officer, Mississippi Squadron, United States Navy}
\secdinkus

\section[Battle of Sikeston, Mo.]{January 28, 1862} % {{{2
    {Battle of Sikeston, Mo}

\begin{toc}[ % Reports {{{3
    caption = {Reports, Etc.},
]{}
No. & 1.  & ---Organization of the Army of the Arkansas \\
No. & 2.  & ---Maj. Gen. Karl Meyer, commanding Army of the Arkansas \\
\end{toc}

\subsection{Organization of the Army of the Arkansas, Maj. Gen. Karl Meyer, % {{{3
U.S. Army, commanding, January 25, 1862--February, 25, 18662}
\footnotetext[1]{
    Arranged according to the numerical designation of the corps, divisions and
    brigades as prescribed in General Orders, No.~5, Headquarters, Army of the
    Arkansas, December~XX, 1861.
}

\begin{fulloob}
    \corps{XIII Corps}{Brig. Gen.}{Carl Schurz} % {{{4

    \division{First Division}{Brig. Gen.}{Nathaniel Lyon} % {{{5
    \begin{leftBde}
        \bde{First Brigade}{Col.}{George Lippitt Andrews}
        \rgt{1st}{United States}{}
        \rgt{2d}{United States}{}
        \rgt{4th}{United States}{}
        \rgt{11th}{United States}{}
    \end{leftBde}
    \begin{rightBde}
        \bde{Second Brigade}{Col.}{Frederick Steele}
        \rgt{12th}{Missouri}{}
        \rgt{15th}{Missouri}{}
        \rgt{29th}{Missouri}{}
        \rgt{30th}{Missouri}{}
    \end{rightBde}
    \begin{middleBde}
        \bde{Third Brigade}{Col.}{Alexander Asboth}
        \rgt{17th}{Missouri}{}
        \rgt{24th}{Missouri}{}
        \rgt{33d}{Missouri}{}
        \rgt{43d}{Missouri}{}
    \end{middleBde}

    \division{Second Division}{Brig. Gen.}{XXX} % {{{5
    \begin{leftBde}
        \bde{First Brigade}{Col.}{Otto Fischer}
        \rgt{3d}{Missouri}{}
        \rgt{11th}{Missouri}{}
        \rgt{13th}{Missouri}{}
        \rgt{26th}{Missouri}{}
    \end{leftBde}
    \begin{rightBde}
        \bde{Second Brigade}{Col.}{Charles Salomon}
        \rgt{1st}{Missouri}{}
        \rgt{2d}{Missouri}{}
        \rgt{5th}{Missouri}{}
        \rgt{47th}{Missouri}{}
    \end{rightBde}
    \begin{middleBde}
        \bde{Third Brigade}{Col.}{Henry Imhauser}
        \rgt{4th}{Missouri}{}
        \rgt{6th}{Missouri}{}
        \rgt{7th}{Missouri}{}
    \end{middleBde}

    \otherbde{Artillery} % {{{5
    \begin{leftBde}
        \otherbde{First Division}
        \rgt{2d}{United States, Battery C}{Capt. Samuel DeGolyer}
    \end{leftBde}
    \begin{rightBde}
        \otherbde{Second Division}
        \rgt{2d}{United States, Battery D}{Capt. James Totten}
    \end{rightBde}
    \begin{middleBde}
        \otherbde{Reserve Artillery}
        \rgt{2d}{United States, Battery A}{Capt. Frank Sands}
        \rgt{2d}{United States, Battery B}{Capt. Albert Powell}
    \end{middleBde}

    \corps{XVth Corps}{Brig. Gen.}{XXX} % {{{4

    \divisionCdrs{First Division}{ % {{{5
        \divOneCdr{Brig. Gen.}{Francis Hassendeubel\footnotemark[1]}
        \divOneCdr{Brig. Gen.}{Franz Backhoff}
    }
    \footnotetext[1]{Captured}
    \begin{leftBde}
        \bde{First Brigade}{Col.}{John Scott}
        \rgt{1st}{Iowa}{}
        \rgt{14th}{Iowa}{}
    \end{leftBde}
    \begin{rightBde}
        \bde{Second Brigade}{Col.}{William Vandever}
        \rgt{8th}{Indiana}{}
        \rgt{18th}{Indiana}{}
        \rgt{22d}{Indiana}{}
    \end{rightBde}
    \begin{middleBde}
        \bde{Third Brigade}{Col.}{James Lane}
        \rgt{1st}{Kansas}{}
        \rgt{2d}{Kansas}{}
        \rgt{10th}{Kansas}{}
        \rgt{13th}{Kansas}{}
    \end{middleBde}

    \division{Second Division}{Brig. Gen.}{Eugene Carr} % {{{5
    \begin{leftBde}
        \bde{First Brigade}{Col.}{John Groesbeck}
        \rgt{5th}{Minnesota}{}
        \rgt{36th}{Illinois}{}
        \rgt{39th}{Ohio}{}
        \rgt{47th}{Illinois}{}
    \end{leftBde}
    \begin{rightBde}
        \bde{Second Brigade}{Col.}{Lucius Hubbard}
        \rgt{9th}{Wisconsin}{}
        \rgt{7th}{Minnesota}{}
        \rgt{10th}{Minnesota}{}
        \rgt{23d}{Wisconsin}{}
    \end{rightBde}
    \begin{middleBde}
        \bde{Third Brigade}{Col.}{Edward Wolfe}
        \rgt{4th}{Iowa}{}
        \rgt{9th}{Iowa}{}
        \rgt{12th}{Iowa}{}
        \rgt{35th}{Iowa}{}
    \end{middleBde}

    \otherbde{Artillery} % {{{5
    \begin{leftBde}
        \otherbde{First Division}
        \rgt{2d}{Illinois Battery}{Capt. Earl St. Jude}
    \end{leftBde}
    \begin{rightBde}
        \otherbde{Second Division}
        \rgt{---}{Missouri Battery}{Capt. Clemens Landgraeber}
    \end{rightBde}
    \begin{middleBde}
        \otherbde{Reserve Artillery}
        \rgt{---}{Missouri Battery}{Capt. Fredrick Schaefer}
        \rgt{---}{Missouri Battery}{Capt. John Du Bois}
    \end{middleBde}

    \corps{Division of Observation}{Brig. Gen.}{Julius White} % {{{4
    \begin{leftBde}
        \bde{First Brigade}{Col.}{James Mulligan}
        \rgt{13th}{Illinois}{}
        \rgt{16th}{Illinois}{}
        \rgt{17th}{Illinois}{}
    \end{leftBde}
    \begin{rightBde}
        \bde{Second Brigade}{Col.}{Grenville Dodge}
        \rgt{37th}{Illinois}{}
        \rgt{44th}{Illinois}{}
        \rgt{59th}{Illinois}{}
    \end{rightBde}
    \begin{middleBde}
        \otherbde{Artillery}
        \rgt{1st}{Illinois Battery}{Capt. Peter Davidson}
    \end{middleBde}

    \corps{Cavalry Division}{Brig. Gen.}{David Stanley} % {{{4
    \begin{leftBde}
        \bde{First Brigade}{Col.}{Charles Farrand}
        \rgt{1st}{United States}{}
        \rgt{2d}{United States Dragoons}{}
        \rgt{5th}{United States}{}
        \rgt{5th}{Ohio}{}
    \end{leftBde}
    \begin{rightBde}
        \bde{Second Brigade}{Col.}{Thomas Marshall}
        \rgt{1st}{Illinois}{}
        \rgt{3d}{Illinois}{}
        \rgt{1st}{Indiana}{}
        \rgt{1st}{Iowa}{}
    \end{rightBde}
    \begin{leftBde}
        \bde{Third Brigade}{Col.}{Benjamin Grover}
        \rgt{2d}{Missouri}{}
        \rgt{3d}{Missouri}{}
        \rgt{7th}{Missouri}{}
        \rgt{27th}{Missouri Mounted Infantry}{}
    \end{leftBde}
    \begin{rightBde}
        \bde{Fourth Brigade}{Col.}{Friedrich von Kleist}
        \rgt{1st}{Missouri}{}
        \rgt{5th}{Missouri}{}
        \rgt{6th}{Missouri}{}
    \end{rightBde}
    \begin{middleBde}
        \otherbde{Artillery}
        \rgt{1st}{Missouri Flying Artillery}{Capt. Gustavus Elbert}
        \rgt{1st}{Missouri Flying Artillery}{Capt. Martin Klauss}
        \rgt{1st}{Missouri Flying Artillery}{Capt. Georg Koch}
        \rgt{1st}{Missouri Flying Artillery}{Capt. Josef Müller}
    \end{middleBde}

    \corps{Artillery Reserve}{}{} % {{{4
    \begin{middleBde}
        \rgt{1st}{Missouri, Battery A}{Capt. Henry Dillon}
        \rgt{1st}{Missouri, Battery B}{Capt. Richard Griffith}
        \rgt{1st}{Missouri, Battery C}{Capt. A. W. Dees}
    \end{middleBde}
\end{fulloob}
% TODO: Remove requirement for this blank line

\subsecdinkus

\subsection*{Reports of Maj. Gen. Karl Meyer, commanding Army of the Arkansas} % {{{3
\gramHeader{Headquarters, Army of the Arkansas} % {{{4
    {Field Headquarters, Charleston, Mo., January}{30th, 1862}
\gramTo{Major Generals}{Cornelius Van Royne and Thomas Smith}
    {Commanding the United States Army and the Army Of The Tennessee, respectively}

\gramHi{Generals} In the recent battle at Sikeston our forces suffered a major
blow at the hands of a large Confederate army bearing down on Ohio City and
Cairo. Our forces were beset by an attack consisting of 10 brigades of infantry,
7 regiments of cavalry, and 11 batteries. While we managed to inflict an
  estimated 2,000 casualties on the enemy, overrunning 3 batteries and putting a
  handful of regiments to flight, fully half of the XIII Corps' 2d Division was
  isolated and surrounded. The XV Corp is still intact, however it too suffered
  losses amounting to 3 infantry regiments and had to retreat to St. Luke

We believe the rebel force currently in Sikeston to represent upwards of 60% of
the rebel Army of the Trans Mississippi under Thomson. There may have also been
elements of Clarke's Army of Mississippi.

We are already redeploying garrisons from the rear to reinforce our positions,
however in the immediate term if Ohio City and Cairo are to be defended we
require the collaboration of the Army of the Tennessee in said defense. XIII
Corps is currently in Charleston separated from XV in St. Luke. The plan at the
current moment is for XIII to attempt delaying actions at Matthew's Prairie to
allow for XV to march north to Cape Girardeau and be transported from the city
to Ohio City in order to reinforce the XIII Corps, alongside anything that
arrives from our garrisons or the Army of the Tennessee.

Additionally, we urgently request that the return of the forces our army loaned
to the Army of the Tennessee and the garrison in place at St. Louis and Cairo.

\gramClosing{I am always, General, and shall ever remain, Your most humble and obedient of servants}
    {Karl Meyer}
    {Maj. Gen. cmdg, Army of the Arkansas}
\reportdinkus

\gramHeader{Headquarters, Army of the Arkansas} % {{{4
    {Charleston, Mo., Afternoon, January}{28th, 1862}
\gramTo{Maj. Gen.}{Cornelius Van Royne}
    {Cmdg Gen. of the United States Army}

\gramHi{General} I will make no excuses for my loss nor will I place blame on
others. I understand the gravity of the current situation. I serve at your
pleasure and that of the President of the United States and offer my resignation
if you deem it necessary.

Please understand that the men of this Army fought valiantly and bravely. All
honor is to be commend upon them and upon Brigadier General Francis Hassendeubel
who fought and stayed on the field of battle with the division until he was
captured, and Colonel Charles E. Salomon and his men who performed beyond the
call of duty.

\gramClosing{I am always, General, and shall ever remain, Your most humble and obedient of servants}
    {Karl Meyer}
    {Maj. Gen. cmdg, Army of the Arkansas}
\subsecdinkus

\section[Siege of Fort Columbus, Ky.]{February 10, 1862--March 17, 1862} % {{{2
    {Siege of Fort Columbus, Ky}

\begin{toc}[
    caption = {Reports, etc.},
]{}
No. & 1. & ---Brig. Gen. Thomas Caldwell, commanding Army of the Tennessee \\
No. & 2. & ---Maj. Gen. Gray, commanding XVIIth Corps \\
\end{toc}

\subsection{Reports of Brig. Gen. Thomas Caldwell, Army of the Tennessee} % {{{3

\gramHeader{} % {{{4
    {(Undelivered) 8.00 am, February}{10, 1862}
\gramTo{Maj. Gen.}{Gray}
    {Commanding, XIIth Corps}

Withdraw from the attack to siege lines. We have made a good probing attack but
we will avoid further losses.

I am making one last assault on the small enemy fortification on my axis and
then will either hold or fall back. 

\gramClosing{}
    {Thomas Caldwell}
    {Brig. Gen.}
\reportdinkus

\gramHeader{} % {{{4
    {(Undelivered) 8.30 am, February}{10, 1862}
\gramTo{Maj. Gen.}{Gray}
    {Commanding, XVIth Corps}

XVIth Corps seems to have a foothold and we could possibly take the outer enemy
fortifications.

If they can make this limited gain we shall hold here unless the enemy fully
breaks.

\gramClosing{}
    {Thomas Caldwell}
    {Brig. Gen.}
\reportdinkus

\gramHeader{Headquarters, Army of the Tennessee} % {{{4
{Field Headquarters, Kentucky City, Ky., February}{24, 1862}
\gramTo{Maj. Gen.}{C. N. Van Royne}
{Commanding General, United States Army}

\gramHi{General} The siege of Fort Columbus has continued uninterrupted. My
forces have managed to eliminate another battery of artillery, leaving two
facing landward and one on the river, at the cost of a regiment and one of my
own batteries. No other rebel forces have been seen or reported in my area.

\gramClosing{Yr. Obdt. Srvt.}
{Thomas Caldwell}
{Brigadier General, Commanding Army of the Tennessee}
\subsecdinkus

\subsection{Report of Maj. Gen. Gray, XVIIth Corps} % {{{3

\gramHeader{} % {{{4
    {7.30 am, February}{10, 1862}
\gramTo{Brig. Gen.}{Thomas Caldwell}
    {Commanding, Army of the Tennessee}

Enemy fire overwhelming. Currently attempting to seize fire superiority with
very limited success. Unlikely to be able to force fort unless you can support.

\gramClosing{}
    {Gray}
    {Maj. Gen.}
\subsecdinkus

\section[Battle of Black Bayou, Mo.]{February 12, 1862} % {{{2
    {Battle of Black Bayou, Mo}

\begin{toc}[
    caption = {Reports, etc.},
]{}
No. & 1. & ---Maj. Gen. Karl Meyer, commanding Army of the Arkansas \\
No. & 2. & ---Brig. Gen. D. S. Stanley, commanding Cavalry Division \\
\end{toc}

\subsection{Reports of Maj. Gen. Karl Meyer, commanding Army of the Arkansas} % {{{3

\gramHeader{Headquarters, Army of the Arkansas} % {{{4
    {11.15 am, February}{12, 1862}
\gramTo{Brig. Gen.}{D. S. Stanley}
    {Commanding, Cavalry Division}

Status Report?

Bucket wiped a regiment. Rebs timid. Reserves are free if you need them.

\gramClosing{}
    {Karl Meyer}
    {Maj. Gen., commanding Army of the Arkansas}
\reportdinkus

\gramHeader{Headquarters, Army of the Arkansas} % {{{4
    {Field Headquarters, Black Bayou, Mo., Feb.}{17th, 1862}
\gramTo{Maj. Gen.}{Cornelius Van Royne}
    {Commanding General of the United States Army}

\gramHi{General} It is my honor to submit the following report of the engagement
fought near Black Bayou on February 12th, resulting in a decisive victory for
the Army of the Arkansas.

After the enemy fell back from East Prairie, the Army of the Arkansas continued
to maintain contact with them. This eventually resulted in Maj. Gen. Thomson and
the Army of the Trans-Mississippi halting their withdrawal and launching attack
against our forces. 

The initial action of the battle was an artillery duel between the batteries of
the 1st Missouri Artillery and rebel cannons directly in the center of the
army's line, on the road to New Madrid. This clash resulted in our only major
casualties in the battle, the loss of a battery and a few scores of injured men. 

To the west, enemy cavalry demonstrated before falling back once the batteries
of the 1st Missouri Flying Artillery came into effect and they discovered our
flank was held. 

Brigadier Brig. Gen. David S. Stanley personally scouted the enemy lines to
discover assault columns being formed. Fifteen regiments of cavalry in tight
formation three ranks wide and five ranks deep. And to their right flank was a
division of infantry two regiments deep. A total of 4,500 cavalry and 4,800
infantry launched their attack, only to run into enfilading fire from the 1st
and 2d brigades of the Cavalry Division. 

Two of the rebel cavalry brigades shattered before the remaining three fell
back, while the 3d brigade of the Cavalry Division shattered the lead
formations of the massed Rebel attack, until the full division fled, guarded by
the remaining Rebel cavalry.

Brig. Gen. Stanley also wished to mention Colonel Grover and 3d Brigade's heroic
stand against an entire Rebel Division.

Field reports and prisoner accounts indicate enemy infantry losses between
1,500--1,800 men killed, wounded, or captured. Cavalry casualties are estimated
at 400-–600. While the extent of enemy artillery losses cannot yet be confirmed,
it is probable that their prolonged exposure to counter-battery fire inflicted
serious damage upon their guns.

Combined with the 2,000 or so losses taken at Sikeston, it is likely that the
Army of the Trans-Mississippi is down to 30,600--31,100 of their initially
estimated force of 35,000 men.

Unfortunately, while our army held the field, it was unable to pursue the enemy
towards New Madrid, due to the city falling four miles outside of our supply lines.
We do not have the resources necessary to create a wagon depot for the entire
army and thus are forced to fall back to East Prairie, though the Cavalry
Division has continued to maintain contact with the Army of the
Trans-Mississippi.

Their reports show that Maj. Gen. Thomson has taken the reprieve to begin
constructing local fortifications around the town. Compounding the bad news is
the fact that increased river traffic along the Mississippi was detected,
raising my suspicions that another Rebel force has come to bolster their
comrades.

Nevertheless, the fact remains that the enemy has been driven back from the
gates of Ohio City, and are now in fact digging in at their local depot of New
Madrid. While the lack of supplies means that any movements on New Madrid must
wait till the spring campaign season or additional reinforcements, the Army of
the Trans-Mississippi remains pinned and rebels forces are being drawn away from
the Western Theater.

I give the honors of victory to Brig. Gen. Norman Ambrose who planed the defense of
the Black Bayou with me and commanded the XVth Corps, Brig. Gen. Stanley, whose
Cavalry Division inflicted the majority of the casualties, and of course the
hard fighting men of the Army of the Arkansas, without whom this victory would
not be possible.

\gramClosing{I am always, General, and shall ever remain, Your most humble and obedient of servants}
    {Karl Meyer}
    {Maj. Gen. cmdg Army of the Arkansas}
\subsecdinkus

\subsection{Reports of Brig. Gen. D. S. Stanley, commanding Cavalry Division} % {{{3

\gramHeader{Headquarters, Cavalry Division} % {{{4
    {11.15 am, February}{12, 1862}
\gramTo{Maj. Gen.}{Karl Meyer}
    {Commanding, Army of the Arkansas}

Enemy formed up three cavalry regiments wide and five deep in assault column,
sabers drawn.

Enemy formed up three infantry brigades in assault column, making ready to attack.

We are ready, canister, Division of Observation and cavalry ready to receive
them.

If the enemy has masked their own guns with the wreckage of their infantry, you
may wish to reoccupy your initial lines.

\gramClosing{I have the honour to remain, Yr obt servt}
    {D. S. Stanley}
    {Brig. Gen., commanding Cavalry Division}
\reportdinkus

\gramHeader{Cavalry Division, Army of the Arkansas} % {{{4
{Two miles southwest of Charleston, 1.00 pm, Feb.}{12th, 1862}
\gramTo{Maj. Gen.}{Karl Meyer}
    {Commanding, Army of the Arkansas}

\gramHi{General} I have the honor to confirm a considerable and resounding
victory of arms by the Army this day. While Brig. Gen. Ambrose conducted a
doughty defense against the Rebel infantry and massed artillery in the center of
the Army's position, my division dismounted behind the creek on the Army's
right; Marshall's and Grover's brigades lined up behind the creek while the 1st
(Regular) Brigade screened our right. The Division of Observation being also
seconded to my command, it was deployed concealed in a second line through the
dense woods behind my division to coax the enemy to conduct a heavy attack upon
our line and expose themselves to a vigorous counter.

The Army's plan proved entirely accurate in the expectation of the enemy's
movement, if anything understating their aggression.

The enemy first demonstrated opposite my position in a broad line at 10.30 am,
taking long range fire from my artillery before fading back west as our flank
proved to be held. Scouting out their line personally at 11 o'clock we observed
the enemy forming up their cavalry in assault column of brigades; three
regiments wide and five deep, a solid mass of horseflesh. To their south a Rebel
infantry division formed up en masse to assault with two regiments deep. The
enemy committed their attack at 11.15 am and took our concentrated fire for the
next hour, artillery and cavalry alike. Two enemy cavalry brigades were
shattered and routed to the rear by 2 (Illinois) Brigade to their fore and 1st
(Regular) Brigade firing into their flank, before the other three enemy brigades
fled. To the south 3d (Missouri) Brigade backed by horse artillery broke and
routed the lead formations of the massed Rebel attack, until the full division
fled, guarded by the remaining Rebel cavalry.

I wish to commend Colonel Grover for his heroic stand; while the enemy failed to
fight their way into our trap, the Colonel showed what brave Missouri patriots
can do when their blood is up!

In my sector the Rebel cavalry and infantry took some 3,000 casualties, while my
division received none and the Division of Observation was wholly uncommitted
beyond the sterling conduct of two batteries of field artillery drawn from the
Division of Observation.

With superiority in cavalry, artillery and infantry, the Army of Arkansas stands
ready to advance.

\gramClosing{Respectfully, your obedient servant}
    {David. S. Stanley}
    {Brigadier-General}
\subsecdinkus

\section[Engagements around Munfordville, Ky.]{February 25, 1862--March 3, 1862} % {{{2
    {Engagements around Munfordville, Ky.}

\subsection*{Report of Maj. Gen. Christopher Stoeffler, commanding Army of the Kentucky} % {{{3

\gramHeader{Headquarters, Army of Kentucky} % {{{4
{Munfordville, Tenn., March}{3, 1862}

\gramTo{Maj. Gen.}{Cornelius Van Royne}
{General In Chief, United States Army}

\gramHi{Sir} General Stoeffel will provide a personal report in due time, but
recent battle exertions and his recovering illness may delay his correspondence.

In the interim, it has been a week of in effect a near bloodless minor victory,
albeit with no advances due to river conditions. We believe in total the enemy
lost about 1,800 infantry, 3 batteries of guns, and 300--600 cavalry. Our own
forces lost 300 cavalry troopers.

The Green River, swollen by rain, was our implacable foe.

Our crossing in the west was so slowed by the  river conditions that we could
scarce get two regiments across before a division of the Army of East Tennessee
arrived with cavalry and infantry, and then debris coming down river fast
destroyed one of our two bridges. We were compelled to withdraw back across the
river under the massed fire of our cannons. While we did not cross, we believe
the enemy lost hundreds of men while trying to pressure our forward force under
the sweep of our guns.

In the east, the enemy tried again to raid our lines of communication with
roughly 1,200 cavalry, but were repulsed by the garrison at Nolin with
``hundreds'' lost. Our own cavalry pursued and inflicted hundreds more before
running into a second enemy force, for a combined estimate of 2,100 troopers. We
lost 300 men, but believe we inflicted at least that, and possibly as many as
600.

In front of Munfordville the Green River prohibited attempts to cross anywhere
but directly in front of the enemy works. This did not seem practicable; instead
we chose extended bombardment, silencing the enemy guns and then beginning the
deliberate punishment of the enemy who chose to remain forward.

Our own force is reporting a sharp uptick in sick rolls, and will need to rest
slightly. Fortunately, river conditions should not settle by the 5th or 6th,
allowing time to recover before a new lunge.

\gramClosingBehalf{Maj. Gen. Christopher Stoeffel}
{James Howard}
{Brig. Gen.}

Force Roster Appendix Attached: \\
Field Force: 36,000 infantry, 3,300 cavalry, 228 guns \\
Garrison Force: 7,200 infantry

\subsecdinkus

\section[Battle of Munfordville, Ky.]{March 8, 1862} %{{{2
    {Battle of Munfordville, Ky.}

\subsection*{Reports of Brig. Gen. James Howard, Army of the Kentucky} % {{{3

\gramHeader{Headquarters, XIth Corps} % {{{4
{Munfordville, Ky., Evening, March}{10, 1862}
\gramTo{Maj. Gen.}{Cornelius Van Royne}
{General in Chief, United States Army}

\gramHi{Sir} I fear the matter of the Green River is not entirely as reported in
the papers. The situation is still in development, and operations over the next
few days will likely determine them.  Maj. Gen. Stoeffel is currently out of
telegram contact and is a hard and distant ride from here, but I shall endeavor
to forward his information to you as I receive it; until then I will seek to
report the status of the army. I anticipate a major clash in the next day or two
that may well seal the fate of the winter campaign.

On the 8th of March,  we conducted a mutual crossing at Mammoth Cave and
Munfordville. Maj. Gen. Stoeffel's force, VIIth Corps, a division of IXth Corps,
and our main battle cavalry was sent against the enemy at Mammoth Cave in order
to turn the enemy position at Munfordville and cut his supply lines. XI Corps,
under myself, was to conduct a fixing action action at Munfordville while our
light cavalry screened the flank.

Maj. Gen. Stoeffel crossed successfully, and as you saw in the papers, routed a
division of the Army of East Tennessee. The speed and totality of his success
prevented a full counting of the enemy beyond ``thousands'', but he captured two
batteries of guns intact and has added them to his train. He followed up this
success on the 9th by driving the enemy division once more with , though with
unclear outcomes other than causing them to reel once more. As of writing, his
VIIth Corps is at Prewitt's knob, astride the enemy line of supply.

The error in over extending the position is entirely mine, having sought to
occupy a wide bridgehead in the face of what became clearly a much superior
enemy, and is no fault of the men's despite any rumors that may arise of the
initial flight of the left. I must also pass the presumed posthumous nomination
for the medal of honor for Gen'l Schoepf, formerly of 2d Division, XIth Corps,
who with the help of his staff dragged me from the front and assumed my
position shortly before it was overwhelmed.

As of today, the enemy abandoned his position to march against Stoeffel. I had
but six regiments fit for duty with most of XIth Corps still being assembled
into good order. I ordered a transfer of the IXth Corps garrison forces under
me, to be replaced by the worst mauled elements of XIth Corps, but they did not
arrive in time to effect a crossing today.

I must also report a third of my force is not fit for battle and is on sick
rolls. Disease, exhaustion, desertion \&c have now lost another regiment's worth
of men permanently. I can only hope that the enemy, having committed to major
assault and then marching towards Stoeffel, is suffering similarly.

Storms are brewing here, and I intend to cross tomorrow to secure and repair the
bridges we have so ardently fought for; I await word from Maj. Gen. Stoeffel
whether a forced march of my one fresh division into the enemy rear may be
profitable or foolish based on his current conditions.  I will keep you informed
as information becomes available. I am also available to send on in ink
dispatches to Maj. Gen. Stoeffel given his lack of access to the telegram.

\gramClosing{Yr Obt. Svt}
{James Howard}
{Brig. Gen.}
\reportdinkus

\gramHeader{Headquarters, XIth Corps} % {{{4
{Munfordville, Ky., Evening, March}{10, 1862}
\gramTo{Maj. Gen.}{Cornelius Van Royne}
{General in Chief, United States Army}

\gramHi{Sir} On the morning of the 8th of March, 1862, XIth Corps of the Army of
Kentucky advanced across the Green River with 20 regiments of men, supported by
nearly 100 guns on the bluffs. This followed a week of bombardment on the enemy
works prior, which had felled many trees and left the Confederate works badly
damaged.

The enemy vanguard, a brigade, collapsed under fire and initially the crossings
were only contested by sporadic artillery fire. Union guns on the bluffs
silenced two enemy batteries, but others continued to shell our forward forces.
The whole of the 20 planned regiments now across with a single brigade left at
Munfordville as a reserve, the corps occupied the former confederate positions
and began bringing up artillery to silence the enemy. In occupying the old
confederate positions, the Union soon gained view of a second line of the enemy
which appeared to be nearly the same strength as the crossing force---perhaps
12,000 men---and well entrenched.

Eschewing an assault on breastworks at even strength was not to the taste of
XIth Corps, which brought up a total of 9 batteries to silence 6 rebel batteries
distant.

By 12.30~am a furious artillery duel had developed, with the Union artillery
gaining the upper hand and silencing several enemy batteries, but not without
losing some pieces. The broken nature of the hills impeded the flow of
ammunition and guns forward, so Brig. Gen. Howard ordered the worst damaged artillery
limbered and withdrawn to allow fresh batteries to operate at the best condition.
About this time enemy cavalry began probing the left flank of the corps. A
couple additional regiments were dispatched in that direction to guard against
catastrophe.

The rebel cavalry had been badly hurt by 1.30~pm and conditions seemed
favorable, but this was not to last.

At 2~o'clock the enemy committed a full corps against the union left, with much
of their original line held in threat against the union right.  The left was in
reverse slope positions, and for reasons not entirely clear, every single
skirmisher and scout ordered to the forward slope hours ago either did not obey
or in some manner missed the massed force of an entire rebel infantry corps as
it made its way forward, resulting in being alerted with them merely a few
hundred yards away. A mystery of friction as Clausewitz might ascribe it.  A
rapid effort to compact the line failed, the enemy carrying forward heedlessly
into volleys delivered from union troops manning the old breastworks. The left
broke.

At this point the XIth Corps immediately committed its reserve brigade and began
drawing back, moving to evacuate the now highly perilous position before it
could be overrun and forced away from the bridges. All tact now gone from the
fight, the Union troops fought on an advantageous line of hills beneath the cover
of guns from their own side of the river, their perimeter shrinking as more and
more troops escaped. The Confederates, not lacking in bravery, pressed into
murderous fire time and again.

Finally, Gen'l Schoepf of the XIth Corps, 2d division forcibly removed Brig.
Gen. Howard from the fighting line and commanded the last desperate defense of the
bridgehead. The final four regiments and three batteries collapsed.  The
confederate force, through some combination of fatigue, casualties, and
understanding the consequence of trying to pursue across the river under a cross
fire of guns from heights while the last reserves of XI corps manned the north
bank to discourage pursuit, withdrew back to their old lines.

The crossing effort had failed. In post mortem, it was assessed the CSA had been
present with 30 or more regiments of infantry---a low estimate, but full
confirmation of the uncommitted enemy center and right urge caution lest numbers
be inflated---4 regiments of cavalry, and 9 batteries arrayed against the 20
regiments and 9 batteries of Union troops that had crossed. The Confederates
lost 2,000--3,000 men, including likely 300 cavalry, and 3 batteries destroyed.
The Union lost a total of 4,200 infantry and 4 batteries, 1,800 known killed or
wounded and six guns dismounted, and 2,400 men missing or captured as well as
eighteen guns.

\gramClosing{Yr Obt. Svt}
{James Howard}
{Brig. Gen.}

\secdinkus

\section[Battle of Prewitt's Knob, Ky.]{March 10, 1862} % {{{2
    {Battle of Prewitt's Knob, Ky.}

\subsection*{Report of Brig. Gen. James Howard, Army of the Kentucky} % {{{3

\gramHeader{Headquarters, XIth Corps, Army of the Kentucky} % {{{4
{Munfordville, Tenn., Morning, March}{11, 1862}
\gramTo{Maj. Gen.}{Cornelius Van Royne}
{General in Chief, U.S. Army}

\gramHi{Sir} I must write with grave news from Maj. Gen. Stoeffel, dated the
10th of March.

His force was concentrated at Prewitt's Knob and Bells, but the enemy made a
feint towards his lines of communication at Mammoth Cave and he withdrew all but
a division in that direction. The division remained at the Knob.

The combined forces of the Army of Tennessee and East Tennessee fell on that
division, still mustering a corps and a half of men for the attack. Three
quarters of the division surrendered or fell. He does not intend an attempt to
relieve the remaining men.

Maj. Gen. Stoeffel is now leading his remaining corps to the town of Bells to
attempt to block the enemy once more. He has sent me orders to advance my fresh
division along the rails to reach him.

Currently there is a thunderstorm which is, quite literally, flattening my tents
and camps. Once it clears I will march west with a fresh division as ordered,
leaving XIth corps to hold Munfordville.

\gramClosing{Yr Obt Svt}
{James Howard}
{Brig. Gen.}

Send copies to Maj. Gen'ls Blake \& Steele.

\subsecdinkus

\section[The Nashville, Tenn. Campaign]{March 18, 1862--May 16, 1862} % {{{2
    {The Nashville, Tenn. Campaign}

\begin{toc}[ % Summary of Events {{{3
    caption = {Summary of the Principal Events.},
]{}
    Mar. & 18, 1862. & --Apr. 11, 1862.---Siege of Fort Defiance, Ky. \\
         & DD, 1862. & ---Capture of Clarksville, Tenn. by Army of the Cumberland \\
    Apr. & 9, 1862.  & ---Cavalry skirmishes at Cloverdale \& Fredonia, Tenn. \\
         & 11, 1862. & ---Battle of Barton's Creek, Tenn., \\
         & 12, 1862. & --- Engagement at Palmyra, Tenn. and the withdrawal to Dover, Tenn. \\
         & 13, 1862. & ---Cavalry skirmish at Fredonia, Tenn. \\
    May  & 2, 1862.  & ---Battle of Cloverdale, Tenn. \\
         & 9, 1862.  & ---Fall of Fort Defiance \\
         & 11, 1862. & ---Capture of Clarksville, Tenn. by 1st Cavalry Division, Army of the Kentucky \\
         & 16, 1862. & ---Fall of Nashville, Tenn. \\
\end{toc}

\begin{toc}[ % Reports {{{3
    caption = {Reports, Etc.},
]{}
No. & 1. & ---Organization of the Department of the Cumberland \\
No. & 2. & ---Organization of the Army of the Cumberland \\
No. & 3. & ---Organization of the Army of the Kanawha \\
No. & 4. & ---Organization of the Army of the Kentucky \\
No. & 5.  & ---Return of casualties in the Union forces after the Battles of Barton's Creek \& Palmyra, April 11---12, 1862 \\
No. & 6.  & ---Return of casualties in the Union forces after the Battle of Cloverdale, May 2, 1862 \\
No. & 7.  & ---Return of casualties in the Union forces after the Battle of Fort Defiance, May 9, 1862 \\
No. & 8. & ---Maj. Gen. James Blake, U.S. Army, commanding Department of the Cumberland \\
No. & 9. & ---Maj. Gen. Harold Fawcett, III, commanding Army of the Cumberland \\
No. & 10. & ---Brig. Gen. Lawrence Graham, commanding Cavalry Corps \\
No. & 11. & ---Brig. Gen. Zebulon Benton, commanding XVIth Corps \\
No. & 12. & ---Maj. Gen. Richard Steele, commanding Army of the Kanawha \\
No. & 13. & ---Brig. Gen. Jacob Ryan, commanding XIth Corps \\
No. & 14. & ---Brig. Gen. Rutherford Hayes, commanding Third Division, XIth Corps \\
No. & 16. & ---Col. J. T. Wilder, commanding First Brigade, First Cavalry Division \\
No. & 17. & ---Maj. Gen. James Howard, commanding Army of the Kentucky \\
\end{toc}

\subsection{Organization of the Department of the Cumberland, Maj. Gen. James W.  Blake, % {{{3
U.S. Army, commanding, April 8, 1862--MM, DD, 186Y*}
\footnotetext[1]{
    Arranged according to the numerical designation of the corps, divisions and
    brigades as prescribed in General Orders, No.~X, Headquarters, Department of
    the Cumberland, MM~DD, 1862.
}

\begin{fulloob}
    \staff{Adjutant General}{Col.}{Walter Chekov}

    \corps{Division of Observation}{Brig. Gen.}{Curran Pope} % {{{4

    \begin{leftBde}
        \bde{First Brigade}{Col.}{John Hardin McHenry, Jr.}
        \rgt{41st}{Ohio}{Lieut. Col. George Mygatt}
        \rgt{49th}{Ohio}{Lieut. Col. Albert Blackman}
        \rgt{40th}{Indiana}{Col. William Wilson}
        \rgt{58th}{Indiana}{Col. George Buell}
        \rgt{17th}{Kentucky}{Col. Alexander Stout}
        \rgt{1st}{Michigan, Battery E}{Capt. John Dennis}
    \end{leftBde}
    \begin{rightBde}
        \bde{Second Brigade}{Col.}{Nathan Kimball}
        \rgt{2d}{Ohio}{Col. William Pittenger}
        \rgt{25th}{Ohio}{Col. James A. Jones}
        \rgt{30th}{Ohio}{Col. Hugh Ewing}
        \rgt{14th}{Indiana}{Col. William Harrow}
        \rgt{11th}{Kentucky}{Col. Pierce Hawkins}
        \rgt{2d}{Illinois, Battery F}{Capt. Charles Keith}
    \end{rightBde}
    \begin{middleBde}
        \bde{Third Brigade}{Col.}{James McMillan}
        \rgt{9th}{Kentucky}{Col. Benjamin Grider}
        \rgt{15th}{Kentucky}{Col. James Forman}
        \rgt{10th}{Michigan}{Col. Charles Lum}
        \rgt{13th}{Indiana}{Col. Jeremiah Sullivan}
        \rgt{21st}{Indiana}{Col. John Keith}
        \rgt{1st}{Ohio, Battery G}{Capt. Joseph Bartlett}
    \end{middleBde}

    \corps{Wool's Division}{Brig. Gen.}{Thomas Wood} % {{{4

    \begin{leftBde}
        \bde{First Brigade}{Col.}{James Shelley}
        \rgt{Middle (5th)}{Tennessee}{Col. Nathaniel Witt}
        \rgt{23d}{Ohio}{Col. Eliakim Scammon}
        \rgt{32d}{Ohio}{Col. Thomas Ford}
        \rgt{7th}{Indiana}{Col. Ira Grover}
    \end{leftBde}
    \begin{rightBde}
        \bde{Second Brigade}{Col.}{James Shackelford}
        \rgt{28th}{Ohio}{Col. August Moor}
        \rgt{64th}{Ohio}{Col. John Ferguson}
        \rgt{21st}{Kentucky}{Col. Ethelbert Dudley}
        \rgt{25th}{Kentucky}{Col. Benjamin Bristow}
    \end{rightBde}
    \begin{middleBde}
        \bde{Third Brigade}{Col.}{Robert Scott}
        \rgt{6th}{Kentucky}{Col. Walter Whitaker}
        \rgt{10th}{Kentucky}{Col. John Harlan}
        \rgt{27th}{Ohio}{Col. John Fuller}
        \rgt{68th}{Ohio}{Col. John Snook}
    \end{middleBde}

    \begin{middleBde}
        \otherbde{Artillery}
        \rgt{8th}{Ohio Battery}{Capt. Louis Markgraf}
        \rgt{2d}{Minnesota Battery}{Capt. William Hotchkiss}
        \rgt{7th}{Indiana Battery}{Capt. George Swallow}
    \end{middleBde}

    \corps{Cavalry Division}{Brig. Gen.}{William Kellogg} % {{{4

    \begin{leftBde}
        \bde{First Brigade}{Col.}{John Farnsworth}
        \rgt{8th}{Illinois}{Col. William Gamble}
        \rgt{12th}{Illinois}{Col. Arno Voss}
        \rgt{15th}{Pennsylvania}{Col. William Palmer}
        \rgt{2d}{Kentucky}{Col. Buckner Board}
    \end{leftBde}
    \begin{rightBde}
        \bde{Second Brigade}{Col.}{Charles Jennison}
        \rgt{6th}{Ohio}{Col. William Lloyd}
        \rgt{3d}{Illinois}{Col. Lafayette McCrillis}
        \rgt{9th}{Pennsylvania}{Col. Edward Williams}
        \rgt{7th}{Kansas}{Col. Daniel Anthony}
    \end{rightBde}
    \begin{middleBde}
        \bde{Third Brigade}{Col.}{Alfred Brackett}
        \rgt{7th}{Illinois}{Col. Edward Prince}
        \rgt{3d}{Michigan}{Col. John Mizner}
        \rgt{3d}{Wisconsin}{Col. William Barstow}
        \rgt{---}{Brackett's Minnesota}{Col. Elias Calkins}
    \end{middleBde}
    \begin{middleBde}
        \otherbde{Artillery}
        \rgt{7th}{Ohio Battery}{Capt. Silas Burnap}
        \rgt{10th}{Indiana Battery}{Capt. Jerome Cox}
    \end{middleBde}

\end{fulloob}
% TODO: Remove requirement for this blank line

\subsecdinkus

\subsection{Organization of the Army of the Cumberland, Maj. Gen. Harold Fawcett, III, % {{{3
commanding, April 8, 1862--MM, DD, 186Y*}
\footnotetext[1]{
    Arranged according to the numerical designation of the corps, divisions and
    brigades as prescribed in General Orders, No.~X, Headquarters, Army of the
    Cumberland, MM~DD, 1862.
}

\begin{fulloob}
    \staff{Chief of Artillery}{Col.}{James Cotter}
    \staff{Adjutant General}{Lieut. Col.}{Tyler Remington}
    
    \corps{Eighth Corps}{Maj. Gen.}{Ptolemy Smith} % {{{4

    \division{First Division}{Brig. Gen.}{John Wool} % {{{5
    \begin{leftBde}
        \bde{First Brigade}{Col.}{Ralph Buckland}
        \rgt{6th}{Ohio}{Col. William Bosley}
        \rgt{24th}{Ohio}{Col. Frederick Jones}
        \rgt{36th}{Indiana}{Col. William Grose}
        \rgt{3d}{Kentucky}{Col. Thomas Bramlette}
    \end{leftBde}
    \begin{rightBde}
        \bde{Second Brigade}{Col.}{William Hazen}
        \rgt{9th}{United States}{Col. Stephen Carpenter}
        \rgt{9th}{Indiana}{Col. Gideon Moody}
        \rgt{17th}{Indiana}{Col. Milo Hascall}
        \rgt{39th}{Illinois}{Col. William Morrison}
    \end{rightBde}
    \begin{middleBde}
        \bde{Third Brigade}{Col.}{Sanders Bruce}
        \rgt{78th}{Pennsylvania}{Col. William Sirwell}
        \rgt{10th}{Ohio}{Col. William Lytle}
        \rgt{13th}{Ohio}{Col. Joseph Hawkins}
        \rgt{7th}{Kentucky}{Col. Reuben May}
    \end{middleBde}

    \division{Second Division}{Brig. Gen.}{Thomas Crittenden} % {{{5
    \begin{leftBde}
        \bdeCdrs{First Brigade}{
            \bdeOneCdr{Brig Gen.}{Jeremiah Boyle}
            \bdeOneCdr{Col.}{Sidney Barnes}
        }
        \rgt{19th}{Ohio}{Col. Charles Manderson}
        \rgt{59th}{Ohio}{Col. James Fyffe}
        \rgtCdrs{8th}{Kentucky}{
            \rgtOneCdr{Col. Sidney Barnes}
            \rgtOneCdr{Lieut. Col. James Mayhew}
        }
        \rgt{12th}{Kentucky}{Col. William Hoskins}
    \end{leftBde}
    \begin{rightBde}
        \bde{Second Brigade}{Col.}{William Smith}
        \rgt{31s}{Ohio}{Col. Moses Walker}
        \rgt{33d}{Ohio}{Col. Joshua Sill}
        \rgt{65th}{Ohio}{Col. Charles Harker}
        \rgt{11th}{Indiana}{Lieut. Col. George McGinnis}
    \end{rightBde}
    \begin{middleBde}
        \bde{Third Brigade}{Col.}{John Pope Cook}
        \rgt{3d}{Ohio}{Col. Warren Keifer}
        \rgt{21st}{Ohio}{Col. Dwella Stoughton}
        \rgt{25th}{Indiana}{Col. James Veatch}
        \rgt{31st}{Indiana}{Col. Charles Cruft}
    \end{middleBde}

    \bde{Artillery}{Lieut. Col.}{Peter Simonson} % {{{5
    \begin{leftBde}
        \otherbde{First Division}
        \rgt{1st}{Kentucky, Battery A}{Capt. David Stone}
        \rgt{1st}{Ohio, Battery F}{Capt. Daniel Cockerill}
    \end{leftBde}
    \begin{rightBde}
        \otherbde{Second Division}
        \rgt{4th}{Ohio Battery}{Capt. Louis Hoffman}
        \rgt{1st}{Kentucky, Battery B}{Capt. John Hewitt}
    \end{rightBde}
    \begin{middleBde}
        \otherbde{Reserve Artillery}
        \rgt{5th}{Ohio Battery}{Capt. Andrew Hickenlooper}
        \rgt{9th}{Ohio Battery}{Capt. Henry Wetmore}
    \end{middleBde}

    \corps{Twelfth Corps}{Brig. Gen.}{Charles Smith} % {{{4

    \division{First Division}{Brig. Gen.}{William Sherman} % {{{5
    \begin{leftBde}
        \bde{First Brigade}{Col.}{Benjamin Smith}
        \rgt{1st}{Ohio}{Lieut. Col. Joab Stafford}
        \rgt{40th}{Ohio}{Col. Edwin Bradley}
        \rgt{8th}{Illinois}{Col. Richard Rowett}
        \rgt{24th}{Illinois}{Col. Friedrich Hecker}
    \end{leftBde}
    \begin{rightBde}
        \bdeCdrs{Second Brigade}{
            \bdeOneCdr{Col.}{George Wagner}
            \bdeOneCdr{Col.}{Thomas Ransom}
        }
        \rgt{15th}{Indiana}{Col. Gustavus Wood}
        \rgt{38th}{Indiana}{Col. Benjamin Scribner}
        \rgtCdrs{11th}{Illinois}{
            \rgtOneCdr{Col. Thomas Ransom}
            \rgtOneCdr{Lieut. Col. James Coates}
        }
        \rgt{15th}{Illinois}{Col. Thomas Turner}
    \end{rightBde}
    \begin{middleBde}
        \bde{Third Brigade}{Col.}{Gustavus Smith}
        \rgt{35th}{Illinois}{Col. William Chandler}
        \rgt{21st}{Illinois}{Col. John Alexander}
        \rgt{15th}{Wisconsin}{Col. Hans Heg}
        \rgt{18th}{United States}{Col. Henry Barrington}
    \end{middleBde}

    \division{Second Division}{Brig. Gen.}{Horatio Van Cleve} % {{{5
    \begin{leftBde}
        \bde{First Brigade}{Col.}{Samuel Beatty}
        \rgt{6th}{Michigan}{Col. Frederick Curtenius}
        \rgt{42d}{Indiana}{Col. James G. Jones}
        \rgt{38th}{Illinois}{Col. William Carlin}
        \rgt{2d}{Minnesota}{Col. James George}
    \end{leftBde}
    \begin{rightBde}
        \bde{Second Brigade}{Col.}{William Stoughton}
        \rgt{9th}{Michigan}{Col. William Duffeld}
        \rgt{11th}{Michigan}{Lieut. Col. Melvin Mudge}
        \rgt{15th}{Michigan}{Col. John Oliver}
        \rgt{1st}{Wisconsin}{Col. John Starkweather}
    \end{rightBde}
    \begin{middleBde}
        \bde{Third Brigade}{Col.}{Jefferson Davis}
        \rgt{8th}{Wisconsin}{Col. George Robbins}
        \rgt{10th}{Wisconsin}{Col. Alfred Chapin}
        \rgt{13th}{Wisconsin}{Col. Maurice Maloney}
        \rgt{35th}{Indiana}{Col. Bernard Mullen}
    \end{middleBde}

    \bde{Artillery}{Lieut. Col.}{Charles Humphrey}% % {{{5
    \begin{leftBde}
        \otherbde{First Division}
        \rgt{1st}{Illinois, Battery D}{Capt. Henry Rogers}
        \rgt{2d}{Illinois, Battery E}{Capt. Adolphus Schwartz}
    \end{leftBde}
    \begin{rightBde}
        \otherbde{Second Division}
        \rgt{1st}{Michigan, Battery B}{Capt. William Ross}
        \rgt{3d}{Wisconsin Battery}{Capt. Lu Drury}
    \end{rightBde}
    \begin{middleBde}
        \otherbde{Reserve Artillery}
        \rgt{5th}{Wisconsin Battery}{Capt. George Gardner}
        \rgt{1st}{Michigan, Battery C}{Capt. Alexander Dees}
    \end{middleBde}

    \corps{Fourteenth Corps}{Brig. Gen.}{John McClernand} % {{{4

    \division{First Division}{Brig. Gen.}{George Thomas} % {{{5
    \begin{leftBde}
        \bde{First Brigade}{Col.}{Samuel Carter}
        \rgt{1st}{Kentucky}{Col. David Enyart}
        \rgt{2d}{Kentucky}{Col. Thomas Sedgewick}
        \rgt{1st}{Tennessee}{Col. Robert Byrd}
        \rgt{13th}{Michigan}{Col. Charles Stuart}
    \end{leftBde}
    \begin{rightBde}
        \bde{Second Brigade}{Col.}{Madison Miller}
        \rgt{10th}{Indiana}{Lieut. Col. William Carroll}
        \rgt{4th}{Kentucky}{Col. John Croxton}
        \rgt{14th}{Ohio}{Col. James Steedman}
        \rgt{17th}{Ohio}{Col. John Connell}
    \end{rightBde}
    \begin{middleBde}
        \bdeCdrs{Third Brigade}{
            \bdeOneCdr{Col.}{Thomas Kilby Smith}
            \bdeOneCdr{Col.}{Karl Sonderson}
        }
        \rgt{44th}{Illinois}{Col. Charles Knobelsdorff}
        \rgtCdrs{9th}{Ohio}{
            \rgtOneCdr{Col. Karl Sonderson}
            \rgtOneCdr{Lieut. Col. Frank Mattice}
        }
        \rgt{35th}{Ohio}{Col. Ferdinand Van Derveer}
        \rgt{38th}{Ohio}{Lieut. Col. William Choate}
    \end{middleBde}

    \division{Second Division}{Brig. Gen.}{William Rosecrans} % {{{5
    \begin{leftBde}
        \bde{First Brigade}{Brig. Gen.}{Benjamin Prentiss}
        \rgt{15th}{United States}{Lieut. Col. John Kung}
        \rgt{6th}{Indiana}{Col. Philemon Baldwin}
        \rgt{77th}{Pennsylvania}{Col. Frederick Stumbaugh}
        \rgt{79th}{Pennsylvania}{Col. Henry Hambright}
    \end{leftBde}
    \begin{rightBde}
        \bde{Second Brigade}{Col.}{Edward Kirk}
        \rgt{16th}{United States}{Col. Edmund Schriver}
        \rgt{29th}{Indiana}{Col. David Dunn}
        \rgt{30th}{Indiana}{Col. Sion Bass}
        \rgt{34th}{Illinois}{Col. Charles Levanway}
    \end{rightBde}
    \begin{middleBde}
        \bde{Third Brigade}{Col.}{William Gibson}
        \rgt{15th}{Ohio}{Col. Moses Dickey}
        \rgt{5th}{Kentucky}{Col. Harvey Buckley}
        \rgt{32d}{Indiana}{Col. Henry von Trebra}
        \rgt{39th}{Indiana}{Col. Thomas Harrison}
    \end{middleBde}

    \bde{Artillery}{Lieut. Col.}{Charles Muehler}% % {{{5
    \begin{leftBde}
        \otherbde{First Division}
        \rgt{---}{Kentucky, Simmond's Battery}{Capt. Seth Simmonds}
        \rgt{1st}{Ohio, battery D}{Capt. Andrew Konkle}
    \end{leftBde}
    \begin{rightBde}
        \otherbde{Second Division}
        \rgt{4th}{United States}{Lieut. S. Canby}
        \rgt{4th}{Indiana Battery}{Capt. Asahel Bush}
    \end{rightBde}
    \begin{middleBde}
        \otherbde{Reserve Artillery}
        \rgt{4th}{United States, Battery M}{Capt. John Brannan}
        \rgt{1st}{Michigan, Battery A}{Capt. Cyrus Loomis}
    \end{middleBde}

    \corpsCdrs{Sixteenth Corps}{ % {{{4
        \corpsOneCdr{Brig. Gen.}{Zebulon Benton}
        \corpsOneCdr{Brig. Gen.}{James Garfield}
    }

    \divisionCdrs{First Division}{ % {{{5
        \divOneCdr{Brig. Gen.}{James Garfield}
        \divOneCdr{Col. (Bvt. Brig. Gen.)}{John De Courcey}
    }
    \begin{leftBde}
        \bdeCdrs{First Brigade}{
            \bdeOneCdr{Col.}{John De Courcey}
            \bdeOneCdr{Col.}{Daniel Lindsey}
        }
        \rgt{51st}{Indiana}{Col. Abel Streight}
        \rgt{57th}{Indiana}{Col. Cyrus Hines}
        \rgtCdrs{22d}{Kentucky}{
            \rgtOneCdr{Col. Daniel Lindsey}
            \rgtOneCdr{Lieut. Col. George Monroe}
        }
        \rgt{16th}{Ohio}{Col. Eli Botsford}
    \end{leftBde}
    \begin{rightBde}
        \bdeCdrs{Second Brigade}{
            \bdeOneCdr{Col.}{Robert Sinclair}
            \bdeOneCdr{Col.}{James Craddock}
        }
        \rgt{14th}{Michigan}{Col. Henry Mizner}
        \rgt{44th}{Indiana}{Col. Hugh Reed}
        \rgt{61st}{Illinois}{Col. Jacob Fry}
        \rgtCdrs{16th}{Kentucky}{
            \rgtOneCdr{Col. James Craddock}
            \rgtOneCdr{Lieut. Col. James Gault}
        }
    \end{rightBde}
    \begin{middleBde}
        \bde{Third Brigade}{Col.}{Oliver Shepherd}
        \rgt{12th}{Michigan}{Col. Francis Quinn}
        \rgt{3d}{Minnesota}{Col. Henry Lester}
        \rgt{14th}{Kentucky}{Col. Laban Moore}
        \rgt{24th}{Kentucky}{Col. Lewis Grigsby}
    \end{middleBde}

    \division{Second Division}{Brig. Gen.}{Richard Oglesby} % {{{5
    \begin{leftBde}
        \bdeCdrs{First Brigade}{
            \bdeOneCdr{Col.}{Marcellus Mundy}
            \bdeOneCdr{Col.}{Timothy Stanley}
        }
        \rgtCdrs{18th}{Ohio}{
            \rgtOneCdr{Col. Timothy Stanley}
            \rgtOneCdr{Lieut. Col. Josiah Given}
        }
        \rgt{58th}{Ohio}{Col. Valentine Bausenwien}
        \rgt{23d}{Kentucky}{Col. J. P. Jackson}
        \rgt{26th}{Kentucky}{Col. Stephen Burbridge}
    \end{leftBde}
    \begin{rightBde}
        \bdeCdrs{Second Brigade}{
            \bdeOneCdr{Col.}{Oliver Wood}
            \bdeOneCdr{Col.}{Joseph St. James}
        }
        \rgt{18th}{Kentucky}{Col. William Warner}
        \rgt{19th}{Kentucky}{Col. William Landram}
        \rgt{58th}{Illinois}{Col. William Lynch}
        \rgtCdrs{22d}{Ohio}{
            \rgtOneCdr{Col. Joseph St. James}
            \rgtOneCdr{Lieut. Col. Theodore Case}
        }
    \end{rightBde}
    \begin{middleBde}
        \bde{Third Brigade\footnotemark[1]}{Col.}{Thomas Matthews}
        \rgt{14th}{Wisconsin}{Col. David Wood}
        \rgt{51st}{Ohio}{Col. Richard McClain}
        \rgt{12th}{Illinois}{Col. John McArthur}
        \rgt{14th}{Illinois}{Col. John Palmer}
    \end{middleBde}
    \footnotetext[1]{Destroyed Apr. 11, 1862}

    \bdeCdrs{Artillery}{ % {{{5
        \bdeOneCdr{Maj.}{Alanson Stevens}
        \bdeOneCdr{Capt.}{Charles Willard}
    }
    \begin{leftBde}
        \otherbde{First Division}
        \rgt{4th}{United States, Battery G}{Capt. Albion Howe}
        \rgt{10th}{Ohio Battery}{Capt. Hamilton White}
    \end{leftBde}
    \begin{rightBde}
        \otherbde{Second Division}
        \rgt{5th}{Indiana Battery}{Capt. Daniel Chandler}
        \rgt{8th}{Indiana Battery}{Capt. George Estep}
    \end{rightBde}
    \begin{middleBde}
        \otherbde{Reserve Artillery\footnotemark[1]}
        \rgt{---}{Pennsylvania, Battery B}{Capt. Samuel McDowell}
        \rgtCdrs{1st}{Illinois, Battery A}{
            \rgtOneCdr{Capt. Charles Willard}
            \rgtOneCdr{Capt. Francis Morgan}
        }
    \end{middleBde}
    \footnotetext[1]{Destroyed Apr. 11, 1862}

    \corps{Cavalry Corps}{Brig. Gen.}{Lawrence Graham} % {{{4

    \division{First Division}{Brig. Gen.}{Theophilus Dickey} % {{{5
    \begin{leftBde}
        \bde{First Brigade}{Col.}{John Bridgeland}
        \rgt{1st}{Kentucky}{Col. Silas Adams}
        \rgt{3d}{Kentucky}{Col. James Jackson}
        \rgt{2d}{Indiana}{Lieut. Col. James Stewart}
        \rgt{4th}{United States}{Col. James Oakes}
    \end{leftBde}
    \begin{rightBde}
        \bde{Second Brigade}{Col.}{Charles Doubleday}
        \rgt{2d}{Ohio}{Col. August Kautz}
        \rgt{3d}{Ohio}{Col. Horace Howland}
        \rgt{4th}{Ohio}{Col. Eli Long}
        \rgt{2d}{Wisconsin}{Col. Cadwallader Washburn}
    \end{rightBde}
    \begin{middleBde}
        \bde{Third Brigade}{Col.}{Arthur Rankin}
        \rgt{2d}{Illinois}{Col. Silas Noble}
        \rgt{4th}{Illinois}{Col. Martin Wallace}
        \rgt{13th}{Illinois}{Col. Joseph Bell}
        \rgt{1st}{U.S. Lancers (Michigan)}{Lieut. Col. James Herrick}
    \end{middleBde}

    \division{Second Division}{Brig. Gen.}{Cyrus Bussey} % {{{5
    \begin{leftBde}
        \bde{First Brigade}{Col.}{Edward McCook}
        \rgt{4th}{Kentucky}{Col. Jesse Bayles}
        \rgt{5th}{Kentucky}{Col. David Haggard}
        \rgt{9th}{Illinois}{Col. Albert Barckett}
        \rgt{3d}{Iowa}{Lieut. Col. Henry Caldwell}
    \end{leftBde}
    \begin{rightBde}
        \bde{Second Brigade}{Col.}{Robert Ingersoll}
        \rgt{5th}{Illinois}{Col. Hall Wilson}
        \rgt{11th}{Illinois}{Lieut. Col. Lucien Kerr}
        \rgt{1st}{Wisconsin}{Col. Edward Daniels}
        \rgt{7th}{Pennsylvania}{Col. Charles Davis}
    \end{rightBde}
    \begin{middleBde}
        \bde{Third Brigade}{Col.}{Own Ransom}
        \rgt{5th}{Iowa}{Col. William Lowe}
        \rgt{2d}{Kansas}{Col. Alson Davis}
        \rgt{10th}{Illinois}{Col. James Barrett}
        \rgt{1st}{Ohio}{Lieut. Col. Minor Millikin}
    \end{middleBde}

    \bde{Artillery}{}{} % {{{5
    \begin{leftBde}
        \otherbde{First Division}
        \rgt{1st}{Ohio, Battery E}{Capt. Warren Edgarton}
        \rgt{5th}{United States, Battery H}{Capt. William Terrill}
        \rgt{4th}{United States, Battery I}{Capt. Oscar Mack}
    \end{leftBde}
    \begin{rightBde}
        \otherbde{Second Division}
        \rgt{1st}{Ohio, Battery M}{Capt. Frederick Schultz}
        \rgt{69th}{Ohio Independent Battery}{Capt. Cullen Bradley}
    \end{rightBde}

    \corps{Artillery Reserve}{Col.}{Charles Cotter} % {{{4
    \begin{leftBde}
        \bde{First Brigade}{Lieut. Col.}{William Standart}
        \rgt{1st}{Ohio, Battery A}{Capt. Wilbur Goodspeed}
        \rgt{1st}{Ohio, Battery B}{Capt. J. Hale Snyder}
        \rgt{1st}{Ohio, Battery C}{Capt. Dennis Kenny, Jr}
    \end{leftBde}
    \begin{rightBde}
        \bde{Second Brigade}{Lieut. Col.}{Alonzo Bidwell}
        \rgt{1st}{Michigan, Battery D}{Capt. Josiah Church}
        \rgt{1st}{Illinois, Battery C}{Capt. Charles Houghtaling}
        \rgt{2d}{Illinois, Battery C}{Capt. Caleb Hopkins}
    \end{rightBde}
\end{fulloob}
% TODO: Remove requirement for this blank line

\subsecdinkus

\subsection{Organization of the Army of the Kanawha, Maj. Gen. Richard Steele, % {{{3
    U.S. Army, commanding, April 8, 1862--MM, DD, 186Y}

\begin{fulloob}
    \corps{Eleventh Corps}{Brig. Gen.}{Jacob Ryan} % {{{4
    \division{First Division}{Brig. Gen.}{Benjamin Kelley}
    \division{Second Division}{Brig. Gen.}{Robert Schenck}
    \division{Third Division\footnotemark[1]}{Brig. Gen.}{Hayes}
    \footnotetext[1]{Deactivated and units reassigned Apr. 21, 1862.}

    \corpsCdrs{Seventeenth Corps}{ % {{{4
        \corpsOneCdr{Brig. Gen.}{Henry Johanson}
        \corpsOneCdr{Brig. Gen.}{Robert Milroy}
    }
    \divisionCdrs{First Division}{
        \divOneCdr{Brig. Gen.}{Robert Milroy}
        \divOneCdr{Brig. Gen.}{Rutherford Hayes}
    }
    \division{Second Division}{Brig. Gen.}{Jacob Cox}

    \corps{Cavalry}{Brig. Gen.}{John Hall\footnotemark[1]} % {{{4
    \footnotetext[1]{Organized into a corps on Apr. 21, 1862.}
    \divisionCdrs{First Division}{
        \divOneCdr{Brig. Gen.}{John Hall}
        \divOneCdr{Col.}{John Wilder}
    }
    \division{Second Division}{Brig. Gen.}{Thomas J. Mosier}

\end{fulloob}
% TODO: Remove requirement for this blank line

\subsecdinkus

\subsection{Organization of the Army of the Kentucky, Maj. Gen. James Howard, % {{{3
    U.S. Army, commanding, April 8, 1862--MM, DD, 186Y}

\begin{fulloob}
    \corps{Seventh Corps}{Brig. Gen.}{Thomas Clancy} % {{{4
    \division{First Division}{Brig. Gen.}{}
    \division{Second Division}{Brig. Gen.}{}
    \division{Third Division}{Brig. Gen.}{}

    \corps{Ninth Corps}{Brig. Gen.}{Aaron Gates} % {{{4
    \division{First Division}{Brig. Gen.}{}
    \division{Second Division}{Brig. Gen.}{}

    \corps{Cavalry}{Maj. Gen.}{James Howard} % {{{4
    \division{First Division}{Brig. Gen.}{Hall}
    \division{Second Division}{Brig. Gen.}{Mosier}

\end{fulloob}
% TODO: Remove requirement for this blank line

\subsecdinkus

\subsection{Return of casualties in the Union forces % {{{3
after the Battles of Barton's Creek \& Palmyra, April 11--12, 1862}

\begin{oob}{}
\oobhdr

\oobtop{Army of the Kanawha}{Maj. Gen.}{Richard Steele} % {{{4

\oobtop{Eleventh Corps}{Brig. Gen.}{Jacob Ryan} % {{{5

\oobdiv{First Division}{Brig. Gen.}{Benjamin Kelley} % {{{6
\oobrgt{}{First Brigade}{}{2,400}{1,200}{50.00}
\oobrgt{}{Second Brigade}{}{2,400}{1,200}{50.00}
\oobrgt{}{Third Brigade}{}{2,400}{600}{25.00}
\oobsum{Total First Division}{7,200}{3,000}{41.67}
\oobdblrule

\oobdiv{Second Division}{Brig. Gen.}{Robert Schenck} % {{{6
\oobrgt{}{First Brigade}{}{2,400}{600}{25.00}
\oobrgt{}{Second Brigade}{}{2,400}{1,200}{50.00}
\oobrgt{}{Third Brigade}{}{2,400}{1,200}{50.00}
\oobsum{Total Second Division}{7,200}{3,000}{41.67}

\oobdiv{Third Division}{Brig. Gen.}{Rutherford Hayes} % {{{6
\oobrgt{}{First Brigade}{}{2,400}{1,200}{50.00}
\oobrgt{}{Second Brigade}{}{2,400}{600}{25.00}
\oobrgt{}{Third Brigade}{}{2,400}{1,200}{50.00}
\oobsum{Total Second Division}{7,200}{3,000}{41.67}

\oobdblrule

\oobdiv{Artillery}{}{} % {{{6

\oobart{First Division Artillery}
\oobtot{Total First Division Artillery}{18}{12}{66.67}

\oobart{Second Division Artillery}
\oobtot{Total Second Division Artillery}{18}{6}{33.33}

\oobart{Third Division Artillery}
\oobtot{Total Third Division Artillery}{18}{\oobnone}{\oobnone}

\oobsum{Total Artillery}{54}{18}{33.33}
\oobdblrule

\oobsum{Eleventh Corps}{21,600}{9,000}{41.67} % {{{6
\oobdblrule

\oobtop{Seventeenth Corps}{Brig. Gen.}{Henry Johanson} % {{{5

\oobdiv{First Division}{Brig. Gen.}{Robert Milroy} % {{{6
\oobrgt{}{First Brigade}{}{2,400}{600}{25.00}
\oobrgt{}{Second Brigade}{}{2,400}{1,200}{50.00}
\oobrgt{}{Third Brigade}{}{2,400}{1,200}{50.00}
\oobsum{Total First Division}{7,200}{3,000}{51.67}
\oobdblrule

\oobdiv{Second Division}{Brig. Gen.}{Jacob Cox} % {{{6
\oobrgt{}{First Brigade}{}{2,400}{600}{25.00}
\oobrgt{}{Second Brigade}{}{2,400}{1,200}{50.00}
\oobrgt{}{Third Brigade}{}{2,400}{1,200}{50.00}
\oobsum{Total Second Division}{7,200}{3,000}{41.67}

\oobdblrule

\oobdiv{Artillery}{}{} % {{{6

\oobart{First Division Artillery}
\oobtot{Total First Division Artillery}{18}{12}{66.67}

\oobart{Second Division Artillery}
\oobtot{Total Second Division Artillery}{18}{12}{66.67}

\oobsum{Total Artillery}{36}{24}{66.67}
\oobdblrule

\oobsum{Seventeenth Corps}{14,400}{6,000}{41.67} % {{{6
\oobdblrule

\oobtop{Cavalry Corps}{}{} % {{{5

\oobdiv{First Division}{Brig. Gen.}{John Hall} % {{{6
\oobrgt{}{First Brigade}{}{900}{300}{33.33}
\oobrgt{}{Second Brigade}{}{900}{300}{33.33}
\oobrgt{}{Third Brigade}{}{900}{300}{33.33}
\oobsum{Total First Division}{2,700}{900}{33.33}
\oobdblrule

\oobdiv{Second Division}{Brig. Gen.}{Thomas J. Mosier} % {{{6
\oobrgt{}{First Brigade}{}{900}{300}{33.33}
\oobrgt{}{Second Brigade}{}{900}{\oobnone}{\oobnone}
\oobrgt{}{Third Brigade}{}{900}{\oobnone}{\oobnone}
\oobsum{Total Second Division}{2,700}{300}{11.11}

\oobdblrule

\oobdiv{Artillery}{}{} % {{{6

\oobart{First Division Artillery}
\oobtot{Total First Division Artillery}{6}{\oobnone}{\oobnone}

\oobart{Second Division Artillery}
\oobtot{Total Second Division Artillery}{6}{\oobnone}{\oobnone}

\oobsum{Total Artillery}{12}{\oobnone}{\oobnone}
\oobdblrule

\oobsum{Cavalry Corps}{5,400}{1,200}{22.22} % {{{6
\oobdblrule

\oobtop{Artillery Reserve}{}{} % {{{5

\oobtot{Total First Brigade Artillery}{18}{\oobnone}{\oobnone}
\oobtot{Total Second Brigade Artillery}{18}{\oobnone}{\oobnone}

\oobsum{Total Artillery Reserve}{36}{\oobnone}{\oobnone} % {{{6

% Overall Totals % {{{5
\oobrecap

\oobsub{Eleventh Corps}{21,600}{9,000}{41.67}
\oobsub{Seventeenth Corps}{14,400}{6,000}{41.67}

\oobdblrule
\oobsum{Grand total}{36,000}{15,000}{41.67}
\oobdblrule

\oobsub{First Cavalry Division}{2,700}{900}{33.33}
\oobsub{Second Cavalry Division}{2,700}{300}{11.11}

\oobdblrule
\oobsum{Grand total}{5,400}{1,200}{22.22}
\oobdblrule

\oobsub{Eleventh Corps Artillery}{54}{18}{33.33}
\oobsub{Seventeenth Corps Artillery}{36}{24}{66.67}
\oobsub{Cavalry Corps Artillery}{12}{\oobnone}{\oobnone}
\oobsub{Artillery Reserve}{36}{\oobnone}{\oobnone}

\oobdblrule
\oobsum{Grand total}{138}{42}{30.43}

\oobtop{Army of the Cumberland}{Maj. Gen.}{Harold Fawcett, III} % {{{4

\oobtop{Eighth Corps}{Maj. Gen.}{Ptolemy Smith} % {{{5

\oobdiv{First Division}{Brig. Gen.}{John Wool} % {{{6
\oobbde{First Brigade}{Col.}{Ralph Buckland} % {{{7
\oobrgt{6th}{Ohio}{Col. William Bosley}{600}{\oobnone}{\oobnone}
\oobrgt{24th}{Ohio}{Col. Frederick Jones}{600}{\oobnone}{\oobnone}
\oobrgt{36th}{Indiana}{Col. William Grose}{600}{\oobnone}{\oobnone}
\oobrgt{3d}{Kentucky}{Col. Thomas Bramlette}{600}{\oobnone}{\oobnone}
\oobtot{Total First Brigade}{2,400}{\oobnone}{\oobnone}

\oobbde{Second Brigade}{Col.}{William Hazen} % {{{7
\oobrgt{9th}{United States}{Col. Stephen Carpenter}{600}{\oobnone}{\oobnone}
\oobrgt{9th}{Indiana}{Col. Gideon Moody}{600}{\oobnone}{\oobnone}
\oobrgt{17th}{Indiana}{Col. Milo Hascall}{600}{\oobnone}{\oobnone}
\oobrgt{39th}{Illinois}{Col. William Morrison}{600}{\oobnone}{\oobnone}
\oobtot{Total Second Brigade}{2,400}{\oobnone}{\oobnone}

\oobbde{Third Brigade}{Col.}{Sanders Bruce} % {{{7
\oobrgt{78th}{Pennsylvania}{Col. William Sirwell}{600}{\oobnone}{\oobnone}
\oobrgt{10th}{Ohio}{Col. William Lytle}{600}{\oobnone}{\oobnone}
\oobrgt{13th}{Ohio}{Col. Joseph Hawkins}{600}{\oobnone}{\oobnone}
\oobrgt{7th}{Kentucky}{Col. Reuben May}{600}{\oobnone}{\oobnone}
\oobtot{Total Third Brigade}{2,400}{\oobnone}{\oobnone}

\oobsum{Total First Division}{7,200}{\oobnone}{\oobnone} % {{{7
\oobdblrule

\oobdiv{Second Division}{Brig. Gen.}{Thomas Crittenden} % {{{6
\oobbde{First Brigade}{Brig. Gen.}{Jeremiah Boyle} % {{{7
\oobrgt{19th}{Ohio}{Col. Charles Manderson}{600}{\oobnone}{\oobnone}
\oobrgt{59th}{Ohio}{Col. James Fyffe}{600}{\oobnone}{\oobnone}
\oobrgt{8th}{Kentucky}{Col. Sidney Barnes}{600}{\oobnone}{\oobnone}
\oobrgt{12th}{Kentucky}{Col. William Hoskins}{600}{\oobnone}{\oobnone}
\oobtot{Total First Brigade}{2,400}{\oobnone}{\oobnone}

\oobbde{Second Brigade}{Col.}{William Smith} % {{{7
\oobrgt{31st}{Ohio}{Col. Moses Walker}{600}{200}{33.33}
\oobrgt{33d}{Ohio}{Col. Joshua Sill}{600}{200}{33.33}
\oobrgt{65th}{Ohio}{Col. Charles Harker}{600}{200}{33.33}
\oobrgt{11th}{Indiana}{Lieut. Col. George McGinnis}{600}{\oobnone}{\oobnone}
\oobtot{Total Second Brigade}{2,400}{600}{25.00}

\oobbde{3d Brigade}{Brig. Gen.}{John Pope Cook} % {{{7
\oobrgt{3d}{Ohio}{Col. Warren Keifer}{600}{\oobnone}{\oobnone}
\oobrgt{21st}{Ohio}{Col. Dwella Stoughton}{600}{\oobnone}{\oobnone}
\oobrgt{25th}{Indiana}{Col. James Veatch}{600}{\oobnone}{\oobnone}
\oobrgt{31st}{Indiana}{Col. Charles Cruft}{600}{\oobnone}{\oobnone}
\oobtot{Total Third Brigade}{2,400}{\oobnone}{\oobnone}

\oobsum{Total Second Division}{7,200}{600}{8.33} % {{{7
\oobdblrule

\oobdiv{Artillery}{Lieut. Col.}{Peter Simonson} % {{{6

\oobart{First Division Artillery} % {{{7
\oobrgt{1st}{Kentucky Light Artillery, Battery A}{Capt. David Stone}{6}{\oobnone}{\oobnone}
\oobrgt{1st}{Ohio Light Artillery, Battery F}{Capt. Daniel Cockerill}{6}{\oobnone}{\oobnone}
\oobtot{Total First Division Artillery}{12}{\oobnone}{\oobnone}

\oobart{Second Division Artillery} % {{{7
\oobrgt{4th}{Ohio Independent Battery}{Capt. Louis Hoffman}{6}{\oobnone}{\oobnone}
\oobrgt{1st}{Kentucky Light Artillery, Battery B}{Capt. John Hewitt}{6}{\oobnone}{\oobnone}
\oobtot{Total Second Division Artillery}{12}{\oobnone}{\oobnone}

\oobart{Reserve Artillery} % {{{7
\oobrgt{5th}{Ohio Independent Battery}{Capt. Andrew Hickenlooper}{6}{\oobnone}{\oobnone}
\oobrgt{9th}{Ohio Independent Battery}{Capt. Henry Wetmore}{6}{\oobnone}{\oobnone}
\oobtot{Total Reserve Artillery}{12}{\oobnone}{\oobnone}

\oobsum{Total Artillery}{36}{\oobnone}{\oobnone} % {{{7
\oobdblrule

% Overall Totals % {{{6
\oobsum{Eighth Corps}{14,400}{600}{4.16}

\oobtop{Twelfth Corps}{Brig. Gen.}{Charles Smith} % {{{5

\oobdiv{First Division}{Brig. Gen.}{William Sherman} % {{{6

\oobbde{First Brigade}{Col.}{Benjamin Smith} % {{{7
\oobrgt{1st}{Ohio}{Lieut. Col. (Bvt. Col.) Joab Stafford}{600}{\oobnone}{\oobnone}
\oobrgt{40th}{Ohio}{Col. Edwin Bradley}{600}{\oobnone}{\oobnone}
\oobrgt{8th}{Illinois}{Col. Richard Rowett}{600}{\oobnone}{\oobnone}
\oobrgt{24th}{Illinois}{Col. Friedrich Hecker}{600}{\oobnone}{\oobnone}
\oobtot{Total First Brigade}{2,400}{\oobnone}{\oobnone}

\oobbde{Second Brigade}{Col.}{George Wagner} % {{{7
\oobrgt{15th}{Indiana}{Col. Gustavus Wood}{600}{\oobnone}{\oobnone}
\oobrgt{38th}{Indiana}{Col. Benjamin Scribner}{600}{\oobnone}{\oobnone}
\oobrgt{11th}{Illinois}{Col. Thomas Ransom}{600}{\oobnone}{\oobnone}
\oobrgt{15th}{Illinois}{Col. Thomas Turner}{600}{\oobnone}{\oobnone}
\oobtot{Total Second Brigade}{2,400}{\oobnone}{\oobnone}

\oobbde{Third Brigade}{Col.}{Gustavus Smith} % {{{7
\oobrgt{35th}{Illinois}{Col. William Chandler}{600}{\oobnone}{\oobnone}
\oobrgt{21st}{Illinois}{Col. John Alexander}{600}{\oobnone}{\oobnone}
\oobrgt{15th}{Wisconsin}{Col. Hans Heg}{600}{\oobnone}{\oobnone}
\oobrgt{18th}{United States}{Col. Henry Barrington}{600}{\oobnone}{\oobnone}
\oobtot{Total Second Brigade}{2,400}{\oobnone}{\oobnone}

% Overall Totals % {{{7
\oobsum{First Division}{7,200}{\oobnone}{\oobnone}
\oobdblrule

\oobdiv{Second Division}{Brig. Gen.}{Horatio Van Cleve} % {{{6
& \SetCell{c}Second Division, XIIth Corps remained at Clarksville, Tenn. and was not present for the engagement.\\
\\ % blank line

\oobdiv{Artillery}{Lieut. Col.}{Charles Humphrey} % {{{6

\oobart{First Division Artillery} % {{{7
\oobrgt{1st}{Illinois Light Artillery, Battery D}{Capt. Henry Rogers}{6}{\oobnone}{\oobnone}
\oobrgt{2d}{Illinois Light Artillery, Battery E}{Capt. Adolphus Schwartz}{6}{\oobnone}{\oobnone}
\oobtot{Total First Division Artillery}{12}{\oobnone}{\oobnone}

\oobart{Second Division Artillery} % {{{7
& \SetCell{c}Second Division Artillery, XIIth Corps remained at Clarksville, Tenn. and was not present for the engagement.\\
\\ % blank line

\oobart{Reserve Artillery} % {{{7
& \SetCell{c}Reserve Artillery, XIIth Corps remained at Clarksville, Tenn. and was not present for the engagement.\\
\\ % blank line

\oobsum{Total Artillery}{12}{\oobnone}{\oobnone}
\oobdblrule

% Overall Totals % {{{6
\oobsum{Twelfth Corps}{7,200}{\oobnone}{\oobnone}
\oobdblrule

\oobtop{Sixteenth Corps}{Brig. Gen.}{Zebulon Benton} % {{{5

\oobdiv{First Division}{Brig. Gen.}{James Garfield} % {{{6
\oobbde{First Brigade}{Col.}{John De Courcey} % {{{7
\oobrgt{51st}{Indiana}{Col. Abel Streight}{600}{300}{50.00}
\oobrgt{57th}{Indiana}{Col. Cyrus Hines}{600}{300}{50.00}
\oobrgt{22d}{Kentucky}{Col. Daniel Lindsey}{600}{\oobnone}{\oobnone}
\oobrgt{16th}{Ohio}{Col. Eli Botsford}{600}{\oobnone}{\oobnone}
\oobtot{Total First Brigade}{2,400}{600}{25.00}

\oobbde{Second Brigade}{Col.}{Robert Sinclair} % {{{7
\oobrgt{14th}{Michigan}{Col. Henry Mizner}{600}{\oobnone}{\oobnone}
\oobrgt{44th}{Indiana}{Col. Hugh Reed}{600}{\oobnone}{\oobnone}
\oobrgt{61st}{Illinois}{Col. Jacob Frey}{600}{\oobnone}{\oobnone}
\oobrgt{16th}{Kentucky}{Col. James Craddock}{600}{\oobnone}{\oobnone}
\oobtot{Total Second Brigade}{2,400}{\oobnone}{\oobnone}

\oobbde{Third Brigade}{Col.}{Oliver Shepherd} % {{{7
\oobrgt{12th}{Michigan}{Col. Francis Quinn}{600}{\oobnone}{\oobnone}
\oobrgt{3d}{Minnesota}{Col. Henry Lester}{600}{\oobnone}{\oobnone}
\oobrgt{14th}{Kentucky}{Col. Laban Moore}{600}{300}{50.00}
\oobrgt{24th}{Kentucky}{Col. Lewis Grigsby}{600}{300}{50.00}
\oobtot{Total Third Brigade}{2,400}{600}{25.00}

\oobsum{Total First Division}{7,200}{1,200}{16.67} % {{{7
\oobdblrule

\oobdiv{Second Division}{Brig. Gen.}{Richard Oglesby} % {{{6
\oobbde{First Brigade}{Col.}{Marcellus Mundy} % {{{7
\oobrgt{18th}{Ohio}{Col. Timothy Stanley}{600}{\oobnone}{\oobnone}
\oobrgt{58th}{Ohio}{Col. Valentine Bausenwien}{600}{\oobnone}{\oobnone}
\oobrgt{23d}{Kentucky}{Col. J. P. Jackson}{600}{\oobnone}{\oobnone}
\oobrgt{26th}{Kentucky}{Col. Stephen Burbridge}{600}{\oobnone}{\oobnone}
\oobtot{Total First Brigade}{2,400}{\oobnone}{\oobnone}

\oobbde{Second Brigade}{Col.}{Oliver Wood} % {{{7
\oobrgt{18th}{Kentucky}{Col. William Warner}{600}{300}{50.00}
\oobrgt{19th}{Kentucky}{Col. William Landram}{600}{300}{50.00}
\oobrgt{58th}{Illinois}{Col. William Lynch}{600}{\oobnone}{\oobnone}
\oobrgt{22d}{Ohio}{Col. Joseph St. James}{600}{\oobnone}{\oobnone}
\oobtot{Total Second Brigade}{2,400}{600}{25.00}

\oobbde{3d Brigade}{Col.}{Thomas Matthews} % {{{7
\oobrgt{14th}{Wisconsin}{Col. David Wood}{600}{600}{100.00}
\oobrgt{51st}{Ohio}{Col. Richard McClain}{600}{600}{100.00}
\oobrgt{12th}{Illinois}{Col. John McArthur}{600}{600}{100.00}
\oobrgt{14th}{Illinois}{Col. John Palmer}{600}{600}{100.00}
\oobtot{Total Third Brigade}{2,400}{2,400}{100.00}

\oobsum{Total Second Division}{7,200}{3,000}{41.67} % {{{7
\oobdblrule

\oobdiv{Artillery}{Maj.}{Alanson Stevens} % {{{6

\oobart{First Division Artillery} % {{{7
\oobrgt{4th}{United States Artillery, Battery G}{Capt. Albion Howe}{6}{\oobnone}{\oobnone}
\oobrgt{10th}{Ohio Independent Battery}{Capt. Hamilton White}{6}{\oobnone}{\oobnone}
\oobtot{Total First Division Artillery}{12}{\oobnone}{\oobnone}

\oobart{Second Division Artillery} % {{{7
\oobrgt{5th}{Indiana Battery}{Capt. Daniel Chandler}{6}{\oobnone}{\oobnone}
\oobrgt{8th}{Indiana Battery}{Capt. George Estep}{6}{\oobnone}{\oobnone}
\oobtot{Total Second Division Artillery}{12}{\oobnone}{\oobnone}

\oobsum{Total Artillery}{24}{\oobnone}{\oobnone}
\oobdblrule

% Overall Totals % {{{6
\oobsum{Sixteenth Corps}{14,400}{3,600}{25.00}

% Overall Totals % {{{5
\oobrecap

\oobsub{Eighth Corps}{14,400}{600}{4.16}
\oobsub{Twelfth Corps}{7,200}{\oobnone}{\oobnone}
\oobsub{Sixteenth Corps}{14,400}{3,600}{25.00}

\oobdblrule
\oobsum{Grand total}{36,000}{4,200}{11.67}
\oobdblrule

\oobsub{Eighth Corps Artillery}{36}{\oobnone}{\oobnone}
\oobsub{Twelfth Corps Artillery}{12}{\oobnone}{\oobnone}
\oobsub{Sixteenth Corps Artillery}{36}{12}{33.33}

\oobdblrule
\oobsum{Grand total}{108}{12}{11.11}

\bottomrule
\end{oob}

\subsecdinkus

\begin{casualties} % {{{4
    \vspace{10pt}\kiawia{Officers Killed}\vspace{5pt}
    Maj. Alanson Stevens \\

    \vspace{10pt}\kiawia{Officers Wounded}\vspace{5pt}
    Brig. Gen. Zebulon Benton \\
    Brig. Gen. Henry Johanson \\
    Brig. Gen. Jacob Ryan \\

    \vspace{10pt}\kiawia{Officers Missing}\vspace{5pt}
    Col. Thomas Matthews \\

    \kiawiaState{Illinois}
    Col. John McArthur \\
    Col. John Palmer \\

    \kiawiaState{Ohio}
    Col. Richard McClain \\

    \kiawiaState{Wisconsin}
    Col. David Wood \\

\end{casualties}
\subsecdinkus

\subsection{Return of casualties in the Union forces % {{{3
    after the Battle of Cloverdale, May 2, 1862}

\begin{oob}{}
\oobhdr

\oobtop{Army of the Cumberland}{Maj. Gen.}{Harold Fawcett, III} % {{{4

\oobtop{Eighth Corps}{Maj. Gen.}{Ptolemy Smith} % {{{5

\oobdiv{First Division}{Brig. Gen.}{John Wool} % {{{6
\oobbde{First Brigade}{Col.}{Ralph Buckland} % {{{7
\oobrgt{6th}{Ohio}{Col. William Bosley}{600}{\oobnone}{\oobnone}
\oobrgt{24th}{Ohio}{Col. Frederick Jones}{600}{\oobnone}{\oobnone}
\oobrgt{36th}{Indiana}{Col. William Grose}{600}{\oobnone}{\oobnone}
\oobrgt{3d}{Kentucky}{Col. Thomas Bramlette}{600}{\oobnone}{\oobnone}
\oobtot{Total First Brigade}{2,400}{\oobnone}{\oobnone}

\oobbde{Second Brigade}{Col.}{William Hazen} % {{{7
\oobrgt{9th}{United States}{Col. Stephen Carpenter}{600}{300}{50.00}
\oobrgt{9th}{Indiana}{Col. Gideon Moody}{600}{300}{50.00}
\oobrgt{17th}{Indiana}{Col. Milo Hascall}{600}{\oobnone}{\oobnone}
\oobrgt{39th}{Illinois}{Col. William Morrison}{600}{\oobnone}{\oobnone}
\oobtot{Total Second Brigade}{2,400}{600}{25.00}

\oobbde{Third Brigade}{Col.}{Sanders Bruce} % {{{7
\oobrgt{78th}{Pennsylvania}{Col. William Sirwell}{600}{\oobnone}{\oobnone}
\oobrgt{10th}{Ohio}{Col. William Lytle}{600}{\oobnone}{\oobnone}
\oobrgt{13th}{Ohio}{Col. Joseph Hawkins}{600}{\oobnone}{\oobnone}
\oobrgt{7th}{Kentucky}{Col. Reuben May}{600}{\oobnone}{\oobnone}
\oobtot{Total Third Brigade}{2,400}{\oobnone}{\oobnone}

\oobsum{Total First Division}{7,200}{600}{8.33} % {{{7
\oobdblrule

\oobdiv{Second Division}{Brig. Gen.}{Thomas Crittenden} % {{{6
\oobbde{First Brigade}{Brig. Gen.}{Jeremiah Boyle} % {{{7
\oobrgt{19th}{Ohio}{Col. Charles Manderson}{600}{200}{33.33}
\oobrgt{59th}{Ohio}{Col. James Fyffe}{600}{200}{33.33}
\oobrgt{8th}{Kentucky}{Col. Sidney Barnes}{600}{200}{33.33}
\oobrgt{12th}{Kentucky}{Col. William Hoskins}{600}{\oobnone}{\oobnone}
\oobtot{Total First Brigade}{2,400}{600}{25.00}

\oobbde{Second Brigade}{Col.}{William Smith} % {{{7
\oobrgt{31st}{Ohio}{Col. Moses Walker}{400}{\oobnone}{\oobnone}
\oobrgt{33d}{Ohio}{Col. Joshua Sill}{400}{\oobnone}{\oobnone}
\oobrgt{65th}{Ohio}{Col. Charles Harker}{400}{\oobnone}{\oobnone}
\oobrgt{11th}{Indiana}{Lieut. Col. George McGinnis}{600}{\oobnone}{\oobnone}
\oobtot{Total Second Brigade}{1,800}{\oobnone}{\oobnone}

\oobbde{3d Brigade}{Brig. Gen.}{John Pope Cook} % {{{7
\oobrgt{3d}{Ohio}{Col. Warren Keifer}{600}{150}{25.00}
\oobrgt{21st}{Ohio}{Col. Dwella Stoughton}{600}{150}{25.00}
\oobrgt{25th}{Indiana}{Col. James Veatch}{600}{150}{25.00}
\oobrgt{31st}{Indiana}{Col. Charles Cruft}{600}{150}{25.00}
\oobtot{Total Third Brigade}{2,400}{600}{33.33}

\oobsum{Total Second Division}{6,600}{1,200}{18.18} % {{{7
\oobdblrule

\oobdiv{Artillery}{Lieut. Col.}{Peter Simonson} % {{{6

\oobart{First Division Artillery} % {{{7
\oobrgt{1st}{Kentucky Light Artillery, Battery A}{Capt. David Stone}{6}{\oobnone}{\oobnone}
\oobrgt{1st}{Ohio Light Artillery, Battery F}{Capt. Daniel Cockerill}{6}{\oobnone}{\oobnone}
\oobtot{Total First Division Artillery}{12}{\oobnone}{\oobnone}

\oobart{Second Division Artillery} % {{{7
\oobrgt{4th}{Ohio Independent Battery}{Capt. Louis Hoffman}{6}{\oobnone}{\oobnone}
\oobrgt{1st}{Kentucky Light Artillery, Battery B}{Capt. John Hewitt}{6}{\oobnone}{\oobnone}
\oobtot{Total Second Division Artillery}{12}{\oobnone}{\oobnone}

\oobart{Reserve Artillery} % {{{7
\oobrgt{5th}{Ohio Independent Battery}{Capt. Andrew Hickenlooper}{6}{\oobnone}{\oobnone}
\oobrgt{9th}{Ohio Independent Battery}{Capt. Henry Wetmore}{6}{\oobnone}{\oobnone}
\oobtot{Total Reserve Artillery}{12}{\oobnone}{\oobnone}

\oobsum{Total Artillery}{36}{\oobnone}{\oobnone}
\oobdblrule

% Overall Totals % {{{6
\oobsum{Eighth Corps}{14,400}{1,800}{12.50}

\oobtop{Twelfth Corps}{Brig. Gen.}{Charles Smith} % {{{5

\oobdiv{First Division}{Brig. Gen.}{William Sherman} % {{{6

\oobbde{First Brigade}{Col.}{Benjamin Smith} % {{{7
\oobrgt{1st}{Ohio}{Lieut. Col. (Bvt. Col.) Joab Stafford}{600}{\oobnone}{\oobnone}
\oobrgt{40th}{Ohio}{Col. Edwin Bradley}{600}{\oobnone}{\oobnone}
\oobrgt{8th}{Illinois}{Col. Richard Rowett}{600}{\oobnone}{\oobnone}
\oobrgt{24th}{Illinois}{Col. Friedrich Hecker}{600}{\oobnone}{\oobnone}
\oobtot{Total First Brigade}{2,400}{\oobnone}{\oobnone}

\oobbde{Second Brigade}{Col.}{George Wagner} % {{{7
\oobrgt{15th}{Indiana}{Col. Gustavus Wood}{600}{\oobnone}{\oobnone}
\oobrgt{38th}{Indiana}{Col. Benjamin Scribner}{600}{200}{33.33}
\oobrgt{11th}{Illinois}{Col. Thomas Ransom}{600}{200}{33.33}
\oobrgt{15th}{Illinois}{Col. Thomas Turner}{600}{200}{33.33}
\oobtot{Total Second Brigade}{2,400}{600}{25.00}

\oobbde{Third Brigade}{Col.}{Gustavus Smith} % {{{7
\oobrgt{35th}{Illinois}{Col. William Chandler}{600}{\oobnone}{\oobnone}
\oobrgt{21st}{Illinois}{Col. John Alexander}{600}{\oobnone}{\oobnone}
\oobrgt{15th}{Wisconsin}{Col. Hans Heg}{600}{\oobnone}{\oobnone}
\oobrgt{18th}{United States}{Col. Henry Barrington}{600}{\oobnone}{\oobnone}
\oobtot{Total Second Brigade}{2,400}{\oobnone}{\oobnone}

% Overall Totals % {{{7
\oobsum{First Division}{7,200}{600}{8.33}
\oobdblrule

\oobdiv{Second Division}{Brig. Gen.}{Horatio Van Cleve} % {{{6
\oobbde{First Brigade}{Col.}{Samuel Beatty} % {{{7
\oobrgt{6th}{Michigan}{Col. Frederick Curtenius}{600}{\oobnone}{\oobnone}
\oobrgt{42d}{Indiana}{Col. James G. Jones}{600}{\oobnone}{\oobnone}
\oobrgt{38th}{Illinois}{Col. William Carline}{600}{\oobnone}{\oobnone}
\oobrgt{2d}{Minnesota}{Col. James George}{600}{\oobnone}{\oobnone}
\oobtot{Total First Brigade}{2,400}{\oobnone}{\oobnone}

\oobbde{Second Brigade}{Col.}{William Stoughton} % {{{7
\oobrgt{9th}{Michigan}{Col. William Duffeld}{600}{150}{25.00}
\oobrgt{11th}{Michigan}{Col. Melvin Mudge}{600}{150}{25.00}
\oobrgt{15th}{Michigan}{Col. John Oliver}{600}{150}{25.00}
\oobrgt{1st}{Wisconsin}{Col. John Starkweather}{600}{150}{25.00}
\oobtot{Total Second Brigade}{2,400}{600}{25.00}

\oobbde{Third Brigade}{Col.}{Jefferson Davis} % {{{7
\oobrgt{8th}{Wisconsin}{Col. George Robbins}{600}{\oobnone}{\oobnone}
\oobrgt{10th}{Wisconsin}{Col. Alfred Chapin}{600}{\oobnone}{\oobnone}
\oobrgt{13th}{Wisconsin}{Col. Maurice Maloney}{600}{\oobnone}{\oobnone}
\oobrgt{35th}{Indiana}{Col. Bernard Mullen}{600}{\oobnone}{\oobnone}
\oobtot{Total Second Brigade}{2,400}{\oobnone}{\oobnone}

% Overall Totals % {{{7
\oobsum{Second Division}{7,200}{600}{8.33}
\oobdblrule

\oobdiv{Artillery}{Lieut. Col.}{Charles Humphrey} % {{{6

\oobart{First Division Artillery} % {{{7
\oobrgt{1st}{Illinois Light Artillery, Battery D}{Capt. Henry Rogers}{6}{3}{50.00}
\oobrgt{2d}{Illinois Light Artillery, Battery E}{Capt. Adolphus Schwartz}{6}{3}{50.00}
\oobtot{Total First Division Artillery}{12}{6}{50.00}

\oobart{Second Division Artillery} % {{{7
\oobrgt{1st}{Michigan Light Artillery, Battery B}{Capt. William Ross}{6}{\oobnone}{\oobnone}
\oobrgt{3d}{Wisconsin Independent Battery}{Capt. Lu Drury}{6}{\oobnone}{\oobnone}
\oobtot{Total Second Division Artillery}{12}{\oobnone}{\oobnone}

\oobart{Reserve Artillery} % {{{7
\oobrgt{5th}{Wisconsin Independent Battery}{Capt. George Gardner}{6}{\oobnone}{\oobnone}
\oobrgt{1st}{Michigan Light Artillery, Battery C}{Capt. Alexander Dees}{6}{\oobnone}{\oobnone}
\oobtot{Total Reserve Artillery}{12}{\oobnone}{\oobnone}

\oobsum{Total Artillery}{36}{6}{16.67}
\oobdblrule

% Overall Totals % {{{6
\oobsum{Twelfth Corps}{14,400}{1,200}{8.33}
\oobdblrule

\oobtop{Fourteenth Corps}{Brig. Gen.}{John McClernand} % {{{5

\oobdiv{First Division}{Brig. Gen.}{George Thomas} % {{{6

\oobbde{First Brigade}{Col.}{Samuel Carter} % {{{7
\oobrgt{1st}{Kentucky}{Col. David Enyart}{600}{\oobnone}{\oobnone}
\oobrgt{2d}{Kentucky}{Col. Thomas Sedgewick}{600}{\oobnone}{\oobnone}
\oobrgt{1st}{Tennessee}{Col. Robert Byrd}{600}{\oobnone}{\oobnone}
\oobrgt{13th}{Michigan}{Col. Charles Stuart}{600}{\oobnone}{\oobnone}
\oobtot{Total First Brigade}{2,400}{\oobnone}{\oobnone}

\oobbde{Second Brigade}{Col.}{Madison Miller} % {{{7
\oobrgt{10th}{Indiana}{Lieut. Col. William Carroll}{600}{\oobnone}{\oobnone}
\oobrgt{4th}{Kentucky}{Col. John Croxton}{600}{300}{50.00}
\oobrgt{14th}{Ohio}{Col. James Steedman}{600}{300}{50.00}
\oobrgt{17th}{Ohio}{Col. John Connell}{600}{\oobnone}{\oobnone}
\oobtot{Total Second Brigade}{2,400}{600}{25.00}

\oobbde{Third Brigade}{Col.}{Thomas Kilby Smith} % {{{7
\oobrgt{44th}{Illinois}{Col. Charles Knobelsdorff}{600}{\oobnone}{\oobnone}
\oobrgt{9th}{Ohio}{Col. Karl Sonderson}{600}{\oobnone}{\oobnone}
\oobrgt{35th}{Ohio}{Col. Ferdinand Van Derveer}{600}{\oobnone}{\oobnone}
\oobrgt{38th}{Ohio}{Col. William Choate}{600}{\oobnone}{\oobnone}
\oobtot{Total Third Brigade}{2,400}{\oobnone}{\oobnone}

\oobsum{Total First Division}{7,200}{600}{8.33} % {{{7
\oobdblrule

\oobdiv{Second Division}{Brig. Gen.}{William Rosecrans} % {{{6

\oobbde{First Brigade}{Brig. Gen.}{Benjamin Prentiss} % {{{7
\oobrgt{15th}{United States}{Lieut. Col. John Kung}{600}{150}{25.00}
\oobrgt{6th}{Indiana}{Col. Philemon Baldwin}{600}{150}{25.00}
\oobrgt{77th}{Pennsylvania}{Col. Frederick Stumbaugh}{600}{150}{25.00}
\oobrgt{79th}{Pennsylvania}{Col. Henry Hambright}{600}{150}{25.00}
\oobtot{Total First Brigade}{2,400}{600}{25.00}

\oobbde{Second Brigade}{Col.}{Edward Kirk} % {{{7
\oobrgt{16th}{United States}{Col. Edmund Schriver}{600}{\oobnone}{\oobnone}
\oobrgt{29th}{Indiana}{Col. David Dunn}{600}{\oobnone}{\oobnone}
\oobrgt{30th}{Indiana}{Col. Sion Bass}{600}{\oobnone}{\oobnone}
\oobrgt{34th}{Illinois}{Col. Charles Levanway}{600}{\oobnone}{\oobnone}
\oobtot{Total Second Brigade}{2,400}{\oobnone}{\oobnone}

\oobbde{Third Brigade}{Col.}{William Gibson} % {{{7
\oobrgt{15th}{Ohio}{Col. Moses Dickey}{600}{\oobnone}{\oobnone}
\oobrgt{5th}{Kentucky}{Col. Harvey Buckley}{600}{300}{50.00}
\oobrgt{32d}{Indiana}{Col. Henry von Trebra}{600}{300}{50.00}
\oobrgt{39th}{Indiana}{Col. Thomas Harrison}{600}{\oobnone}{\oobnone}
\oobtot{Total Third Brigade}{\oobnone}{\oobnone}{\oobnone}

\oobsum{Total Second Division}{7,200}{1,200}{16.67} % {{{7
\oobdblrule

\oobdiv{Artillery}{Lieut. Col.}{Charles Muehler} % {{{6

\oobart{First Division Artillery} % {{{7
\oobrgt{---}{Kentucky Light Artillery, Simmond's Battery}{Capt. Seth Simmonds}{6}{\oobnone}{\oobnone}
\oobrgt{1st}{Ohio Light Artillery, Battery D}{Capt. Andrew Konkle}{6}{\oobnone}{\oobnone}
\oobtot{Total First Division Artillery}{12}{\oobnone}{\oobnone}

\oobart{Second Division Artillery} % {{{7
\oobrgt{4th}{United States Artillery, Battery H}{Capt. S.  Canby}{6}{\oobnone}{\oobnone}
\oobrgt{4th}{Indiana Battery}{Capt. Asahel Bush}{6}{\oobnone}{\oobnone}
\oobtot{Total Second Division Artillery}{12}{\oobnone}{\oobnone}

\oobart{Reserve Artillery} % {{{7
\oobrgt{4th}{United States, Battery M}{Capt. John Brannan}{6}{\oobnone}{\oobnone}
\oobrgt{1st}{Michigan Light Artillery, Battery A}{Capt. Cyrus Loomis}{3}{\oobnone}{\oobnone}
\oobtot{Total Reserve Artillery}{12}{\oobnone}{\oobnone}

\oobsum{Total Artillery}{36}{\oobnone}{\oobnone}
\oobdblrule

\oobsum{Fourteenth Corps}{14,400}{1,800}{12.50} % {{{6
\oobdblrule

\oobtop{Sixteenth Corps}{Brig. Gen.}{Zebulon Benton} % {{{5

\oobdiv{First Division}{Brig. Gen.}{James Garfield} % {{{6
\oobbde{First Brigade}{Col.}{John De Courcey} % {{{7
\oobrgt{51st}{Indiana}{Col. Abel Streight}{300}{\oobnone}{\oobnone}
\oobrgt{57th}{Indiana}{Col. Cyrus Hines}{300}{\oobnone}{\oobnone}
\oobrgt{22d}{Kentucky}{Col. Daniel Lindsey}{600}{\oobnone}{\oobnone}
\oobrgt{16th}{Ohio}{Col. Eli Botsford}{600}{\oobnone}{\oobnone}
\oobtot{Total First Brigade}{1,800}{\oobnone}{\oobnone}

\oobbde{Second Brigade}{Col.}{Robert Sinclair} % {{{7
\oobrgt{14th}{Michigan}{Col. Henry Mizner}{600}{300}{50.00}
\oobrgt{44th}{Indiana}{Col. Hugh Reed}{600}{\oobnone}{\oobnone}
\oobrgt{61st}{Illinois}{Col. Jacob Frey}{600}{\oobnone}{\oobnone}
\oobrgt{16th}{Kentucky}{Col. James Craddock}{600}{300}{50.00}
\oobtot{Total Second Brigade}{2,400}{600}{25.00}

\oobbde{Third Brigade}{Col.}{Oliver Shepherd} % {{{7
\oobrgt{12th}{Michigan}{Col. Francis Quinn}{600}{300}{50.00}
\oobrgt{3d}{Minnesota}{Col. Henry Lester}{600}{300}{50.00}
\oobrgt{14th}{Kentucky}{Col. Laban Moore}{300}{\oobnone}{\oobnone}
\oobrgt{24th}{Kentucky}{Col. Lewis Grigsby}{300}{\oobnone}{\oobnone}
\oobtot{Total Third Brigade}{1,800}{600}{33.33}

\oobsum{Total First Division}{6,000}{1,800}{30.00} % {{{7
\oobdblrule

\oobdiv{Second Division}{Brig. Gen.}{Richard Oglesby} % {{{6
\oobbde{First Brigade}{Col.}{Marcellus Mundy} % {{{7
\oobrgt{18th}{Ohio}{Col. Timothy Stanley}{600}{\oobnone}{\oobnone}
\oobrgt{58th}{Ohio}{Col. Valentine Bausenwien}{600}{\oobnone}{\oobnone}
\oobrgt{23d}{Kentucky}{Col. J. P. Jackson}{600}{\oobnone}{\oobnone}
\oobrgt{26th}{Kentucky}{Col. Stephen Burbridge}{600}{\oobnone}{\oobnone}
\oobtot{Total First Brigade}{2,400}{\oobnone}{\oobnone}

\oobbde{Second Brigade}{Col.}{Oliver Wood} % {{{7
\oobrgt{18th}{Kentucky}{Col. William Warner}{300}{\oobnone}{\oobnone}
\oobrgt{19th}{Kentucky}{Col. William Landram}{300}{\oobnone}{\oobnone}
\oobrgt{58th}{Illinois}{Col. William Lynch}{600}{300}{50.00}
\oobrgt{22d}{Ohio}{Col. Joseph St. James}{600}{300}{50.00}
\oobtot{Total Second Brigade}{1,800}{600}{33.33}

\oobsum{Total Second Division}{4,200}{1,800}{42.86} % {{{7
\oobdblrule

\oobdiv{Artillery}{Maj.}{Alanson Stevens} % {{{6

\oobart{First Division Artillery} % {{{7
\oobrgt{4th}{United States Artillery, Battery G}{Capt. Albion Howe}{6}{\oobnone}{\oobnone}
\oobrgt{10th}{Ohio Independent Battery}{Capt. Hamilton White}{6}{\oobnone}{\oobnone}
\oobtot{Total First Division Artillery}{12}{\oobnone}{\oobnone}

\oobart{Second Division Artillery} % {{{7
\oobrgt{5th}{Indiana Battery}{Capt. Daniel Chandler}{6}{\oobnone}{\oobnone}
\oobrgt{8th}{Indiana Battery}{Capt. George Estep}{6}{\oobnone}{\oobnone}
\oobtot{Total Second Division Artillery}{12}{\oobnone}{\oobnone}

\oobsum{Total Artillery}{24}{\oobnone}{\oobnone}
\oobdblrule

% Overall Totals % {{{6
\oobsum{Sixteenth Corps}{14,400}{1,800}{25.00}

\oobtop{Cavalry Corps}{Brig. Gen.}{Lawrence Graham} % {{{5

\oobtop{First Division}{Brig. Gen.}{Theophilus Dickey} % {{{6

\oobbde{First Brigade}{Col.}{John Bridgeland} % {{{7
\oobrgt{1st}{Kentucky}{Lieut. Col. Silas Adams}{300}{75}{25.00}
\oobrgt{3d}{Kentucky}{Col. James Jackson}{300}{75}{25.00}
\oobrgt{2d}{Indiana}{Lieut. Col. James Stewart}{300}{75}{25.00}
\oobrgt{4th}{United States}{Col. James Oakes}{300}{75}{25.00}
\oobtot{Total First Brigade}{1,200}{300}{25.00}

\oobbde{Second Brigade}{Col.}{Charles Doubleday} % {{{7
\oobrgt{2d}{Ohio}{Col. August Kautz}{300}{75}{25.00}
\oobrgt{3d}{Ohio}{Lieut. Col. Horace Howland}{300}{75}{25.00}
\oobrgt{4th}{Ohio}{Col. Eli Long}{300}{75}{25.00}
\oobrgt{2d}{Wisconsin}{Col. Cadwallader Washburn}{300}{75}{25.00}
\oobtot{Total Second Brigade}{1,200}{300}{25.00}

\oobbde{Third Brigade}{Col.}{Arthur Rankin} % {{{7
\oobrgt{2d}{Illinois}{Col. Silas Noble}{300}{\oobnone}{\oobnone}
\oobrgt{4th}{Illinois}{Col. Martin Wallace}{300}{\oobnone}{\oobnone}
\oobrgt{13th}{Illinois}{Col. Joseph Bell}{300}{\oobnone}{\oobnone}
\oobrgt{1st}{U.S. Lancers (Michigan)}{Lieut. Col. James Herrick}{300}{\oobnone}{\oobnone}
\oobtot{Total Third Brigade}{1,200}{\oobnone}{\oobnone}

\oobart{Division Artillery} % {{{7
\oobrgt{1st}{Ohio Light Artillery, Battery E}{Capt. Warren Edgarton}{6}{\oobnone}{\oobnone}
\oobrgt{5th}{United States Artillery, Battery H}{Capt. William Terrill}{6}{\oobnone}{\oobnone}
\oobrgt{4th}{United States Artillery, Battery I}{Capt. Oscar Mack}{6}{\oobnone}{\oobnone}
\oobtot{Total Division Artillery}{18}{\oobnone}{\oobnone}

% Overall Totals % {{{7
\oobsum{First Division}{3,600}{600}{16.67}
\oobdblrule

\oobtop{Second Division}{Brig. Gen.}{Cyrus Bussey} % {{{6

\oobbde{First Brigade}{Col.}{Edward McCook} % {{{7
\oobrgt{4th}{Kentucky}{Col. Jesse Bayles}{300}{\oobnone}{\oobnone}
\oobrgt{5th}{Kentucky}{Col. David Haggard}{300}{\oobnone}{\oobnone}
\oobrgt{9th}{Illinois}{Col. Albert Barckett}{300}{\oobnone}{\oobnone}
\oobrgt{3d}{Iowa}{Lieut. Col. Henry Caldwell}{300}{\oobnone}{\oobnone}
\oobtot{Total First Brigade}{1,200}{\oobnone}{\oobnone}

\oobbde{Second Brigade}{Col.}{Robert Ingersoll} % {{{7
\oobrgt{5th}{Illinois}{Col. Hall Wilson}{300}{\oobnone}{\oobnone}
\oobrgt{11th}{Illinois}{Lieut. Col. Lucien Kerr}{300}{\oobnone}{\oobnone}
\oobrgt{1st}{Wisconsin}{Col. Edward Daniels}{300}{\oobnone}{\oobnone}
\oobrgt{7th}{Pennsylvania}{Col. Charles Davis}{300}{\oobnone}{\oobnone}
\oobtot{Total Second Brigade}{1,200}{\oobnone}{\oobnone}

\oobbde{Third Brigade}{Col.}{Owen Ransom} % {{{7
\oobrgt{5th}{Iowa}{Col. William Lowe}{300}{\oobnone}{\oobnone}
\oobrgt{2d}{Kansas}{Col. Alson Davis}{300}{\oobnone}{\oobnone}
\oobrgt{10th}{Illinois}{Col. James Barrett}{300}{\oobnone}{\oobnone}
\oobrgt{1st}{Ohio}{Lieut. Col. Minor Millikin}{300}{\oobnone}{\oobnone}
\oobtot{Total Third Brigade}{1,200}{\oobnone}{\oobnone}

\oobart{Division Artillery} % {{{7
\oobrgt{1st}{Ohio Light Artillery, Battery M}{Capt. Frederick Schultz}{6}{\oobnone}{\oobnone}
\oobrgt{6th}{Ohio Independent Battery}{Capt. Cullen Bradley}{6}{\oobnone}{\oobnone}
\oobtot{Total Division Artillery}{12}{\oobnone}{\oobnone}

% Overall Totals % {{{7
\oobsum{Second Division}{3,600}{\oobnone}{\oobnone}
\oobdblrule

\oobtop{Kellogg's Division}{Brig. Gen.}{William Kellogg} % {{{6

\oobbde{First Brigade}{Col.}{John Farnsworth} % {{{7
\oobrgt{8th}{Illinois}{Col. William Gamble}{300}{75}{25.00}
\oobrgt{12th}{Illinois}{Col. Arno Voss}{300}{75}{25.00}
\oobrgt{15th}{Pennsylvania}{Col. William Palmer}{300}{75}{25.00}
\oobrgt{2d}{Kentucky}{Col. Buckner Board}{300}{75}{25.00}
\oobtot{Total First Brigade}{1,200}{300}{25.00}

\oobbde{Second Brigade}{Col.}{Charles Jennison} % {{{7
\oobrgt{6th}{Ohio}{Col. William Lloyd}{300}{\oobnone}{\oobnone}
\oobrgt{3d}{Illinois}{Col. Lafayette McCrillis}{300}{\oobnone}{\oobnone}
\oobrgt{9th}{Pennsylvania}{Col. Edward Williams}{300}{\oobnone}{\oobnone}
\oobrgt{7th}{Kansas}{Col. Daniel Anthony}{300}{\oobnone}{\oobnone}
\oobtot{Total Second Brigade}{1,200}{\oobnone}{\oobnone}

\oobbde{Third Brigade}{Col.}{Alfred Brackett} % {{{7
\oobrgt{7th}{Illinois}{Col. Edward Prince}{300}{\oobnone}{\oobnone}
\oobrgt{3d}{Michigan}{Col. John Mizner}{300}{\oobnone}{\oobnone}
\oobrgt{3d}{Wisconsin}{Col. William Barstow}{300}{\oobnone}{\oobnone}
\oobrgt{--}{Brackett's Minnesota}{Col. Elias Calkins}{300}{\oobnone}{\oobnone}
\oobtot{Total Third Brigade}{1,200}{\oobnone}{\oobnone}

\oobart{Division Artillery} % {{{7
\oobrgt{7th}{Ohio Independent Battery}{Capt. Silas Burnap}{6}{\oobnone}{\oobnone}
\oobrgt{10th}{Indiana Battery}{Capt. Jerome Cox}{6}{\oobnone}{\oobnone}
\oobtot{Total Division Artillery}{12}{\oobnone}{\oobnone}

% Overall Totals % {{{7
\oobsum{Kellogg's Division}{3,600}{300}{8.33}
\oobdblrule

% % Overall Totals % {{{6
\oobsum{Cavalry Corps}{10,800}{300}{2.78}

\oobtop{Artillery Reserve}{Col.}{Charles Cotter} % {{{4
\oobbde{First Brigade}{Lieut. Col.}{William Standart} % {{{5
\oobrgt{1st}{Ohio Light Artillery, Battery A}{Capt. Wilbur Goodspeed}{6}{\oobnone}{\oobnone}
\oobrgt{1st}{Ohio Light Artillery, Battery B}{Capt. J. Hale Sypher}{6}{\oobnone}{\oobnone}
\oobrgt{1st}{Ohio Light Artillery, Battery F}{Capt. Dennis Kenny, Jr.}{6}{\oobnone}{\oobnone}
\oobtot{Total First Brigade Artillery}{18}{\oobnone}{\oobnone}

\oobbde{Second Brigade}{Lieut. Col.}{Alonzo Bidwell} % {{{5
\oobrgt{1st}{Michigan Light Artillery, Battery D}{Capt. Josiah Church}{6}{\oobnone}{\oobnone}
\oobrgt{1st}{Illinois Light Artillery, Battery C}{Capt. Charles Houghtaling}{6}{\oobnone}{\oobnone}
\oobrgt{2d}{Illinois Light Artillery, Battery C}{Capt. Caleb Hopkins}{6}{\oobnone}{\oobnone}
\oobtot{Total Second Brigade Artillery}{18}{\oobnone}{\oobnone}

\oobsum{Total Artillery Reserve}{36}{\oobnone}{\oobnone} % {{{5

% Overall Totals % {{{5
\oobrecap

\oobsub{Eighth Corps}{13,800}{1,800}{13.04}
\oobsub{Twelfth Corps}{14,400}{1,200}{8.33}
\oobsub{Fourteenth Corps}{14,400}{1,800}{12.50}
\oobsub{Sixteenth Corps}{10,200}{1,800}{17.65}

\oobdblrule
\oobsum{Grand total}{52,800}{6,600}{12.50}
\oobdblrule

\oobsub{Cavalry Corps}{10,800}{900}{8.33}

\oobdblrule
\oobsub{Grand Total}{10,800}{900}{8.33}
\oobdblrule

\oobsub{Eighth Corps Artillery}{36}{\oobnone}{\oobnone}
\oobsub{Twelfth Corps Artillery}{36}{6}{16.67}
\oobsub{Fourteenth Corps Artillery}{36}{\oobnone}{\oobnone}
\oobsub{Sixteenth Corps Artillery}{24}{\oobnone}{\oobnone}
\oobsub{Cavalry Corps Artillery}{30}{\oobnone}{\oobnone}
\oobsub{Artillery Reserve}{36}{\oobnone}{\oobnone}

\oobdblrule
\oobsum{Grand total}{198}{6}{3.03}

\oobtop{Army of the Kanawha}{Maj. Gen.}{Richard Steele} % {{{4

\oobtop{Eleventh Corps}{Brig. Gen.}{Jacob Ryan} % {{{5

\oobdiv{First Division}{Brig. Gen.}{Benjamin Kelley} % {{{6
\oobrgt{}{First Brigade}{}{1,800}{\oobnone}{\oobnone}
\oobrgt{}{Second Brigade}{}{1,800}{\oobnone}{\oobnone}
\oobrgt{}{Third Brigade}{}{1,800}{\oobnone}{\oobnone}
\oobsum{Total First Division}{5,400}{\oobnone}{\oobnone}
\oobdblrule

\oobdiv{Second Division}{Brig. Gen.}{Robert Schenck} % {{{6
\oobrgt{}{First Brigade}{}{1,800}{\oobnone}{\oobnone}
\oobrgt{}{Second Brigade}{}{1,800}{\oobnone}{\oobnone}
\oobrgt{}{Third Brigade}{}{1,200}{\oobnone}{\oobnone}
\oobsum{Total Second Division}{4,800}{\oobnone}{\oobnone}

\oobdblrule

\oobdiv{Artillery}{}{} % {{{6

\oobart{First Division Artillery}
\oobtot{Total First Division Artillery}{18}{\oobnone}{\oobnone}

\oobart{Second Division Artillery}
\oobtot{Total Second Division Artillery}{18}{\oobnone}{\oobnone}

\oobsum{Total Artillery}{36}{\oobnone}{\oobnone}
\oobdblrule

\oobsum{Eleventh Corps}{10,200}{\oobnone}{\oobnone} % {{{6
\oobdblrule

\oobtop{Seventeenth Corps}{Brig. Gen.}{Robert Milroy} % {{{5

\oobdiv{First Division}{Brig. Gen.}{Rutherford Hayes} % {{{6
\oobrgt{}{First Brigade}{}{1,800}{\oobnone}{\oobnone}
\oobrgt{}{Second Brigade}{}{1,800}{\oobnone}{\oobnone}
\oobrgt{}{Third Brigade}{}{1,800}{\oobnone}{\oobnone}
\oobsum{Total First Division}{5,400}{\oobnone}{\oobnone}
\oobdblrule

\oobdiv{Second Division}{Brig. Gen.}{Jacob Cox} % {{{6
\oobrgt{}{First Brigade}{}{1,800}{600}{33.33}
\oobrgt{}{Second Brigade}{}{1,800}{\oobnone}{\oobnone}
\oobrgt{}{Third Brigade}{}{1,800}{\oobnone}{\oobnone}
\oobsum{Total Second Division}{5,400}{600}{10.11}

\oobdblrule

\oobdiv{Artillery}{}{} % {{{6

\oobart{First Division Artillery}
\oobtot{Total First Division Artillery}{18}{\oobnone}{\oobnone}

\oobart{Second Division Artillery}
\oobtot{Total Second Division Artillery}{18}{\oobnone}{\oobnone}

\oobsum{Total Artillery}{36}{\oobnone}{\oobnone}
\oobdblrule

\oobsum{Seventeenth Corps}{10,800}{600}{5.56} % {{{6
\oobdblrule

\oobtop{Cavalry Corps}{Brig. Gen.}{John Hall} % {{{5

\oobdiv{First Division}{Col.}{John Wilder} % {{{6
\oobrgt{}{First Brigade}{}{600}{\oobnone}{\oobnone}
\oobrgt{}{Second Brigade}{}{600}{\oobnone}{\oobnone}
\oobrgt{}{Third Brigade}{}{600}{\oobnone}{\oobnone}
\oobsum{Total First Division}{1,800}{\oobnone}{\oobnone}
\oobdblrule

\oobdiv{Second Division}{Brig. Gen.}{Thomas J. Mosier} % {{{6
\oobrgt{}{First Brigade}{}{600}{\oobnone}{\oobnone}
\oobrgt{}{Second Brigade}{}{600}{\oobnone}{\oobnone}
\oobsum{Total Second Division}{1,200}{\oobnone}{\oobnone}

\oobdblrule

\oobdiv{Artillery}{}{} % {{{6

\oobart{First Division Artillery}
\oobtot{Total First Division Artillery}{12}{\oobnone}{\oobnone}

\oobart{Second Division Artillery}
\oobtot{Total Second Division Artillery}{12}{\oobnone}{\oobnone}

\oobsum{Total Artillery}{24}{\oobnone}{\oobnone}
\oobdblrule

\oobsum{Cavalry Corps}{3,000}{\oobnone}{\oobnone} % {{{6

% Overall Totals % {{{5
\oobrecap

\oobsub{Eleventh Corps}{10,200}{\oobnone}{\oobnone}
\oobsub{Seventeenth Corps}{10,800}{600}{5.56}

\oobdblrule
\oobsum{Grand total}{21,000}{600}{2.86}
\oobdblrule

\oobsub{First Cavalry Division}{1,800}{\oobnone}{\oobnone}
\oobsub{Second Cavalry Division}{1,200}{\oobnone}{\oobnone}

\oobdblrule
\oobsum{Grand total}{3,000}{\oobnone}{\oobnone}
\oobdblrule

\oobsub{Eleventh Corps Artillery}{36}{\oobnone}{\oobnone}
\oobsub{Seventeenth Corps Artillery}{36}{\oobnone}{\oobnone}
\oobsub{Cavalry Corps Artillery}{24}{\oobnone}{\oobnone}

\oobdblrule
\oobsum{Grand total}{132}{\oobnone}{\oobnone}

\bottomrule
\end{oob}

\begin{casualties} % {{{4
    \vspace{10pt}\kiawia{Officers Killed}\vspace{5pt}
    Col. Thomas Kilby Smith \\
    Col. George Wagner \\
    Col. Oliver Wood \\

    \vspace{10pt}\kiawia{Officers Wounded}\vspace{5pt}
    Maj. Gen. Harold K. Fawcett, III \\
    Brig. Gen. Jeremiah Boyle \\
    Col. Marcellus Mundy \\
    Col. Robert Sinclair \\
\end{casualties}
\subsecdinkus

\subsection{Return of casualties in the Union forces % {{{3
    after the Battle of Fort Defiance, May 9, 1862}

\begin{oob}{}
\oobhdr

\oobtop{Army of the Kanawha}{Maj. Gen.}{Richard Steele} % {{{4

\oobtop{Eleventh Corps}{Brig. Gen.}{Jacob Ryan} % {{{5

\oobdiv{First Division}{Brig. Gen.}{Benjamin Kelley} % {{{6
\oobrgt{}{First Brigade}{}{1,800}{\oobnone}{\oobnone}
\oobrgt{}{Second Brigade}{}{1,800}{600}{33.33}
\oobrgt{}{Third Brigade}{}{1,800}{\oobnone}{\oobnone}
\oobsum{Total First Division}{5,400}{600}{11.11}
\oobdblrule

\oobdiv{Second Division}{Brig. Gen.}{Robert Schenck} % {{{6
\oobrgt{}{First Brigade}{}{1,800}{600}{33.33}
\oobrgt{}{Second Brigade}{}{1,800}{\oobnone}{\oobnone}
\oobrgt{}{Third Brigade}{}{1,200}{\oobnone}{\oobnone}
\oobsum{Total Second Division}{4,800}{600}{12.60}

\oobdblrule

\oobdiv{Artillery}{}{} % {{{6

\oobart{First Division Artillery}
\oobtot{Total First Division Artillery}{18}{\oobnone}{\oobnone}

\oobart{Second Division Artillery}
\oobtot{Total Second Division Artillery}{18}{\oobnone}{\oobnone}

\oobsum{Total Artillery}{36}{\oobnone}{\oobnone}
\oobdblrule

\oobsum{Eleventh Corps}{10,200}{\oobnone}{\oobnone} % {{{6
\oobdblrule

\oobtop{Seventeenth Corps}{Brig. Gen.}{Robert Milroy} % {{{5

\oobdiv{First Division}{Brig. Gen.}{Rutherford Hayes} % {{{6
\oobrgt{}{First Brigade}{}{1,800}{\oobnone}{\oobnone}
\oobrgt{}{Second Brigade}{}{1,800}{\oobnone}{\oobnone}
\oobrgt{}{Third Brigade}{}{1,800}{\oobnone}{\oobnone}
\oobsum{Total First Division}{5,400}{\oobnone}{\oobnone}
\oobdblrule

\oobdiv{Second Division}{Brig. Gen.}{Jacob Cox} % {{{6
\oobrgt{}{First Brigade}{}{1,200}{\oobnone}{\oobnone}
\oobrgt{}{Second Brigade}{}{1,800}{\oobnone}{\oobnone}
\oobrgt{}{Third Brigade}{}{1,800}{\oobnone}{\oobnone}
\oobsum{Total Second Division}{4,800}{\oobnone}{\oobnone}

\oobdblrule

\oobdiv{Artillery}{}{} % {{{6

\oobart{First Division Artillery}
\oobtot{Total First Division Artillery}{18}{\oobnone}{\oobnone}

\oobart{Second Division Artillery}
\oobtot{Total Second Division Artillery}{18}{\oobnone}{\oobnone}

\oobsum{Total Artillery}{36}{\oobnone}{\oobnone}
\oobdblrule

\oobsum{Seventeenth Corps}{10,200}{\oobnone}{\oobnone} % {{{6
\oobdblrule


\oobtop{Dep't of the Cumberland}{Brig. Gen.}{Thomas Wood} % {{{5

\oobdiv{Wood's Division}{Brig. Gen.}{Thomas Wood} % {{{6
\oobbde{First Brigade}{Col.}{James Shelley} % {{{7
\oobrgt{Middle (5th)}{Tennessee}{Col. Nathaniel Witt}{600}{150}{25.00}
\oobrgt{23d}{Ohio}{Col. Eliakim Scammon}{600}{150}{25.00}
\oobrgt{32d}{Ohio}{Col. Thomas Ford}{600}{150}{25.00}
\oobrgt{7th}{Indiana}{Col. Ira Grover}{600}{150}{25.00}
\oobtot{Total First Brigade}{2,400}{600}{25.00}

\oobbde{Second Brigade}{Col.}{James Shackelford} % {{{7
\oobrgt{28th}{Ohio}{Col. August Moor}{600}{\oobnone}{\oobnone}
\oobrgt{64th}{Ohio}{Col. John Ferguson}{600}{\oobnone}{\oobnone}
\oobrgt{21st}{Kentucky}{Col. Ethelbert Dudley}{600}{\oobnone}{\oobnone}
\oobrgt{25th}{Kentucky}{Col. Benjamin Bristow}{600}{\oobnone}{\oobnone}
\oobtot{Total Second Brigade}{2,400}{\oobnone}{\oobnone}

\oobbde{Third Brigade}{Col.}{Robert Scott} % {{{7
\oobrgt{6th}{Kentucky}{Col. Walter Whitaker}{600}{\oobnone}{\oobnone}
\oobrgt{10th}{Kentucky}{Col. John Harlan}{600}{\oobnone}{\oobnone}
\oobrgt{27th}{Ohio}{Col. John Fuller}{600}{\oobnone}{\oobnone}
\oobrgt{68th}{Ohio}{Col. John Snook}{600}{\oobnone}{\oobnone}
\oobtot{Total Third Brigade}{2,400}{\oobnone}{\oobnone}

\oobsum{Total Wood's Division}{7,200}{\oobnone}{\oobnone} % {{{7
\oobdblrule

\oobdiv{Division of Observation}{Col.}{John Hardin McHenry, Jr.} % {{{6
\oobbde{First Brigade}{Col.}{John Hardin McHenry, Jr.} % {{{7
\oobrgt{41st}{Ohio}{Lieut. Col. George Mygatt}{600}{\oobnone}{\oobnone}
\oobrgt{49th}{Ohio}{Lieut. Col. Albert Blackman}{600}{\oobnone}{\oobnone}
\oobrgt{40th}{Indiana}{Col. William Wilson}{600}{\oobnone}{\oobnone}
\oobrgt{58th}{Indiana}{Col. George Buell}{\textsuperscript{*}}{\textsuperscript{*}}{\textsuperscript{*}}
\oobrgt{17th}{Kentucky}{Col. Alexander Stout}{\textsuperscript{*}}{\textsuperscript{*}}{\textsuperscript{*}}
\oobtot{Total First Brigade}{1,800}{\oobnone}{\oobnone}
% TODO: This should only span the first two columns
\SetCell[c=5]{c}\footnotesize{\textsuperscript{*}
    The 58th Ind. and 17th Ky. were guarding the railroad between Crossing
    and McKenzie, Tenn. and not present at the engagement.
} \\

\oobbde{Third Brigade}{Col.}{James McMillan} % {{{7
\oobrgt{9th}{Kentucky}{Col. Benjamin Grider}{600}{\oobnone}{\oobnone}
\oobrgt{15th}{Kentucky}{Col. James Forman}{600}{\oobnone}{\oobnone}
\oobrgt{10th}{Michigan}{Col. Charles Lum}{600}{\oobnone}{\oobnone}
\oobrgt{13th}{Indiana}{Col. Jeremiah Sullivan}{600}{\oobnone}{\oobnone}
\oobrgt{21st}{Indiana}{Col. John Keith}{600}{\oobnone}{\oobnone}
\oobtot{Total Second Brigade}{3,000}{\oobnone}{\oobnone}

\oobsum{Total Division of Observation}{4,800}{\oobnone}{\oobnone} % {{{7
\oobdblrule

\oobdiv{Artillery}{}{} % {{{6

\oobart{Wood's Division Artillery} % {{{7
\oobrgt{8th}{Ohio Independent Battery}{Capt. Louis Markgraf}{6}{\oobnone}{\oobnone}
\oobrgt{2d}{Minnesota Light Artillery Battery}{Capt. William Hotchkiss}{6}{\oobnone}{\oobnone}
\oobrgt{7th}{Indiana Battery}{Capt. George Swallow}{6}{\oobnone}{\oobnone}
\oobtot{Total Wood's Division Artillery}{18}{\oobnone}{\oobnone}

\oobart{Division of Observation Artillery} % {{{7
\oobrgt{1st}{Michigan Light Artillery, Battery E}{Capt. John Dennis}{6}{\oobnone}{\oobnone}
\oobrgt{1st}{Ohio Light Artillery, Battery G}{Capt. Joseph Bartlett}{6}{\oobnone}{\oobnone}
\oobtot{Total Division of Observation Artillery}{12}{\oobnone}{\oobnone}

\oobsum{Total Artillery}{30}{\oobnone}{\oobnone} % {{{7
\oobdblrule

% Overall Totals % {{{6
\oobsum{Dep't of the Cumberland}{12,000}{600}{5.00}

\oobtop{Cavalry Corps}{Brig. Gen.}{John Hall} % {{{5

\oobdiv{First Division}{Col.}{John Wilder} % {{{6
\oobrgt{}{First Brigade}{}{600}{\oobnone}{\oobnone}
\oobrgt{}{Second Brigade}{}{600}{\oobnone}{\oobnone}
\oobrgt{}{Third Brigade}{}{600}{\oobnone}{\oobnone}
\oobsum{Total First Division}{1,800}{\oobnone}{\oobnone}
\oobdblrule

\oobdiv{Second Division}{Brig. Gen.}{Thomas J. Mosier} % {{{6
\oobrgt{}{First Brigade}{}{600}{\oobnone}{\oobnone}
\oobrgt{}{Second Brigade}{}{600}{\oobnone}{\oobnone}
\oobsum{Total Second Division}{1,200}{\oobnone}{\oobnone}

\oobdblrule

\oobdiv{Artillery}{}{} % {{{6

\oobart{First Division Artillery}
\oobtot{Total First Division Artillery}{12}{\oobnone}{\oobnone}

\oobart{Second Division Artillery}
\oobtot{Total Second Division Artillery}{12}{\oobnone}{\oobnone}

\oobsum{Total Artillery}{24}{\oobnone}{\oobnone}
\oobdblrule

\oobsum{Cavalry Corps}{3,000}{\oobnone}{\oobnone} % {{{6

% Overall Totals % {{{5
\oobrecap

\oobsub{Eleventh Corps}{10,200}{1,200}{11.76}
\oobsub{Seventeenth Corps}{10,200}{\oobnone}{\oobnone}
\oobsub{Dep't of the Cumberland}{12,000}{600}{5.00}

\oobdblrule
\oobsum{Grand total}{32,400}{1,800}{5.56}
\oobdblrule

\oobsub{First Cavalry Division}{1,800}{\oobnone}{\oobnone}
\oobsub{Second Cavalry Division}{1,200}{\oobnone}{\oobnone}

\oobdblrule
\oobsum{Grand total}{3,000}{\oobnone}{\oobnone}
\oobdblrule

\oobsub{Eleventh Corps Artillery}{36}{\oobnone}{\oobnone}
\oobsub{Seventeenth Corps Artillery}{36}{\oobnone}{\oobnone}
\oobsub{Dep't of the Cumberland Artillery}{30}{\oobnone}{\oobnone}
\oobsub{Cavalry Corps Artillery}{24}{\oobnone}{\oobnone}

\oobdblrule
\oobsum{Grand total}{126}{\oobnone}{\oobnone}

\bottomrule
\end{oob}

\subsecdinkus

\subsection{Reports of Maj. Gen. James Blake, Department of the Cumberland} % {{{3

\gramHeader{Headquarters, Army of the Cumberland} % {{{4
{Clarksville, Tenn., March}{24, 1862}
\gramTo{Maj. Gen.}{Cornelius Van Royne}
{Commanding General, United States Army}

\gramHi{Sir} This army, in cooperation with Gen'l Steele's has seized
Clarksville, Tenn. and invested the nearby enemy fort.

On the 18th ult., XIIth and XIVth Corps, screened by this army's cavalry,
crossed the Cumberland River and marched on Clarksville, where an enemy depot,
was found burned to the ground. These forces now occupy positions around the
town.

At the same time, VIIIth Corps advanced on the cavalry division of the Army of
Mississippi, positioned at Sailor's Rest, just east of Dover. The enemy was able
to slip through Gen'l Steele's positions although it is believed the enemy
suffered some loss to horses in order to escape the trap.

VIIIth Corps, along with the Army of the Kanawha, has now completely cut off the
enemy fort, located on the neck of land across the river from Clarksville. The
garrison is estimated at 12,000--20,000 men of Whisper's Army of Mississippi.
Our combined movements have trapped the garrison and leave it able to be
supplied only via riverboat from Nashville.

Gen'l Steele and I will discuss our options but I do not believe an assault on
the enemy works is at all advisable. Rather, I intend to keep the fort isolated
while maneuvering to render the fort useless and force the garrison to
surrender.

\gramClosing{Respectfully}
{J. W. Blake}
{Maj. Gen., Commanding}

Send copy to Steele.

\reportdinkus

\gramOrdersHeader{Headquarters, Department of the Cumberland} % {{{4
{Clarksville, Tenn., April}{1, 1862}
{Special Field Order}{1}

The goal of this Department for the coming months is the seizure of Nashville,
Tenn. and the opening of the railroad from Louisville, Ky. into central
Tennessee. Before this can be attempted, the enemy position at Fort Defiance
must be eliminated in order to open the Cumberland River for riverboat traffic.

A sizable enemy force is gathering in and around Nashville and it is the opinion
of this headquarters that the enemy intends to relieve the garrison of Fort
Defiance by an attack along either the north or south bank of the Cumberland
River. The southern route against Gen'l Steele poses the greatest risk as the
Army of the Kanawha could be trapped between an army marching from Nashville and
the garrison of the fort. The northern route is also dangerous as such a
maneuver could be supported by portions of the armies facing Gen'l Howard around
Munfordville.

Thus, the first priority of this Department is to defeat the expected enemy
attack. Once the enemy has been defeated, Fort Defiance can be isolated or taken
by assault, clearing the Cumberland River for an advance on Nashville. To wit:

I. Reinforcements for the armies of this Department are assembling in Cairo,
Ill. and Louisville, Ky. as stated in previous correspondence. These troops will
be sent to their respective field armies as soon as possible.  The Chief
Quartermaster, Maj. Mackay is to coordinate with the District of the Ohio to
immediately bring up four batteries of siege guns to Clarksville for the purpose
of bombarding the rebel works.  The pontoon bridges at Dover will immediately be
disassembled and sent to the Army of the Cumberland near Sugar Creek Landing.
Gen'l Howard is to release three batteries of artillery to the Army of the
Cumberland.

II. Maj. Mackay is directed to release sufficient supply [210 LP] to Gen'l
Howard for establishment of a supply line to support his army in the field and
enable an advance on Bowling Green, Ky. if the enemy in front of him should
withdraw. Maj. Mackay will establish river landings at Smithland, Ky., Dover,
Tenn., Sailor's Rest, Tenn. and Sugar Creek Landing, Tenn. A wagon depot will
be established at Cloverdale [220 LP].

III. As soon as practicable, Gen'l Steele is to advance against the enemy
cavalry to his front and establish positions near Chestnut Grove to identify and
delay any movement against his army from Nashville. Gen'l Fawcett will likewise
establish positions to do the same along the north bank of the river.

IV. Gen'l Howard is to hold positions along the Green River so that the enemy is
unable to withdraw significant forces from around Bowling Green to aid in the
relief of Fort Defiance. If the enemy should do so, and if Gen'l Howard believes
it prudent, an advance on Bowling Green will be conducted.

V. The Division of Observation will establish garrisons at Smithland, Ky.,
Frankfort, Ky., Dover, Tenn. and Crossing, Tenn. Wood's Division will move to
Camp Carter as the reserve for the armies around Fort Defiance. Kellogg's
Cavalry Division will position one brigade at Frankfort, Ky. and the remainder
around Crossing, Tenn. to identify enemy forces attempting to cross the
Tennessee River.

VI. Cdre. Lewis' Cumberland Squadron will be requested to maintain patrols
to identify and prevent enemy crossings of the Tennessee and Cumberland Rivers.

\gramClosingBy{Maj. Gen. Blake}
{Walter Chekov}
{Col., Adjutant General}
\reportdinkus

\gramHeader{Headquarters, Army of the Cumberland} % {{{4
{Clarksville, Tenn., April}{11, 1862}
\gramTo{Maj. Gen.}{Richard Steele}
{Commanding, Army of the Kanawha}

XXX: Undelivered

\gramHi{General} Your report is well received. By the time this message reaches
you, the 7,200 men of Benton's XVIth Corps should be well on its way to your
assistance. The artillery will lag behind as crossing them on pontoon boats will
be quite slow.

I have also ordered Gen'l Kellogg to move his division of cavalry from Crossing
to White Oak. If you believe it prudent for him to reposition, you may order him
to do so although I am loath to commit our only cavalry reserve to combat unless
absolutely necessary.

I approve of your orders to VIIIth Corps but ask that as soon as you feel you
have enough force to contain Gen'l Jackson's attack that VIIIth Corps be
returned to their positions. An attack on your rear by Gen'l Whisper could prove
disastrous. Gen'l Fawcett is most certainly prepared, however, to take advantage
of such a move by immediately moving across the bridge against the rebel
position.

\gramClosing{Respectfully}
{James. W. Blake}
{Maj. Gen., Commanding}

Copy to Kellogg

\reportdinkus

\gramHeader{Headquarters, Army of the Cumberland} % {{{4
{Clarksville, Tenn., Midnight, April}{11, 1862}
\gramTo{Maj. Gen.}{Richard Steele}
{Commanding, Army of the Kanawha}

XXX: Undelivered

\gramHi{General} Your report is received. I agree with your inclination to
withdraw. Preserving the army is paramount. Gen'l Fawcett is presently near
Sailor's Rest and can support you with some troops. Kellogg's cavalry is at
White Oak and Wood's Division is at Dover.

You are ordered to withdraw towards Dover in whatever manner you deem fitting.
Do your best to extricate VIIIth Corps from its positions facing the fort but if
they must be sacrificed as a rearguard, so be it.

I intend to establish a line around Dover and hold the enemy there. The brigade
at the bridge at Crossing will hold as long as possible but if that position
cannot be held, the bridge will be burned.

The remainder of the army here at Clarksville will fire the bridge here and then
withdraw as well, crossing the river at Dover.

Your army fought bravely today and you must preserve it to do so again.

\gramClosing{For the Union}
{J. W. Blake}
Commanding.

Copy to Fawcett

\reportdinkus

\gramHeader{Headquarters, Army of the Cumberland} % {{{4
{Clarksville, Tenn., Midnight, April}{11, 1862}
\gramTo{Maj. Gen.}{Harold Fawcett, III}
{Commanding, Army of the Cumberland}

\gramHi{General} I have ordered Gen'l Steele to withdraw towards Dover and am
issuing orders to your forces here to do the same after firing the bridge to
Clarksville.

I am uncertain if my messages have been reaching Gen'l Steele so be prepared to
issue appropriate orders to him in my stead if necessary.

I am leaving immediately to join you at Sailor's Rest.

\gramClosing{Respectfully}
{J. W. Blake}
{Commanding}
\reportdinkus

\gramOrdersHeader{Headquarters, Army of the Cumberland} % {{{4
{Sailor's Rest Depot, Tenn., 3 am, April}{12, 1862}
{Special Field Orders}{2}

Due to the precarious position of the Army of the Kanawha, the priority of this
Department is to effect the withdrawal of said army from its positions near Fort
Defiance and the establishment of new defensive positions around Dover and
Crossing.

I. Gen'l Steele will immediately withdraw the Army of the Kanawha along with
VIIIth and XVIth Corps of the Army of the Cumberland towards Dover. Gen'l
Fawcett is to order the pontoon bridge to be established at Hematite, Tenn. to
facilitate the withdrawal if necessary. However, Gen'l Steele is at his own
discretion as to the direction and method of withdrawal.

II. Gen'l Fawcett is to immediately destroy the bridge at Clarksville and
withdraw the remainder of the Army of the Cumberland, less Sherman's Division to
Dover and begin establishing defensive positions there. Gen'l Wood  and Col.
McHenry, Jr.  will report to him until this headquarters is established at
Dover. Gen'l Fawcett is authorized to conduct any operations he deems prudent to
support these orders.

III. Gen'l Kellogg will aggressively probe Charlotte and Cloverdale and destroy
any depots discovered there. Once complete, his division is to center on Yellow
Creek and harass the enemy facing Steele in order to disrupt their operations.

IV. Sherman's Division  will secure the depot against attacks by cavalry and
light forces. Gen'l Sherman will destroy the depot and withdraw towards Dover as
soon as it is clear the enemy is advancing on him in force.  Once so ordered,
this headquarters will immediately relocate to Dover.

\gramClosingBy{Gen'l Blake}
{Walter Chekov}
{Col., Adjutant General}
\reportdinkus

\gramHeader{Headquarters, Dep't of the Cumberland} % {{{4
{In the field, April}{12, 1862}
\gramTo{Brig. Gen.}{William Sherman}
{Commanding, 1st Division, XIIth Corps}

\gramHi{General} You are ordered to take temporary command of Wool's Division
and Kellogg's cavalry and provide the rearguard for the withdrawal of the Army
of Kanawha and attached forces to the Dover line.

You will hold this position only if lightly pressed and only if in no danger of
being cut off from your line of retreat; you will immediately withdraw as far as
necessary if the enemy comes upon you in force or maneuvers  to cut you off from
Dover.

\gramClosingBy{Gen'l Blake}
{Walter Chekov}
{Col., Adjutant General}

Copy to Kellogg.

\reportdinkus

\gramHeader{Headquarters, Dep't of the Cumberland} % {{{4
{Camp Carter, Tenn., April}{14, 1862}
\gramTo{Maj. Gen.}{Cornelius Van Royne}
{Commanding United States Army}

\gramHi{Sir} A full accounting of the battle at Barton's Creek and the
subsequent report must wait until Gen'l Steele has attended to his army and
Gen'l Fawcett has rejoined this headquarters at Dover. However, a brief summary
of the key particulars follows which may, at least somewhat, assuage any
concerns as to the conduct of Gen'l Steele and the men under his command.

All reinforcements to this Dep't. having been received by the 8th inst., Gen'l
Steele stepped off as planned to establish forward positions defending the
southern approach to Fort Defiance. On the 9th inst., the two cavalry divisions
of the Army of the Cumberland were engaged around Fredonia, Tenn., east of
Clarksville by an equivalent number of enemy troopers of the Army of Tennessee.
Gen'l Steele's two cavalry divisions were likewise engaged by roughly four
divisions of enemy horse the next day, the 10th. As it was yet uncertain where
the enemy would attack. Benton's Corps of the Army of the Cumberland was in camp
at Sugar Creek on the north bank of the Cumberland, ready to move to assist
either army. At the same time, Kellogg's cavalry of the Dep't reserve was sent
forward to White Oak, Tenn. to identify any movement by the enemy directly
towards Dover, around Gen'l Steele.

Early in the morning of the 11th inst., an urgent message was received from Gen'
Steele, dated the previous evening, that the enemy had been spotted in strength
and that reinforcements were urgently needed as he expected his five divisions
were outnumbered three to one. Benton was immediately dispatched south and
Crittenden's Division of VIIIth Corps was pulled away from the positions south
of Fort Defiance to aid the Army of the Kanawha. Once arriving on the field this
gave Gen'l Steele a force of eight divisions supported by two of cavalry. In
Gen'l Steele's estimation this reduced the enemy's superiority to a two to one
advantage. However, these forces did not arrive at the same time and were
committed piecemeal Gen'l Fawcett immediately moved an additional division to
the depot Sailor's Rest as an additional reserve. Gen'l Steele resolved to hold
his position on good ground to prevent the relief of Fort Defiance.

In the event, Gen'l Jackson came at Gen'l Steele on the 11th inst. with a force
firmly identified as five corps totaling 15 divisions of infantry and a cavalry
corps of four divisions. Outnumbered two to one and with Gen'l Whisper's force
threatening to break out and attack its rear, the Army of the Kanawha fought
better than can be expected, making the enemy pay dearly for driving our forces
back. However, Gen'l Steele reported that he was in danger of being cut off from
his line of retreat by enemy cavalry that had ranged as far as Palmyra. The
order was given and during the night of the 11th to 12th inst., the forces that
could withdraw west did so.

Our total losses were 25 infantry regiments, eight cavalry regiments and seven
artillery batteries from the Army of the Kanawha and an additional eight
infantry regiments and two artillery batteries from the Army of the Cumberland.

Gen'l Fawcett is still marching west to Dover with the balance of his army,
having burned the rail bridge at Clarksville. Two divisions of infantry and
Kellogg's cavalry are posted around Sailor's Rest to lightly screen against any
advance towards Dover. McHenry, Jr's brigade still holds the rail bridge across
the Tennessee and will destroy that bridge if forced to abandon it. The
remainder of the forces here in the west are now at Camp Carter.

In summation, the enemy came at Gen'l Steele with a force greater than the
combined strength of his and Gen'l Fawcett's armies. This does not include Gen'l
Whisper's Army of Mississippi which, as far as is known, was unengaged at
Barton's Creek. The enemy's cavalry superiority, however, was most critical as
this headquarters did not have the time expected to react to the enemy advance.
The rebel's continual advantage in cavalry cannot continue as it leaves our
armies unable to gain timely and accurate intelligence on enemy forces.

In addition, unless the enemy is able to raise far more manpower than we, this
leads me to believe that he has stripped troops from multiple areas of the
Confederacy, leaving us free to advance in other areas. The numerical
superiority that is being brought against me, and the enemy's consistent
superiority in cavalry makes any offensive taken in this area quite difficult.
However, once Gen'l Fawcett arrives this headquarters will develop new plans to
take the fight to the enemy although it seems the direct route to Nashville from
Dover may be impossible.

\gramClosing{Respectfully}
{J. W. Blake}
{Maj. Gen., Commanding}
\reportdinkus

\gramHeader{Headquarters, Dep't of the Cumberland} % {{{4
{Camp Carter, Tenn., April}{21, 1862}
\gramTo{Maj. Gen.}{Cornelius Van Royne}
{Commanding United States Army}

\gramHi{Sir} In response to your last I am forwarding Gen'l Steele's report on
friendly and enemy strength and losses at Barton's Creek. In addition to his
report I am able to report that the cavalry arm of the Army of Tennessee,
consists of roughly six divisions.

Gen'l Graham's Army of the Cumberland cavalry defeated two divisions of such,
commanded by Gen'l Forrest, near Fredonia in two separate engagements on the 9th
and 13th inst. The remaining four divisions faced Gen'l Steele at Barton's
Creek.

I have moved Gen'l Kellogg's cavalry from the Dep't reserve and placed him in
the field, but this still leaves the enemy with a superiority of one division of
cavalry in the area between the rivers.

\gramClosing{Respectfully}
{J. W. Blake}
{Maj. Gen., Commanding}
\reportdinkus

\gramHeader{Headquarters, Dep't of the Cumberland} % {{{4
{Woods Valley, 7.00 am, May}{2, 1862}
\gramTo{Maj. Gen.}{Richard Steele}
{Commanding Army of the Kanawha}

\gramHi{General} Gen'l Fawcett appears to be making progress with VIII and XII
Corps reporting the enemy lines beginning to falter. He is sending in the second
wave now. His cavalry is probing along the enemy flanks, attempting to cut off
lines of retreat.

I have reports that your forces are advancing---if so, continue; if not, begin
advancing and drive the enemy back.

\gramClosing{Respectfully}
{J. W. Blake}
{Maj. Gen., Commanding}
\reportdinkus

\gramHeader{Headquarters, Dep't of the Cumberland} % {{{4
{Woods Valley, 7.45 am, May}{2, 1862}
\gramTo{Maj. Gen.}{Richard Steele}
{Commanding Army of the Kanawha}

\gramHi{General} Gen'l Fawcett is wounded and I am taking direct command of his
army. The two second wave corps are still largely fresh and continuing the
attack. With your assault and Graham's cavalry harassing the enemy left I
believe we may break the position shortly.

Recommend your cavalry strikes the enemy right or otherwise continues into his
rear.

\gramClosing{Respectfully}
{J. W. Blake}
{Maj. Gen., Commanding}
\reportdinkus

\gramHeader{Headquarters, Dep't of the Cumberland} % {{{4
{Near Cloverdale, Tenn., 7.45 am, May}{2, 1862}
\gramTo{Brig. Gen.}{Graham}
{Commanding, Army of the Cumberland Cavalry Corps}

\gramHi{General} If possible, aim to cut the road from Cloverdale to Charlotte
as that is the enemy's main line of retreat.

\gramClosing{Respectfully}
{J. W. Blake}
{Maj. Gen., Commanding}
\reportdinkus

\gramHeader{Headquarters, Dep't of the Cumberland} % {{{4
{Cloverdale, Tenn., May}{12, 1862}
\gramTo{Maj. Gen.}{Cornelius Van Royne}
{Commanding General, United States Army}

\gramHi{Sir} I have the honor to submit to you the report of the overall actions
of the forces of this Department as well as the specifics relating to the Army
of the Cumberland during the engagement at Cloverdale, Tenn. on the 2d inst.

On the 28th of April, this headquarters issued orders to the Armies of the
Cumberland and Kanawha to move against the enemy centered on Cloverdale,
intending to secure the crossroads there to prevent the enemy from interfering
with the siege and destruction of Fort Defiance. During the previous week, while
the bulk of the two armies remained quartered at Camp Carter, Maj.  Gen.
Graham's cavalry, augmented by Brig. Gen. Kellogg's division had moved forward
while the enemy had withdrawn to Cloverdale, allowing our cavalry to occupy
Carbondale, Maysville and a position across Yellow Creek on the road to
Cloverdale. Brig. Gen'ls Sherman and Wool had remained at Sailor's Rest while
Brig. Gen. Wood's Division had advanced to Mill Spring to cover the approach to
Dover from that direction.

With the entirety of the Army of the Cumberland now at Dover, Gen'l Fawcett was
to march east to Cloverdale, approaching in two wings from Yellow Creek and
Maysville.  C. Smith's and Garfield's Corps composed the northern wing while P.
Smith's and McClernand's Corps composed the southern.  Both wings would be
preceded by their detached divisions at Sailor's Rest and Kellogg's cavalry in
the north and Bussey's in the south. Brig. Gen. Dickey's cavalry division was to
remain in the rear as a reserve.

At the same time, Gen'l Steele's Army of the Kanawha would move east towards
McAllister's and, after leaving sufficient force to contain the garrison of Fort
Defiance would turn south towards Cloverdale, attempting to catch the enemy in
his right flank while he was decisively engaged with Gen'l Fawcett.

Wood's Division was to advance from Mill Spring to White Oak to cover Gen'l
Fawcett's right flank.

I moved with Gen'l Fawcett so as to quickly gain observation of the enemy
positions and be able to adjust the plan of battle if necessary before Gen'l
Steele became engaged. This decision proved fortuitous as I was able to quickly
exercise command over the Army of the Cumberland once Gen'l Fawcett was wounded.

After a three day march from Dover, the Army of the Cumberland (henceforth
referred to as ``this army''), crossed the Yellow Creek, approaching the enemy
lines in front of Cloverdale by the evening of the 1st. Brig. Gen. Kellogg's
cavalry had destroyed a rebel supply depot on the south bank of the Cumberland
River, across from Fredonia, Tenn. and reported that Gen'l Steele's army was
close behind, approaching McAllister's. Gen'l Kellogg and Maj. Gen.  Graham both
reported the enemy was positioned behind Little Barton's Creek, defending
Cloverdale.

Gen'l Fawcett gave orders to begin the attack in the early dawn with C. Smith's
XIIth and P. Smith's VIIth Corps, from left to right, leading with Garfield's
XVIth and McClernand's XIVth Corps, from left to right, in reserve. Kellogg's
Cavalry was on the left with Dickey and Bussey on the right. A courier was sent
to Gen'l Steele requesting that he begin his attack between 10 o'clock and noon
the next day to which he responded in the affirmative.

The army broke camp as the sun rose and was advancing towards the enemy when,
less than 20 minutes later, they were engaged by Rebel infantry advancing
towards them. It appeared the enemy intended to drive off our assault by an
attack of his own.

Gen'l Fawcett gave orders for the leading corps to shift to the defensive,
bringing up his artillery to stiffen the line. The cavalry was ordered to feel
around the enemy flanks.

By 7 o'clock Brig. Gen. Garfield, whose corps remained in reserve, had reported
observing men of the Army of the Kanawha, on his left, advancing against the
enemy while Brig. Gen'ls C. \& P. Smith reported observing several enemy
brigades rout from the field, allowing their corps, though battered, to steadily
advance. By this point the army had advanced to Big Barton's Creek and onto the
foot of Butcher Ridge.

I sent a rider to Gen'l Steele advising him to begin a general advance if he had
not already done so. Gen'l Fawcett ordered his reserve corps committed to the
battle and his cavalry to make their probes around the flanks more aggressive
while he moved forward to direct his men personally. I regret that I did not
order him to remain in safety towards the rear as his loss has been keenly felt.

By 7.45~am fighting in the center had died down somewhat as the Confederates
continued in the face of two fresh corps, withdrew behind Furnace Creek, ceding
the town of Cloverdale. It was at about this time that the report of Gen'l
Fawcett's wounding reached me and I took direct command of his army.

With the Army of the Kanawha now fully committed, the enemy right finally
collapsed and, for the next three hours, both our armies continued to drive the
enemy back while our cavalry endeavored to trap and defeat any routing units. 

The enemy was able to withdraw to new lines in front of Charlotte as our
infantry was unable to maintain contact with them during the retreat.

Our total losses numbered 600 infantry from Gen'l Steele's army, largely from
his XVIIth Corps, and 6,600, 900 cavalry troopers and 6 six guns from the Army
of the Cumberland.  Among those losses are Maj. Gen. Harold K. Fawcett, III, as
previously mentioned, Brig. Gen. Jeremiah Boyle and Col's Marcellus Mundy,
Robert Sinclair, Thomas Kilby Smith, George Wagner and Oliver Wood.

Based on captured standards and reports from engaged units, I can confirm that
we engaged the enemy's Army of Tennessee, commanded by Gen'l Jackson, composed of
three wings under Bragg, Breckinridge and Harrison with a total of six divisions
and an additional two divisions of cavalry. The enemy certainly sustained
casualties in excess of our own and over 1,800 enemy were captured by our
cavalry during the retreat to Charlotte in addition to 54 enemy cannon retrieved
from the field.

Now that the enemy has voluntarily withdrawn from Charlotte as well, I believe
that his combined losses from Barton's Creek and Cloverdale as well as the
appearance of Gen'l Fawcett's fresh army convinced Gen'l Jackson that the best
defense of Nashville, and likely the only one feasible, is to withdraw tight
around the bastion there and attempt to delay its capture as long as possible.

\gramClosing{Respectfully}
{J. W. Blake}
{Maj. Gen., Commanding}

\subsecdinkus

\subsection{Reports of Maj. Gen. Harold Fawcett, III, Army of the Cumberland} % {{{3

\gramOrdersHeader{Headquarters, Army of the Cumberland} % {{{4
{Clarksville, Tenn., April}{1, 1862}
{Special Field Order}{XX}

I. XIIth and XIVth Corps are to prepare fieldworks along the Red River and on
the east bank of the Cumberland River between Red River and the Cumberland.  One
brigade of XIVth Corps will be dedicated to manning fieldworks at the rail
bridge at Clarksville. Each Corps is to keep half its strength on the line and
half in reserve at all times.

II. The cavalry corps is to be organized into six brigades of four regiments
each. First Division will establish a screen between Hampton's, Port Royal and
Thomasville: 1st Brigade will strike at and tear up the rail junction at Line,
Tenn., and 3d Brigade will move to and fire the ferry at Williams, Tenn. in
small sallies beyond the normal depot range. Second Division will mass north of
Clarksville to commit against a substantial enemy incursion, to ablate and delay
a main body or work to eliminate a cavalry force.

III. VIIIth Corps will prepare field lines to trap and enclose Fort Defiance.
Half the corps will man the works while the other half remains in reserve.

IV. XVIth Corps is to assemble at the Sugar Creek Landing, ready to move
northeast to join the army's main body or to cross the Cumberland to the south
to join VIIIth Corps and the Army of the Kanawha depending on which way the
enemy comes from Nashville.

V. Once received, siege mortars will be deployed at Clarksville and commence
shelling all Rebel concentrations at Fort Defiance, day on day. Focus on any
troops concentrations, supply or artillery positions---the goal is to degrade and
attrition the enemy rather than to annihilate them.

VI. If the fort does not surrender within a month, and the enemy has not come at our
cavalry lines, these orders will be reviewed and amended as necessary.

\gramClosing{}
{Harold Fawcett, III}
{Maj. Gen., Commanding}
\reportdinkus

\gramHeader{Headquarters, Army of the Cumberland} % {{{4
{Clarksville, Tenn., 2.00 pm, April}{11, 1862}
\gramTo{Brig. Gen.}{Zebulon Benton}
{Commanding, XVIth Corps}

\gramHi{General} As discussed in your brief for the campaign, the advance of
substantive enemy infantry forces against the Army of the Kanawha places your
most urgent duty as crossing the river with those prepared vessels and then
marching to the close support of General Steele and the Army of the Kanawha.

I trust this message will find you marching to the sound of the guns, if you are
not already engaging.

While you remain a part of the Army of the Cumberland, you should consider
yourself under General Steele's tactical command for the duration of the battle.
Once the enemy pulls away, you should support VIIIth Corps in place to free
General Steele for maneuvers.

\gramClosing{Rspy yrs}
{Harold Fawcett, III}
{Maj. Gen., Commanding}

Copies to Gen'ls. Blake \& Steele

\reportdinkus

\gramHeader{Headquarters, Army of the Cumberland} % {{{4
{Clarksville, Tenn., 2.00 pm, April}{11, 1862}
\gramTo{Maj. Gen.}{Ptolemy Smith}
{Commanding, VIIIth Corps}

\gramHi{General} I understand the Army of the Kanawha is closely engaged to your
south. While you may send scouting parties and staffers to plan a movement to
support General Steele, you must maintain your reserve division with your main
body against a breakout attempt by General Whisper.

We will send an additional division to General Steele's support, although it
likely will not arrive until tomorrow morning.

\gramClosing{I have the honour to remain}
{Harold Fawcett, III}
{Maj. Gen., Commanding}

Copies to Gen'ls. Blake \& Steele

\reportdinkus

\subsection{Report of Brig. Gen. Lawrence Graham, Cavalry Corps} % {{{3

\gramHeader{Headquarters, Cavalry Corps, Army of the Cumberland} % {{{4
{Near Fredonia, Ten., April}{10, 1862}
\gramTo{Maj. Gen.}{Harold K. Fawcett III}
{Commander, Army of the Cumberland}

\gramHi{General} I have the honor to report the particulars of a cavalry
engagement which occurred near the village of Fredonia, Tennessee, on the 9th
instant, involving the 1st Cavalry Division of this Corps and a superior force
of the enemy's horse belonging to the Army of Tennessee.

At about eight o'clock in the forenoon, while the 1st Division was holding the
approaches south and west of Fredonia for scouting and screening purposes, the
pickets were briskly driven in by a strong body of Confederate cavalry advancing
in force. The enemy's strength was variously estimated by officers on the field
at from five to six thousand men, moving with evident design to overwhelm our
division before support could arrive.

The 1st Division, commanded by Brig. Gen. Theophilus Dickey, promptly formed
line upon favorable ground and received the attack with steadiness and
resolution. For more than an hour the contest was sharp and closely pressed. The
enemy made repeated efforts to turn both flanks and press our center, but were
met at every point with determined resistance.  Though heavily outnumbered, our
troopers held their ground firmly, delivering their fire with coolness and
retiring only when compelled by the weight of the attack, and then in good
order.

At this critical juncture, the 2d Cavalry Division, under Brig. Gen.  Cyrus
Bussey, came rapidly upon the field from the north, moving with great gallantry
and precision into supporting position. Upon the appearance of this fresh force
the enemy's fire sensibly slackened. A brief advance by our reinforced line was
sufficient to convince the Confederate commander that further effort would be
fruitless. After a short skirmish with General Bussey's leading brigade, the
enemy withdrew from the field and called off the attack, retiring toward the
south in good order.

Our loss in the action is estimated at about sixty killed and wounded. From the
best information obtainable, and from reports of citizens and scouts following
the engagement, the enemy's loss is believed to be not less than one hundred and
fifty nor more than two hundred killed and wounded.

It is proper to state that no enemy infantry was observed upon the field at any
point, nor have our scouts, sent well out upon all surrounding roads, been able
to discover any supporting foot troops in the vicinity. The movement appears to
have been a purely cavalry enterprise.

The cavalry pickets have been re-established upon advanced lines, and the
country in the direction of the enemy's retreat is now being actively
reconnoitered.

\gramClosing{I am, General, very respectfully, Your obedient servant}
{Lawrence Graham}
{Brigadier General, Commanding Cavalry Corps, Army of the Cumberland}
\reportdinkus

\subsection{Report of Brig. Gen. Zebulon Benton, XVIth Corps} % {{{3

\gramHeader{War Department} % {{{4
{Washington City, April}{14, 1862}
\gramTo{Maj. Gen.}{C. Van Royne}
{U.S. Army, Commanding}

\gramHi{Sir} I have the honor of providing this accounting of the number and
disposition of rebel troops encountered by my headquarters during the Battle of
Barton's Creek, primarily occurring during the afternoon and evening of the 11th
inst.

As has been communicated in prior reports, on the day I was in command of XVIth
Corps, Army of the Cumberland.  The Corps commenced the day in reserve, on the
north bank of the Cumberland River opposite Palmyra. At half-past 5~o’clock in
the morning, I received urgent notice from General Blake that the enemy had been
sighted in force south of the positions occupied by Maj. Gen. Steele’s Army of
the Kanawha, then engaged in besieging the rebel forces opposite Clarksville in
Fort Defiance.  Upon receipt of this news, I gave orders that the Corps should
be roused to undertake an immediate crossing of the Cumberland River, and a
march to General Steele's relief via Cunningham, Tenn.

My infantry was able to effect a crossing in pontoon barges by brigade, and the
approximately eleven-mile march to Cunningham completed with such rapidity that
I and my lead division under Brig. Gen. Oglesby reached the rear of Maj. Gen.
Steele's line at approximately noon.  Due to the urgency of the moment I left
my artillery behind, with instructions to cross and follow after the infantry as
possible.

I and Oglesby's division arrived at the right-center of General Steele's line,
on wooded heights south of the Louise creek, and north of the Mount Zion road
junction. Oglesby's division, with all three brigades suffering heavily from
straggling and disorganization thanks to the urgency and speed of the morning's
march, took up position in support of the 1st and 2d divisions of XIth Corps.
Prior to the arrival of Oglesby's division, General Steele's battle line
consisted of five brigades in battle line, with no local reserves.

From General Oglesby's position at noon, I was able to confirm the presence of
no fewer than ten  rebel infantry brigades, organized into three divisions with
a corps headquarters present, occupying the self-same frontage that the eight
combined brigades of First and Second Divisions of XIth Corps and Second
Division of XVIth Corps had in place.

The rebel division aiming straight north up the road from Mount Zion appeared to
be in a column of brigades, while the two divisions on either side had shaken
out with two brigades to the front, and their third brigade in reserve.
Additionally, the rebels had at least three artillery batteries in action,
firing on the two batteries posted with XIth Corps' divisions. I note that these
may have been of two separate corps---tentative identification was made of
``Wesley's Wing'' on the road from Mt. Zion, and a separate ``wing'' to the
east, though no solid name was ever associated with it.

Unfortunately, due to the woods and terrain, I was unable to ascertain what
formations were either to the left or right of this rebel corps (though there
was evidence that troops in battle line were present on both flanks), or if
there was more depth to the enemy position to their rear.

After these initial observations I rode back to Brig. Gen. Garfield's 1st
Division, and per Maj. Gen. Steele's instructions, conducted Garfield's brigades
to the western flank, into a gap between Second and Third Divisions, XIth Corps
which had opened up as Rebel cavalry sought to flank our position and cut us off
from the river to the west.

I reached that position at approximately 2.30~pm, as a serious infantry assault
was being made on First and Second Divisions of XIth Corps to the east. From
that position I was able to observe five brigades of rebel cavalry spread in
an arc from north to south, facing three brigades of our own cavalry at
Oakridge, soon reinforced with the third brigade of Garfield's division. I had
under my command two brigades of Garfield's division, with the three
brigades of Third Division, XIth Corps to my west.

At this point I saw a rebel infantry brigade, of ``Robertson's Wing,'' or
possibly ``Harrison's Wing'' maneuvering to my left front, with its western
flank uncovered, and no ready supports, though I was not able to see its
east-rear or eastern flank. Based on the noise at issue, I believe it to be an
entirely separate corps in front of Second Division, XIth Corps to our east.
Seeing potentially an opportunity to disorder the attack being made against XIth
Corps, and make a nuisance of myself out of proportion to my numbers, I ordered
Garfield's division over to the attack against this flank, and led the assault
personally.  We met with some initial success, but significant numbers of
reinforcing troops were directed to our front and the assault ground to a halt
under a most galling fire. It was at this point I was wounded and removed from
the field.

I was not able, in the din of battle, to obtain further information on any
further enemy units which may have fought in this action; perhaps General
Garfield will be able to provide further information.

This is all the information I am able to provide personally, and I solemnly
attest and swear that it is true, so help me God.

\gramClosing{Yr. Obt. Svt.}
{Zebulon H. Benton}
{Brig. Gen., U.S. Volunteers}

\subsection{Reports of Maj. Gen. Richard Steele, Army of the Kanawha} % {{{3

\gramOrdersHeader{Headquarters, Army of the Kanawha} % {{{4
{Carbondale, Tenn., April}{1, 1862}
{Special Field Order}{XX}

It is the intention of the Army of the Kanawha to drive back the enemy cavalry
screen and establish a forward defensive line at Jones Creek, with our own
cavalry projected out toward Chestnut Grove. This will put us into a position to
detect and hinder any enemy movement from Nashville to relieve the garrison of
Fort Defiance.

\textbf{Opening Shaping Operation}

The cavalry, backed by both corps of infantry, will push forward against the
enemy cavalry. Direct confrontation is preferably avoided, using turning
movements to compel retreat. If the enemy stands fast we will compel them with
force.  The army will converge on Cloverdale where a wagon depot capable of
sustaining the entire army will be established.

\textbf{Second Stage of Operation---Cavalry}

I. The cavalry will actively scout for enemy activity and alert this
headquarters of an advance by enemy forces from Nashville and then fall back
upon friendly lines if compelled while hindering enemy forward movement.

II. First Division is to proceed southeast through Jones' Cross Road and occupy
the intersection north of Chestnut Grove.

III. Second Division will proceed south through Charlotte, then cross Turnbull
Creek and occupy the intersection west of Chestnut Grove.

\textbf{Third Stage of Operation---Infantry}

I. The infantry will form a defensive line behind Jones' Creek to interdict any
enemy attempts to relieve the garrison of Fort Defiance. Once behind the creek,
defensive lines will be prepared. We will attempt to hold the enemy here unless
overwhelming force compels us to fall back toward Cloverdale.

II. XIth Corps will march south through Charlotte and set up a defensive line
behind Jones Creek. Third Division will serve as the army reserve on the road
midway between Charlotte and Jones Cross Road.

III. XVIIth Corps marches southeast through Jones Cross Road and set up a
defensive line behind Jones Creek.

\gramClosing{}
{Richard Steele}
{Maj. Gen., Commanding}
\reportdinkus

\gramHeader{Headquarters, Army of the Kanawha} % {{{4
{In the field, Noon, April}{10, 1862}
\gramTo{Maj. Gen.}{James Blake}
{Commanding, Dep't of the Cumberland}

\gramHi{Sir} I must report to you about developments on our front.

We finally stepped off on April 8th but the cavalry reported seeing no enemy in
our path. On the 9th we briefly encountered Confederate Cavalry from Army of the
West near Cloverdale and drove them off with a loss of 40 men and inflicting 100
to 140 upon the enemy.

This morning our cavalry was pushed back by the arrival of heavily reinforced
Rebel cavalry. We are still attempting to determine the size of the enemy force
but we are currently behind Louise Creek, northwest of McCallister's. More
information will be delivered as it becomes available.

\gramClosing{Respectfully}
{Steele}
{Maj. Gen.}
\reportdinkus

\gramHeader{Headquarters, Army of the Kanawha} % {{{4
{In the field, April}{10, 1862}
\gramTo{Maj. Gen.}{James Blake}
{Commanding, Dep't of the Cumberland}

My cavalry reports they are outnumbered by a ratio of two to one and were unable
to hold the enemy back. Unfortunately, we were unable to reach more than halfway
to Cloverdale and must attempt to hold the enemy here. Whatever reinforcements
can be spared are urgently needed.

I do not believe we can afford to give too much ground before our own supply
line will be endangered, or before we fall back to close to the siege lines
around Ft. Defiance.

I am actively seeking the most advantageous ground to seize before the battle,
hoping it will give the enemy pause, and deny them the initiative to attack me
in a manner of their choosing. Based on the reports, I believe I am outnumbered
so it is unlikely I will sweep the enemy from the field. But I do not intend to
be easily pushed aside either.

I am enclosing a copy of my orders for your records.

\gramClosing{Respectfully}
{Steele}
{Maj. Gen.}

Copy to General Fawcett, Army of the Cumberland

\reportdinkus

\gramHeader{Headquarters, Army of the Kanawha} % {{{4
{In the field, April}{10, 1862}
\gramTo{Maj. Gen.}{James Blake}
{Commanding, Dep't of the Cumberland}

To prevent the relief of the enemy force at Ft. Defiance, this army must attempt
to make a stand and buy time.

The Cavalry will hold their ground until relieved by the infantry, then they
will fall back to redeploy to the flanks.

XIth Corps will advance south to secure, or otherwise deny usage by the enemy,
of the Marion--Mt. Zion road. Care must be taken to find good defensive ground
with superior elevation and clear lines of sight.

XVIIth Corps will advance south to secure the east-west road running through
McAllister's. Care must be taken to find good defensive ground with
superior elevation and clear lines of sight. This line should be in advance of
the road, allowing for redeployment from one flank to the other.

First Cavalry Division, once relieved, will redeploy to secure our western flank
at with one brigade at Shiloh and the rest in reserve at the intersection of the
Shiloh-Marion road at Baggett-Bryant Branch

Second Cavalry Division, once relieved, will redeploy to our eastern flank behind
Louise Creek southeast of Southside. One brigade will overwatch where the road
crosses the creek, the other brigades will wait in reserve a short distance
north.

\gramClosing{Respectfully}
{Steele}
{Maj. Gen.}
\reportdinkus

\gramHeader{Headquarters, Army of the Kanawha} % {{{4
{In the field, 8.00 am, April}{10, 1862}
\gramTo{Maj. Gen.}{James Blake}
{Commanding, Dep't of the Cumberland}

\gramHi{General} We have engaged the enemy at McAllister's. General
Ward's cavalry fought bravely and held the high ground against enemy cavalry
attacks around the crossroads long enough for XVIIth Corps to relieve them and
hold the position. XIth Corps is on my right flank northwest of Mount Zion. The
enemy infantry is gathering but not yet advanced.

At this point in time I can report the presence of the enemy Army of the
Tennessee, under the command of General Jackson. He has at least two corps of
infantry currently along the ridge south of Barton's Creek, and his corps of
cavalry. It is possible Jackson has more men out of sight, but that is only
conjecture.

We hold good ground with strong lines of sight. The cavalry is protecting our
flanks. I feel reasonably confident we have done what we can to receive the
enemy in the most advantageous position we can.

You should know that I have also asked VIIIth Corps from Army of the Cumberland
to send their reserve division our way in case we need them. I have requested
that they position themselves at Cunningham, several miles to our north, so they
can either act as a last reserve for my army, or return to Ft. Defiance if
General Whisper attempts to break out.

We request all of the reinforcements you are capable of providing.

\gramClosing{For the Union}
{Steele}
{Maj. Gen.}

Copy to General Fawcett, Army of the Cumberland

\reportdinkus

\gramHeader{Headquarters, Army of the Kanawha} % {{{4
{In the field, 11.00 pm, April}{11, 1862}
\gramTo{Maj. Gen.}{James Blake}
{Commanding, Dep't of the Cumberland}

\gramHi{General} We met the enemy today at Barton Creek and were faced with
overwhelming force. I believe the enemy numbered at least five corps of infantry
and another corps of cavalry. There were moments in the battle where single
divisions found themselves engaged with entire enemy corps, all along the line.

Our men fought like lions, and frankly kicked ass. The problem was they had too
many asses.

Worse, every one of my corps commanders fell in battle. Johanson has lost a leg,
Ryan is shot through the foot. Benton brought his two divisions to reinforce us
but he has lost an arm.

The enemy threw their weight upon my left, attempting to drive us back from the
river. For the most part they failed, at great cost to themselves, and to us. My
cavalry did a splendid job protecting them from completely turning our left
flank while outnumbered 2 to 1.

On my right flank, however, the enemy also enjoyed a 2 to 1 advantage in cavalry
and without the river to hem them in they were able to range all the way to the
Cumberland in our rear, and I believe they were able to capture Palymra in the
afternoon as reports of cannon fire were reported from that area around 4~pm. Any
communications sent after 4~pm, if not earlier, were likely intercepted by the
enemy. As of this writing, I have received no communiques from your side of the
river since April 8th.

I will provide the details of the battle below, but currently we occupy a rough
line some miles north of where we began this morning. Our right is anchored at
Budds Creek with our newly placed supply base at Hematite. The line goes from
Hackberry eastward across Orgains Crossroads to Camp Creek on our left. 1st
Division of VIIIth Corps still occupies the neck of the peninsula along Ussery
Branch.

Considering the size of the Army of Tennessee, we would be hard pressed to
withstand another assault tomorrow of the scale we witnessed today. But when you
add in the numbers of Whisper's army to my immediate rear our position is
untenable. I have no doubt whatsoever that Whisper will launch his own attack in
the morning, in coordination with another assault by Jackson. Unless you are
planning to bring the bulk of your army across the river tonight, I must attempt
to withdraw my army towards Dover to extract as much of my army as possible.

\textbf{Details of the Battle}

In the morning we found ourselves in line behind Louise Creek. There was more
formidable high ground to my south along McAllister's so we began by
seizing this high ground before the enemy could. In this, General Hall and the
rest of the cavalry performed exemplary in holding back the enemy, which doubled
them in strength, long enough for us to capture this important terrain. The
cavalry then relocated to our flanks. Ryan's XIth Corps was put on my right
while Johansen's XVIIth Corps formed my left.

At this time, I requested the assistance of Smith's 2d Division from VIIIth
Corps, which proved instrumental in reinforcing Johanson against two enemy corps
with additional cavalry. Also arriving was General Benton at the head of his own
XVIth Corps. When it became clear the enemy was massing on both of my flanks I
dispatched Benton to reinforce my right and do all he could to hold open the
line to Palmyra.

Our left began to falter first through no fault of Johanson. He led his men with
utmost courage and zeal. It is due to his leadership that his corps was able to
withstand five or six assaults of both Hardee's and Bragg's Corps, only giving
ground to overwhelming waves of enemy. Johanson eventually fell critically
wounded in the early evening, and was replaced by Milroy.

At 2.30~pm an opportunity arrived for Ryan and Benton to make a local
counter-attack on our right to relieve some pressure. This action they performed
in exemplary fashion, but it was met by enemy reinforcements and pushed back. In
this fight, General Ryan fell to his wounds and command of that area of the line
fell to General Garfield of XVI Corps. At this time General Benton also fell.
Generals Schenck and Hayes of XI Corps distinguished themselves here. 

Late afternoon Jackson sent another 25,000 men at my center. General Kelley of
1st Division, XIth Corps did the work of God here, and was only forced to retire
due to overwhelming numbers to either flank. 

Eventually, under the crush of overwhelming numbers, we retreated back to
Orgains Crossroads. We are still attempting to organize for the morning but we
have, at best, two divisions adequate for battle. Three if you count Smith's 1st
Division holding the line at Ft. Defiance. I estimate eleven brigades are
depleted, seven brigades severely depleted. Artillery is in rough shape. Up to
half our batteries have not reported in this evening, and those who have are
quite worked over. Benton was unable to bring his artillery across the river.

Of the enemy, we have identified ``wings'' under the command of Hardee, Bragg,
Johnston, Wesley, and Harrison. I believe the cavalry is commanded by Anderson
but I am not certain. There is also a rumor that General Harrison fell in
battle, but this is unconfirmed. They appeared to have four brigades per
division.

As I stated above, our position is untenable. At full strength we were
outnumbered 2 to 1 or or 3 to 1 without even adding Whisper's army to the tally.
I will attempt to reach Dover in the morning unless ordered to remain. But
without a significant reinforcement overnight I do not believe we will be able
to withstand a combined attack from Jackson and Whisper.

\gramClosing{Respectfully yours}
{Richard Steele}
{Maj. Gen., Army of the Kanawha}

Copy To General Fawcett, Army of the Cumberland

\reportdinkus

\gramHeader{Headquarters, Army of the Kanawha} % {{{4
{Dover, Tenn., April}{21, 1862}
\gramTo{Maj. Gen.}{James Blake}
{Commanding Department of the Cumberland}

\gramHi{General} With regards to the known enemy strength of General Jackson's
Army of the Tennessee at Barton's Creek, I will provide as much information as I
am able, based upon my own observations and those relayed to me by my
subordinates.

I can confirm with certainty that we identified five corps, designated as
``wings'' by the enemy. These five corps were commanded by Generals Bragg,
Hardee, Johnston, Harrison, and Vexley, with the enemy cavalry commanded by
Anderson. It is my strong suspicion that General Harrison was killed or
otherwise incapacitated in battle. 

Of Bragg's Wing, we are reasonably certain that two of his divisions were
commanded by Ruggles and Withers.

Of Hardee's Wing, we are reasonably certain that one division was commanded by
Breckinridge and another, with less certainty, was possibly identified as being
commanded by McCown.

Of Johnston's Wing, I can confirm that he had three divisions. One division was
commanded by Polk, another by Floyd and a third as-yet unidentified.

The Army of the Kanawha entered the campaign with 36,000 infantry, 5,400
cavalry, and 138 guns manned by 2,300 men for a total force of 43,700 men.

Johnston's Wing is confirmed to have three divisions although it is possible
that other wings were also so organized. Each enemy division is believed to
consist of three brigades.  With five enemy corps on the field, if each were the
equivalent size to our own that alone would be 100,000 men. They also fielded
twice as many cavalry as we did, placing Jackson's army at approx. 110,000.

Over the course of the battle, 2d Division, VIIIth Corps and 1st \& 2d
Divisions of XVIth Corps arrived on the field, increasing our forces by an
additional 22,000 men. This was, at best, only able to reduce our disadvantage
to 2:1, and arriving one division at a time meant their numbers were not as
keenly felt as they would have had they been able to arrive concurrently.

As it was, our roll calls currently indicate a loss of some 23,000 men out of
65,000 engaged. No doubt many of those are captured. In actual battlefield
casualties, we believe we gave as good as we got, likely even better. If we
estimate a third of our losses were captured in the retreat, we could also
estimate anywhere from 12,000--18,000 casualties inflicted on the enemy. 

As they held the ground at the conclusion of combat, they will be able to
recover some of these losses after a period of recuperation. These losses would
also be offset by the rescue of General Whisper's army, which was itself
estimated to be in the range of 12,000--20,000 men.

With these figures, I believe that by April 14, 1862, the enemy still fielded an
army exceeding 100,000 men on paper. This would not represent their full
effective strength at this exact moment as Jackson's army no doubt requires a
period of recuperation for his own forces. I believe he was forced to commit the
entirety of his forces on hand to defeat the Army of the Kanawha.

This report represents the best of my knowledge as of the date of this writing. 

\gramClosing{Respectfully yours}
{Richard Steele}
{Maj. General, Commanding Army of the Kanawha}
\reportdinkus

\gramHeader{Headquarters, Army of the Kanawha} % {{{4
{Evening, May}{1, 1862}
\gramTo{Maj. Gen.}{H. Fawcett}
{Commanding, Army of the Cumberland}

\gramHi{General} Your message this evening is received. We will join your
attack.

With Kellogg currently screening my force as well, I will deploy my First
Cavalry Division to find their rear and cut their line of communications.

We will attack with two divisions with another two in reserve to exploit
developments.

I will also report that I have four regiments of cavalry screening the fort.
They report drastically reduced troop levels there. While still prepared to
defend itself, they pose no threat to our rear.

\gramClosing{For the Union}
{Richard Steele}
{Major General, Army of the Kanawha}
\reportdinkus

\gramHeader{Headquarters, Army of the Kanawha} % {{{4
{7.00 am, May}{1, 1862}
\gramTo{Maj. Gen.}{H. Fawcett}
{Commanding, Army of the Cumberland}

Kellogg has engaged some rebel cavalry to our front. I am sending in XVII Corps
now.

I am also recalling Second Cavalry Division from picket duty at the fort. They
report they are no threat to our rear and four regiments of cavalry might come
in handy here

\gramClosing{}
{Steele}
{}
\reportdinkus

\gramHeader{Headquarters, Army of the Kanawha} % {{{4
{Near Fort Defiance, Tenn., May}{9, 1862}
\gramTo{Maj. Gen.}{Blake}
{Dep't of the Cumberland}

\gramHi{General} It is my pleasure to report the capture of Ft. Defiance. The
river to Nashville is now clear.

With the assistance of General Wood, General Ryan of XIth Corps led the assault
upon the enemy works. Surely this country has no finer men than the souls in
this army.

Charging through the artillery and fire if enemy gunboats, then men of XIth
Corps reached the enemy ramparts and carried the fight to the enemy at the point
of the bayonet.

The foothold they were able to gain was then capitalized upon by Milroy’s XVIIth
Corps, who forced a breach and stormed through, compelling the enemy to
surrender the fort.

Losses reported are 1,800 men killed, wounded, and missing. The enemy left 300
casualties and 900 more surrendered their arms and are now held prisoner.

The Navy has now moved past the bastion and is in control of this portion of the
river.

We will take stock of this position, and take the necessary steps to recover
from the assault as we await your further instructions.

\gramClosing{Your obedient servant}
{Richard Steele}
{Major General, Commanding Army of the Kanawha}

\subsecdinkus

\subsection{Reports of Brig. Gen. Jacob Ryan, XIth Corps} % {{{3

\gramHeader{Headquarters, XIth Corps} % {{{4
{Dover, Tenn., April}{11, 1862}
\gramTo{Maj. Gen.}{Richard Steele}
{Commanding, Army of the Kanawha}

\gramHi{Sir} A response to the request for report of the state of XIth Corps and
its subordinate formations after battle yesterday. The information presented
here is based on supposition of enemy and friendly strength at around
2.30~pm, after which this headquarters was out of action due to wounds.

First Division, XIth Corps status was last known to be fully combat capable
beyond losses suffered to its artillery. Further updates after 3.00~pm are
entirely unknown but would suggest that the Commanding General would be more
aware of their current status and capabilities than this headquarters.

Second Division, XIth Corps is the formation most well known to this Commanding
Officer. As of 2.30~pm, the Division had suffered severe losses to one Brigade,
another with significant losses, while the third remained essentially fresh. No
artillery losses to report at that time. It is the estimation of this
headquarters that the Second is to be considered heavily degraded, likely bled
white, and committed to battle only in the most desperate circumstances after
3.00~pm. Brigadier General Schenck and his Division should be commended for
buying this Army time for the arrival of XVIth Corps reinforcements. They gave
as good as those Rebels gave back, and while outnumbered two to one. This
headquarters believes this worthy of distinction.

Third Division, XIth Corps is estimated to be a bit bloodied but fully combat
capable. The transfer of the division against Rebel Cavalry on the flank likely
preserved most of their combat capability. If the First took significant losses
in the fighting, the Third likely remains the most combat capable formation in
the entirety of XIth Corps. These troops should be ready and reliable should
another engagement take place.

Overall, this headquarters has taken significant losses but remains a capable,
if battered, fighting force.

\gramClosing{Very respectfully}
{Jacob Ryan}
{Brig. Gen., Commanding}

\gramHeader{Headquarters, XIth Corps} % {{{4
{Dover, Tenn., April}{12, 1862}
\gramTo{Maj. Gen.}{Richard Steele}
{Commanding, Army of the Kanawha}

\gramHi{Sir} A response to the request for report of the state of XIth Corps and
its subordinate formations after the withdrawal to Dover yesterday. The
information presented here is based the best estimation of this headquarters at
the time of writing

First Division, XIth Corps status is known to be heavily degraded with extensive
losses in guns and men. The First and Second Divisions both participated in the
initial breakthrough action towards Palmyra behind friendly cavalry elements,
which proved successful. In the process, the Division reported the shattering of
one rebel cavalry division before forcing elements of a second to retire before
them. Upon reaching their road junction objective, the Division then
participated in an effort to push back enemy cavalry and infantry reinforcements
before linking up with Third, which was hard pressed at the time. This proved
unsuccessful, but thankfully also unnecessary.

Third Division, XIth Corps participated in a holding action on the immediate
southern flank of the Army during its withdrawal. Reports indicate the Third
Division held against elements of two to three rebel divisions that were
attempting to prevent the breakout of friendly forces during the night. In this
effort, the Third performed above and beyond all expectations of this
headquarters, holding against three times its numbers in sharp nighttime
fighting with its flank in the air. The performance of these men under fire
should not be understated.

Overall, this headquarters has taken massive losses over these last days, but we
remain. On another day, our time will come. On that field, we will make the
enemy regret that fact.

\gramClosing{Very respectfully}
{Jacob Ryan}
{Brig. Gen., Commanding}

\gramHeader{Headquarters, XIth Corps} % {{{4
{Cloverdale, Tenn., May}{3, 1862}
\gramTo{Maj. Gen.}{Richard Steele}
{Commanding, Army of the Kanawha}

\gramHi{Sir} A response to the request for report of the state of XIth Corps and
its subordinate formations after the Battle of Cloverdale yesterday. The
information presented here is based the best estimation of this headquarters at
the time of writing.

First Division, XIth Corps status is known to be fully combat effective.

Second Division, XIth Corps status is known to be fully combat effective.

This headquarters can firmly report its participation in the Battle of
Cloverdale as minor, and mainly limited to the exploitation of the breakthrough
of enemy lines achieved at great cost by XVIIth Corps, Army of the Kanawha and
the Army of the Cumberland elements on our left flank. From the commencement of
battle until 7.45~am this headquarters served as the standing reserve of the
Army of the Kanawha.  Upon the prompt identification of the collapse of the
Rebel right and a quick survey of the ground, XIth Corps was committed to the
exploitation of the collapse and pursuit of the enemy. Unfortunately, effective
rearguard actions by enemy remnants and difficult terrain limited close pursuit
of the enemy to cavalry elements only, limiting the losses inflicted upon us and
the enemy to only prisoners and scattered Rebel soldiers.

Overall, this headquarters remains fully capable and ready for further combat,
should higher headquarters desire it.

\gramClosing{Very respectfully}
{Jacob Ryan}
{Brig. Gen., Commanding}

\gramHeader{Headquarters, XIth Corps} % {{{4
{Cloverdale, Tenn., May}{10, 1862}
\gramTo{Maj. Gen.}{Richard Steele}
{Commanding, Army of the Kanawha}

\gramHi{Sir} A response to the request for report of the state of XIth Corps and
its subordinate formations after the assault against Fort Defiance yesterday.
The information presented here is based the best estimation of this headquarters
at the time of writing

First Division, XIth Corps status has taken approximately 600 casualties as a
result of its participation in the assault.

Second Division, XIth Corps status has taken approximately 600 casualties as a
result of its participation in the assault.

An hour before dawn, friendly elements under General Wood began our effort by
pushing in the enemy picket line around the Bastion back towards their ramparts.
Upon the direction of General Steele, XIth Corps then assembled into assault
column, with each element in echelon behind the first regiments in the line.
Once word was received that our covering artillery was ready and that General
Wood had been demonstrating preparations for an assault on the Bastion's West
for some time, XIth Corps was given direction to advance at dawn against the
Eastern ramparts. Enemy fire proved heavy, augmented by a trio of Rebel gunboats
stationed in the river; heavy fighting ensued. The first and second assault
waves proved unable to wither the volume of enemy shot before withdrawing under
pressure, however, they paved the way for the third to breach and carry the
enemy's works, at which point Defiance's garrison laid down its arms. All in
all, fifteen minutes passed between the commencement of the assault and the
surrender of the enemy. XIth Corps reports 300 enemy casualties with an
additional
900 prisoners as a result of the successful action.

Overall, this headquarters remains fully combat capable, however, rest will be
required in order to ensure that all subordinate formations remain ready for
action in strength.

\gramClosing{Very respectfully}
{Jacob Ryan}
{Brig. Gen., Commanding}

\subsecdinkus

\subsection{Report of Brig. Gen. Rutherford Hayes, Third Division, XIth Corps} % {{{3

\gramHeader{Headquarters, Third Division, XIth Corps} % {{{4
{Near Dover, Tenn., April}{14, 1862}
\gramTo{Maj. Gen.}{Richard Steele}
{Commanding Army of the Kanawha}
\gramTo{Brig. Gen.}{Jacob C. Ryan}
{Commanding XIth Corps}

\gramHi{Generals} I have the honor to submit the following report of the
operations of my command in the engagement at Oak Ridge, Tenn., on the
morning of the 12th instant.

In obedience to orders received from Brigadier General Ryan, directing me to
cover the southern flank of the Army of the Kanawha during its withdrawal from
Hackberry, my division was put in motion shortly after 1 o'clock~a.m., the
troops forming quietly and with commendable steadiness for a night movement of
some difficulty. At 4 o'clock~a.m. the column advanced toward Oak Ridge, and at
about 4.30~a.m., while still moving in darkness, the leading elements
encountered Confederate infantry occupying the ground near the ridge.

The proximity of the enemy being thus unexpectedly disclosed, I at once ordered
the division to form line of battle. The First Brigade, under Colonel Joseph
Thoburn, and the Second Brigade, under Colonel Isaac H. Duval, were promptly
brought into position. The Third Brigade, under Colonel Carr B. White, was
moving up in rear, with the division artillery, commanded by Captain Henry A. du
Pont, following along the road.

For the next two hours, a severe and confused engagement ensued, fought almost
entirely in darkness. During this time the division successfully resisted the
repeated assaults of two Confederate infantry divisions, supported by artillery.
From prisoners and other information since obtained, I am satisfied that the
troops immediately engaged belonged to Floyd's division of General Albert
Johnston's wing, with support from Polk's division, and that later a further
division was brought forward to press the attack.

The enemy advanced repeatedly and with great determination. In the obscurity and
confusion of the night, two Confederate brigades were observed to give way in
disorder, their lines becoming broken under the steady fire of our troops. Our
own men, though heavily pressed and aware of the enemy's superiority in numbers,
maintained their ground with firmness and discipline worthy of the highest
commendation.

At length, however, the increasing pressure upon our front, combined with the
arrival of an additional Confederate division upon the right flank of my line,
rendered the position no longer tenable. Orders were therefore given to withdraw
in good order toward our original line of march near Hackberry. In this
movement, the First Brigade, Colonel Thoburn commanding, displayed marked
steadiness and gallantry, holding its position until the remainder of the
division had safely disengaged, and retiring last without confusion.

During the withdrawal toward Hackberry, the division narrowly avoided collision
with Confederate columns advancing upon that point. The timely appearance and
effective action of Wilder's Cavalry Brigade, operating upon our flank and rear,
materially assisted in checking pursuit and enabling the division to continue
its movement without further serious loss.

The division ultimately reached Sailor's Rest at about 10.30~a.m., being among
the last organized elements of the Army of the Kanawha to arrive at that
position. Despite the severity of the engagement and the exhaustion of the men
after continuous night operations, the command remained in good order upon its
arrival.

I desire to call special attention to the conduct of the brigade commanders,
Colonels Thoburn, Duval, and White, and to Captain du Pont and the officers and
men of the artillery, all of whom performed their duties with coolness and
resolution under circumstances of unusual difficulty. The endurance and
discipline displayed by the division in night combat against greatly superior
numbers reflect the highest credit upon the troops.

A detailed report of casualties will be forwarded as soon as they can be
accurately compiled.

\gramClosing{I am, Generals, very respectfully, Your obedient servant}
{Rutherford B. Hayes}
{Brig. Gen., Commanding Third Division, XIth Corps}
\reportdinkus

\subsection{Report of Col. J. T. Wilder, First Brigade, First Cavalry Division} % {{{3

\gramHeader{Headquarters, First Brigade, First Cavalry Division} % {{{4
{In the Field, near Sailor's Rest, Tenn., April}{14, 1862}
\gramTo{Capt.}{W. T. Hobbins}
{Asst. Adjutant General, First Cavalry Division}

\gramHi{Captain} I have the honor to submit the following report of the
operations of this brigade during the recent Barton's Creek Campaign, extending
from the 8th to the 12th of April, inclusive.

Pursuant to orders from Brig. Gen. Hall, my command, consisting of the 17th
Indiana, 72d Indiana, and 98th Illinois Volunteer Mounted Infantry, broke camp
at Cunningham on the morning of April 8th.  We joined the cavalry advance moving
toward Cloverdale. No enemy was encountered during the day's march, and the
brigade bivouacked for the night in the vicinity of Mt. Zion Church.

On the morning of the 9th, scouts and loyal citizens having reported the enemy's
cavalry in possession of Cloverdale, this brigade was assigned the advance of
the divisional column. At early dawn, near Big Horse Branch, we uncovered the
enemy's pickets. I immediately ordered the brigade to dismount and deploy in
loose order. Captain Eli Lilly, commanding the divisional horse artillery,
unlimbered his guns with his usual alacrity and opened a well-directed fire.
Under cover of this metal, my men advanced with a cheer, driving the enemy's
skirmishers from the creek. We quickly remounted and pursued the fleeing rebels
toward the Furnace.

Upon reaching Cloverdale, we came upon the camps of Kryznowski's Wing, Army of
the West. The enemy had fled in such haste as to leave much of his equipment,
though he soon established a formidable line just beyond the works. I deployed
the brigade to harass his front while the remainder of the division moved up.
Being reinforced by the 2d Division under Brig. Gen. Mosier, I was ordered to
move to the right and turn the enemy's flank. We found the junction of Furnace
Creek and Big Barton's Creek to be the key. Again dismounting, my men advanced
through dense brush at the oblique, falling upon the unsuspecting flank of the
enemy. The surprise was complete; the enemy broke in disorder, and we pursued
them until the entire Confederate line withdrew to the southeast. We encamped
upon the field of our victory.

The morning of April 10th found the brigade on the advanced line. Shortly after
posting pickets, they returned at a gallop, reporting the enemy advancing in
great force. My brigade took position to the left of the road, with the 2d
Brigade on our right. We maintained a stubborn resistance throughout the
morning, but upon receiving intelligence that the 2d Cavalry Division's line on
the Charlotte Road was collapsing, we were compelled to withdraw. This
retrograde movement was conducted in good order. As the rear guard, my brigade
contested every natural obstacle, making stands at Furnace Creek, Big Horse
Branch, and Little Horse Branch, finally halting at Little Barton's Creek. That
night we slept on our arms; the horizon was brilliantly illuminated by the
campfires of a vast host in our front. At 4~a.m. on the 11th, hearing heavy
firing on the picket line, I observed my outposts being driven down the ridge
toward the creek. I directed Captain Lilly to form a false line with the pickets
between the creek and McAllister's to delude the enemy. The brigade
remained hidden in the tall grass and corn north of the crossing. As the enemy's
dismounted line entered the creek bed, I gave the signal. The entire line
erupted in a sheet of flame. The enemy, staggered by this sudden volley in the
predawn darkness, fell back. They reformed and attacked again at 6.30~a.m. with
at least four brigades. Our left was momentarily threatened, but the timely
arrival of the 2d Brigade secured the position.

Following a vigorous counterattack ordered by General Hall, we drove the enemy
back across the creek, only to find them supported by heavy masses of infantry.
At 8~a.m., under orders to guard the army's left, we displaced to the confluence
of Barton's and Louise Creeks. Seeing a massive thrust materializing against
General Milroy's division of the XIth Corps, I moved the brigade to a position
of enfilade. We poured a destructive fire into the enemy's flank until relieved
by a division of the VIIIth Corps.

As the day progressed and the infantry's situation became desperate, this
brigade was detached to support the extreme left of the army. Dismounted, and
using our rapid-firing carbines to terrible effect, we repulsed two massed
infantry assaults. When the enemy pressed within a few yards, the men used their
revolvers with such deadly precision that the rebel line withered and broke.
During the general withdrawal toward Cumberland Heights, we harried the enemy
until nightfall. I must confess to such exhaustion that I fell asleep in my
saddle during the final movement to camp.

At midnight on the 12th, I was roused by orders to secure the road from
Cunningham to Hackberry Landing to cover the army's retreat to Sailor's Rest. At
4.30 o'clock, we encountered a massed formation of enemy infantry. We fought a
running battle against overwhelming odds. Near a critical crossroads, I
encountered Brig. Gen. Hayes, who informed me his division must pass this
point to escape. I ordered my men to hold at all hazards. As the enemy reached
pistol range, I directed the men to discharge their revolvers rapidly to
simulate a much larger force. The ruse succeeded; the enemy hesitated, and
Hayes' division passed to safety. We reached Thomas Creek at 10~a.m.

A muster taken this morning shows 612 of all ranks present for duty out of 907
who entered the action on the 8th.

I cannot close this report without mentioning the gallant conduct of Colonel
John J. Funkhouser, 98th Illinois, and Lieut. Col. Edward Mitchell, who
handled their commands with great skill. Sergeant Henry Heller of the 17th
Indiana deserves special mention for carrying a message through a galling fire
when our lines were nearly turned at the crossroads. The coolness of Captain
Lilly and his cannoneers under fire was the admiration of the entire command.

\gramClosing{I have the honor to be, Captain, very respectfully, your obedient servant}
{J. T. Wilder}
{Colonel, Commanding Brigade}

\reportdinkus

\subsection{Report of Maj. Gen. James Howard, Army of the Kentucky} % {{{3

\gramOrdersHeader{Headquarters, Army of the Kentucky} % {{{4
{Munfordville, Ky., April}{1, 1862}
{Special Field Order}{XX}

The spring campaign will unfold broadly in three phases: re-assembly, fixing,
and major operations.

I. Initially the  army will re-assemble at Munfordville, reflagging into the new
corps structure and assembling reinforcements. During this time cavalry, and
potentially advanced guard, operations will go as far west as Dripping Springs
to search for the enemy and alert him to the potential of our advance. Should we
encounter no resistance, short reconnaissance efforts will seek to determine if
the enemy is in position at Bowling Green and the Barren River.

II. Having found the enemy, we will transition to fixing him in place. The Army
of the Kentucky will seek to develop the situation and present a threat to
prevent enemy forces from reinforcing the defense of the Cumberland and
Nashville.  Unless the enemy has left only token forces in Bowling Green, this
will mean moving up the main body to as far as Dripping Springs and establishing
a small depot capable of maintaining a permanent cavalry presence forward. If
Bowling Green can be easily seized, this army will consume most of its
logistical potential for the season to conduct an army size advance on it. We
will confirm this decision with Department headquarters first.

III. Once the rail bridge is repaired over Munfordville, this army will be
prepared to conduct major operations---either a general engagement against an
enemy near Bowling Green, or exploitation into Tennessee if resistance is
limited.

\gramClosing{}
{James Howard}
{Maj. Gen., Commanding}
\reportdinkus

\section[Second Battle of New Madrid, Mo.]{March 24, 1862} % {{{2
    {Battle of New Madrid, Mo.}

\subsection*{Report of Maj. Gen. Karl Meyer, Army of the Arkansas} % {{{3

\gramHeader{Headquarters, Army of the Arkansas} % {{{4
{Charleston, Mo., March}{24, 1862}
\gramTo{Maj. Gen.}{Cornelius Van Royne}
{Commanding General of the United States Army}
\gramTo{Brig. Gen.}{Thomas Caldwell}
{Commanding General of the Army of the Tennessee}

\gramHi{Generals} It is my great pleasure to offer to you on this day, the town
of New Madrid. Our forces marched with great haste upon the city and found
improvised earthen ramparts prepared against us. Fortunately, due to our
significant advantage in artillery, most of the enemy's counter battery fire was
silenced after almost two hours.

At this point our artillery continued to bombard the enemy for a short period of
time before the infantry advanced upon the defenses, the cavalry screening their
left flank near the swamp. For a moment, the enemy held their positions before
Generals Ambrose and Schurz ordered the XIIIth Corps to fix bayonets and
personally led the charge into the enemy lines, engaging in half an hour of hand
to hand fighting before driving the enemy out of their defenses.

The enemy was then forced to retire to their second line of defenses on the
outer limits of the town, but were quickly set upon by our boys who had already
made a breach. The cavalry division as well as the infantry reserves were
committed to the attack and finally captured the defenses after an hour, during
which they repulsed an enemy counter attack at great cost.

The capture of the second line of defenses resulted in the routing of General
Thomson's infantry, but sadly the opportunity could not be capitalized upon, due
to the rebel cavalry screening their retreat as well as the presence of enemy
gunboats shelling any advance we attempted to make.

As an aside, I would like to inform you that Flag Officer Foote encountered guns
of the heaviest caliber, and formidable defenses at Island Number 10. It is
highly likely we will have to siege the island if we are to continue down the
Mississippi.

At the end of the battle, the total casualties for the Army of the Arkansas were
estimated to amount to 3,600 men and 300 troopers, or 6 regiments of infantry
and a regiment of cavalry, with 23\% of the army missing muster this past week
due to illness, injury, and other causes. Of the enemy, it is estimated that
they took 4,500 to 5,500 casualties and lost 24 cannons, which I am pleased to
report were captured intact.

By the estimates of my staff, this most recent battle brings the total strength
of the Army of the Trans-Mississippi down to 24,800 to 26,300 men. The enemy
most likely still retains 15 to 16 regiments of cavalry given they avoided the
majority of the battle, leaving them with 33 or 35 regiments of infantry and 9
batteries of artillery. For comparison, my own Army of the Arkansas has 42
regiments of infantry, 9 regiments of cavalry, and 16 batteries of artillery.

During the morning of the 24th, my men spotted spot butternut and grey infantry
across the river from New Madrid. It is certain that by next week, the enemy
army will be fully whole again. Nevertheless, we will not allow him to regain
what we have so dearly liberated.

I wish to once again reiterate the bravery of Generals Norman Ambrose and Carl
Schurz, as well as the actions of the XIIIth Corps as a whole. In particular,
General Ambrose has been a most capable and intuitive commander, who I would
attribute a good deal of this army's success to. He is my right hand which holds
the saber cutting through the insurrectionists grasp on the Mississippi. It is
for this reason, that I would like to submit the name of the good general to be
considered for promotion to Major General of the Volunteers.

\gramClosing{I am always, Generals, and shall ever remain, Your most humble and obedient of servants}
{Karl Meyer}
{Maj. Gen. cmdg, Army of the Arkansas}

\secdinkus

\section[The Memphis Campaign]{April 1, 1862--MM DD, 1862} % {{{2
    {The Memphis Campaign}

\begin{toc}[ % Summary of Events {{{3
    caption = {Summary of the Principal Events.},
]{}
Apr.  & 22, 1862.  & --M. D, 1862.---Siege of Island Number 10, Ky. \\
      & 25, 1862.  & ---Capture of Humboldt, Tenn. \\
May.  & 7, 1862.   &---Battle of Holly Grove, Tenn. \\
      & 19, 1962.  &---Battle of Middle Deer Fork, Tenn. \\

\end{toc}

\begin{toc}[ % Reports {{{3
    caption = {Reports, Etc.},
]{}
No. & 1.  & ---Organization of the Army of the Arkansas \\
No. & 2.  & ---Maj. Gen. Karl Meyer, Army of the Arkansas \\
\end{toc}

\subsection{Organization of the Army of the Arkansas, Maj. Gen. Norman Ambrose, % {{{3
U.S. Army, commanding, April 23, 1862--MM, DD, 18662}
\footnotetext[1]{
    Arranged according to the numerical designation of the corps, divisions and
    brigades as prescribed in General Orders, No.~5, Headquarters, Army of the
    Arkansas, MM~DD, 1862.
}

\begin{fulloob}
    \corps{XVth Corps}{Maj. Gen.}{Carl Schurz} % {{{4

    \division{First Division}{ Brig. Gen.}{XXX}% {{{5
    \begin{leftBde}
        \bde{First Brigade}{Col.}{John Scott}
        \rgt{1st}{Iowa}{}
        \rgt{14th}{Iowa}{}
        \rgt{1st}{Kansas}{}
        \rgt{2d}{Kansas}{}
    \end{leftBde}
    \begin{rightBde}
        \bde{Second Brigade}{Col.}{William Vandever}
        \rgt{8th}{Indiana}{}
        \rgt{18th}{Indiana}{}
        \rgt{22d}{Indiana}{}
        \rgt{13th}{Kansas}{}
    \end{rightBde}
    \begin{middleBde}
        \bde{Third Brigade}{Col.}{Edward Wolfe}
        \rgt{4th}{Iowa}{}
        \rgt{12th}{Iowa}{}
        \rgt{35th}{Iowa}{}
        \rgt{47th}{Missouri}{}
    \end{middleBde}

    \division{Second Division}{Brig. Gen.}{Nathaniel Lyon} % {{{5
    \begin{leftBde}
        \bde{First Brigade}{Col.}{George Lippitt Andrews}
        \rgt{2d}{United States}{}
        \rgt{4th}{United States}{}
        \rgt{15th}{Missouri}{}
        \rgt{29th}{Missouri}{}
    \end{leftBde}
    \begin{rightBde}
        \bde{Second Brigade}{Col.}{Charles Salomon}
        \rgt{1st}{Missouri}{}
        \rgt{2d}{Missouri}{}
        \rgt{6th}{Missouri}{}
        \rgt{7th}{Missouri}{}
    \end{rightBde}
    \begin{leftBde}
        \bde{Third Brigade}{Col.}{Alexander Asboth}
        \rgt{30th}{Missouri}{}
        \rgt{24th}{Missouri}{}
        \rgt{33d}{Missouri}{}
        \rgt{43d}{Missouri}{}
    \end{leftBde}
    \begin{rightBde}
        \bde{Fourth Brigade}{Col.}{Otto Fischer}
        \rgt{3d}{Missouri}
        \rgt{4th}{Missouri}
        \rgt{11th}{Missouri}
        \rgt{26th}{Missouri}
    \end{rightBde}

    \otherbde{Artillery} % {{{5
    \begin{leftBde}
        \otherbde{First Division}
        \rgt{---}{Missouri Battery}{Capt. Fredrick Schaefer}
        \rgt{---}{Missouri Battery}{Capt. John Du Bois}
    \end{leftBde}
    \begin{rightBde}
        \otherbde{Second Division}
        \rgt{1st}{Illinois Battery}{Capt. Peter Davidson}
        \rgt{2d}{Illinois Battery}{Capt. Earl St. Jude}
    \end{rightBde}
    \begin{middleBde}
        \otherbde{Reserve Artillery}
        \rgt{1st}{Missouri, Battery A}{Capt. Henry Dillon}
        \rgt{1st}{Missouri, Battery B}{Capt. Richard Griffith}
        \rgt{1st}{Missouri, Battery C}{Capt. A. W. Dees}
    \end{middleBde}

    \corps{Division of Observation}{Brig. Gen.}{Julius White} % {{{4
    \begin{leftBde}
        \bde{First Brigade}{Col.}{James Mulligan}
        \rgt{13th}{Illinois}{}
        \rgt{16th}{Illinois}{}
        \rgt{17th}{Illinois}{}
        \rgt{36th}{Illinois}{}
    \end{leftBde}
    \begin{rightBde}
        \bde{Second Brigade}{Col.}{Grenville Dodge}
        \rgt{37th}{Illinois}{}
        \rgt{44th}{Illinois}{}
        \rgt{47th}{Illinois}{}
        \rgt{59th}{Illinois}{}
    \end{rightBde}
    \begin{middleBde}
        \otherbde{Artillery}
        \rgt{1st}{Missouri Flying Artillery}{Capt. Josef Müller}
        \rgt{---}{Missouri Battery}{Capt. Clemens Landgraeber}
    \end{middleBde}

    \corps{Cavalry Corps}{Brig. Gen.}{David Stanley} % {{{4

    \division{First Division}{Brig. Gen.}{Franz Backhoff} % {{{5
    \begin{leftBde}
        \bde{First Brigade}{Col.}{Charles Farrand}
        \rgt{1st}{United States}{}
        \rgt{2d}{United States Dragoons}{}
    \end{leftBde}
    \begin{rightBde}
        \bde{Second Brigade}{Col.}{Thomas Marshall}
        \rgt{3d}{United States}{}
        \rgt{5th}{United States}{}
    \end{rightBde}
    \begin{leftBde}
        \bde{Third Brigade}{Col.}{Benjamin Grover}
        \rgt{1st}{Missouri}{}
        \rgt{2d}{Missouri}{}
    \end{leftBde}
    \begin{rightBde}
        \bde{Fourth Brigade}{Col.}{Friedrich von Kleist}
        \rgt{5th}{Missouri}{}
        \rgt{6th}{Missouri}{}
    \end{rightBde}
    \begin{middleBde}
        \otherbde{Artillery}
        \rgt{1st}{Missouri Flying Artillery}{Capt. Gustavus Elbert}
    \end{middleBde}

    \division{Second Division}{Brig. Gen.}{Eugene Carr} % {{{5
    \begin{leftBde}
        \bde{First Brigade}{Col.}{Henry Imhauser}
        \rgt{7th}{Missouri}{}
        \rgt{27th}{Missouri Mounted Infantry}{}
    \end{leftBde}
    \begin{rightBde}
        \bde{Second Brigade}{Col.}{Frederick Steele}
        \rgt{8th}{Kansas}{}
        \rgt{9th}{Kansas}{}
    \end{rightBde}
    \begin{leftBde}
        \bde{Third Brigade}{Col.}{John Groesbeck}
        \rgt{4th}{Missouri}{}
        \rgt{8th}{Missouri}{}
        \rgt{9th}{Missouri}{}
    \end{leftBde}
    \begin{middleBde}
        \otherbde{Artillery}
        \rgt{1st}{Missouri Flying Artillery}{Capt. Georg Koch}
    \end{middleBde}

    \corps{Artillery Reserve}{}{} % {{{4
    \begin{middleBde}
        \rgt{1st}{Missouri, Battery A}{Capt. Henry Dillon}
        \rgt{1st}{Missouri, Battery B}{Capt. Richard Griffith}
        \rgt{1st}{Missouri, Battery C}{Capt. A. W. Dees}
    \end{middleBde}
\end{fulloob}
% TODO: Remove requirement for this blank line

\subsecdinkus

\subsection*{Reports of Maj. Gen. Karl Meyer, commanding Department of the West} % {{{3

\gramHeader{Headquarters, Dep't of the West} % {{{4
{Bell, Tenn., May}{12, 1862}
\gramTo{Maj. Gen.}{Cornelius Van Royne}
{Commanding General, United States Army}

\gramHi{Sir} The Army of the Tennessee is marching down the Memphis and Ohio
Railroad, pursuing elements of Gen’l Thomson’s cavalry back to Memphis. The
evening of May 6th, 1862, while the XVIIIth and XIXth Corps are camped at Bell
on the north bank of the South Fork River, scouts report that Rebel infantry
some 16--20,000 strong have been spotted marching up to Brownsville. In
addition, rebel infantry have been encountered by the 33d Cavalry Division at
the road crossing at Cherryville.

The decision is made to have as much of the Army of the Tennessee cross the Bell
rail bridge to the south bank of the South Fork River, and by morning the entity
to the XVIIIth Corps and the 1st Division of the XIXth Corps are across. By
8~o'clock first contact is made with the Rebel line and the batteries of the
XVIIIth Corps open up on their rebel counter parts, while the 2d Division of
XIXth Corps crosses the river and moves towards the left flank of their sister
division.

By 9~o'clock both divisions of the XVIIIth Corps and the 1st Division of the
XIXth Corps advance forward, driving the rebels back up the hill before running
into a renewed line of infantry and cannons. Meanwhile the 1st Division of the
XIXth Corps runs into enemy cavalry to their left flank, forcing them to refuse
their line.

Again the artillery batteries are brought up to hammer away at the enemy line
while the 1st and 2d Divisions of the XVIIIth Corps cycle in fresh brigades to
renew the attack, while the 2d Division of the XIXth Corps forms up as a
reserve behind the 1st Division.

The renewed assault lasts for another hour, advancing slowly and receiving quite
the battering as the rebel line is slowly driven back at Holly Grove, a few
yards at a time. On the right, the 1st Division of XIXth Corps reports that the
cavalry facing them has slowly created distance between themselves and the wall
of Union blue.

Believing that the enemy is on their last legs, Gen’ls Meyer and Caldwell agree
to throw forward the 2d Division of XIXth Corps in a flanking attack on the
enemy’s right flank. At the same time, the battered XVIIIth Corps will advance
forwards, pinning the enemy to their front in place in order to allow the 2d
Division, XIXth Corps to roll up the flank.

At first the attack makes great headway, routing several rebel regiments and
rolling up the line, but by 11.30~am, thirty minutes into the assault, fresh
reinforcements for the rebels counterattack into the 2d Division, XIXth Corps’
flank, forcing them to recoil and creating a bend as Union artillery and
infantry reserves bring the enemy assault to a halt. To their left, the 1st
Division, XIXth Corps has extended their own lines leftwards in order to match
the Confederate cavalry taking to the field.

Realizing the danger of the position, Gen’l Meyer orders the entire line of
battle to redress itself as he rallies his fleeing men. By 12.45~pm, the Union
forms a worn but sturdy line to ward off any enemy counter attack only to find
the Rebels having drawn back yet further down the hill. Having noticed this
recent change, Gen’l Meyer sends a courier to Gen’l Lyon and the Deutsch
Division at Jackson, ordering them to march to Bell and across the river to join
the rest of the army.

The end to the first day’s battle reveals that the Army of the Tennessee has
taken some 4,500 casualties and scouts report that the enemy withdrew completely
off the hill, moving into a valley behind a small creek on the foot of a
neighboring hill.

By 11~o'clock the next day, Gen’l Lyon and his division arrive to the
battlefield, only for the Gen’ls present on the field to have decided that
further efforts to ford the creek and assault a second hill would be too costly
for the army which is now at 55-60\% effectiveness and completely out of solid
shot. Instead the army falls back to the bridgehead, with the XVIIIth Corps
crossing the bridge to encamp at Bell.

By the end of the week field returns confirm that the Army of the Tennessee has
lost 2,400 men and 6 guns during the Battle of Holly Grove. Enemy casualties are
much more vague with the officers and men who participated in the attack feeling
like they inflicted more losses than they took, and taking out a number of guns.

\gramClosing{I am always, General, and shall ever remain, Your most humble and obedient of servants}
{Karl Meyer}
{Maj. Gen. cmdg Dept. of the West}

\subsecdinkus

\section[Capture of Bowling Green, Ky.]{April 21, 1862} % {{{2
    {Capture of Bowling Green, Ky.}

\subsection*{Report of Maj. Gen. James Howard, commanding Army of the Kentucky} % {{{3

\gramHeader{Headquarters, Army of the Kentucky} % {{{4
{Dripping Springs, Ky., April}{21, 1862}
\gramTo{Maj. Gen.}{James W. Blake}
{Commanding Dep't of the Cumberland}

\gramHi{Sir} We have occupied Bowling Green and burnt the Russellville and
Franklin North rail bridges without any sign of the enemy. My cavalry has
withdrawn to Bowling Green and my main body remains near Dripping Springs in
order to economize on logistical expenditure.

While the enemy remains absent, the burning I believe prophylactically ensures
against any sudden bold moves on their part. At worst we will rebuild them later
in our advance; if nothing else our engineers have had some practice in such
matters.

To that end, a parsing of the news and your reports indicates the enemy suffered
some 12,000 losses at Barton's Creek. While not as grievous as our own losses,
that seems mostly a matter of missing men; their field force must be in poor
condition and likely will take weeks to restore itself to a full operational
capacity.

In short, I expect them to reach a state of campaigning readiness at roughly the
same time the Munfordsville rail bridge is finished, with some smaller portion
of their force being ready before then.

I presume the rest of our department will be on a similar schedule, perhaps
ready for efforts again in late May as opposed to the rebels early to mid May.

With that in mind, I ask about your overall plan for that period so that I can
begin preparations accordingly. I intend to advance the main body to Bowling
Green almost regardless, but how I set the stage from there will depend on
circumstances and guidance.

\gramClosing{Yr Obt Svt}
{James Howard}
{Maj. Gen., Army of the Kentucky, Commanding}

Copies to: Maj. Gen's Harold K. Fawcett, III \& Richard Steele

\subsecdinkus

\section[The Corinth Campaign]{May 20, 1862--MM DD, 1862} % {{{2
    {The Corinth Campaign}

\begin{toc}[ % Summary of Events {{{3
    caption = {Summary of the Principal Events.},
]{}
May. & 20, 1862. & ---Maj. Gen. James Howard assumes command of the Army of
     & 20, 1862. & Skirmish at Harpeth River, Tenn. \\
     & 21, 1862. & ---Occupation of Franklin, Tenn. \\
     & 31, 1862. & --June 1, 1862.---Engagement at Bethesda, Tenn. \\

\end{toc}

\begin{toc}[ % Reports {{{3
    caption = {Reports, Etc.},
]{}
No. & 1.  & ---Organization of the Department of the Cumberland \\
No. & 2.  & ---Organization of the Army of the Cumberland \\
No. & 3.  & ---Organization of the Army of the Kanawha \\
No. & 4.  & ---Return of casualties in the Union forces after the skirmish at Harpeth River, Tenn., May, 20, 1862. \\
No. & X. & ---Maj. Gen. James Blake, U.S. Army, commanding Department of the Cumberland \\
No. & X. & ---Maj. Gen. James Howard, commanding Army of the Cumberland \\
No. & X. & ---Maj. Gen. Richard Steele, commanding Army of the Kanawha \\
\end{toc}

\subsection{Organization of the Department of the Cumberland, Maj. Gen. James W.  Blake, % {{{3
U.S. Army, commanding, May 20, 1862--MM, DD, 186Y*}
\footnotetext[1]{
    Arranged according to the numerical designation of the corps, divisions and
    brigades as prescribed in General Orders, No.~8, Headquarters, Department of
    the Cumberland, May~20, 1862.
}

\begin{fulloob}
    \staff{Adjutant General}{Col.}{Walter Chekov}

    \corps{Division of Observation}{Brig. Gen.}{Curran Pope} % {{{4

    \begin{leftBde}
        \bde{First Brigade}{Col.}{John Hardin McHenry, Jr.}
        \rgt{41st}{Ohio}{Lieut. Col. George Mygatt}
        \rgt{49th}{Ohio}{Lieut. Col. Albert Blackman}
        \rgt{40th}{Indiana}{Col. William Wilson}
        \rgt{58th}{Indiana}{Col. George Buell}
        \rgt{17th}{Kentucky}{Col. Alexander Stout}
        \rgt{1st}{Michigan, Battery E}{Capt. John Dennis}
    \end{leftBde}
    \begin{rightBde}
        \bde{Second Brigade}{Col.}{Nathan Kimball}
        \rgt{2d}{Ohio}{Col. William Pittenger}
        \rgt{25th}{Ohio}{Col. James A. Jones}
    \end{rightBde}
    \begin{middleBde}
        \bde{Third Brigade}{Col.}{James McMillan}
        \rgt{9th}{Kentucky}{Col. Benjamin Grider}
        \rgt{15th}{Kentucky}{Col. James Forman}
        \rgt{10th}{Michigan}{Col. Charles Lum}
        \rgt{13th}{Indiana}{Col. Jeremiah Sullivan}
        \rgt{21st}{Indiana}{Col. John Keith}
        \rgt{1st}{Ohio, Battery G}{Capt. Joseph Bartlett}
    \end{middleBde}
    \begin{middleBde}
        \bde{Fourth Brigade}{Col.}{Pierce Hawkins}
        \rgt{30th}{Ohio}{Col. Hugh Ewing}
        \rgt{14th}{Indiana}{Col. William Harrow}
        \rgt{11th}{Kentucky}{Col. S. P. Love}
        \rgt{2d}{Illinois, Battery F}{Capt. Charles Keith}
    \end{middleBde}

    \corps{Wool's Division}{Brig. Gen.}{Thomas Wood} % {{{4

    \begin{leftBde}
        \bde{First Brigade}{Col.}{James Shelley}
        \rgt{Middle (5th)}{Tennessee}{Col. Nathaniel Witt}
        \rgt{23d}{Ohio}{Col. Eliakim Scammon}
        \rgt{32d}{Ohio}{Col. Thomas Ford}
        \rgt{7th}{Indiana}{Col. Ira Grover}
    \end{leftBde}
    \begin{rightBde}
        \bde{Second Brigade}{Col.}{James Shackelford}
        \rgt{28th}{Ohio}{Col. August Moor}
        \rgt{64th}{Ohio}{Col. John Ferguson}
        \rgt{21st}{Kentucky}{Col. Ethelbert Dudley}
        \rgt{25th}{Kentucky}{Col. Benjamin Bristow}
    \end{rightBde}
    \begin{middleBde}
        \bde{Third Brigade}{Col.}{Robert Scott}
        \rgt{6th}{Kentucky}{Col. Walter Whitaker}
        \rgt{10th}{Kentucky}{Col. John Harlan}
        \rgt{27th}{Ohio}{Col. John Fuller}
        \rgt{68th}{Ohio}{Col. John Snook}
    \end{middleBde}

    \begin{middleBde}
        \otherbde{Artillery}
        \rgt{8th}{Ohio Battery}{Capt. Louis Markgraf}
        \rgt{2d}{Minnesota Battery}{Capt. William Hotchkiss}
        \rgt{7th}{Indiana Battery}{Capt. George Swallow}
    \end{middleBde}

    \corps{Cavalry Division}{Brig. Gen.}{William Kellogg} % {{{4

    \begin{leftBde}
        \bde{First Brigade}{Col.}{John Farnsworth}
        \rgt{8th}{Illinois}{Col. William Gamble}
        \rgt{12th}{Illinois}{Col. Arno Voss}
        \rgt{15th}{Pennsylvania}{Col. William Palmer}
        \rgt{2d}{Kentucky}{Col. Buckner Board}
    \end{leftBde}
    \begin{rightBde}
        \bde{Second Brigade}{Col.}{Charles Jennison}
        \rgt{6th}{Ohio}{Col. William Lloyd}
        \rgt{3d}{Illinois}{Col. Lafayette McCrillis}
        \rgt{9th}{Pennsylvania}{Col. Edward Williams}
        \rgt{7th}{Kansas}{Col. Daniel Anthony}
    \end{rightBde}
    \begin{middleBde}
        \bde{Third Brigade}{Col.}{Alfred Brackett}
        \rgt{7th}{Illinois}{Col. Edward Prince}
        \rgt{3d}{Michigan}{Col. John Mizner}
        \rgt{3d}{Wisconsin}{Col. William Barstow}
        \rgt{---}{Brackett's Minnesota}{Col. Elias Calkins}
    \end{middleBde}
    \begin{middleBde}
        \otherbde{Artillery}
        \rgt{7th}{Ohio Battery}{Capt. Silas Burnap}
        \rgt{10th}{Indiana Battery}{Capt. Jerome Cox}
    \end{middleBde}

\end{fulloob}
% TODO: Remove requirement for this blank line

\subsecdinkus

\subsection{Organization of the Army of the Cumberland, Maj. Gen. James Howard, % {{{3
commanding, May 20, 1862--MM, DD, 186Y*}
\footnotetext[1]{
    Arranged according to the numerical designation of the corps, divisions and
    brigades as prescribed in General Orders, No.~X, Headquarters, Army of the
    Cumberland, MM~DD, 1862.
}

\begin{fulloob}
    \staff{Chief of Artillery}{Col.}{James Cotter}
    \staff{Adjutant General}{Lieut. Col.}{Tyler Remington}
    
    \corps{Eighth Corps}{Maj. Gen.}{Oskar Gerritsen} % {{{4

    \division{First Division}{Brig. Gen.}{John Wool} % {{{5
    \begin{leftBde}
        \bde{First Brigade}{Col.}{Ralph Buckland}
        \rgt{6th}{Ohio}{Col. William Bosley}
        \rgt{24th}{Ohio}{Col. Frederick Jones}
        \rgt{36th}{Indiana}{Col. William Grose}
        \rgt{3d}{Kentucky}{Col. Thomas Bramlette}
    \end{leftBde}
    \begin{rightBde}
        \bde{Second Brigade}{Col.}{William Hazen}
        \rgt{9th}{United States}{Col. Stephen Carpenter}
        \rgt{9th}{Indiana}{Col. Gideon Moody}
        \rgt{17th}{Indiana}{Col. Milo Hascall}
        \rgt{39th}{Illinois}{Col. William Morrison}
    \end{rightBde}
    \begin{middleBde}
        \bde{Third Brigade}{Col.}{Sanders Bruce}
        \rgt{78th}{Pennsylvania}{Col. William Sirwell}
        \rgt{10th}{Ohio}{Col. William Lytle}
        \rgt{13th}{Ohio}{Col. Joseph Hawkins}
        \rgt{7th}{Kentucky}{Col. Reuben May}
    \end{middleBde}

    \division{Second Division}{Brig. Gen.}{Thomas Crittenden} % {{{5
    \begin{leftBde}
        \bde{First Brigade}{Col.}{Sidney Barnes}
        \rgt{19th}{Ohio}{Col. Charles Manderson}
        \rgt{59th}{Ohio}{Col. James Fyffe}
        \rgt{8th}{Kentucky}{Lieut. Col. James Mayhew}
        \rgt{12th}{Kentucky}{Col. William Hoskins}
    \end{leftBde}
    \begin{rightBde}
        \bde{Second Brigade}{Col.}{William Smith}
        \rgt{31s}{Ohio}{Col. Moses Walker}
        \rgt{33d}{Ohio}{Col. Joshua Sill}
        \rgt{65th}{Ohio}{Col. Charles Harker}
        \rgt{11th}{Indiana}{Lieut. Col. George McGinnis}
    \end{rightBde}
    \begin{middleBde}
        \bde{Third Brigade}{Col.}{John Pope Cook}
        \rgt{3d}{Ohio}{Col. Warren Keifer}
        \rgt{21st}{Ohio}{Col. Dwella Stoughton}
        \rgt{25th}{Indiana}{Col. James Veatch}
        \rgt{31st}{Indiana}{Col. Charles Cruft}
    \end{middleBde}

    \bde{Artillery}{Lieut. Col.}{Peter Simonson} % {{{5
    \begin{leftBde}
        \otherbde{First Division}
        \rgt{1st}{Kentucky, Battery A}{Capt. David Stone}
        \rgt{1st}{Ohio, Battery F}{Capt. Daniel Cockerill}
    \end{leftBde}
    \begin{rightBde}
        \otherbde{Second Division}
        \rgt{4th}{Ohio Battery}{Capt. Louis Hoffman}
        \rgt{1st}{Kentucky, Battery B}{Capt. John Hewitt}
    \end{rightBde}
    \begin{middleBde}
        \otherbde{Reserve Artillery}
        \rgt{5th}{Ohio Battery}{Capt. Andrew Hickenlooper}
        \rgt{9th}{Ohio Battery}{Capt. Henry Wetmore}
    \end{middleBde}

    \corps{Fourteenth Corps}{Maj. Gen.}{John McClernand} % {{{4

    \division{First Division}{Brig. Gen.}{George Thomas} % {{{5
    \begin{leftBde}
        \bde{First Brigade}{Col.}{Samuel Carter}
        \rgt{1st}{Kentucky}{Col. David Enyart}
        \rgt{2d}{Kentucky}{Col. Thomas Sedgewick}
        \rgt{1st}{Tennessee}{Col. Robert Byrd}
        \rgt{13th}{Michigan}{Col. Charles Stuart}
    \end{leftBde}
    \begin{rightBde}
        \bde{Second Brigade}{Col.}{Madison Miller}
        \rgt{10th}{Indiana}{Lieut. Col. William Carroll}
        \rgt{4th}{Kentucky}{Col. John Croxton}
        \rgt{14th}{Ohio}{Col. James Steedman}
        \rgt{17th}{Ohio}{Col. John Connell}
    \end{rightBde}
    \begin{middleBde}
        \bde{Third Brigade}{Col.}{Karl Sonderson}
        \rgt{44th}{Illinois}{Col. Charles Knobelsdorff}
        \rgt{9th}{Ohio}{Lieut. Col. Frank Mattice}
        \rgt{35th}{Ohio}{Col. Ferdinand Van Derveer}
        \rgt{38th}{Ohio}{Lieut. Col. William Choate}
    \end{middleBde}

    \division{Second Division}{Brig. Gen.}{William Rosecrans} % {{{5
    \begin{leftBde}
        \bde{First Brigade}{Brig. Gen.}{Benjamin Prentiss}
        \rgt{15th}{United States}{Lieut. Col. John Kung}
        \rgt{6th}{Indiana}{Col. Philemon Baldwin}
        \rgt{77th}{Pennsylvania}{Col. Frederick Stumbaugh}
        \rgt{79th}{Pennsylvania}{Col. Henry Hambright}
    \end{leftBde}
    \begin{rightBde}
        \bde{Second Brigade}{Col.}{Edward Kirk}
        \rgt{16th}{United States}{Col. Edmund Schriver}
        \rgt{29th}{Indiana}{Col. David Dunn}
        \rgt{30th}{Indiana}{Col. Sion Bass}
        \rgt{34th}{Illinois}{Col. Charles Levanway}
    \end{rightBde}
    \begin{middleBde}
        \bde{Third Brigade}{Col.}{William Gibson}
        \rgt{15th}{Ohio}{Col. Moses Dickey}
        \rgt{5th}{Kentucky}{Col. Harvey Buckley}
        \rgt{32d}{Indiana}{Col. Henry von Trebra}
        \rgt{39th}{Indiana}{Col. Thomas Harrison}
    \end{middleBde}

    \bde{Artillery}{Lieut. Col.}{Charles Muehler}% % {{{5
    \begin{leftBde}
        \otherbde{First Division}
        \rgt{---}{Kentucky, Simmond's Battery}{Capt. Seth Simmonds}
        \rgt{1st}{Ohio, battery D}{Capt. Andrew Konkle}
    \end{leftBde}
    \begin{rightBde}
        \otherbde{Second Division}
        \rgt{4th}{United States}{Lieut. S. Canby}
        \rgt{4th}{Indiana Battery}{Capt. Asahel Bush}
    \end{rightBde}
    \begin{middleBde}
        \otherbde{Reserve Artillery}
        \rgt{4th}{United States, Battery M}{Capt. John Brannan}
        \rgt{1st}{Michigan, Battery A}{Capt. Cyrus Loomis}
    \end{middleBde}

    \corps{Sixteenth Corps}{Brig. Gen.}{James Garfield} % {{{4

    \division{First Division}{Brig. Gen.}{John De Courcey} % {{{5
    \begin{leftBde}
        \bde{First Brigade}{Col.}{Daniel Lindsey}
        \rgt{51st}{Indiana}{Col. Abel Streight}
        \rgt{57th}{Indiana}{Col. Cyrus Hines}
        \rgt{22d}{Kentucky}{Lieut. Col. George Monroe}
        \rgt{16th}{Ohio}{Col. Eli Botsford}
    \end{leftBde}
    \begin{rightBde}
        \bde{Second Brigade}{Col.}{James Craddock}
        \rgt{14th}{Michigan}{Col. Henry Mizner}
        \rgt{44th}{Indiana}{Col. Hugh Reed}
        \rgt{61st}{Illinois}{Col. Jacob Fry}
        \rgt{16th}{Kentucky}{Lieut. Col. James Gault}
    \end{rightBde}
    \begin{middleBde}
        \bde{Third Brigade}{Col.}{Oliver Shepherd}
        \rgt{12th}{Michigan}{Col. Francis Quinn}
        \rgt{3d}{Minnesota}{Col. Henry Lester}
        \rgt{14th}{Kentucky}{Col. Laban Moore}
        \rgt{24th}{Kentucky}{Col. Lewis Grigsby}
    \end{middleBde}

    \division{Second Division}{Brig. Gen.}{Richard Oglesby} % {{{5
    \begin{leftBde}
        \bde{First Brigade}{Col.}{Timothy Stanley}
        \rgt{18th}{Ohio}{Lieut. Col. Josiah Given}
        \rgt{58th}{Ohio}{Col. Valentine Bausenwien}
        \rgt{23d}{Kentucky}{Col. J. P. Jackson}
        \rgt{26th}{Kentucky}{Col. Stephen Burbridge}
    \end{leftBde}
    \begin{rightBde}
        \bde{Second Brigade}{Col.}{Joseph St. James}
        \rgt{18th}{Kentucky}{Col. William Warner}
        \rgt{19th}{Kentucky}{Col. William Landram}
        \rgt{58th}{Illinois}{Col. William Lynch}
        \rgt{22d}{Ohio}{Lieut. Col. Theodore Case}
    \end{rightBde}

    \bde{Artillery}{Capt.}{Charles Willard} % {{{5
    \begin{leftBde}
        \otherbde{First Division}
        \rgt{4th}{United States, Battery G}{Capt. Albion Howe}
        \rgt{10th}{Ohio Battery}{Capt. Hamilton White}
    \end{leftBde}
    \begin{rightBde}
        \otherbde{Second Division}
        \rgt{5th}{Indiana Battery}{Capt. Daniel Chandler}
        \rgt{8th}{Indiana Battery}{Capt. George Estep}
    \end{rightBde}

    \corps{Cavalry Corps}{Maj. Gen.}{Lawrence Graham} % {{{4

    \division{First Division}{Brig. Gen.}{Theophilus Dickey} % {{{5
    \begin{leftBde}
        \bde{First Brigade}{Col.}{John Bridgeland}
        \rgt{1st}{Kentucky}{Col. Silas Adams}
        \rgt{3d}{Kentucky}{Col. James Jackson}
        \rgt{2d}{Indiana}{Lieut. Col. James Stewart}
        \rgt{4th}{United States}{Col. James Oakes}
    \end{leftBde}
    \begin{rightBde}
        \bde{Second Brigade}{Col.}{Charles Doubleday}
        \rgt{2d}{Ohio}{Col. August Kautz}
        \rgt{3d}{Ohio}{Col. Horace Howland}
        \rgt{4th}{Ohio}{Col. Eli Long}
        \rgt{2d}{Wisconsin}{Col. Cadwallader Washburn}
    \end{rightBde}
    \begin{middleBde}
        \bde{Third Brigade}{Col.}{Arthur Rankin}
        \rgt{2d}{Illinois}{Col. Silas Noble}
        \rgt{4th}{Illinois}{Col. Martin Wallace}
        \rgt{13th}{Illinois}{Col. Joseph Bell}
        \rgt{1st}{U.S. Lancers (Michigan)}{Lieut. Col. James Herrick}
    \end{middleBde}

    \division{Second Division}{Brig. Gen.}{Cyrus Bussey} % {{{5
    \begin{leftBde}
        \bde{First Brigade}{Col.}{Edward McCook}
        \rgt{4th}{Kentucky}{Col. Jesse Bayles}
        \rgt{5th}{Kentucky}{Col. David Haggard}
        \rgt{9th}{Illinois}{Col. Albert Barckett}
        \rgt{3d}{Iowa}{Lieut. Col. Henry Caldwell}
    \end{leftBde}
    \begin{rightBde}
        \bde{Second Brigade}{Col.}{Robert Ingersoll}
        \rgt{5th}{Illinois}{Col. Hall Wilson}
        \rgt{11th}{Illinois}{Lieut. Col. Lucien Kerr}
        \rgt{1st}{Wisconsin}{Col. Edward Daniels}
        \rgt{7th}{Pennsylvania}{Col. Charles Davis}
    \end{rightBde}
    \begin{middleBde}
        \bde{Third Brigade}{Col.}{Own Ransom}
        \rgt{5th}{Iowa}{Col. William Lowe}
        \rgt{2d}{Kansas}{Col. Alson Davis}
        \rgt{10th}{Illinois}{Col. James Barrett}
        \rgt{1st}{Ohio}{Lieut. Col. Minor Millikin}
    \end{middleBde}

    \bde{Artillery}{}{} % {{{5
    \begin{leftBde}
        \otherbde{First Division}
        \rgt{1st}{Ohio, Battery E}{Capt. Warren Edgarton}
        \rgt{5th}{United States, Battery H}{Capt. William Terrill}
        \rgt{4th}{United States, Battery I}{Capt. Oscar Mack}
    \end{leftBde}
    \begin{rightBde}
        \otherbde{Second Division}
        \rgt{1st}{Ohio, Battery M}{Capt. Frederick Schultz}
        \rgt{69th}{Ohio Independent Battery}{Capt. Cullen Bradley}
    \end{rightBde}

    \corps{Artillery Reserve}{Col.}{Charles Cotter} % {{{4
    \begin{leftBde}
        \bde{First Brigade}{Lieut. Col.}{William Standart}
        \rgt{1st}{Ohio, Battery A}{Capt. Wilbur Goodspeed}
        \rgt{1st}{Ohio, Battery B}{Capt. J. Hale Snyder}
        \rgt{1st}{Ohio, Battery C}{Capt. Dennis Kenny, Jr}
    \end{leftBde}
    \begin{rightBde}
        \bde{Second Brigade}{Lieut. Col.}{Alonzo Bidwell}
        \rgt{1st}{Michigan, Battery D}{Capt. Josiah Church}
        \rgt{1st}{Illinois, Battery C}{Capt. Charles Houghtaling}
        \rgt{2d}{Illinois, Battery C}{Capt. Caleb Hopkins}
    \end{rightBde}
\end{fulloob}
% TODO: Remove requirement for this blank line

\subsecdinkus

\subsection{Organization of the Army of the Kanawha, Maj. Gen. Richard Steele, % {{{3
    U.S. Army, commanding, May 20, 1862--MM, DD, 186Y}

\begin{fulloob}
    \corps{Ninth Corps}{Brig. Gen.}{Aaron Gates} % {{{4
    \division{First Division}{Brig. Gen.}{}
    \division{Second Division}{Brig. Gen.}{}

    \corps{Eleventh Corps}{Brig. Gen.}{Jacob Ryan} % {{{4
    \division{First Division}{Brig. Gen.}{Benjamin Kelley}
    \division{Second Division}{Brig. Gen.}{Robert Schenck}

    \corps{Twelfth Corps}{Brig. Gen.}{Charles Smith} % {{{4

    \division{First Division}{Brig. Gen.}{William Sherman} % {{{5
    \begin{leftBde}
        \bde{First Brigade}{Col.}{Benjamin Smith}
        \rgt{1st}{Ohio}{Lieut. Col. Joab Stafford}
        \rgt{40th}{Ohio}{Col. Edwin Bradley}
        \rgt{8th}{Illinois}{Col. Richard Rowett}
        \rgt{24th}{Illinois}{Col. Friedrich Hecker}
    \end{leftBde}
    \begin{rightBde}
        \bde{Second Brigade}{Col.}{Thomas Ransom}
        \rgt{15th}{Indiana}{Col. Gustavus Wood}
        \rgt{38th}{Indiana}{Col. Benjamin Scribner}
        \rgt{11th}{Illinois}{Lieut. Col. James Coates}
        \rgt{15th}{Illinois}{Col. Thomas Turner}
    \end{rightBde}
    \begin{middleBde}
        \bde{Third Brigade}{Col.}{Gustavus Smith}
        \rgt{35th}{Illinois}{Col. William Chandler}
        \rgt{21st}{Illinois}{Col. John Alexander}
        \rgt{15th}{Wisconsin}{Col. Hans Heg}
        \rgt{18th}{United States}{Col. Henry Barrington}
    \end{middleBde}

    \division{Second Division}{Brig. Gen.}{Horatio Van Cleve} % {{{5
    \begin{leftBde}
        \bde{First Brigade}{Col.}{Samuel Beatty}
        \rgt{6th}{Michigan}{Col. Frederick Curtenius}
        \rgt{42d}{Indiana}{Col. James G. Jones}
        \rgt{38th}{Illinois}{Col. William Carlin}
        \rgt{2d}{Minnesota}{Col. James George}
    \end{leftBde}
    \begin{rightBde}
        \bde{Second Brigade}{Col.}{William Stoughton}
        \rgt{9th}{Michigan}{Col. William Duffeld}
        \rgt{11th}{Michigan}{Lieut. Col. Melvin Mudge}
        \rgt{15th}{Michigan}{Col. John Oliver}
        \rgt{1st}{Wisconsin}{Col. John Starkweather}
    \end{rightBde}
    \begin{middleBde}
        \bde{Third Brigade}{Col.}{Jefferson Davis}
        \rgt{8th}{Wisconsin}{Col. George Robbins}
        \rgt{10th}{Wisconsin}{Col. Alfred Chapin}
        \rgt{13th}{Wisconsin}{Col. Maurice Maloney}
        \rgt{35th}{Indiana}{Col. Bernard Mullen}
    \end{middleBde}

    \bde{Artillery}{Lieut. Col.}{Charles Humphrey}% % {{{5
    \begin{leftBde}
        \otherbde{First Division}
        \rgt{1st}{Illinois, Battery D}{Capt. Henry Rogers}
        \rgt{2d}{Illinois, Battery E}{Capt. Adolphus Schwartz}
    \end{leftBde}
    \begin{rightBde}
        \otherbde{Second Division}
        \rgt{1st}{Michigan, Battery B}{Capt. William Ross}
        \rgt{3d}{Wisconsin Battery}{Capt. Lu Drury}
    \end{rightBde}
    \begin{middleBde}
        \otherbde{Reserve Artillery}
        \rgt{5th}{Wisconsin Battery}{Capt. George Gardner}
        \rgt{1st}{Michigan, Battery C}{Capt. Alexander Dees}
    \end{middleBde}

    \corps{Seventeenth Corps}{Brig. Gen.}{Robert Milroy} % {{{4
    \division{First Division}{Brig. Gen.}{Rutherford Hayes}
    \division{Second Division}{Brig. Gen.}{Jacob Cox}

    \corps{Cavalry}{Brig. Gen.}{John Hall} % {{{4
    \division{First Division}{Col.}{John Wilder}
    \division{Second Division}{Brig. Gen.}{Thomas J. Mosier}
    \division{Third Division}{Brig. Gen.}{XX}
    \division{Fourth Division}{Brig. Gen.}{XX}

    \corps{Artillery Reserve}{}{} % {{{4
    \otherbde{Artillery Reserve}
\end{fulloob}
% TODO: Remove requirement for this blank line

\subsecdinkus

\subsection{Return of casualties in the Union forces % {{{3
    after the skirmish at Harpeth River, Tenn., May 20, 1862}

\begin{oob}{}
\oobhdr

\oobtop{Cavalry Corps}{Maj. Gen.}{Lawrence Graham} % {{{5

\oobtop{First Division}{Brig. Gen.}{Theophilus Dickey} % {{{6

\oobbde{First Brigade}{Col.}{John Bridgeland} % {{{7
\oobrgt{1st}{Kentucky}{Col. Silas Adams}{225}{\oobnone}{\oobnone}
\oobrgt{3d}{Kentucky}{Col. James Jackson}{225}{\oobnone}{\oobnone}
\oobrgt{2d}{Indiana}{Lieut. Col. James Stewart}{225}{\oobnone}{\oobnone}
\oobrgt{4th}{United States}{Col. James Oakes}{225}{\oobnone}{\oobnone}
\oobtot{Total First Brigade}{900}{\oobnone}{\oobnone}

\oobbde{Second Brigade}{Col.}{Charles Doubleday} % {{{7
\oobrgt{2d}{Ohio}{Col. August Kautz}{225}{\oobnone}{\oobnone}
\oobrgt{3d}{Ohio}{Col. Horace Howland}{225}{\oobnone}{\oobnone}
\oobrgt{4th}{Ohio}{Col. Eli Long}{225}{\oobnone}{\oobnone}
\oobrgt{2d}{Wisconsin}{Col. Cadwallader Washburn}{225}{\oobnone}{\oobnone}
\oobtot{Total Second Brigade}{900}{\oobnone}{\oobnone}

\oobbde{Third Brigade}{Col.}{Arthur Rankin} % {{{7
\oobrgt{2d}{Illinois}{Col. Silas Noble}{300}{\oobnone}{\oobnone}
\oobrgt{4th}{Illinois}{Col. Martin Wallace}{300}{\oobnone}{\oobnone}
\oobrgt{13th}{Illinois}{Col. Joseph Bell}{300}{\oobnone}{\oobnone}
\oobrgt{1st}{U.S. Lancers (Michigan)}{Lieut. Col. James Herrick}{300}{\oobnone}{\oobnone}
\oobtot{Total Third Brigade}{1,200}{\oobnone}{\oobnone}

\oobart{Division Artillery} % {{{7
\oobrgt{1st}{Ohio Light Artillery, Battery E}{Capt. Warren Edgarton}{6}{\oobnone}{\oobnone}
\oobrgt{5th}{United States Artillery, Battery H}{Capt. William Terrill}{6}{\oobnone}{\oobnone}
\oobrgt{4th}{United States Artillery, Battery I}{Capt. Oscar Mack}{6}{\oobnone}{\oobnone}
\oobtot{Total Division Artillery}{18}{\oobnone}{\oobnone}

% Overall Totals % {{{7
\oobsum{First Division}{3,000}{\oobnone}{\oobnone}
\oobdblrule

\oobtop{Second Division}{Brig. Gen.}{Cyrus Bussey} % {{{6

\oobbde{First Brigade}{Col.}{Edward McCook} % {{{7
\oobrgt{4th}{Kentucky}{Col. Jesse Bayles}{300}{\oobnone}{\oobnone}
\oobrgt{5th}{Kentucky}{Col. David Haggard}{300}{\oobnone}{\oobnone}
\oobrgt{9th}{Illinois}{Col. Albert Barckett}{300}{\oobnone}{\oobnone}
\oobrgt{3d}{Iowa}{Lieut. Col. Henry Caldwell}{300}{\oobnone}{\oobnone}
\oobtot{Total First Brigade}{1,200}{\oobnone}{\oobnone}

\oobbde{Second Brigade}{Col.}{Robert Ingersoll} % {{{7
\oobrgt{5th}{Illinois}{Col. Hall Wilson}{300}{\oobnone}{\oobnone}
\oobrgt{11th}{Illinois}{Lieut. Col. Lucien Kerr}{300}{\oobnone}{\oobnone}
\oobrgt{1st}{Wisconsin}{Col. Edward Daniels}{300}{\oobnone}{\oobnone}
\oobrgt{7th}{Pennsylvania}{Col. Charles Davis}{300}{\oobnone}{\oobnone}
\oobtot{Total Second Brigade}{1,200}{\oobnone}{\oobnone}

\oobbde{Third Brigade}{Col.}{Owen Ransom} % {{{7
\oobrgt{5th}{Iowa}{Col. William Lowe}{300}{\oobnone}{\oobnone}
\oobrgt{2d}{Kansas}{Col. Alson Davis}{300}{\oobnone}{\oobnone}
\oobrgt{10th}{Illinois}{Col. James Barrett}{300}{\oobnone}{\oobnone}
\oobrgt{1st}{Ohio}{Lieut. Col. Minor Millikin}{300}{\oobnone}{\oobnone}
\oobtot{Total Third Brigade}{1,200}{\oobnone}{\oobnone}

\oobart{Division Artillery} % {{{7
\oobrgt{1st}{Ohio Light Artillery, Battery M}{Capt. Frederick Schultz}{6}{\oobnone}{\oobnone}
\oobrgt{6th}{Ohio Independent Battery}{Capt. Cullen Bradley}{6}{\oobnone}{\oobnone}
\oobtot{Total Division Artillery}{12}{\oobnone}{\oobnone}

% Overall Totals % {{{7
\oobsum{Second Division}{3,600}{\oobnone}{\oobnone}
\oobdblrule

% % Overall Totals % {{{6
\oobsum{Cavalry Corps}{6,600}{\oobnone}{\oobnone}

% Overall Totals % {{{5
\oobrecap

\oobsub{First Cavalry Division}{3,00}{\oobnone}{\oobnone}
\oobsub{Second Cavalry Division}{3,600}{\oobnone}{\oobnone}

\oobdblrule
\oobsum{Grand total}{6,600}{\oobnone}{\oobnone}
\oobdblrule

\oobsub{Cavalry Corps Artillery}{30}{\oobnone}{\oobnone}

\oobdblrule
\oobsum{Grand total}{30}{\oobnone}{\oobnone}

\bottomrule
\end{oob}

\subsection{Return of casualties in the Union forces % {{{3
    after the engagement at Bethesda, Tenn., May 30--June 1, 1862}

\begin{oob}{}
\oobhdr

\oobtop{Sixteenth Corps}{Brig. Gen.}{Zebulon Benton} % {{{5
& \SetCell{c}First Division, XVIth Corps remained at Franklin, Tenn. and was not present for the engagement.\\
\\ % blank line

\oobdiv{Second Division}{Brig. Gen.}{Richard Oglesby} % {{{6
\oobbde{First Brigade}{Col.}{Timothy Stanley} % {{{7
\oobrgt{18th}{Ohio}{Lieut. Col. Josiah Given}{600}{\oobnone}{\oobnone}
\oobrgt{58th}{Ohio}{Col. Valentine Bausenwien}{600}{\oobnone}{\oobnone}
\oobrgt{23d}{Kentucky}{Col. J. P. Jackson}{600}{\oobnone}{\oobnone}
\oobrgt{26th}{Kentucky}{Col. Stephen Burbridge}{600}{\oobnone}{\oobnone}
\oobtot{Total First Brigade}{2,400}{\oobnone}{\oobnone}

\oobbde{Second Brigade}{Col.}{Joseph St. James} % {{{7
\oobrgt{18th}{Kentucky}{Col. William Warner}{300}{\oobnone}{\oobnone}
\oobrgt{19th}{Kentucky}{Col. William Landram}{300}{\oobnone}{\oobnone}
\oobrgt{58th}{Illinois}{Col. William Lynch}{300}{\oobnone}{\oobnone}
\oobrgt{22d}{Ohio}{Lieut. Col. Theodore Case}{300}{\oobnone}{\oobnone}
\oobtot{Total Second Brigade}{1,200}{\oobnone}{\oobnone}

\oobsum{Total Second Division}{3,600}{\oobnone}{\oobnone} % {{{7
\oobdblrule

\oobdiv{Artillery}{Capt.}{Charles Willard} % {{{6

\oobart{Second Division Artillery} % {{{7
\oobrgt{5th}{Indiana Battery}{Capt. Daniel Chandler}{6}{\oobnone}{\oobnone}
\oobrgt{8th}{Indiana Battery}{Capt. George Estep}{6}{\oobnone}{\oobnone}
\oobtot{Total Second Division Artillery}{12}{\oobnone}{\oobnone}

\oobsum{Total Artillery}{12}{\oobnone}{\oobnone}
\oobdblrule

% Overall Totals % {{{6
\oobsum{Sixteenth Corps}{3,600}{\oobnone}{\oobnone}

\oobtop{Cavalry Corps}{Maj. Gen.}{Lawrence Graham} % {{{5

\oobtop{First Division}{Brig. Gen.}{Theophilus Dickey} % {{{6

\oobbde{First Brigade}{Col.}{John Bridgeland} % {{{7
\oobrgt{1st}{Kentucky}{Col. Silas Adams}{225}{\oobnone}{\oobnone}
\oobrgt{3d}{Kentucky}{Col. James Jackson}{225}{\oobnone}{\oobnone}
\oobrgt{2d}{Indiana}{Lieut. Col. James Stewart}{225}{\oobnone}{\oobnone}
\oobrgt{4th}{United States}{Col. James Oakes}{225}{\oobnone}{\oobnone}
\oobtot{Total First Brigade}{900}{\oobnone}{\oobnone}

\oobbde{Second Brigade}{Col.}{Charles Doubleday} % {{{7
\oobrgt{2d}{Ohio}{Col. August Kautz}{225}{\oobnone}{\oobnone}
\oobrgt{3d}{Ohio}{Col. Horace Howland}{225}{\oobnone}{\oobnone}
\oobrgt{4th}{Ohio}{Col. Eli Long}{225}{\oobnone}{\oobnone}
\oobrgt{2d}{Wisconsin}{Col. Cadwallader Washburn}{225}{\oobnone}{\oobnone}
\oobtot{Total Second Brigade}{900}{\oobnone}{\oobnone}

\oobbde{Third Brigade}{Col.}{Arthur Rankin} % {{{7
\oobrgt{2d}{Illinois}{Col. Silas Noble}{300}{\oobnone}{\oobnone}
\oobrgt{4th}{Illinois}{Col. Martin Wallace}{300}{\oobnone}{\oobnone}
\oobrgt{13th}{Illinois}{Col. Joseph Bell}{300}{\oobnone}{\oobnone}
\oobrgt{1st}{U.S. Lancers (Michigan)}{Lieut. Col. James Herrick}{300}{\oobnone}{\oobnone}
\oobtot{Total Third Brigade}{1,200}{\oobnone}{\oobnone}

\oobart{Division Artillery} % {{{7
\oobrgt{1st}{Ohio Light Artillery, Battery E}{Capt. Warren Edgarton}{6}{\oobnone}{\oobnone}
\oobrgt{5th}{United States Artillery, Battery H}{Capt. William Terrill}{6}{\oobnone}{\oobnone}
\oobrgt{4th}{United States Artillery, Battery I}{Capt. Oscar Mack}{6}{\oobnone}{\oobnone}
\oobtot{Total Division Artillery}{18}{\oobnone}{\oobnone}

% Overall Totals % {{{7
\oobsum{First Division}{3,000}{\oobnone}{\oobnone}
\oobdblrule

\oobtop{Second Division}{Brig. Gen.}{Cyrus Bussey} % {{{6

\oobbde{First Brigade}{Col.}{Edward McCook} % {{{7
\oobrgt{4th}{Kentucky}{Col. Jesse Bayles}{300}{\oobnone}{\oobnone}
\oobrgt{5th}{Kentucky}{Col. David Haggard}{300}{\oobnone}{\oobnone}
\oobrgt{9th}{Illinois}{Col. Albert Barckett}{300}{\oobnone}{\oobnone}
\oobrgt{3d}{Iowa}{Lieut. Col. Henry Caldwell}{300}{\oobnone}{\oobnone}
\oobtot{Total First Brigade}{1,200}{\oobnone}{\oobnone}

\oobbde{Second Brigade}{Col.}{Robert Ingersoll} % {{{7
\oobrgt{5th}{Illinois}{Col. Hall Wilson}{300}{\oobnone}{\oobnone}
\oobrgt{11th}{Illinois}{Lieut. Col. Lucien Kerr}{300}{\oobnone}{\oobnone}
\oobrgt{1st}{Wisconsin}{Col. Edward Daniels}{300}{\oobnone}{\oobnone}
\oobrgt{7th}{Pennsylvania}{Col. Charles Davis}{300}{\oobnone}{\oobnone}
\oobtot{Total Second Brigade}{1,200}{\oobnone}{\oobnone}

\oobbde{Third Brigade}{Col.}{Owen Ransom} % {{{7
\oobrgt{5th}{Iowa}{Col. William Lowe}{225}{75}{25.0}
\oobrgt{2d}{Kansas}{Col. Alson Davis}{225}{75}{25.0}
\oobrgt{10th}{Illinois}{Col. James Barrett}{225}{75}{25.0}
\oobrgt{1st}{Ohio}{Lieut. Col. Minor Millikin}{225}{75}{25.0}
\oobtot{Total Third Brigade}{1,200}{300}{25.0}

\oobart{Division Artillery} % {{{7
\oobrgt{1st}{Ohio Light Artillery, Battery M}{Capt. Frederick Schultz}{6}{\oobnone}{\oobnone}
\oobrgt{6th}{Ohio Independent Battery}{Capt. Cullen Bradley}{6}{\oobnone}{\oobnone}
\oobtot{Total Division Artillery}{12}{\oobnone}{\oobnone}

% Overall Totals % {{{7
\oobsum{Second Division}{3,600}{300}{8.33}
\oobdblrule

% % Overall Totals % {{{6
\oobsum{Cavalry Corps}{6,600}{300}{4.55}

% Overall Totals % {{{5
\oobrecap

\oobsub{Sixteenth Corps}{3,600}{\oobnone}{\oobnone}

\oobdblrule
\oobsum{Grand Total}{3,600}{\oobnone}{\oobnone}
\oobdblrule

\oobsub{First Cavalry Division}{3,00}{\oobnone}{\oobnone}
\oobsub{Second Cavalry Division}{3,600}{300}{8.33}

\oobdblrule
\oobsum{Grand total}{6,600}{300}{4.55}
\oobdblrule

\oobsub{Sixteenth Corps Artillery}{12}{\oobnone}{\oobnone}
\oobsub{Cavalry Corps Artillery}{30}{\oobnone}{\oobnone}

\oobdblrule
\oobsum{Grand total}{42}{\oobnone}{\oobnone}

\bottomrule
\end{oob}

\subsection{Reports of Maj. Gen. James Blake, Department of the Cumberland} % {{{3

\gramHeader{Headquarters, Dep't of the Cumberland} % {{{4
{Franklin, Tenn., May}{26, 1862}
\gramTo{Maj. Gen.}{Cornelius Van Royne}
{Commanding General, United States Army}

\gramHi{Sir} My cavalry skirmished with cavalry from the Army of Tennessee on
the 20th inst., causing an estimated 300 losses to the enemy to only a few of
our own. The enemy withdrew south, burning bridges as they went. The Army of the
Cumberland was able to advance to Duck River just north of Columbia, Tenn.

As Columbia was at the limit of supply until the bridges can be repaired and
concerned that an enemy movement north through Murfreesborough would cut me off
from Nashville, I left P. Smith's corps at Duck River and withdrew the remainder
to Franklin.

I cannot ascertain anything about the strength of the enemy at Columbia except
that infantry is present. If significant force has been pulled away west, I am
unaware.

Gen'l Steele is now assembled around Clarksville; I will forward his anticipated
landing date when able.

\gramClosing{Respectfully}
{J. W. Blake}
{Maj. Gen., Commanding}

Copies to Meyer \& Steele.

\subsecdinkus

\subsection{Reports of Maj. Gen. Richard Steele, Army of the Kanawha} % {{{3

\subsecdinkus

\subsection{Reports of Maj. Gen. James Howard, Army of the Kentucky} % {{{3

\subsecdinkus

\section[Correspondence, Etc.---January 1--March 31, 1862] % {{{2
{Correspondence, Orders, and Returns Relating to Operations in
    Missouri, Kentucky and Tennessee from January 1, 1862, to March 31, 1862}{}

\subsection*{January 1, 1862}{} % {{{3
\gramOrdersHeader{Headquarters, Army of the Cumberland} % {{{4
{Evansville, Ind., January}{1, 1862}
{General Orders}{1}

In pursuance of instructions from Maj. Gen. Van Royne, commanding United States
Army, the undersigned hereby assumes command of the Army of the Cumberland.

\gramClosing{}
{J. W. Blake}
{Maj. Gen., U. S. Army}
\reportdinkus

\gramHeader{Headquarters, Army of the Cumberland} % {{{4
    {Evansville, Ind., Morning, January}{1st, 1862}
\gramTo{Maj. Gen.}{James W. Blake}
    {Commanding General, Army of the Cumberland}

\gramHi{General} I have the honor to lay before you the latest intelligence
regarding the forces of the enemy now arrayed in this theater. It is credibly
reported that the Army of the Tennessee, commanded by Maj. Gen. John Jackson,
lies concentrated at Bowling Green, numbering about seventeen thousand troops of
all arms.

The Army of the Mississippi, under Maj. Gen. Alexander Clarke, is entrenched at
Columbus. The strength of this host is judged to be thirty-six thousand
infantry, seven thousand cavalry, and between ninety and one hundred guns.
Considerable works are confirmed not only at Columbus itself but likewise at
Forts Henry and Donelson, commanding the Tennessee and Cumberland Rivers.

It is further believed that Maj. Gen. Robert Anderson exercises general
authority over Confederate forces in the Western Department, though the
direction of the armies named rests immediately with Generals Jackson and
Clarke.

At present, the enemy's formations remain stationary, and no certain indications
of their intended movements have yet been observed. I shall continue to gather
and transmit such intelligence as may be obtained.

\gramClosing{I am, General, very respectfully, Your obedient servant}
    {Col. Walter Chekov}
    {Adjutant General, Army of the Cumberland}
\subsecdinkus

\subsection*{January 15, 1862}{} % {{{3
\gramHeader{Headquarters, Army of the Cumberland} % {{{4
    {Evansville, Ind., January}{15th, 1862}
\gramTo{Maj. Gen.}{Thomas Smith}
    {Commanding General, Army of the Tennessee}

\gramHi{General} The commanding general is troubled by reports of a depot and
fortifications being established at Paducah and then abandoned to the enemy.
Please advise if there is any veracity to this claim, or whether the newspapers
are guilty of their usual flights of fancy. What is your situation, intelligence
and intention at this time?

\gramClosing{I am your most obedient servant}
    {Harold Fawcett, III}
    {Brigadier General, Chief of Staff, Army of the Cumberland}
\subsecdinkus

\gramHeader{Headquarters, Army of the Tennessee} % {{{4
    {Ohio City, Ohio, January}{15th, 1862}
\gramTo{Maj. Gen.}{Harold K Fawcett III}
    {Army Of The Cumberland}

\gramHi{General} To my current information the raid on Paducah was just that, a
raid, my chief of intelligence reports it was at most a brigade. I am put at an
odd position caught between my orders and the enemy, I do believe assistance
from Gen. Meyer may be required as I turn my army northward to set the governors
at ease.

\gramClosing{I am your most obedient servant}
    {Thomas Smith}
    {Major General, Commander, Army Of The Tennessee}
\subsecdinkus

\gramHeader{United States Army} % {{{4
    {City of Washington, January}{15, 1862}
\gramTo{Maj. Gens.}{Christopher Stoeffler, James Blake, Thomas Smith, \& Karl Meyer}
    {Commanding depts. of the Kentucky, Cumberland, Tennessee, \& Arkansas respy.}

\gramHi{Sir} I wish to hear from you on the progress of your campaigns.  The
mood in the capital is anxious, and the Congress grows impatient and frustrated
with initial, if minor, rebel successes.

\gramClosing{I remain vy rspy yours \&c}
    {C. N. Van Royne}
    {Major General cmdg, USA}
\subsecdinkus

\gramHeader{Headquarters, Army of the Arkansas} % {{{4
    {St. Louis, Mo., Afternoon, January}{15th, 1862}
\gramTo{Maj. Gen.}{Cornelius Van Royne}
    {Commanding General of the United States Army}

\gramHi{General} I humbly report that the Army of the Arkansas has reached its
initial starting locations. The XII Corps and the Cavalry Division are at
Ironton, and XV Corps has made it to St. Luke. No progress on extending the rail
road has occurred as the men are still setting up patrols and getting the lay of
the land. Rolla and Jefferson City are each garrisoned with one brigade from the
Division of Observation.

The pickets of the XV Corps have also made contact with pickets from XVI Corps
of the Army of Tennessee, with General Smith's forces near Matthews Prairie to
Ohio City. This leaves White Water and its swamps between the majority of my
forces and General Smith. I intend to send couriers to General Smith in order to
coordinate with the his forces on the west side of the Mississippi. I am also
concerned about the reports about the loss of Paducah. If necessary, I can
divert my own forces to take up more of the burden of taking New Madrid should
portions of General Smith's forces need to be shifted back to the east bank.

Initial reports are conflicting in regards to nearby enemy activity along the
river and to our south. In order to get a better picture of what is happening to
our south, I intend to strip most of the Cavalry Division from Ironton save for
a regiment or two, with the purpose of sending them south to make contact with
the enemy.

I have not received any word from Rolla nor Jefferson City, but believe that
construction on Fort Ellsworth to be in progress. If you deem it necessary, I
will send Captain Louis to lay eyes on the situation and report back to me.

I shall endeavor to keep you promptly informed of any alteration in the state of
affairs.

\gramClosing{I am always, General, and shall ever remain, Your most humble and obedient of servants}
    {Karl Meyer}
    {Maj. Gen., Army of the Arkansas, cmdg.}
\subsecdinkus

\gramHeader{Headquarters, Army of the Arkansas} % {{{4
    {St. Louis, Mo., Afternoon, January}{15th, 1862}
\gramTo{Maj. Gen.}{Thomas Smith}
    {Commanding General of the Army of the Tennessee}

\gramHi{General} My scouts from the XV Corps have reported that they have made
contact with elements of your army near Matthews Prairie. I would like to ask in
what numbers you have made the crossing and when you intend to strike south
towards New Madrid. I currently have 24 regiments of infantry and seven
batteries of artillery located in the vicinity of St. Luke.

Equally as important is coordinating patrols so that the enemy may not find a
gap between us to exploit, and also to avoid wasting time covering the same
ground that the other army's pickets have trodden. I would suggest.the White
Water to be the line which my most eastern of pickets will patrol.

If necessary and with permission from General Van Royne, I can shift additional
elements of the XV and XIII Corps to help with the assault on New Madrid. The
journey from St. Luke is 29 miles across roads if the map is to believed, or
roughly three days, four at most. I would also appreciate confirmation as to the
status of the city of Paducah.

\gramClosing{I am always, General, and shall ever remain, Your most humble and obedient of servants}
    {Karl Meyer}
    {Maj. Gen., Army of the Arkansas, cmdg.}
\subsecdinkus

\gramHeader{Headquarters, Army of the Arkansas} % {{{4
    {St. Louis, Mo., Morning, January}{15th, 1862}
\gramTo{Maj. Gen'ls}{Cornelius Van Royne and Thomas Smith}
    {Cmdg Gen. of the United States Armies and of the Tennessee respy}

\gramHi{Generals} As of a few hours ago, scouts of the XV Corps have reported
that there is indeed a rebel depot at New Madrid. The size of the enemy
defending the town is unknown but I would expect at least a brigade so as to not
allow the important depot to fall so easily.

\gramClosing{I am always, General, and shall ever remain, Your most humble and obedient of servants}
    {Karl Meyer}
    {Maj. Gen., Army of the Arkansas, cmdg.}
\subsecdinkus

\gramHeader{Headquarters, Army of the Kentucky} % {{{4
    {Louisville, Ky., Afternoon, January}{15th, 1862}
\gramTo{Maj. Gen.}{Cornelius Van Royne}
    {Commanding General of the United States Army}

\gramHi{General} Enemy cavalry has used the delay to our operations due to the
winter weather to push aggressively up the rail line towards Louisville.  Our
own cavalry has been inadequate to stop their advance.

The enemy cavalry force is now located at the tunnel through the hills south
west of Sheperdsville. We fear they may attempt to blow it, which would
significantly delay any further operations and may see us divert to the green
river.

The initial redeployment of the infantry corps was not successful and they
remain concentrated around Louisville. I will march at speed in an attempt to
again secure the tunnel. I will reassess the situation on the ground and decide
whether to repair the tunnel/find a workaround or instead use the green.

There are no reports from the pickets in the south east.

\gramClosing{I am your most obedient servant,}
    {Christopher Stoeffel}
    {Major General, Commanding, Army of the Kentucky}
\subsecdinkus

\gramHeader{Army of the Tennessee} % {{{4
    {Cairo, Ill., Morning, January}{15th, 186}
\gramTo{Maj. Gen.}{Cornelius Van Royne}
    {Cmdg Gen. US Army}

\gramHi{General} To my current information the raid on Paducah was just a minor
raid. I am put at an odd position caught between my orders and this disturbance
however minor, I firmly believe that I can not both secure Paducah and continue
on current orders. Would you prefer me to secure Paducah in a long march or
continue on current orders? We have made steady time.

\gramClosing{I am your most obedient servant}
    {Major General Thomas Smith}
    {Commander, Army Of The Tennessee}
\subsecdinkus

\gramHeader{United States Army} % {{{4
    {City of Washington, January}{15, 1862}
\gramTo{Maj. Gen.}{Christopher Stoeffel}
    {Commanding, Army of the Kentucky}

\gramHi{Sir} Yours of this date just received. It must be impressed upon you the
seriousness of the need for your command to decamp at the earliest possible
moment toward Bowling Green. The loss of Paducah has become intolerable to the
civil administration and the Congress, and the loss of the Louisville \&
Nashville through Elizabethtown, on the doorstep of Shephersdville, will
similarly become intolerable to this head-quarters, and significantly delay
future progress in securing the Commonwealth from the rebellion.

You must force the enemy from his position by whatever means, using the southern
passes along Muldraugh's Hill east of Hodgensville as necessary. General Jackson
must be kept busy, and gains eventually made, such that General Blake may tend
to his own duties along the Cumberland and Tennessee.

\gramClosing{I remain vy rspy yours \&c}
    {C. N. Van Royne}
    {Major General, USA, cmdg}
\subsecdinkus

\subsection*{January 18, 1862}{} % {{{3
\gramHeader{United States Army} % {{{4
    {City of Washington, January}{18, 1862}
\gramTo{Maj. Gen.}{Thomas Smith}
    {Commanding Dep't of the Tennessee}

\gramHi{Sir} Yrs the 15th inst. just received. Paducah falls into your
department, and is therefore your immediate responsibility; but I do not think
the recent raid, which has received outsized attention in the papers for want of
other news, need be placed at the feet of any one person. In any event such
misplacement of troops and attention should estimably be the responsibility of
the Army.

Your current mission, that is to take New Madrid and, if practicable, Island No.
5, is to be your primary object in spite of this development. General Blake's
cutting movement onto enemy soil shall provide all the animus necessary for the
enemy to return Paducah to our control. The enemy will not win the war by the
seizure of small points on the Ohio River for but a few days, in spite of what
the papers may say. General Meyer has offered, and will be instructed, to render
some assistance to you in the movement against New Madrid.

\gramClosing{I remain vy rspy yours \&c}
    {C. N. Van Royne}
    {Major General, USA, cmdg}
\subsecdinkus

\subsection*{January 19, 1862}{} % {{{3
\gramHeader{Headquarters, Army of the Arkansas} % {{{4
    {St. Louis, Mo., Morning, January}{19th, 1862}
\gramTo{Maj. Gens.}{Cornelius Van Royne and Thomas Smith}
    {Cmdg Gen. of the United States Army and of the Tennessee respy}

\gramHi{Generals} I wish to inform you that the Cavalry Division under Brigadier
General John Louis has been ordered to make its way to St. Louis, where it will
board steamships bound for Ohio City. The Division should arrive in two to three
days, at which point they will board the train bound for Sikeston. The 1st and
2d Brigades with the artillery will debark at the town, while the 3d Brigade
will continue on to St. Luke, where they will move south to scout out the area
for enemy forces.

Roughly around the same time as the Cavalry arrives at Ohio City, two brigades
of Brigadier General Norman Ambrose ['s XV Corps] with the divisional and army
artillery, will be heading to Sikeston from St. Luke. It should take both
forces approximately two to four hours to reach Sikeston. Once Generals Louis and
Ambrose have joined forces, the four brigades will march down the road to New
Madrid by way of the road passing Winchester. This should take another three to
four days.

It is my hope that by the start of 1st prox., the town of New Madrid will be in
Union hands, baring anything short of significant defensive works or a corps
occupying the town.

\gramClosing{I am always, General, and shall ever remain, Your most humble and obedient of servants}
    {Karl Meyer}
    {Maj. Gen. cmdg, Army of the Arkansas}
\subsecdinkus

\gramHeader{Headquarters, Army of the Arkansas} % {{{4
    {St. Louis, Mo., Morning, January}{19th, 1862}
\gramTo{Brig. Gen.}{Thomas Caldwell}
    {Army of the Tennessee}

\gramHi{General} In the off chance that General Smith does not receive my
telegram, I am electing to inform you that the Army of the Arkansas' Cavalry
Division will be arriving in Ohio City in approximately three days.

They will then board a train bound for Sikeston where they will meet with two
brigades of the XV Corps and proceed to march on New Madrid with two brigades of
infantry, two brigades of cavalry, and six batteries of artillery.

I expect to be holding New Madrid within seven days baring significant defensive
works or an entire corps worth of troops.

\gramClosing{I am always, General, and shall ever remain, Your most humble and obedient of servants}
    {Karl Meyer}
    {Maj. Gen. cmdg, Army of the Arkansas}
\subsecdinkus

\subsection*{January 23, 1862}{} % {{{3
\gramHeader{Headquarters, Army of the Cumberland} % {{{4
    {Smithland, Ky., January}{23, 1862}
\gramTo{Maj. Gen.}{Cornelius Van Royne,}
    {United States Army, Commanding}

\gramHi{General} In order to avoid another Paducah, I intend to garrison more
locations than originally intended. Please inform me if the following meets your
approval.

One regiment will be placed in Evansville as I expect nothing more than a
cavalry raid and even that is unlikely as we control the Ohio River. One brigade
will be placed in Smithland. It is not at any great risk, protected as it is by
the three rivers but it is the source of my supply.

Ideally a combined force drawn from army reserves or a combination of my and
Maj. Gen. Smith's armies could garrison both Smithland and Paducah together,
perhaps along with Cairo as well. Once the advance into Kentucky and Tennessee
is well under way it may be wise to shift responsibility for garrisoning rear
areas to another command entirely.

I plan no further garrisons outside of my line of supply and critical locations
as we advance south.

\gramClosing{Respectfully}
    {J. W. Blake}
    {Maj. Gen., Commanding}
\subsecdinkus

\gramHeader{Headquarters, Army of the Cumberland} % {{{4
    {Smithland, Ky., January}{23, 1862}
\gramTo{Maj. Gen.}{Cornelius Van Royne,}
    {United States Army, Commanding}

\gramHi{General} In response to reports of enemy cavalry raiding across western
Kentucky and the fall of Paducah, my own cavalry division headed south a full
three weeks early while my VIII Corps has already arrived in Smithland. An
inconclusive skirmish was fought at Golden Pond on the 6th instant between my
cavalry and an enemy cavalry division. The advance has been halted for the
moment.

Our navy reports a score of guns at Forts Henry and Donelson. Between this and
the cavalry division at Golden Pond I expect the forts to be at least moderately
held.

Enemy cavalry has also been sighted across the Ohio River from Evansville. I
believe they will withdraw once the Army of the Kentucky begins its advance but
with the enemy pressing against the Ohio to my east and west the navy will be
busy patrolling to prevent a raid across the river.

I intend to leave small garrisons at Smithland and Evansville but otherwise my
plan remains the same, although I expect movement to begin a week earlier than
intended.

\gramClosing{Respectfully}
    {J. W. Blake}
    {Maj. Gen., Commanding}
\subsecdinkus

\subsection*{January 24, 1862}{} % {{{3
\gramHeader{United States Army} % {{{4
    {City of Washington, January}{24, 1862}
\gramTo{Maj. Gen.}{James Blake}
    {Commanding Dep't of the Cumberland}

\gramHi{Sir} Yours both of the 23d inst. just received. Continue to expedite at
best speed the transport of the rest of your command onto Kentucky soil.

You may wish to consider a combined movement up both banks of the Cumberland,
with gunboats on both that river and the Tennessee, so as to neutralize Henry
and Donelson simultaneously. If such places are indeed held with fortifications
and batteries, it is likely there is a force disposed to defend both on the
landward side. A movement to cross the Cumberland upriver of Dover with some
force may compel the enemy to abandon both works outright. Receive these words
merely to inspire thought on how to reduce the forts, rather than as orders. You
enjoy our confidence.

General Stoeffel has been delayed by the activities of enemy cavalry and poor
weather, and has been ordered to move upon Bowling Green without delay as soon
as fair judgment permits him.

There appears to be little intelligence, if any at all, on whether General
Clarke has decamped from Columbus with any body of his force. At the least,
General Jackson has dispatched some or perhaps most of his cavalry northward
toward Elizabethtown.

We are certain that a serious movement against Henry and Donelson, and by that
path Nashville, will compel not only the cession of the enemy occupation of
Paducah, but possibly even the works at Columbus itself, and the enemy position
at Bowling Green.

\gramClosing{I remain vy rspy yours \&c}
    {C. N. Van Royne}
    {Major General cmdg, USA}
\subsecdinkus

\gramHeader{United States Army} % {{{4
    {City of Washington, January}{24, 1862}
\gramTo{Maj. Gen.}{James Blake}
    {Commanding Dep't of the Cumberland}

\gramHi{Sir} Addendum to last. Loss of Paducah may complicate movement of
gunboats up the Tennessee, but naval assistance may not be necessary to take
Henry. Worthy of further investigation as you push further south. In any event,
combined movements of yours and Gen. Smith's command should compel enemy
evacuation of Paducah.

\gramClosing{I remain vy rspy yours \&c}
    {C. N. Van Royne}
    {Major General cmdg, USA}
\subsecdinkus

\gramHeader{Headquarters, Army of the Cumberland} % {{{4
    {Smithland, Ky., January}{24, 1862}
\gramTo{Maj. Gen.}{Cornelius Van Royne}
    {United States Army, Commanding}

\gramHi{General} The navy reports that river traffic near Paducah and Evansville
is only receiving intermittent and mild interruptions from rebel guns.  Unless
the enemy emplaces larger guns the impact to our operations should be minimal.

\gramClosing{Respectfully}
    {J. W. Blake}
    {Maj. Gen., Commanding}
\subsecdinkus

\subsection*{January 25, 1862}{} % {{{3
\gramHeader{United States Army} % {{{4
    {City of Washington, January}{25, 1862}
\gramTo{Maj. Gen.}{Karl Meyer}
    {Commanding, Dep't of the West}

\gramHi{Sir} Yrs of this date just received. Remember that it is Gen.  Thomas'
objective to seize New Madrid and cut off the upper Mississippi to the enemy.
You may aid him in whatever manner you best see fit with the aforementioned
detachment of troops, and may exercise command by seniority in the field should
you see fit, but are of course reminded to tend also to your own department, and
to not get too carried away with affairs furthest to your east.

\gramClosing{I remain vy rspy yours \&c}
    {C. N. Van Royne}
    {Major General, USA, cmdg}
\subsecdinkus

\subsection*{January 26, 1862}{} % {{{3
\gramHeader{United States Army} % {{{4
    {City of Washington, January}{26, 1862}
\gramTo{Maj. Gen.}{Thomas Smith}
    {Commanding Dep't of the Tennessee}

\gramHi{Sir} Yrs of this date just received. How many of infantry \& cavalry
were lost? Please also provide estimate on the enemy's strength \& organization.
Do you believe the enemy intent on pursuing you?

\gramClosing{I remain vy rspy yours \&c}
    {C. N. Van Royne}
    {Major General, USA, cmdg}
\subsecdinkus

\gramHeader{United States Army} % {{{4
    {City of Washington, January}{26, 1862}
\gramTo{Maj. Gen.}{Thomas Smith}
    {Commanding Dep't of the Tennessee}

Addendum. Has any attempt been made on securing Island No. 5 and establishing a
battery controlling the length of the Mississippi there?

\gramClosing{I remain vy rspy yours \&c}
    {C. N. Van Royne}
    {Major General, USA, cmdg}
\subsecdinkus

\gramHeader{United States Army} % {{{4
    {City of Washington, January}{26, 1862}
\gramTo{Maj. Gen.}{Thomas Smith}
    {Commanding Dep't of the Tennessee}

\gramHi{Sir} We have begun to hear that there was a battle in which you were
engaged. I wish to hear from you, and, if this is true, a report on the
engagement, its result, and any losses sustained.

\gramClosing{I remain vy rspy yours \&c}
    {C. N. Van Royne}
    {Major General, USA, cmdg}
\subsecdinkus

\gramHeader{Army Of The Tennessee} % {{{4
    {City of Charleston, January}{26, 1862}
\gramTo{Maj. Gen.}{C. N. Van Royne}
    {Commander, US Army}

\gramHi{Sir} Four regiments of infantry and two of cavalry were lost from my
army. The enemy seems to have at least a division but we can't accurately
estimate. They are harassing our rearguard but I don't know if it's some
ragamuffins  intent on roughing up some pickets or an actual committed force.
The work on Number 5 has not begun yet as my orders were misinterpreted.  I
believe with your consent I may move New Madrid to a next month goal and instead
focusing on Number 5 and training my army which has become apparent after the
last battle is untrained and nervous under fire.

\gramClosing{I remain vy rspy yours}
    {Thomas Smith}
    {Commander, Army Of The Tennessee}
\subsecdinkus

\gramHeader{United States Army} % {{{4
    {City of Washington, January}{26, 1862}
\gramTo{Maj. Gen.}{Thomas Smith}
    {Commanding Dep't of the Tennessee}

\gramHi{General} Yours of this date received. The effort to take Island No. 5
must begin without delay. Your cavalry now numbers 1,800 men, and there are
estimates the enemy at New Madrid has no fewer than 4,800 troopers, and likely
5,400, and possibly more. If orders to take Island No. 5 were misunderstood, I
should like to know by whom and the reasoning for such misunderstanding. The
army and country can no longer tolerate any such misunderstandings if the Union
is to be restored. Inquiries by the nat'l gov't are intensifying and we stand to
lose the faith of the general public if we are met by further disaster.

I will sustain you as best my means allow, but I must stress that I cannot long
tolerate, by any dept commander, missteps that will cost the country 3,000 men
with no profit whatsoever; not because I am not without understanding, but
because the country simply cannot accept it.

\gramClosing{I remain vy rspy yours \&c}
    {C. N. Van Royne}
    {Major General, USA cmdg}
\subsecdinkus

\gramHeader{Headquarters, Army of the Arkansas} % {{{4
    {St. Louis, Mo., Morning, January}{26th, 1862}
\gramTo{Maj. Gens.}{Cornelius Van Royne and Thomas Smith,}
    {Cmdg Gen. of the United States Army and of the Tennessee respy}

\gramHi{Generals} With news of the Battle of New Madrid and the rough
disposition of the enemy ascertained. I am holding off from marching on the
town. Especially with only two brigades of infantry and two brigades of cavalry.

I have received word that the Rebel cavalry force numbered over five thousand,
with likely infantry support of over four thousand. Elements known to have been
present are the 2d Division and Cavalry Division of the Confederate Army of the
Trans Mississippi.

I am again, requesting that General Smith reach out to me to coordinate any
further attacks. If the good general had simply waited for my brigades to
arrive, it is possible that the extra division worth of soldiers may have been
the deciding factor. Especially considering that between the Army of the
Tennessee and the Army of the Arkansas, we would have most certainly held the
advantage in the number of cavalry present on the field and would have been able
to scout their disposition far more effectively. The truth of the matter is that
I was not aware that General Smith was even intending to attack New Madrid as he
has given me no such prior warning despite reaching out to him about my own
attack.

I understand General Van Royne, that New Madrid is part of General Smith's
Department, but I cannot afford leave the rail line that supply's my force so
vulnerable against what is at the very least six regiments of infantry and
sixteen to seventeen cavalry regiments. These would essentially be operating in
my rear lines as I move down the Black River in the months to come.

I am once again requesting that General Smith contact me for future plans on
moving on the enemy town. I will be moving to Ohio City as soon as possible to
meet with the good General in person.

\gramClosing{I am always, General, and shall ever remain, Your most humble and obedient of servants}
    {Karl Meyer}
    {Maj. Gen. cmdg, Army of the Arkansas}
\subsecdinkus

\gramHeader{United States Army} % {{{4
    {City of Washington, January}{26, 1862}
\gramTo{Maj. Gen.}{Karl Meyer}
    {Commanding, Dep't of the West}

\gramHi{General} Yours of this date just received. Are the brigades of infantry
from the Division of Observation? What is their precise location, along with the
detached cavalry brigades?

\gramClosing{I remain \&c}
    {C. N. Van Royne}
    {Major General, USA cmdg}
\subsecdinkus

\gramHeader{Headquarters, Army of the Arkansas} % {{{4
    {St. Louis, Mo., Morning, January}{26th, 1862}
\gramTo{Maj. Gen.}{Cornelius Van Royne}
    {Cmdg Gen. of the United States Army}

\gramHi{General} I have received your message. The two brigades of infantry are
from the XV which is already at St. Luke. They are being ordered back to St.
Luke. The Cavalry Division was previously at Ironton with the XIII Corps to
patrol the head of navigation of the Current River. The entire Division is being
ordered to continue on to St. Luke instead of stopping at Sikeston.

Half of the Division of Observation is still at Rolla while the other half is at
Jefferson City.

\gramClosing{I am always, General, and shall ever remain, Your most humble and obedient of servants}
    {Karl Meyer}
    {Maj. Gen. cmdg, Army of the Arkansas}
\subsecdinkus

\gramHeader{United States Army} % {{{4
    {City of Washington, January}{26, 1862}
\gramTo{Maj. Gen.}{Karl Meyer}
    {Commanding, Dep't of the West}

\gramHi{General} Yours of this date just received. For the time being, you will
detach the XV Corps brigades of infantry, along with all the Saint Luke cavalry,
for service with the Army of the Tennessee. They will report to Gen.  Smith
until such time as they are no longer required, hopefully by the end of March.
Focus your energies for now on securing southern Missouri and ascertaining, by
whatever means reasonable and available to you, the disposition of enemy forces
in the remainder of Arkansas along the state line. I will be in communication
with Gen. Smith on this matter.

\gramClosing{I remain vy rspy yr obt svt \&c}
    {C. N. Van Royne}
    {Major General, USA, cmdg}
\subsecdinkus

\gramHeader{United States Army} % {{{4
    {City of Washington, January}{26, 1862}
\gramTo{Maj. Gen.}{Thomas Smith}
    {Commanding Dep't of the Tennessee}

\gramHi{General} You will be in receipt of two brigades each of infantry and
cavalry, detached from the Dep't of the West. They are currently headed
toward or away from Saint Luke. You will dispose of them as you require for the
operations against Island No. 5 and New Madrid, and will be remanded back to the
Dep't of the West at the conclusion of the season [by end of March].

You are also reminded to respond to the communications of Gen. Meyer when you
are in receipt of them. A lack of communication can prove fatal to our
operations in the region.

\gramClosing{I remain vy rspy yours \&c}
    {C. N. Van Royne}
    {Major General, USA, cmdg}
\subsecdinkus

\gramHeader{Headquarters, Army of the Arkansas} % {{{4
    {St. Louis, Mo., Morning, January}{26th, 1862}
\gramTo{Maj. Gen.}{Cornelius Van Royne}
    {Cmdg Gen. of the United States Army}

\gramHi{General} I have received your message. I will abide by your decision but
must inform you that any sort of ascertaining of the enemy forces without a
cavalry force is not possible in my department. We simply do not have the supply
to stretch out our infantry that far from the rail lines. It will also take
another week to move the XIII Corps from Ironton to replace the XV Corps at St.
Luke. I do not wish to leave the rail construction unguarded and so the transfer
will have to be done on the 2d prox.

I wish to confirm that you desire for me to detach the entire XV Corps of six
brigades of infantry from my command as well as the entire Cavalry Division of
three brigades which are currently located at St. Luke.

\gramClosing{I am always, General, and shall ever remain, Your most humble and obedient of servants}
    {Karl Meyer}
    {Maj. Gen., cmdg, Army of the Arkansas}
\subsecdinkus

\gramHeader{United States Army} % {{{4
    {City of Washington, January}{26, 1862}
\gramTo{Maj. Gen.}{Karl Meyer}
    {Commanding, Dep't of the West}

\gramHi{General} Yours in response to mine of this date just received.  The two
brigades you dispatched to St. Luke are the brigades in question to be detached
for service with the Dep't of the Tennessee, as are the two brigades of cavalry
sent for the same purpose. The remaining four regiments of the Cavalry Division
may be reformed into two smaller brigades for your purposes. General Smith is
outnumbered in horseflesh by a factor of four and will be rendered completely
incapable of offensive operations without such support.

Those four combined brigades, having fallen under Gen. Smith's command, will
have the responsibility of their supply fall to his department for the duration
of the detachment, until they are again returned to you.

With fewer troops in your department, you may find the strains of supply
requirements much eased, and perhaps even find yourself freer in action and
ability thereby.

\gramClosing{I remain vy rspy yr obt svt \&c}
    {C. N. Van Royne}
    {Major General, USA, cmdg}
\subsecdinkus

\gramHeader{Headquarters, Army of the Arkansas} % {{{4
    {St. Louis, Mo., Morning, January}{26th, 1862}
\gramTo{Maj. Gen.}{Cornelius Van Royne}
    {Cmdg Gen. of the United States Army}

\gramHi{General} In response to the prior message. I thank you for the
clarification. I will order the two brigades of infantry and two brigades of
cavalry to General Smith post haste. It may still take a few days for them to
arrive in Matthew's Prairie for General Smith's use.

I wish to bring up that even with my contributions, with the General Smith's
loss of two cavalry regiments, it is highly likely that he will be outnumbered
in terms of cavalry by at least two to three regiments if not more. The reports
that arrived to me suggested the enemy division had over 5,000 cavalry troopers.
It may benefit him that we move in tandem upon New Madrid when he tries again,
so that he may match the enemy in terms of cavalry and hopefully outnumber them
in terms of infantry.

However, I lay the greater planning of this war in your hands and will
concentrate on continuing to extend the rail road towards Poplar Bluff as is my
remit.

\gramClosing{I am always, General, and shall ever remain, Your most humble and obedient of servants}
    {Karl Meyer}
    {Maj. Gen., cmdg, Army of the Arkansas}
\subsecdinkus

P.S.---Two of my cavalry regiments are at Ironton. With the loss of
the 1st and 2d Cavalry Brigades, my operations in the south will only have two
regiments available.

\gramHeader{United States Army} % {{{4
    {City of Washington, January}{26, 1862}
\gramTo{Maj. Gen.}{Christopher Stoeffel}
    {Commanding, Army of the Kentucky}

\gramHi{Sir} I wish to hear from you at the earliest. What progress have you
made in pushing out beyond Muldraugh's Mountain? What progress is made against
the enemy citadel at Bowling Green?

\gramClosing{I remain vy rspy yours \&c}
    {C. N. Van Royne}
    {Maj. Gen., USA, cmdg}
\subsecdinkus

\subsection*{January 27, 1862}{} % {{{3
\gramHeader{Headquarters, Army of the Cumberland} % {{{4
    {Field Headquarters, Near Fort Donelson, Tenn., January}{27, 1862}
\gramTo{Maj. Gen.}{Cornelius Van Royne}
    {United States Army, Commanding}

\gramHi{General} My entire army, save five regiments garrisoning our supply line
has now arrived near Fort Donelson. Unfortunately, the enemy has also brought up
reinforcements. Current reports indicate Gen'l Whisper's Army of East Tennessee,
numbered at 7,000 to 9,000 men plus both the Left and Right Wings of Gen'l
Clarke's Army of Mississippi, numbered at 10,000 to 12,000 are now encamped
around the fort. These reports put the total enemy strength at 17,000 to 21,000
men against my 37,200 infantry and 3,300 cavalrymen.

Enemy cavalry continues to operate along the east bank of the Cumberland River
and the west bank of the Tennessee River but is unable to affect our operations
due to Cdre. Davis' gunboat patrols.

The enemy clearly intends to make a strong defense of this place. I still
believe I can drive him off but it will be more difficult than expected. After
the bloody repulse of the 25th instant, I will need to proceed more cautiously.

However, I believe the enemy concentration here gives us opportunity elsewhere.
If Gen'l Clarke has indeed brought such a strong force here to oppose me, then,
by logic, Columbus must have a much weaker garrison than anticipated. Perhaps
our own Gen'l Smith may be able to seize the city outright?

Additionally, I have been informed that the locks on the Green River at
Rochester and Calhoun have been destroyed by Confederate raiders. The extent of
the damage cannot presently be ascertained but this avenue of advance towards
Bowling Green seems as difficult as the rail line currently being traveled by
Gen'l Stoeffler.

\gramClosing{I remain, as always, yours rspy}
    {J. W. Blake}
    {Maj. Gen., Commanding}
\subsecdinkus

\gramHeader{United States Army} % {{{4
    {Army City of Washington, January}{27, 1862}
\gramTo{Maj. Gen.}{James Blake}
    {Commanding Dep't of the Cumberland}

\gramHi{Sir} Yours of this date just received. Has your final corps arrived? You
should, if so, have sufficient strength to invest Donelson and Dover and reduce
it, assuming the place is indeed held by 20 or 25 thousand men.

I will impress upon Gen. Stoeffel the seriousness of the situation and urge him
toward Bowling Green. Pressure on the enemy must be universal and strong at
every point to compel his defeat.

If a portion of Clarke's force is detached for service at Donelson, as you say,
then he still has at least 20 thousand men at Columbus, and possibly up to 30
thousand, which would make Gen. Smith's task far more difficult than initially
proposed. At best, he would have numerical parity with the enemy before him, and
to deploy him into Kentucky would completely remove the threat we have thus far
posed to New Madrid and Island Number 10.

\gramClosing{I remain vy rspy yours \&c}
    {C. N. Van Royne}
    {Major General, USA, cmdg}
\subsecdinkus

\gramHeader{United States Army} % {{{4
    {City of Washington, January}{27, 1862}
\gramTo{Maj. Gen.}{Thomas Smith}
    {Commanding Dep't of the Tennessee}

\gramHi{General} Are you aware how far north the fortifications at Columbus
stretch? Do they reach near Blandville? Do you have any intelligence whatsoever
on enemy dispositions across the Mississippi?

\gramClosing{I remain vy rspy yours \&c}
    {C. N. Van Royne}
    {Major General, USA, cmdg}
\subsecdinkus

\gramHeader{Army Of The Tennessee} % {{{4
    {City of Charleston, January}{27, 1862}
\gramTo{Maj. C. N.}{Van Royne}
    {Commander US Army}

\gramHi{Sir} Yours of this date received. The Army Of Tennessee's headquarters
was there last week but I do not know if anything has changed. There is at least
one fort covering the river, however when I return to Number 5 I suspect I will
have a better look. The enemy was not outside the hills of Columbus and nowhere
near Blandville last week but that may have changed.

\gramClosing{I remain vy rspy yours}
    {Thomas Smith}
    {Commander, Army Of The Tennessee}
\subsecdinkus

\gramHeader{Headquarters, Army of the Cumberland} % {{{4
    {Field Headquarters, Near Fort Donelson, Tenn., January}{27, 1862}
\gramTo{Maj. Gen.}{Cornelius Van Royne}
    {United States Army, Commanding}

\gramHi{General} Cdre. Lewis commanding the Mississippi Squadron informed me
that rebel gunboats did conduct a desultory bombardment of Cairo but rapidly
withdrew before they could be engaged by the bulk of our gunboats. He has
adjusted his patrols to ensure the enemy is not able to repeat this incident.
Once Gen'l Smith completes the battery at Island Number 5 that will also reduce
the enemies ability to threaten Cairo by gunboat.

\gramClosing{Respectfully}
    {J. W. Blake}
    {Maj. Gen., Commanding}
\subsecdinkus

\gramHeader{Headquarters, Army of the Cumberland} % {{{4
    {Field Headquarters, Near Fort Donelson, Tenn., January}{27, 1862}
\gramTo{Maj. Gen.}{Cornelius Van Royne}
    {United States Army, Commanding}

\gramHi{General} With poor weather approaching and the lack of clear numerical
superiority over the enemy I am temporarily shifting to a defensive posture. The
belief of this headquarters is that the enemy, having yet to identify XIV Corps'
arrival on the field will endeavor to attack what they believe is our right
flank in an attempt to cut us off from our line of supply. However, I have
placed XIV Corps on my right, screened by my cavalry (of which the enemy has
none), and hope, by the construction of field works, to entice me in a perceived
position of weakness.

If the enemy does not attack then, once the weather clears, I will attempt to
resume the offensive.

Additionally, I currently maintain garrisons of Evansville, Indiana and
Smithland, Ky. amounting to an entire brigade. I request leave to reduce or
eliminate those garrisons so as to increase my total strength here in Tennessee.

\gramClosing{Respectfully}
    {J. W. Blake}
    {Maj. Gen., Commanding}
\subsecdinkus

\gramHeader{Headquarters, Army of the Cumberland} % {{{4
    {Field Headquarters, Near Fort Donelson, Tenn., January}{27, 1862}
\gramTo{Maj. Gen's.}{Thomas Smith and Karl Meyer}
    {Commanding armies of the Tennessee \& Arkansas respy.}

\gramHi{Gen'ls} My cavalry reports that at least some elements of the Left Wing
of the Army of Mississippi has arrived on the field here, that army's Right Wing
being already present. As I try to ascertain the number of forces arrayed
against me this would indicate up to four divisions of the Army of Mississippi
are now present around Fort Donelson.

If that is correct, it would mean Columbus is nigh unguarded. Have you any
intelligence regarding the number of forces currently occupying Columbus and its
environs? Any such information would help me form a better estimate of the
amount of force here in Tennessee.

\gramClosing{Yours, respectfully}
    {J. W. Blake}
    {Maj. Gen., Commanding}
\subsecdinkus

\subsection*{January 28, 1862}{} % {{{3
\gramHeader{Headquarters, Army of the Arkansas} % {{{4
    {Charleston, Mo., Morning, January}{28th, 1862}
\gramTo{Maj. Gen.}{Cornelius Van Royne}
    {Cmdg Gen. of the United States Army}

\gramHi{General} I have just arrived at General Smith's headquarters though the
general is currently nowhere to be found. Having gained access to the camp's
telegraph machine, I must inform you of recent unfortunate news, that being that
Rebel gunboats are encroaching higher up the river all the way to Cairo. 

I am concerned that this may prevent me from bringing up further reinforcements
from St. Louis to St. Lukes by way of Ohio City, due to the risk of the enemy
preying upon my transports. But as of yet, there is nothing to be done but to
continue building the rail line and coordinating with General Smith.

\gramClosing{I am always, General, and shall ever remain, Your most humble and obedient of servants}
    {Karl Meyer}
    {Maj. Gen. cmdg Army of the Arkansas}
\subsecdinkus

\gramHeader{United States Army} % {{{4
    {City of Washington, January}{28, 1862}
\gramTo{Maj. Gen.}{Karl Meyer}
    {Commanding, Dep't of the West}

\gramHi{General} Yours this date just received. Am I to understand from your
dispatch that General Smith has abandoned his post? He last wrote to me
yesterday.

I should like to know what our riverine squadrons are doing if not preventing a
raid on Cairo by whatever pulled-together steamers the rebels have managed to
mount some guns on. I understand the ironclads are with General Blake, but are
there not other boats and ships available to help defend the confluence, or to
prevent a breakout onto the Ohio? Are there not some boats in the area, within
the jurisdiction of the Department of the Tennessee?

\gramClosing{I remain vy rspy yr obt svt \&c}
    {C. N. Van Royne}
    {Major General, USA, cmdg}
\subsecdinkus

\gramHeader{Headquarters, Army of the Arkansas} % {{{4
    {Charleston, Mo., Morning, January}{28th, 1862}
\gramTo{Maj. Gen.}{Cornelius Van Royne}
    {Cmdg Gen. of the United States Army}

\gramHi{General} I have just received your letter. I do not believe that General
Smith has abandoned his post but rather may be preoccupied with other duties. I
will endeavor to keep searching for him.

As for the state of the riverine squadrons, I do not know the situation on the
river front, nor have I even been informed of who the flag officer is, in order
to coordinate with him. I will be certain to ask General Smith if he has been
coordinating with any ships in the region, but it is my belief that your earlier
orders regarding Is. 5 will be the only certain method of defending the
confluence.

A battery or fort must be constructed to prevent the rebel gunboats from
reaching Cairo and Ohio City. My only concern is possible shelling from the
opposite bank of the river from Columbus which would frustrate the construction.

\gramClosing{I am always, General, and shall ever remain, Your most humble and obedient of servants}
    {Karl Meyer}
    {Maj. Gen. cmdg, Army of the Arkansas}
\subsecdinkus

\gramHeader{Headquarters, Army of the Kentucky} % {{{4
    {Louisville, Ky., January}{28th, 1862}
\gramTo{Maj. Gen.}{Cornelius Van Royne}
    {Commanding General of the United States Army}

\gramHi{Sir} Yours of the 26th instant received. I write to you now that I have
received news.

My Cavalry was unable to displace enemy cavalry holding tunnel. Once infantry
support arrived, enemy retreated to Elizabethville. Tunnel supports burned,
track removed, tunnel itself intact. Estimated duration of repair 5 weeks if 5
labor parties (LP) are employed, 3 weeks if 15 labor parties (LP) are employed.

My current disposition has VII Corps posted at the tunnel, IX corps is encamped
east of Howell Springs, six miles north of Elizabethtown. XI Corps is in
Lebanon. My cavalry division is northeast of Elizabethtown.

The enemy forces encountered belong to the Army of Tennessee, under General
Jackson. The Cavalry is commanded by General Forrest. Current estimates place
two enemy cavalry divisions in Elizabethtown, 2,500--4,000 sabres
strong. No Infantry has been spotted, they are assumed to be further towards
Bowling Green along the railway.

After throwing the enemy back from the Tunnel, progress has been slow.
Engagements were limited to smaller cavalry skirmishes, the weather not
permitting more.

Current reports indicate that the weather is likely to get worse before it gets
better. This means the damaged tunnel has limited impact, I will have to wait
for better weather anyway. Mud is already present.

Once weather allows, I will advance on Elizabethville and take the town. I will
have to then wait for the tunnel to be repaired before further advances.

There are no reports from the pickets in the south east.

\gramClosing{I am your most obedient servant}
    {Christopher Stoeffel}
    {Major General, Commanding, Army of the Kentucky}
\subsecdinkus

\gramHeader{Headquarters, Army of the Arkansas} % {{{4
    {Charleston, Mo., Afternoon, January}{28th, 1862}
\gramTo{Maj. Gen.}{James W. Blake}
    {Cmdg Gen. of the Army of the Cumberland}

\gramHi{General} Have received your letter. Initial reports earlier this month
suggested that the army of the Mississippi is 45,000 strong. With at least
7-9,000 troops at New Madrid, and 20-22,000 present at Fort Donelson by your
reports, that leaves 12-18,000 troops unaccounted for. I would also request
reports on what the initial strength of the Army of East Tennessee and their
location so as to better understand their disposition. Worryingly, the Army of
Tennessee some 40,000 strong has not been sighted, though reports suggest they
are operating in central Kentucky.

I will consult with General Van Royne on the possibility of attacking Columbus,
but even so, I cannot assist the Army of the Tennessee in holding it, as the
35,000 strong Army of the Trans-Mississippi is also yet to be located and most
likely somewhere in my Department.

\gramClosing{I am always, General, and shall ever remain, Your most humble and obedient of servants}
    {Karl Meyer}
    {Maj. Gen. cmdg Army of the Arkansas}
\subsecdinkus

\gramHeader{Headquarters, Army of the Arkansas} % {{{4
    {Charleston, Mo., Afternoon, January}{28th, 1862}
\gramTo{Maj. Gen.}{Cornelius Van Royne}
    {Cmdg Gen. of the United States Army}

\gramHi{General} I have managed to meet with General Smith. Having just received
General Blake's reports on the whereabouts of the Army of Mississippi, I can
safely state that only 12,000 to 18,000 of General Clarke's forces are
unaccounted for, the rest either present at New Madrid or Fort Donelson. More
over, with anywhere from 4-5,000 forces at New Madrid being composed of Cavalry,
it seems unlikely that the Army of Mississippi has much more cavalry unless they
have more than three brigades or an entirely cavalry corps.

With the Army of Tennessee being reported earlier in the month to be operating
in central Kentucky, and the Army of East Tennessee at least partially tied up
at Fort Donelson, it seems increasingly likely that Columbus is defended with
only 18,000 soldiers at most. More likely that number is lower if there are
other garrisons along the Mississippi.

In regards to your previous question about the state of our Riverine Squadrons,
General Smith has told me that the entire force is currently operating with
General Blake and the Army of the Cumberland. It may be that the General and I
will have to find a way to construct our own riverine force to defend the river
ways.

The good general has also requested my aid in occupying Columbus. If you are
agreeable to this idea, it may be possible for me to have General Louis take the
XIII Corps into Kentucky along with General Smith's forces, save for a Division
which he will leave at Is. 5. Our joint force would number around 33,600
infantry, along with possibly a brigade or two of cavalry to scout and screen
our advance.

Please inform me if you are against this idea or wish to make changes.

\gramClosing{I am always, General, and shall ever remain, Your most humble and obedient of servants}
    {Karl Meyer}
    {Maj. Gen., cmdg, Army of the Arkansas}
\subsecdinkus

\gramHeader{United States Army} % {{{4
    {City of Washington, January}{28, 1862}
\gramTo{Maj. Gen.}{Christopher Stoeffel}
    {Commanding, Army of the Kentucky}

\gramHi{Sir} Yours of this date just received. You have been given a copious
amount of supply and are urged to use it to repair the railroad in that manner
which best supports your rapid advance upon Bowling Green. General Blake may
soon face danger on the banks of the Cumberland, should the enemy choose to
abandon Bowling Green and combine against him.

Execute that advance at the earliest opportunity. Timely pressure on every front
will force the enemy to yield ground or otherwise commit to a battle under
unfavorable circumstances.

\gramClosing{I remain vy rspy yours \&c}
    {C. N. Van Royne}
    {Major General, USA, cmdg}
\subsecdinkus

\gramHeader{United States Army} % {{{4
    {City of Washington, January}{28, 1862}
\gramTo{Maj. Gen.}{Karl Meyer}
    {Commanding, Dep't of the West}

\gramHi{General} Yours this date just received. For the time being you may
exercise your seniority by date of rank and assume tactical and operational
command over both your department and the Department of the Tennessee, with
General Thomas Smith to remain in command of his department.

At earliest opportunity take your forces as you described, cross the Mississippi
at whatever point you deem best (closest to Cairo may be best), and, using your
best judgment, either block the union of Clarke's Columbus forces with those at
Nashville, or, should the enemy make no such attempt, otherwise move directly
upon Columbus. To compel the city's abandonment is preferred, but offensive
action is also permitted.

Provide in some regard for the defense of Ohio City, as the enemy may choose to
become saucy and put guns on the confluence of the Mississippi and Ohio. We are
already overseeing the beginning of the construction of Fort Sumner at Cairo.

General Blake has been forced to assume a defensive posture at Donelson. We must
apply pressure elsewhere to ensure the enemy does not concentrate to Blake's
front and put his army at risk.

Also consider the possibility of retaking Paducah. Take all of Smith's cavalry
and give serious consideration to bringing at least one of the brigades you have
brought over from your own cavalry division. This should give you numerical
superiority in saddles.

Garrisoning of Island Number 5 is to our eye good, so long as it is properly
screened on the landward side, has an avenue of escape should it be pressured
strongly by the enemy, and if a battery is there to guard the length of the
river there, as originally ordered.

\gramClosing{I remain vy rspy yr obt svt \&c}
    {C. N. Van Royne}
    {Major General, USA, cmdg}
\subsecdinkus

\gramHeader{United States Army} % {{{4
    {City of Washington, January}{28, 1862}
\gramTo{Maj. Gen.}{James Blake}
    {Commanding Dep't of the Cumberland}

\gramHi{Sir} Yours of this date just received. I have instructed gens.  Stoeffel
and Meyer, the latter assuming temporary seniority over Gen. Smith, to move
southward and pressure Bowling Green and Columbus, respectively. If either army
fails to properly keep the enemy's attention there is a serious risk that Clarke
is reinforced significantly and puts you at a fatal disadvantage. As always you
must have an eye toward the survival of your command, and in all respects you
enjoy our confidence.

If you can spare some small portion of your supporting fleet of gunboats,
without fatally weakening it, then you must send them posthaste to Cairo, to
defend the confluence of the Mississippi and Ohio.

Should you find the enemy's imminent attack handsomely repulsed, you are
encouraged to drive on Donelson and secure it without delay. With each day that
passes, the enemy's position at Nashville grows stronger.

\gramClosing{I remain vy rspy yours \&c}
    {C. N. Van Royne}
    {Major General USA, cmdg}
\subsecdinkus

\gramHeader{Headquarters, Army of the Cumberland} % {{{4
    {Field Headquarters, Near Fort Donelson, Tenn., January}{28, 1862}
\gramTo{Maj. Gen.}{Cornelius Van Royne}
    {United States Army, Commanding}

\gramHi{Sir} I have emphasized to Cdre. Lewis that the specific requests of my
Army do not override his primary task of securing freedom of navigation along
the Mississippi, Ohio, Tennessee and Cumberland Rivers as well as preventing
enemy incursion on those rivers. Requests for specific support are merely
supplemental and will be conducted only as practicable by his small force of
gunboats.

As resources allow I humbly request that additional gunboats be procured so as
to increase Cdre. Lewis' ability to support operations here in the West.

Have you considered my request to reduce or eliminate the garrisons of
Evansville and Smithland? The presence of a few additional regiments here in
Tennessee would be greatly appreciated.

\gramClosing{Respectfully}
    {J. W. Blake}
    {Maj. Gen., Commanding}
\subsecdinkus

\gramHeader{Headquarters, Army of the Kentucky} % {{{4
    {Louisville, Ky., January}{28th, 1862}
\gramTo{Maj. Gen.}{Cornelius Van Royne}
    {Commanding General of the United States Army}

\gramHi{Sir} Yours of today received. We have started repairing the tunnel with
utmost speed. Weather permitting, we will attempt to throw the enemy cavalry out
of Elizabethville, which is at the edge of what our supply network allows while
the tunnel is damaged.

After that is achieved, any further progress will have to wait until the tunnel
is repaired. The weather has not shown signs of improving, so the effective
delay by the tunnel might be minor.

\gramClosing{I am your most obedient servant}
    {Christopher Stoeffel}
    {Major General, Commanding, Army of the Kentucky}
\subsecdinkus

\gramHeader{United States Army} % {{{4
    {City of Washington, January}{28, 1862}
\gramTo{Maj. Gen.}{James Blake}
    {Commanding Dep't of the Cumberland}

\gramHi{Sir} Yours of this date just received. News of the enemy's repulse is
excellent and cause for happiness at our headquarters. You believe you outnumber
the enemy, and have inflicted much heavier loss on the enemy than you have
received? Your strength must now be no less than 45 thousand, and the enemy's
less than 35 thousand.

Our forces west of the Mississippi have suffered dreadfully at the hands of the
Army of the Trans-Mississippi. The enemy probably feels confident in sending all
of the Army of Mississippi to Donelson to destroy or otherwise repel you from
beneath the works of Donelson. I am awaiting news from Gen. Stoeffel on his
efforts in moving out toward Bowling Green to prevent the union of Jackson's
army with the main body to your front.

\gramClosing{I remain vy rspy yours \&c}
    {C. N. Van Royne}
    {Major General, USA, cmdg}
\subsecdinkus

\gramHeader{United States Army} % {{{4
    {City of Washington, January}{28, 1862}
\gramTo{Maj. Gen.}{James Blake}
    {Commanding Dep't of the Cumberland}

\gramHi{Sir} Addendum to last. When practicable please provide conservative
estimate on enemy loss in most recent engagement.

It is best to draw up immediately plans to take Henry but, as you expect renewed
efforts from the enemy to dislodge you, best not to execute them right now. Will
advise again once Gen. Stoeffel provides report on his own efforts.

\gramClosing{I remain vy rspy yours \&c}
    {C. N. Van Royne}
    {Major General, USA, cmdg}
\subsecdinkus

\gramHeader{United States Army} % {{{4
    {City of Washington, January}{28, 1862}
\gramTo{Maj. Gen.}{James Blake}
    {Commanding Dep't of the Cumberland}

\gramHi{Sir} Final addendum. You requested permission to withdraw your garrisons
and re-attach them to your army to bolster your field strength. That request is
belatedly but wholeheartedly approved.

All efforts are being made to increase the size and strength of the riverine
squadrons. The enemy presently threatens Cairo by land \& sea and its defense
will require, for a time, the balance of all boats not currently assigned to
your operations up the Cumberland. You may retain use of your boats there for
the present campaign.

\gramClosing{I remain vy rspy yours \&c}
    {C. N. Van Royne}
    {Major General, USA, cmdg}
\subsecdinkus

\subsection*{January 30, 1862}{} %{{{3
\gramHeader{United States Army} % {{{4
    {City of Washington, January}{30, 1862}
\gramTo{Maj. Gen'ls.}{Karl Meyer \& Thomas Smith}
    {Commanding Dep'ts of the West \& Tennessee respectively}

\gramHi{General Meyer} Yrs both of this date just received. Please provide at
earliest possible opportunity a casualty estimate, in cavalry, infantry, and
artillery, and what number of the guns were of the horse artillery rather than
the infantry. Also provide list of all units present at the fight.

How much of Gen. Smith's army was present? Where is the rest of his army? Gen.
Smith: were you made aware of the enemy's advance toward Charleston \& Ohio
City?  If yes, why does Gen. Meyer imply you have not been rendering aid or
answering to his seniority? Can the latter place (Ohio City) be saved from an
advance by this enemy army?

Gen. Meyer, you may wish to prepare a movement of your troops in central
Missouri closer to the banks of the Mississippi itself. It is possible the enemy
seeks to destroy our position at Ohio City, neutralize any defenses at Cairo,
and then move northward toward St. Louis, relying on support from whatever
gunboats they have available.

Cairo must not be permitted to be put to siege or fall to the enemy. Its
existence is absolutely crucial for our operations in the West.

\gramClosing{I remain vy rspy yr obt svt \&c}
    {C. N. Van Royne}
    {Major General, USA, cmdg}
\subsecdinkus

\gramHeader{Headquarters, Army of the Arkansas} % {{{4
    {Charleston, Mo., Afternoon, January}{30th, 1862}
\gramTo{Maj. Gen.}{Cornelius Van Royne}
    {Cmdg Gen. of the United States Army}

\gramHi{General} I have received your message. No cavalry nor their artillery
was lost as they were still forming up to head to the battlefield by the time
the battle ended.

Present during the battle was the entirety of the XII Corps and the 2d Division
of the XIII Corps. 3 regiments of infantry from the XV corps and a battery of
guns and 6 regiments of infantry from the XIII corps were casualties during the
battle. All told some 5,500 soldiers in total.

Forces that can immediately move to defend Ohio City are the 1st Division of
XIII Corps which is comprised of 12 infantry regiments and 4 batteries of
artillery, 7,600 strong. Three regiments of cavalry, 1 regiment from the 1st
Brigade and two from the 3d Brigade totaling 900 troopers. And the remnants of
2d Division which comprise of 6 regiments, 3,600 strong. All of these are at
Charleston. The total force will be 11,700 strong in the face of an enemy
roughly 20,4000 strong plus an enemy cavalry division south of Charleston. I do
not know if any of these forces were the original 9,000 strong force General
Clarke met during his march on New Madrid.

It pains my heart to say this but all prior operational goals must be discarded.
I intend to destroy the depot at St. Luke and march the XV Corps and the other
three regiments of 1st Cavalry Brigade up to Bloomfield and then Cape Girardeau
which will take roughly five days before steaming them down to Ohio City. This
will produce 21 regiments of infantry, three regiments of cavalry, and 6
batteries for a force of 14,100 strong.

I am also strongly requesting stripping the Division of Observation from Middle
Missouri to move to Ohio City as well. This would present another six regiments
and an artillery battery, 3,700 strong. I would also request support from your
garrison at Cairo for XIII Corps until the XV Corps lands at Ohio City.

\gramClosing{I am always, General, and shall ever remain, Your most humble and obedient of servants}
    {Karl Meyer}
    {Maj. Gen. cmdg, Army of the Arkansas}
\subsecdinkus

\gramHeader{Headquarters, Army of the Cumberland} % {{{4
    {Field Headquarters, Near Fort Donelson, Tenn., January}{30 1862}
\gramTo{Maj. Gen.}{Cornelius Van Royne}
    {Commanding, United States Army}

\gramHi{Sir} I estimate that the enemy suffered around 1,200 casualties during
the action on the 28th instant. The two divisions of the Left Wing of the Army
of Mississippi have been identified as being commanded by Gen'ls Breckenridge
and Cleburne. General Whisper's Division has been spotted as of this morning
near Fort Henry, having left the field here.

I believe the enemy line to my south to be quite thin, held by two divisions of
the Right Wing of the Army of Mississippi, with a possible division of the Army
of East Tennessee in reserve.

Our debated withdrawal north was delayed due to heavy rain. If we remain on the
field I will have field fortifications along my southern line completed on the
3d prox. I am currently considering options to exploit the perceived enemy
weakness to my south as well as his extended lines overall. Withdrawal however
is still an option.

\gramClosing{Respectfully}
    {J. W. Blake}
    {Maj. Gen., commanding}
\subsecdinkus

\gramHeader{United States Army} % {{{4
    {City of Washington, January}{30, 1862}
\gramTo{Brig. Gen'ls.}{Thomas Caldwell \& Jeremiah Wentworth}
    {Army of the Tennessee}

\gramHi{Sirs} I have not heard from Gen. Smith. Gen. Meyer is repulsed with
great loss at Sikeston. I must know the dispositions of the Army of the
Tennessee and where Gen. Smith is. If he is indisposed by illness I must know
immediately so that the army is not leaderless in the field. We are facing a
crisis.

\gramClosing{I remain}
    {C. N. Van Royne}
    {Maj Gen'l, cmdg, USA}
\subsecdinkus

\gramHeader{Headquarters, Army of the Tennessee} % {{{4
    {Field Headquarters, Paducah, Ky., Evening, January}{30th, 1862}
\gramTo{Maj. Gen.}{C. N. Van Royne}
    {Commanding General, United States Army}

\gramHi{General} The Army of the Tennessee, currently formed as the XVIth and
XVIIth Corps, has retaken Paducah with further intent to prosecute a campaign
against rebel forces present in the Mayfield and Columbus areas.  Columbus in
particular has a confederate fort which we intend to isolate and reduce to
enable further actions down the Mississippi. General Smith is not at present
incapacitated, so I am unsure as to the lack of response to you. I am currently
confident, due to rebel presence west of the Mississippi and east of the
Tennessee, that we may yet find success for our cause in this campaign.
However, with recent developments on the West bank of the Mississippi I have
encouraged my General to communicate with Gen Meyer to ascertain whether our
immediate return is necessary.

\gramClosing{Yr. Obdt. Srvt.}
    {Brig. Gen. Thomas Caldwell}
    {Army Of The Tennessee}
\subsecdinkus

\gramHeader{Headquarters, Army of the Cumberland} % {{{4
    {Field Headquarters, Near Fort Donelson, Tenn., January}{30, 1862}
\gramTo{Maj. Gen.}{Cornelius Van Royne}
    {United States Army, Commanding}

\gramHi{Sir} I have the honor to report to you that my army in the field
currently numbers 37,148 infantry, 3,271  cavalry and 210 guns. Once the
Smithland and Evansville garrisons arrive I will have 39,548 infantry in the
field.

I believe the enemy has four divisions present around Fort Donelson with a full
strength of 28,800  infantry although I believe this number to have been reduced
to somewhat less than 27,000 by this point.

The enemy, according to reports, has no cavalry present here and I have a
significant advantage in artillery.

\gramClosing{Respectfully}
    {J. W. Blake}
    {Maj. Gen., Commanding}
\subsecdinkus

\gramHeader{Headquarters, Army of the Cumberland} % {{{4
    {Field Headquarters, Near Fort Donelson, Tenn., January}{30, 1862}
\gramTo{Generals}{T. Smith \& T. Caldwell}
    {Army of the Tennessee}

\gramHi{General} We understand there is no significant Rebel presence to your
front at this time, with the presence of your immediate opponents to our own
front.

What are your intentions and time frame towards either striking south with
aggression towards Columbus or else shifting your weight west in support of
General Meyer?

With two armies to our own front, we have an opportunity we can exploit.

\gramClosing{Your humble servt.}
    {Brig. Gen. H. Fawcett}
    {Army of the Cumberland}
\subsecdinkus

\gramHeader{Headquarters, Army of the Cumberland} % {{{4
    {Field Headquarters, Near Fort Donelson, Tenn., January}{30, 1862}
\gramTo{Major General}{K. Meyer}
    {Commanding the Army of the Arkansas}

\gramHi{General} We understand you have faced some adversity there in recent
weeks, with a relentless attack from your south.

What is your overall situation in terms of your own condition and strength and
the enemy presence to your front.

What are your readiness and campaign plans?

\gramClosing{Your humble servt}
    {Brig. Gen. H. Fawcett}
    {Army of the Cumberland}
\subsecdinkus

\gramHeader{Headquarters, Army of the Tennessee} % {{{4
    {Paducah, Ky., Morning, January}{30th, 1862}
\gramTo{Major General}{K. Meyer}
    {Commander, Army of the Arkansas}

\gramHi{General} I have heard of your situation, do you believe you can hold
Cairo or should I return to Cairo? 

\gramClosing{I am your most obedient servant}
    {Thomas Smith}
    {Major General, Commander, Army Of The Tennessee}
\subsecdinkus

\subsection*{February 3, 1862}{} %{{{3
\gramHeader{Headquarters, Army of the Arkansas} % {{{4
    {Charleston, Mo., Afternoon, February}{3d, 1862}
\gramTo{Maj. Gen.}{Cornelius Van Royne}
    {Cmdg Gen. of the United States Army}

\gramHi{General} The XVth Corps has destroyed the depot at St. Luke and managed
to make their way up to Winkler's Mill about 5--8 miles from Cape
Girardeau. The remainder of the XIIIth Corps is currently all at Charleston.

It has been raining the entire week and the roads have turned to mud, which may
be the reason why the enemy does not seemed to have moved against me. The only
thing spotted since the 29th has been the Confederate Cavalry south of
Charleston which has maintained a presence but not pressed at all.

I am confirming your permission to remove the Division of Observation from
Central Missouri and to send them down to Ohio City. I intend for the XV to join
them there as well.

\gramClosing{I am always, General, and shall ever remain, Your most humble and obedient of servants}
    {Karl Meyer}
    {Maj. Gen. cmdg, Army of the Arkansas}
\subsecdinkus

\gramHeader{United States Army} % {{{4
    {City of Washington, February}{3, 1862}
\gramTo{Maj. Gen.}{Karl Meyer}
    {Commanding, Dep't of the West}

\gramHi{General} Yours of this date just received. Your request to remove the
Division of Observation to Ohio City is approved forthwith.

Must request clarification on whether or not you can reach Gen. Thomas. We have
been unable to do so and are unable to reach any other person on his staff or
within the Army of the Tennessee.

\gramClosing{I remain vy rspy yr obt svt \&c}
    {C. N. Van Royne}
    {Major General, USA, cmdg}
\subsecdinkus

\gramHeader{Headquarters, Army of the Arkansas} % {{{4
    {Charleston, Mo., Afternoon, February}{3d, 1862}
\gramTo{Maj. Gen.}{Cornelius Van Royne}
    {Cmdg Gen. of the United States Army}

\gramHi{General} I have received your message. I have been unable to contact
General Smith, last that I spoke with him is when his army left on trains the
morning of the 29th. I assume from our last discussion that he is moving the
direction of Paducah.

\gramClosing{I am always, General, and shall ever remain, Your most humble and obedient of servants}
    {Karl Meyer}
    {Maj. Gen. cmdg, Army of the Arkansas}
\subsecdinkus

\gramHeader{Headquarters, Army of the Arkansas} % {{{4
    {Charleston, Mo., Afternoon, February}{3d, 1862}
\gramTo{Maj. Gen.}{Cornelius Van Royne}
    {Cmdg Gen. of the United States Army}

\gramHi{Sir} Addendum to last. I want to confirm that you are aware that there
will be no forces of mine in Rolla and Jefferson City when I remove the
Division.

\gramClosing{I am always, General, and shall ever remain, Your most humble and obedient of servants}
    {Karl Meyer}
    {Maj. Gen. cmdg, Army of the Arkansas}
\subsecdinkus

\gramHeader{United States Army} % {{{4
    {City of Washington, February}{3, 1862}
\gramTo{Maj. Gen.}{Karl Meyer}
    {Commanding, Dep't of the West}

\gramHi{General} Yours of this date just received and understood. You may
commence with the movement toward Cape Girardeau and Ohio City as soon as
practicable.

\gramClosing{I remain vy rspy yr obt svt \&c}
    {C. N. Van Royne}
    {Major General, USA, cmdg}
\subsecdinkus

\gramHeader{Headquarters, Army of the Kentucky} % {{{4
    {Louisville, Ky., February}{3d, 1862}
\gramTo{Maj. Gen.}{Cornelius Van Royne}
    {Commanding General of the United States Army}

\gramHi{Sir} Yours received. The tunnel repairs are proceeding on schedule,
predicted to require two more weeks.

We have successfully secured Elizabethtown, winning a battle there and pushing
the enemy back. They have fallen back southward.

The weather has not shown signs of improving, there is supposed to be rain all
week.

Once the tunnel is back in working condition, a Depot will be established in
Elizabethtown, which will accelerate our campaign forward.

\gramClosing{I am your most obedient servant,}
    {Christopher Stoeffel}
    {Major General, Commanding, Army of the Kentucky}
\subsecdinkus

P.S.---Recent publications of enemy communications and discussions in
the newspaper makes it clear to me that the enemy command is amateurish at best,
and I believe that their luck must run out soon.

\gramHeader{Headquarters, Army of the Kentucky} % {{{4
    {Louisville, Ky., February}{3d, 1862}
\gramTo{Maj. Gen.}{Cornelius Van Royne}
    {Commanding General of the United States Army}

\gramHi{Sir} We are planning to establish a wagon depot at Elizabethtown large
enough to support the cavalry division. Once the rail line is connected, this
will count as a rail depot.

With this support, we plan to push the enemy down the rail line to the extent of
this new depot. This should put us in position to leapfrog forward once the rail
tunnel is repaired.

There is still no information on where the enemy main body is located. I am
certain you have, compared to myself, a much better picture of the overall
situation. Do you also have up to date force estimates of the various enemy
armies?

\gramClosing{I am your most obedient servant}
    {Christopher Stoeffel}
    {Major General, Commanding, Army of the Kentucky}
\subsecdinkus

\gramHeader{United States Army} % {{{4
    {City of Washington, February}{3, 1862}
\gramTo{Maj. Gen.}{Christopher Stoeffel}
    {Commanding, Army of the Kentucky}

\gramHi{General} Yours of this date just received. It is a relief to hear of
your successful efforts thus far. Must impress upon you again the importance
that you continue your advice with the fullest vigor. Gen. Blake is confident in
future success, but such success relies on your splendid army also fulfilling
its orders.

Gen. Jackson so far as we are aware remains in the Bowling Green area opposing
you, with a minimum of 17 thousand men and an upper estimate of about 35
thousand. The composition of this force is unknown. Gen. Clarke's Army of
Mississippi faces Blake in its entirety, with perhaps 25 thousand, aided by Gen.
Whisper's Army of East Tennessee, which likely does not exceed 10 thousand. Both
of these forces are deployed in defense of Dover and Fort Donelson, with
apparently some detachment holding Fort Henry in Gen. Blake's rear.

No other Confederate force in the region is known to oppose you. Jackson's force
will be compelled to forfeit Bowling Green if Nashville is directly threatened,
which is why it is critical that you press forward and pin him in place. If
Jackson is found to have already abandoned Bowling Green, you must hurry even
further, as that will mean Gen. Blake will be outnumbered on the banks of the
Cumberland \& at risk of destruction. Under such circumstances, you must then
move beyond Bowling Green \& advance toward Nashville up the Louisville \&
Nashville RR.

Gen. Smith has been sent over the Mississippi and has retaken Paducah for the
Nat'l flag, \& was ordered to demonstrate against Columbus, take it if possible,
and prevent a junction of any forces there with Clarke \& Whisper; but at least
part, \& maybe all, of his force must now be recalled to defend Ohio City, as
the enemy is moving northward on the west bank of the Mississippi.

\gramClosing{I remain vy rspy yr obt svt \&c}
    {C. N. Van Royne}
    {Major General, USA, cmdg}
\subsecdinkus

\gramHeader{Headquarters, Army of the Arkansas} % {{{4
    {Field Headquarters, Charleston, Mo., February}{3d, 1862}
\gramTo{Brig. Gen.}{H. Fawcett}
    {Army of the Cumberland}

\gramHi{General} Yours of this date just received. The XV Corps is currently
nearing Cape Girardeau and will soon land at Ohio City, as will a division of
six infantry regiments, and a cavalry regiment from central Missouri. I have
currently lost a quarter of XIII Corps and an eighth of the XV corps. More over
an entire cavalry brigade is currently attached to General Smith's Army of the
Tennessee.

The prior attempts to build a rail line to Poplar Bluff are now infeasible, and
instead, my immediate goal is to gather my entire force in one location so as to
avoid another situation like Sikeston occurring again.

As of this date, what is left of the XIII Corps stands between the enemy and
Ohio City. The 1st Division of XIII Corps which is comprised of 12 infantry
regiments and 4 batteries of artillery, 7,600 strong. Three regiments of
cavalry, 1 regiment from the 1st Brigade and two from the 3d Brigade totaling
900 troopers. And the remnants of 2d Division which comprise of 6 regiments,
3,600 strong.

All of these are at Charleston. The total force will be 11,700 strong in the
face of an enemy roughly 20,4000 strong plus an enemy cavalry division south of
Charleston. I do not know if any of these forces were the original 9,000 strong
force General Smith met during his march on New Madrid.

Once the Division of Observation and the XV Corps rejoin the XIII Corps, I will
have an additional 27 infantry regiments, 4 cavalry regiments, and 6 batteries
of artillery, adding another 18,000 soldiers to the defense of Ohio City.

I am endeavoring to send out scouts to probe for rebel activity at Sikeston and
south of Charleston. Once the entire army is in place, I may attempt to strike
at the enemy army, but it may also prove more useful to simply try and pin the
enemy army in place with a threat towards New Madrid, while General Smith
endeavors to take Columbus.

\gramClosing{I am always, General, and shall ever remain, Your most humble and obedient of servants}
    {Karl Meyer}
    {Maj. Gen., cmdg Army of the Arkansas}
\subsecdinkus

\gramHeader{Headquarters, Army of the Cumberland} % {{{4
    {Field Headquarters, Near Fort Donelson, Tenn., February}{3, 1862}
\gramTo{Major General}{K. Meyer}
    {Commanding the Army of the Arkansas}

\gramHi{General} Yours of this 3d inst. received and shared with General Blake.

For myself I heartily endorse your intentions to concentrate while preparing
infrastructure for a campaign later in the year.

Fixing the enemy army before a position he cannot take accomplishes the needful
at this time while simplifying your logistical constraints and exacerbating
his.

\gramClosing{Your humble servt}
    {Brig. Gen. H. Fawcett}
    {Army of the Cumberland}
\subsecdinkus

\gramHeader{United States Army} % {{{4
    {City of Washington, February}{3, 1862}
\gramTo{Maj. Gen.}{Karl Meyer}
    {Commanding, Dep't of the West}

\gramHi{General} I will be releasing to your command the three regiments of
infantry and three batteries of artillery from the reserve garrison at Cairo, to
be attached to the Department of the West forthwith.

I wish to know your appraisal of your ability to defend Ohio City with all the
forces you are marshaling to that point, plus the reinforcements aforementioned,
against the known enemy force against you, should it continue its advance toward
Ohio City and the confluence of the Mississippi and Ohio rivers. Respond
urgently.

\gramClosing{I remain vy rspy yr obt svt \&c}
    {C. N. Van Royne}
    {Major General cmdg, USA}
\subsecdinkus

\gramHeader{United States Army} % {{{4
    {City of Washington, February}{3, 1862}
\gramTo{Maj. Gen.}{Thomas Smith}
    {Commanding Dep't of the Tennessee}

\gramHi{General} I wish to hear from you. General Caldwell says you have moved
your army toward Columbus and have relieved Paducah. For this you have the
thanks of the Nation. Do you believe it practicable, with supply and weather
being what it is, to move aggressively on Columbus, determine the strength of
its landward fortifications, and, if circumstances favorable, to take the city
\& its defenses?

I have two cavalry regiments at Cairo available for service, but am waiting to
hear from Gen. Meyer before releasing them into his service or yours.

\gramClosing{I remain vy rspy yours \&c}
    {C. N. Van Royne}
    {Major General cmdg, USA}
\subsecdinkus

\gramHeader{Headquarters, Army of the Arkansas} % {{{4
    {Field Headquarters, Charleston, Mo. February}{3d, 1862}
\gramTo{Maj. Gen'ls.}{Cornelius Van Royne, James W. Blake, Thomas Smith}
    {Commanding the United States Army, Dep't of the Cumberland and Dep't of the Tennessee, respy}

\gramHi{General} Yours of this date just received. I am most appreciative of the
reinforcements for they are desperately needed. I am fairly confident that with
your reinforcements added to my ranks, my army of some 31,500 soldiers would be
able to resist the enemy, even if their entire force of 33,000 were brought to
bear against me. 

Furthermore, I believe that it is important that you and Generals Blake and
Smith are aware of a discrepancy that was made earlier last year after the
Battle of New Madrid. General Smith reported that the enemy force of 4,000
infantry and 5,000 cavalry that he encountered were from the Army of Trans
Mississippi. At the time, it was thought that this was General Clarke's army,
but recent information has revealed that this is in fact General Thomson's army
coming out of Arkansas.

This means that the portion of General Clarke's army still not accounted for is
23-25,000 strong rather than 14-16,000 strong. General Smith should be advised
that the force against him may be stronger than I first believed. 

\gramClosing{I am always, General, and shall ever remain, Your most humble and obedient of servants}
    {Karl Meyer}
    {Maj. Gen. cmdg Army of the Arkansas}
\subsecdinkus

\gramHeader{Headquarters, Army of the Tennessee} % {{{4
    {Field Headquarters, Paducah, Ky., Midday, February}{3, 1862}
\gramTo{Maj. Gen.}{C. N. Van Royne}
    {Commanding, United States Army}

\gramHi{General} Our two Corps have advanced southward from Paducah and taken
both the town of Viola and the Yellow Bluff just to its West. Our scouts have
thus far encountered no rebel infantry force; however, a large force of
approximately 4,500 rebel cavalry has presented itself just north of the town of
Melvine. We are currently planning offensive operations to either push the enemy
away from the town, or isolate and destroy their force completely. It is my hope
that reports relayed by the enemy cavalry will result in pressure being relieved
from our flanking armies. If not, we will continue to exploit this gap we have
found as far as the enemy allows.

\gramClosing{Yr. Obdt. Srvt.}
    {Thomas Caldwell}
    {Brig. Gen., Army of the Tennessee}
\subsecdinkus

\gramHeader{United States Army} % {{{4
    {City of Washington, February}{3, 1862}
\gramTo{Brig. Gen'ls.}{Thomas Caldwell \& Jeremiah Wentworth}
    {Army of the Tennessee}

\gramHi{Sirs} Yours of this date just received. Intelligence from Gen. Blake
suggests two large Confederate divisions may be in the Columbus vicinity, plus
the aforementioned cavalry. This cavalry may be the entirety of the mounted
contingent under Clarke's command, as Gen. Blake reports little if any cavalry
on his front.

You will be heavily outnumbered in saddles. You are strongly advised to proceed
cautiously, and to fall back on Paducah as your base if you can. Gen. Meyer has
indicated his confidence in repelling any further advances on Ohio City by the
rebels under Thomson, and Gen. Stoeffel has reported good progress in driving on
Bowling Green. If you find the opportunity to fall upon an isolated portion of
the enemy's strength, you must do so, but otherwise you should keep in mind that
you are likely equal in overall numbers, or close to it, to the enemy you are
facing.

\gramClosing{I remain vy rspy yours \&c}
    {C. N. Van Royne}
    {Major General cmdg, USA}
\subsecdinkus

\gramHeader{Headquarters, Army of the Arkansas} % {{{4
    {Field Headquarters, Charleston, Mo., Feb}{3d, 1862}
\gramTo{Generals}{Thomas Smith and Thomas Caldwell}
    {Army Of The Tennessee}

\gramHi{Generals} Yours of this date just received. I believe that as long as my
entire army arrives at Matthew's Prairie before the enemy attacks, I should be
able to protect Ohio City from enemy occupation.

General Van Royne has offered two regiments of cavalry for my forces at
Matthew's Prairie. Should I refuse, these units would be redirected to your own
army. I am currently outnumbered 13 to 7 in terms of cavalry whereas you are
outnumbered 15 to 10. Given that your attack is to relieve pressure on General
Blake and that taking Columbus would be important, I am inclined to tell General
Van Royne to direct that cavalry to the Army of the Tennessee, should you
believe that the additional two regiments could prove vital to your operations.

\gramClosing{I am always, General, and shall ever remain, Your most humble and obedient of servants}
    {Karl Meyer}
    {Maj. Gen. cmdg Army of the Arkansas}
\subsecdinkus

\gramHeader{Headquarters, Army of the Arkansas} % {{{4
    {Field Headquarters, Charleston, Mo. Feb}{3d, 1862}
\gramTo{Generals}{Thomas Smith and Thomas Caldwell}
    {Army Of The Tennessee}

\gramHi{Generals} Addendum to last. Given that General Smith faced more than
5,000 cavalry when he previously moved towards New Madrid, some 17 regiments
worth, I believe I am outnumbered 17 to 7 rather than 13 to 7. In light of this
realization, I will be accepting General Van Royne's offer of the cavalry lest I
be completely overwhelmed in terms of maneuver and screening.

\gramClosing{I am always, General, and shall ever remain, Your most humble and obedient of servants}
    {Karl Meyer}
    {Maj. Gen. cmdg Army of the Arkansas}
\subsecdinkus

\gramHeader{Headquarters, Army of the Cumberland} % {{{4
    {Near Fort Donelson, February}{3, 1862}
\gramTo{Maj. Gen.}{Cornelius Van Royne}
    {Commander, United States Army}

\gramHi{General} I have the honor to submit the following officers to be
considered for promotion:

{
    \centering
    \begin{dispatch}{
        colspec        = {l|l},
        cell{1}{1}     = {c},
        cell{1,3,5}{2} = {c},
    }

        \textit{To Colonel of Volunteers:} & \textit{To Colonel of Regulars:} \\
        Lieut. Col. William Choate            & Lieut. Col. Stephen D. Carpenter \\
        Lieut. Col. David Dunn                & \textit{To Colonel of Volunteers (Cavalry):} \\
        Lieut. Col. William Kise              & Lieut. Col. Edward McCook \\
        Lieut. Col. Charles Levanway          & \textit{To Captain of Regulars (Artillery):} \\
        Lieut. Col. Joseph Hawkins            & Lieut. S. Canby \\
        Lieut. Col. Melvin Mudge \\
        Lieut. Col. Edmund Schriver \\
        Lieut. Col. Karl Sonderson \\
        Lieut. Col. Dwella Stoughton \\
        Lieut. Col. Henry von Trebra \\
    \end{dispatch}
    \par
}

These officers have served faithfully in command positions commensurate with the
rank of their requested promotions. I request that they be promoted with a date
of rank of February 1st, 1862.

\gramClosing{Respectfully}
    {J. W. Blake}
    {Maj. Gen., Commander, Army of the Cumberland}
\subsecdinkus

\gramHeader{United States Army} % {{{4
    {City of Washington, February}{3, 1862}
\gramTo{Maj. Gen.}{James Blake}
    {Commanding Dep't of the Cumberland}

\gramHi{Sir} Yours of this date just received. Your requests for promotion for
the listed officers are approved forthwith.

\gramClosing{I remain vy rspy yours \&c}
{C. N. Van Royne}
{Major General cmdg, USA}\subsecdinkus

\gramHeader{United States Army} % {{{4
    {City of Washington, February}{3, 1862}
\gramTo{Maj. Gen.}{James Blake}
    {Commanding Dep't of the Cumberland}

\gramHi{Sir} Addendum to last. With all intelligence gathered to us, it is clear
all enemy cavalry in the area is concentrated in the defense of New Madrid and
Columbus. Your cavalry division is thus unopposed. This is a rare opportunity
that must be capitalized on. I urge you to threaten the supply lifeline to Fort
Donelson by whatever means with your troopers and to press your numerical
advantage in guns and infantry at whatever point. 

\gramClosing{I remain vy rspy yours \&c}
    {C. N. Van Royne}
    {Major General cmdg, USA}
\subsecdinkus

\gramHeader{Headquarters, Army of the Cumberland} % {{{4
    {Near Fort Donelson, Tenn., February}{3, 1862}
\gramTo{Maj. Gen.}{Cornelius Van Royne}
    {Commander, United States Army}

\gramHi{General} I write to regretfully inform you that no such cavalry
operation will be possible at least until Spring as the horses would find it
quite difficult to graze at this time of year. Regardless, the terrain here is
not as conducive to deep raids as I would like. Once we clear the forts however,
there will be more possibility to use our cavalry to pressure the enemy behind
his lines.

\gramClosing{Respectfully}
    {J. W. Blake}
    {Maj. Gen., Commanding}
\subsecdinkus

\gramHeader{United States Army} % {{{4
    {City of Washington, February}{3, 1862}
\gramTo{Brig. Gen'ls.}{Thomas Caldwell \& Jeremiah Wentworth}
    {Army of the Tennessee}

\gramHi{Sir} Yours of this date just received. Intelligence from Gen. Blake
suggests two large Confederate divisions may be in the Columbus vicinity, plus
the aforementioned cavalry. This cavalry may be the entirety of the mounted
contingent under Clarke's command, as Gen. Blake reports little if any cavalry
on his front.

You will be heavily outnumbered in saddles. You are strongly advised to proceed
cautiously, and to fall back on Paducah as your base if you can. Gen. Meyer has
indicated his confidence in repelling any further advances on Ohio City by the
rebels under Thomson, and Gen. Stoeffel has reported good progress in driving on
Bowling Green. If you find the opportunity to fall upon an isolated portion of
the enemy's strength, you must do so, but otherwise you should keep in mind that
you are likely equal in overall numbers, or close to it, to the enemy you are
facing.

\gramClosing{I remain vy rspy yours \&c}
    {C. N. Van Royne}
    {Major General cmdg USA}
\subsecdinkus

\gramHeader{Headquarters, Army of the Cumberland} % {{{4
    {Near Fort Donelson, Tenn., February}{3, 1862}
\gramTo{Maj. Gen.}{Cornelius Van Royne}
    {Commanding General, United States Army}

\gramHi{Sir} I am writing to apprise you of the options before us here at Fort
Donelson. With the withdrawal of Whisper's Division, I believe the enemy force
here numbers around 30,000 compared to my nearly 40,000 infantry. However, the
enemy enjoys the advantage of excellent terrain likely making any direct assault
upon their positions quite costly.

To our advantage the enemy positions to my west and south are separated by about
two miles of rough terrain; movement between the two wings of the enemy is only
via a route farther behind his lines. If we could move through this gap there is
an opportunity to drive a wedge in the center of the enemy, forcing him to
either attack or withdraw his western force sitting astride the road to Fort
Henry. However, the rough terrain that precludes easy movement between the
enemy's two wings would also make such a movement of my army difficult.

A second option is to fix half the enemy in place while concentrating for an
attack against an exposed flank in his center. Such a movement would not be as
difficult as driving between the wings but would be more costly given the
terrain and because bringing significant amounts of artillery to bear is likely
not possible.

Either option, if successful, would unhinge the enemy position but could result
in significant casualties requiring quick reinforcement to maintain our
position.

A third option is to retreat to a defensible position against the Cumberland
River, construct defenses for a single corps, and cross the river with the
remainder of the Army to move on Bowling Green in support of Maj. Gen.
Stoeffler's operation.  This option is a significant departure from your
strategic plan and would only be attempted with your approval.

Finally, my army could defend its  current position, waiting for events
elsewhere to force the enemy to weaken his lines. This would, however, give the
enemy the time to complete his fortifications, making any subsequent attack more
difficult.

\gramClosing{Respectfully}
    {J. W. Blake}
    {Maj. Gen., Commanding}
\subsecdinkus

\gramHeader{United States Army} % {{{4
    {City of Washington, February}{3, 1862}
\gramTo{Maj. Gen.}{James Blake}
    {Commanding Dep't of the Cumberland}

\gramHi{Sir} Yours this date just received. Between the two options presented,
that is engaging directly the enemy before you, I wish to know which of the two
you believe promises the greater chance of success; not just of the plan itself,
tactically, but toward the greater design of the seizure of Donelson and the
liberation of the Cumberland through to Nashville.

I have not yet ruled out the option to have you redeploy the bulk or all of your
strength eastward, to fall upon Bowling Green from the West, though to do so
would require a great deal more expenditure in supply to supply you by rail than
by the supply boats coming from Cairo.

\gramClosing{I remain vy rspy yours \&c}
{C. N. Van Royne}
{Major general cmdg USA}\subsecdinkus

\gramHeader{Headquarters, Army of the Cumberland} % {{{4
    {Field Camp, Near Fort Donelson, Tenn., February}{3, 1862}
\gramTo{Maj. Gen.}{Cornelius Van Royne}
    {Commander, United States Army}

\gramHi{Sir} In response to your last, I provide the following:

Either offensive option, if successful, would further your operational goals
here in Tennessee if the enemy were forced to withdraw or attack into our own
strong position. Tactically, I believe a strike directly into the gap between
the enemy positions is more likely to succeed but also carries the highest risk
as the attacking force would itself be somewhat isolated from support. An attack
into an exposed flank is more conservative but could also result in greater loss
of men as the enemy, presumably has his flanks defended.

In any case, until the weather improves, the ground dries, and the creeks lower,
neither of these options is feasible. Such a delay does give the enemy more time
to complete his field works, increasing the difficulty of any frontal attack.

I regret that my response only further complicates matters instead of providing
a clearly superior option. The situation here, unfortunately, either requires
success by the Armies to my flanks or an acceptance of losses equivalent to
those of the action of the 25th ult. as the enemy is determined to fight for
this ground.

\gramClosing{Respectfully}
    {J. W. Blake}
    {Maj. Gen., Commanding}
\subsecdinkus

\gramHeader{Headquarters, Army of the Arkansas} % {{{4
    {Field Headquarters, Charleston, Mo., Feb}{3d, 1862}
\gramTo{Maj. Gen.}{Cornelius Van Royne}
    {Commanding General of the United States Army}

\gramHi{General} I would like to inquire upon rumors that General Steele and the
Army of the Kanawha are to join the Western Theater. The presence of an
additional 30,000 or so men would certainly assist General Stoeffler's advance
on Bowling Green, leaving General Blake free to concentrate on Fort Donelson or
else assist General Smith in facing off against enemies on the eastern bank of
the Mississippi. 

\gramClosing{I am always, General, and shall ever remain, Your most humble and obedient of servants}
    {Karl Meyer}
    {Maj. Gen. cmdg, Army of the Arkansas}
\reportdinkus

\gramHeader{Headquarters, Army of the Tennessee} % {{{4
    {Paducah, Ky., Evening, February}{3, 1862}
\gramTo{Maj. Gen.}{James W. Blake}
    {Commander, Army of the Cumberland}

\gramHi{General} Our Army's operations south of Paducah have, so far, borne
success. We have confirmed a large enemy cavalry force operating just East of
Columbus and plan to engage this coming week. This message is to inquire as to
whether you have seen rebel forces moving West from your current siege of Fort
Donelson and to request that, if they are not needed for other operations, your
cavalry be deployed to the Wadesborough or Murray areas to act as a warning
against any rebel force moving to counter our advance. I wish you success in
your endeavor at Fort Donelson and victory for our cause on either side of the
Tennessee.

\gramClosing{Yr. Obdt. Srvt.}
    {Thomas Caldwell}
    {Brig. Gen., Army of the Tennessee}
\reportdinkus

\gramHeader{Headquarters, Army of the Cumberland} % {{{4
    {Near Fort Donelson, Tenn., February}{3, 1862}
\gramTo{Brig. Gen.}{Thomas Caldwell}
    {Army of the Tennessee}

\gramHi{Sir} My cavalry reports an force of enemy cavalry of unknown size was
seen near Fort Henry on the east bank of the Tennessee River yesterday.
Whisper's Division of the Army of the Tennessee is also known to be in that
location.

I have no indication that any forces have decamped from the Fort Donelson area
since the departure of Whisper's Division on the 30th ult.

I regret that I cannot send my cavalry to support you as that would leave my own
army blind. I will continue to keep you informed as to any enemy forces moving
westward.

\gramClosing{Respectfully}
    {J. W. Blake}
    {Maj. Gen., Commanding}
\reportdinkus

\gramHeader{Headquarters, Army of the Cumberland} % {{{4
    {Near Fort Donelson, Tenn., February}{3, 1862}
\gramTo{Brig. Gen'ls.}{Ptolemy Smith, Charles Smith, John McClernand, Lawrence Graham}
    {Commanding VIIIth, XIIth and XIVth Corps, and the Cavalry Division, resp'y}
\gramTo{Col.}{Charles Cotter}
    {Commanding the Artillery Reserve}

\gramHi{Sirs} The army will remain here near Fort Donelson at least until the
weather improves. We will maintain our positions in the west and south but
adjust somewhat so that 1st Division, XIIth Corps can be moved into a reserve
position. You will array your divisions and brigades as laid forth here.

XIVth Corps will maintain its general position along Bufford's Hill and near
the Sawmill.

1st Division, XIIth Corps will move into a reserve position from where it can
easily move to support our western or southern positions.

VIIIth Corps and 2d Division XIIth Corps will extend their lines to cover the
position vacated by 1st Division, XIIth Corps.

1st Brigade, Artillery Reserve will move to a reserve position supporting VIIIth
and XIIth Corps while 2d Brigade will remain in its current position, ready to
support XIVth Corps or move south if ordered.

The Cavalry Division is responsible for screening our northern flank as well as
the gap between VIIIth and XIVth Corps.

\gramClosing{Respectfully}
    {J. W. Blake}
    {Maj. Gen., Commanding}
\reportdinkus

\gramHeader{United States Army} % {{{4
    {City of Washington, February}{3, 1862}
\gramTo{Maj. Gen.}{James Blake}
    {Commanding Dep't of the Cumberland}

\gramHi{Sir} Yours this date just received. For the time being, keep all options
open to you and attempt to develop the situation as you see fit. You enjoy our
confidence and we are therefore happy with any attempt you make against the
enemy position. Keep open option of withdrawal if necessary. Gen'ls Smith \&
Stoeffel are pressing the enemy before them, so odds of enemy reinforcement are
low but remain possible.

\gramClosing{I remain vy rspy yours \&c}
    {C. N. Van Royne}
    {Major general cmdg, USA}
\secdinkus

\subsection*{February 7, 1862}{} %{{{3

\gramHeader{Headquarters, Army of the Tennessee} % {{{4
    {Paducah, Ky., Morning, February}{7, 1862}

\gramTo{Maj. Gen'ls.}{C. N. Van Royne, Karl Meyer}
    {Commanding United States Army and Army of Arkansas, respectively}

\gramHi{Generals} The past few days have borne success. The enemy cavalry in our
area chose not to give battle and instead withdrew to Columbus. Accordingly, we
have liberated the towns of Milburn and Mayfield. We now intend to complete the
landward isolation of Columbus by taking Clinton and it's environs. We currently
have no intelligence of any enemy presence to our East, but also no means to
acquire any due to our cavalry being needed for our operations. Regardless, the
Cumberland reports no forces have left their front, so we shall continue and
attempt bring this small campaign to a close.

\gramClosing{Yr. Obdt. Srvt.}
    {Thomas Caldwell}
    {Brig. Gen., Army of the Tennessee}
\secdinkus

\subsection*{February 10, 1862}{} %{{{3

\gramHeader{Headquarters, Army of the Arkansas} % {{{4
    {Field Headquarters, Charleston, Mo., Feb.}{10th, 1862}
\gramTo{Maj. Gen.}{Cornelius Van Royne}
    {Commanding General of the United States Army}

\gramHi{General} I have managed to bring together my entire army, which is now
positioned at Charleston between the enemy and Ohio City. There has been an
increase in enemy cavalry -at least two divisions confirmed, possibly three- in
front of my positions in the last few days and reports of large dust clouds some
six to eight miles southwest of Charleston. No attacks have come in but minor
skirmishing has commenced. I believe that the enemy is forming up to attack my
position from southwest of Charleston and march for Ohio City within the next
few days, though the light rain present may impede them. 

\gramClosing{I am always, General, and shall ever remain, Your most humble and obedient of servants}
    {Karl Meyer}
    {Maj. Gen. cmdg Army of the Arkansas}
\reportdinkus

\gramHeader{Headquarters, Army of the Cumberland} % {{{4
    {Near Fort Donelson, Tenn., February}{10, 1862}
\gramTo{Maj. Gen.}{Cornelius Van Royne}
    {Commanding, United States Army}

\gramHi{Sir} As stated in my last message to you, I am now facing portions of
three different Confederate armies with strength roughly equal to my own. As
such, although I believe I can continue to hold this ground, driving the enemy
back from Fort Donelson could prove to be next to impossible.

As suggested previously, a landing to take Fort Henry by the Army of the
Tennessee, could find success, although Whisper's Division may be retreating in
that direction.

If that is not possible I ask that my army receive additional forces to tip the
balance in our favor. And if not that, I urge that the armies to my east and
west move with all haste so that the enemy will be forced to react and withdraw
troops from the area between the rivers.

\gramClosing{Respectfully}
    {J. W. Blake}
    {Maj. Gen., Commanding}
\reportdinkus

\gramHeader{United States Army} % {{{4
    {City of Washington, February}{16, 1862}
\gramTo{Maj. Gen.}{James Blake}
    {Commanding Dep't of the Cumberland}

\gramHi{Sir} Yours of the 8th \& 10th inst. just received. I am in communication
with Gen. Steele to expedite the transfer of part of his command to your relief,
as well as to determine whether any movement by a large part of his army toward
Fort Henry is viable, and will furthermore encourage Gen. Stoeffel to drive hard
on Bowling Green.

Gen. Meyer is forced to deal with a saucy attack against Ohio City from the
Missouri panhandle, and Gen. Smith's command I have instructed to apply maximal
pressure against Columbus.

Will you be able to continue commanding the length of the Cumberland from the
confluence with the Ohio down to your present position? Are you able to
adequately supply your army for the coming weeks?

\gramClosing{I remain vy rspy yours \&c}
{C. N. Van Royne}
{Major General, cmdg USA}
\reportdinkus

\gramHeader{United States Army} % {{{4
    {City of Washington, February}{10, 1862}
\gramTo{Maj. Gen.}{Christopher Stoeffel}
    {Commanding, Army of the Kentucky}

\gramHi{Sir} Gen. Blake's predicament is cause for great concern in the capital.
I have seen fit to ask Gen. Steele to detach a part of his force and to
accompany it westward for possible operations up the length of the Tennessee.

Gen. Smith is applying great pressure against Columbus, and now Gen. Blake
reports a division of cavalry facing him. It is as yet unknown if that cavalry
is from Whisper's army, from Clarke's or detached from Jackson's. I must again
impress upon you the need to advance with boldness and celerity against the
enemy in Bowling Green, and request at your earliest possible convenience an
update on your efforts to that end. I am confident you enjoy strong numerical
superiority against the enemy to your front.

\gramClosing{I remain vy rspy yours \&c}
{C. N. Van Royne}
{Major General, cmdg USA}
\reportdinkus

\gramHeader{Headquarters, Army of the Cumberland} % {{{4
    {Near Fort Donelson, Tenn., February}{10, 1862}
\gramTo{Maj. Gen.}{Cornelius Van Royne}
    {Commander, United States Army}

\gramHi{Sir} With the new ironclad recently delivered to the Mississippi
Squadron, the Cumberland River will remain secure. Any attempt by the enemy to
construct batteries to interfere with our control would be identified in time to
react.

Supply to my army is secure as long as I can keep the enemy away from the
landing. With my shortened front and good interior lines I believe I can hold
for now. Reinforcement from Maj. Gen. Steele would be most welcome as would
serious advances on Columbus and Bowling Green.

\gramClosing{Respectfully}
    {J. W. Blake}
    {Maj. Gen., Commanding}
\reportdinkus

\gramHeader{Headquarters, Army of the Arkansas} % {{{4
    {Field Headquarters, Charleston, Mo., Feb}{10th, 1862}
\gramTo{Maj. Gen.}{Cornelius Van Royne}
    {Commanding General of the United States Army}

\gramHi{General} Yours this date received. After consulting cavalry scouts as
well as the maps in the region, there is a strong suspicion that the enemy is
currently at East Prairie some 6-8 miles southwest of Charleston, where he is
flanked by significant marshland to the east and minor marshland to the west.
General Ambrose is of the opinion that with all our forces marshalled as one,
there is no better time to advance upon the enemy and engage them in battle,
where his cavalry will be of less use and our artillery can be formed into a
grand battery to fire into them.

I must admit that I myself am of a more phlegmatic nature, hesitant to engage in
another major battle if one can be avoided and the road to Ohio City still
protected. I do not wish to squander your reinforcements on an attack if the
enemy is content to sit in place instead of redeploying. It must be said
however, that the region around Charleston is far more open, and therefor the
enemy will most likely gain much more use of their cavalry advantage were they
to attack us. General Ambrose is also convinced that a battle no matter
offensive or defensive will prove to be quite bloody either way.

If you have any strong opinions of either course of action, I would be content
to follow them. If not, I will take the counsel of General Ambrose and advance
upon the enemy once good weather arrives.

\gramClosing{I am always, General, and shall ever remain, Your most humble and obedient of servants}
{Karl Meyer}
{Maj. Gen. cmdg Army of the Arkansas}
\reportdinkus

\gramHeader{United States Army} % {{{4
    {City of Washington, February}{10, 1862}
\gramTo{Maj. Gen.}{Christopher Stoeffel}
    {Commanding, Army of the Kentucky}

\gramHi{Sir} Addendum to last. You must forgive the flurry of messages but Gen.
Caldwell (cmdg a wing of the Army of the Tennessee) reports that the defenses of
Columbus are weaker than expected. The whole of Clarke's army, plus all of
Whisper's, is committed to defending Donelson. It is likely Jackson has detached
some force as well. It is likely you outnumber him two to one, perhaps more.
Cavalry may be more of a concern for you but I trust you to employ whatever
means to present as menacing a threat as possible to Bowling Green.

\gramClosing{I remain vy rspy yours \&c}
    {C. N. Van Royne}
    {Major General, cmdg USA}
\reportdinkus

\gramHeader{United States Army} % {{{4
    {City of Washington, February}{10, 1862}
\gramTo{Maj. Gen.}{Karl Meyer}
    {Commanding, Dep't of the West}

\gramHi{General} Yours of this date just received. It is understandable to wish
to capitalize on the enemy's inability to utilize the whole of the numerically
superior body of cavalry. However, if the enemy has halted in position, it may
very well be an invitation by them to attack.

A defeat could mean a vigorous enemy pursuit, and your smaller cavalry
contingent may not be capable of protecting the army should it have to retreat
to Ohio City.

No doubt the number of guns you have is superior to the enemy's. As hesitant as
I am for further bloodshed, particularly after the recent calamity; to strike
while the enemy's cavalry cannot fully be brought to bear may be the wisest
course of action. I nevertheless leave this decision to you, and trust you fully
no matter the course you take.

\gramClosing{I remain vy rspy yr obt svt \&c}
{C. N. Van Royne}
{Major General, cmdg USA}
\reportdinkus

\gramHeader{Headquarters, Army of the Arkansas} % {{{4
{Field Headquarters, Charleston, Mo., Feb}{10th, 1862}
\gramTo{Maj. Gen.}{Cornelius Van Royne}
    {Commanding General of the United States Army}

\gramHi{General} Addendum to last. I thank you for your trust, and seek to add a
secondary reason for a possible attack on the enemy.

When General Smith and the Army of the Tennessee launch their attack upon
Columbus, I do not wish for elements of General Thomson's Army of the
Trans-Mississippi to come to the city's aid. Therefore I will be attacking in
conjunction with General Smith so as to pin down the enemy present on the west
side of the Mississippi, and prevent them from ferrying forces across the river.

\gramClosing{I am always, General, and shall ever remain, Your most humble and obedient of servants}
    {Karl Meyer}
    {Maj. Gen. cmdg Army of the Arkansas}
\reportdinkus

\gramHeader{United States Army} % {{{4
    {City of Washington, February}{10, 1862}
\gramTo{Maj. Gen.}{Karl Meyer}
    {Commanding, Dep't of the West}

\gramHi{General} Yrs this date just received. Do you have any vessels on the
Mississippi which could prevent a junction of Thomson's force with the garrison
of Columbus, presuming you are unable for whatever reason to make the attack
south of Charleston?

\gramClosing{I remain vy rspy yr obt svt \&c}
    {C. N. Van Royne}
    {Major General, cmdg USA}
\reportdinkus

\gramHeader{United States Army} % {{{4
    {City of Washington, February}{10, 1862}
\gramTo{Brig. Gen'ls.}{Thomas Caldwell \& Jeremiah Wentworth}
    {Army of the Tennessee}

\gramHi{Sirs} Yrs of 3d inst. just received. Gen. Blake's predicament has
worsened. He has repeatedly beaten back the enemy with favorable losses, but is
beleaguered, and Fort Donelson does not appear that it will fall soon. Gen.
Stoeffel is being encouraged to press boldly against Jackson's army in Bowling
Green, and I have seen fit to order Gen. Steele to prepare the movement of most
of his command from western Virginia to Tennessee.

Blake now reports some body of cavalry to his north, on the left bank of the
Cumberland. This must be either from Whisper's, Jackson's, or Clarke's army.
Columbus may be mightily weakened. You must press your advantage while the
opportunity presents.

The enemy advances against Gen. Meyer in great strength toward Charleston \& Ohio
City and likely aims to put batteries on the confluence of the Ohio \&
Mississippi. The seizure of Columbus will preclude any such designs by the
enemy. If Gen. Steele's army can be moved to Blake's support, you will find your
rear completely clear of the enemy, and can enjoy full freedom of maneuver.

\gramClosing{I remain vy rspy yours \&c}
    {C. N. Van Royne}
    {Major General, cmdg USA}
\reportdinkus

\gramHeader{Headquarters, Army of the Arkansas}
    {Field Headquarters, Charleston, Mo., Feb}{3d, 1862}
\gramTo{Gen'ls}{Thomas Smith and Thomas Caldwell}
    {Army Of The Tennessee}

\gramHi{Generals} I wish to inquire to your intent and progress to siege
Columbus. At present, I am encountering what may be the entirety, or at least
the majority of the Army of the Trans-Mississippi. I do not want to hasten your
proceedings, but seek to coordinate with your movements as well as understand
what possible enemy forces you are encountering.

\gramClosing{I am always, General, and shall ever remain, Your most humble and obedient of servants}
    {Karl Meyer}
    {Maj. Gen. cmdg Army of the Arkansas}
\reportdinkus

\gramHeader{Headquarters, Army of the Tennessee}
    {Field Headquarters, Kentucky City, Ky., February}{10, 1862}
\gramTo{Maj. Gen'ls.}{C. N. Van Royne \& Karl Meyer}
    {Commanding United States Army and Army of the Arkansas, respectively}

\gramHi{Generals} We have liberated the town of Clinton and invested Columbus
and it's accompanying fort. The previously reported force of approximately 4,500
cavalry sadly avoided the encirclement and rode south, with a report from a
civilian that at least some of their number rode east through the town of
Morrisville. General Smith has yet to issue the final order for an assault to
take place, but based on messages received from your offices I anticipate action
in the coming days. I have advised to order our garrison at Paducah move to
Viola in order to protect our immediate supply line. This will leave Paducah
unprotected, but may prove necessary due to the enemy cavalry's escape and
reported eastern movement.

I regret that the rebels, whether through ignorance or arrogance, have neglected
to reallocate forces to counter our advance. But I hope in doing so they have
spelled the end of their influence on this part of the Mississippi.

\gramClosing{Yr. Obdt. Srvt.}
    {Thomas Caldwell}
    {Brig. Gen., Army of the Tennessee}
\reportdinkus

\gramHeader{Headquarters, Army of the Arkansas} % {{{4
    {Field Headquarters, Charleston, Mo., Feb}{10th, 1862}
\gramTo{Gen'ls}{Thomas Smith and Thomas Caldwell}
    {Army Of The Tennessee}

\gramHi{Generals} Yours this date received. I am gladdened that your army has
managed to invest Columbus and will be ordering the Mississippi Squadron to
begin bombardment operations against the city's defenses as well as the fort to
assist in your endeavor.

I understand that you have need to protect your supply line from secessionist
raiders. Do you believe that you need the 3d Brigade to assist with those
endeavors now that the enemy cavalry force seems to have dispersed?

If General Smith does intend to move upon the enemy within a matter of days,
then I shall begin an attack of my own to pin down the enemy army in front of me
in order to prevent their redeployment. I would be most grateful for the extra
four regiments, though would equally understand if you deem them essential for
protecting Mayfield, Viola, and Paducah from enemy raids.

\gramClosing{I am always, General, and shall ever remain, Your most humble and obedient of servants}
    {Karl Meyer}
    {Maj. Gen. cmdg Army of the Arkansas}
\reportdinkus

\gramHeader{Headquarters, Army of the Tennessee} % {{{4
    {Field Headquarters, Kentucky City, Ky., February}{10, 1862}
\gramTo{Maj. Gen.}{Karl Meyer}
    {Army of Arkansas}

\gramHi{General} I do not currently believe the enemy cavalry has dispersed
completely, and in fact their riding east may be an attempt to strike at our
rear or supply depots. I also believe that their movement back to your force
would take a number of crucial days. Therefore, I request that they remain
attached for the coming week. If Columbus can be taken, a second source of
supply can be used and therefore our supply line to Paducah will no longer be as
vital to defend. If that comes to pass I will recommend to General Smith that we
dispatch them back to you with haste.

\gramClosing{Yr. Obdt. Srvt.}
    {Thomas Caldwell}
    {Brig. Gen., Army of the Tennessee}
\reportdinkus

\gramHeader{Headquarters, Army of the Cumberland} % {{{4
    {Near Fort Donelson, Tenn., February}{10, 1862}
\gramTo{Brig. Gen.}{Thomas Caldwell}
    {Army of the Tennessee}

\gramHi{General} I welcome your news of the investiture of Columbus. If you
believe you will need gunboat support for your assault I ask that you coordinate
such through my headquarters so I may advise Cdre. Lewis.

The priority will, of course, remain security of navigation along the rivers,
but the Mississippi Squadron should be fully capable of bombarding Columbus,
especially with the recent arrival of the ironclad.

\gramClosing{Respectfully}
    {J. W. Blake}
    {Maj. Gen., Commanding}
\reportdinkus

\gramHeader{Headquarters, Army of the Cumberland} % {{{4
    {Near Fort Donelson, Tenn., February}{10, 1862}
\gramTo{Cdre.}{Daniel Lewis}
    {U. S. N.}

\gramHi{Sir} I write to humbly ask if there has been any change in the
relationship between your Mississippi Squadron and the armies of the West. Are
requests for naval support still intended to be routed through the headquarters
of this army or has another arrangement been ordered?

The arrival of the ironclad St. Louis off the landing here in Tennessee was a
welcome sight. I wish to inform you that Maj. Gen. Smith's Army of the Tennessee
may have need of her shortly to assist in his upcoming assault on the enemy
first at Columbus, Ky. We are currently waiting for Gen'l Smith to confirm the
date of his attack and will forward that to you soonest.

\gramClosing{For Maj. Gen. J. Blake,}
    {George Campbell}
    {Capt., Aide de Camp}
\reportdinkus

\gramHeader{Headquarters, Army of the Arkansas} % {{{4
    {Field Headquarters, Charleston, Mo., Feb}{10th, 1862}
\gramTo{Gen'ls}{Thomas Smith and Thomas Caldwell}
    {Army Of The Tennessee}

\gramHi{Generals} Yours this date received. I understand your position, and wish
you the greatest of success to carry the walls of Columbus. Please inform me
what date General Smith plans to attacks the city so that I may launch my own
attack in conjunction with his own. I intend to pin down the Army of the
Trans-Mississippi so that they will not be able to send reinforcements to
Columbus, or else suffer for it when the Army of the Arkansas marches for New
Madrid.

\gramClosing{I am always, General, and shall ever remain, Your most humble and obedient of servants}
    {Karl Meyer}
    {Maj. Gen. cmdg Army of the Arkansas}
\reportdinkus

\gramHeader{United States Army} % {{{4
    {City of Washington, February}{10, 1862}
\gramTo{Brig. Gen.}{Thomas Caldwell}
    {Army of the Tennessee}

\gramHi{Sir} You are hereby ordered to assume command of the Army of the
Tennessee, effective immediately. Maj. Gen'l Smith has been relieved of command
and recalled to Washington to await further orders.

\gramClosing{I remain vy rspy yours \&c}
    {C. N. Van Royne}
    {Major General, cmdg USA}
\reportdinkus

\gramHeader{Headquarters, Army of the Arkansas} % {{{4
    {Field Headquarters, Charleston, Mo., Feb}{10th, 1862}
\gramTo{Gen'ls}{Cornelius Van Royne, Thomas Smith, Thomas Caldwell and James W.  Blake}
    {Commanding General of the United States Army, Army of the Tennessee and Army of the Cumberland, resp'y}

\gramHi{Generals} Yrs this date just received. Unfortunately, all ships upon the
Mississippi River are currently attached to General Blake's command through Flag
Officer Lewis. It is for that reason that I request that an attack by the
Mississippi Squadron be made on the city of Columbus preceding or at least in
conjunction with General Smith's attack.

As for interdicting enemy movements across the river, it is my regret to inform
you that the fort at Columbus prevents any watercraft from passing the city. The
Army of the Trans-Mississippi would most likely take a ferry from New Madrid and
then land on the peninsula near the town of Obionville, far from the reach of
our gunboats.

It is my belief that the only possibility of pinning them is to launch a full
scale assault upon them so as to prevent them from moving forces away from
Charleston, or else suffer the consequences of engaging in a battle without
their full force, a force that is currently in a state of parity with my own
army.

\gramClosing{I am always, General, and shall ever remain, Your most humble and obedient of servants}
    {Karl Meyer}
    {Maj. Gen. cmdg Army of the Arkansas}
\reportdinkus

\gramHeader{United States Army} % {{{4
    {City of Washington, February}{10, 1862}
\gramTo{Brig. Gen.}{Thomas Caldwell}
    {Army of the Tennessee cmdg}

\gramHi{Sir} Was a battery ever established on Island No. 5?

\gramClosing{I remain vy rspy yours \&c}
    {C. N. Van Royne}
    {Major General, cmdg USA}
\reportdinkus

\gramHeader{Headquarters, Army of the Tennessee} % {{{4
    {Field Headquarters, Kentucky City, Ky., February}{10, 1862}
\gramTo{Maj. Gen'ls.}{Karl Meyer \& James W. Blake}
    {Commanding the Army of Arkansas \& the Army of the Cumberland, resp'y}

\gramHi{Generals} I plan to move forward with an attack against the enemy's
works as soon as the XVI Corps can move into position and prepare for the
assault, estimated February 12th. Scouts have confirmed that no small amount of
large bore artillery is emplaced on the river side of the fort and would caution
against a heavy gunboat presence if no real benefit would result from a
bombardment. With my current timeline I do not believe that any force on the
west bank of the Mississippi could redeploy to Columbus before the issue is
resolved.

\gramClosing{Yr. Obdt. Srvt.}
    {Thomas Caldwell}
    {Brigadier General, Commanding Army of the Tennessee}
\reportdinkus

\gramHeader{Headquarters, Army of the Tennessee} % {{{4
    {Field Headquarters, Kentucky City, Ky., February}{10, 1862}
\gramTo{Maj. Gen.}{C. N. Van Royne}
    {Commanding General, United States Army}

\gramHi{General} From my position near Columbus I do not see any such
construction on the island. I believe this was determined to be impracticable
after the engagement at Sikeston. Part of my motivation in not commencing a long
term bombardment of Columbus fort is to take the fort reasonably intact with its
heavy pieces to control the Mississippi traffic from this bank.

\gramClosing{Yr. Obdt. Srvt.}
    {Thomas Caldwell}
    {Brigadier General, Commanding Army of the Tennessee}
\reportdinkus

\gramHeader{United States Army} % {{{4
    {City of Washington, February}{10, 1862}
\gramTo{Brig. Gen.}{Thomas Caldwell}
    {Army of the Tennessee cmdg}

\gramHi{Sir} Yrs this date just received. Capture of the enemy pieces at
Columbus would be of great help to this Nation, but ultimately, should you find
any method by which the city can be taken in the most efficient manner possible,
even if it risks the permanent loss of the guns, then the attempt must be made.
The captured guns could be used to establish forts elsewhere, but in any case
the loss of Columbus would make enemy riverine traffic north of Memphis
exceedingly difficult.

\gramClosing{I remain vy rspy yours \&c}
    {C. N. Van Royne}
    {Major General, cmdg USA}
\reportdinkus

\gramHeader{Headquarters, Army of the Arkansas} % {{{4
    {Field Headquarters, Charleston, Mo., Feb}{10th, 1862}
\gramTo{Brig. Gen.}{Thomas Caldwell}
    {Army Of The Tennessee}

\gramHi{General} Yours this date just received. If you believe that no forces on
the west bank are likely to impede you, then in light of the rain that is likely
to continue the entire week, I am holding off on any attacks until at least the
start of next week. Instead, I will be continuing to probe the enemy lines with
my cavalry.

\gramClosing{I am always, General, and shall ever remain, Your most humble and obedient of servants}
    {Karl Meyer}
    {Maj. Gen. cmdg Army of the Arkansas}
\reportdinkus

\gramHeader{Headquarters, Army of the Arkansas} % {{{4
    {Field Headquarters, Charleston, Mo., February}{10th, 1862}
\gramTo{Gen'ls}{Cornelius Van Royne and Thomas Caldwell}
    {Commanding Generals of the United States Army and the Army Of The Tennessee, resp'y}

\gramHi{Generals} My staff officers have informed me that the current weather
will not impede any attack on East Prairie. For that reason I will be advancing
upon the enemy as soon as my final cavalry regiment from Ironton rejoins my
army. I estimate that the regiment will arrive on the 12th and that I will be
making contact with the enemy on the 13th or 14th instant.

\gramClosing{I am always, General, and shall ever remain, Your most humble and obedient of servants}
{Karl Meyer}
{Maj. Gen. cmdg Army of the Arkansas}
\reportdinkus

\subsection*{February 17, 1862}{} % {{{3
\gramHeader{Headquarters, Army of the Cumberland} % {{{4
{Happy Hollow, Tenn., February}{17, 1862}
\gramTo{Maj. Gen.}{Cornelius Van Royne}
    {United States Army, Commanding}

\gramHi{Sir} I have the honor to submit the following report on the situation
here near Fort Donelson. The Army of the Cumberland currently numbers 37,742
infantry, 3,271 cavalry and 162 guns. All three corps now benefit from field
works constructed over the past week. Our supply situation is good and the men's
morale remains high.

Whisper's Division has returned to the enemy lines to our south, having
abandoned the attempted maneuver to our north. An unknown amount of cavalry from
the Army of Mississippi has arrived to join the Army of East Tennessee cavalry
division already present. Estimates of enemy losses so far would give the enemy
a total force of around 36,000 infantry and anywhere from 3,000--6,000 cavalry.
With this strengthening of the enemy force we anticipate further attacks in the
days to come as the enemy continues to try to force us away from Fort Donelson.
However, the siege of Columbus by the Army of the Tennessee should limit the
time the enemy has before troops must be withdrawn to relieve Columbus.

I respectfully request an update on the status of Maj. Gen. Steele's force that
has been ordered to assist us. When is he expected to arrive and has he been
directed to move to threaten Fort Henry or to move directly to reinforce this
army?

\gramClosing{Yr. Obt. Svt.}
    {Walter Chekov}
    {Col., Adjutant General}
\reportdinkus

\gramHeader{Headquarters, Army of the Tennessee} % {{{4
    {Field Headquarters, Kentucky City, Ky., February}{17, 1862}
\gramTo{Maj. Gen.}{C. N. Van Royne}
    {Commander, United States Army}

\gramHi{General} I regret to inform you that, due to the strength and layout of
the enemy works, my assault of Fort Columbus on February 12th was repulsed with
the loss of five regiments and one artillery battery. With this loss, XVIIth
Corps maintains a siege of the works and prevents their resupply. XVIth Corps
has been redeployed to Clinton and will camp at the town to prevent any rebel
column from approaching unchecked. I accept all responsibility for the assault
failure and my command remains ready for future engagements with the rebels.

\gramClosing{Yr. Obdt. Srvt.}
    {Thomas Caldwell}
    {Brigadier General, Commanding Army of the Tennessee}
\reportdinkus

\gramHeader{United States Army} % {{{4
    {City of Washington, February}{17, 1862}
\gramTo{Maj. Gen.}{Christopher Stoeffel}
    {Commanding, Army of the Kentucky}

\gramHi{Sir} I wish to hear from you. Inform without delay of the developments
in your department and in your push toward Bowling Green.

\gramClosing{I remain vy rspy yours \&c}
    {C. N. Van Royne}
    {Major general, cmdg USA}
\reportdinkus

\gramHeader{United States Army} % {{{4
    {City of Washington, February}{17, 1862}
\gramTo{Maj. Gen.}{Steele}
    {Army of the Kanawha cmdg.}

\gramHi{General} I have heard that the balance of your command is now in the
vicinity of Cairo. Are you in sufficient shape to begin immediate operations up
the Tennessee? Gen'l Caldwell has invested Columbus but seemingly cannot take
the works by storming. Gen'l Blake has repulsed numerous attempts on his works
and remains northwest of Donelson but cannot take the city. I have not heard
from Gen'l Stoeffel on his efforts to take Bowling Green for about two weeks.

\gramClosing{I remain vy rspy \&c.}
    {C. N. Van Royne}
    {Maj. Gen. cmdg USA}
\reportdinkus

\gramHeader{Headquarters, Army of the Kanawha} % {{{4
    {Cairo, Ill., Feb.}{17, 1862}
\gramTo{Maj. Gen.}{Cornelius van Royne}
    {Commanding, United States Army}

\gramHi{General} I can now report my army of three infantry divisions and one
light division of cavalry have arrived in Cairo, where I will begin operations
to capture Fort Henry. I have with me 36 regiments of infantry, 6 regiments of
cavalry, and 11 batteries with which to conduct operations.

\gramClosing{Respectfully}
    {Richard Steele}
    {Maj. Gen., Commanding Army of the Kanawha}
\reportdinkus

\gramHeader{Headquarters, Army of the Arkansas} % {{{4
    {Field Headquarters, Black Bayou, Mo., Feb.}{17th, 1862}
\gramTo{Maj. Gen.}{Cornelius Van Royne}
    {Commanding General of the United States Army}
\gramTo{Brig. Gen.}{Thomas Caldwell}
    {Commanding, Army Of The Tennessee}

\gramHi{Generals} With the victory at Black Bayou and the withdrawal of the
rebels to New Madrid, I would intend to continue to put pressure on the enemy
while they remain in place instead of being able to maneuver.

For that reason I would request supplies from General Caldwell in order to
create a rail depot at Sikeston which could support the entire army.  New Madrid
lies roughly twenty two miles away from Sikeston and the defenses the enemy is
building are most likely closer than that.

It is my intention to begin constructing siege works at the limits of my supply
line and then to begin bombarding the enemy fortifications. Any attempt by the
enemy to move away from the city will weaken their position and either allow me
to pivot to attack the force leaving from the city or the if enough of the enemy
force has left, possibly carry the city itself.

The route by Sikeston is also composed of grassland as opposed to the swamps and
marshes of Bayouville which while advantageous to us on the defensive, will
offer the enemy the same benefits if we were to attack through the swamp from
that direction.

If additional supply is unavailable or the plan does not facilitate the greater
operational goals of this season, I will continue to hold East Prairie will
launching raids and scouting parties towards New Madrid and the various roads
leading to Charleston.

\gramClosing{I am always, Generals, and shall ever remain, Your most humble and obedient of servants}
    {Karl Meyer}
    {Maj. Gen. cmdg Army of the Arkansas}
\reportdinkus

\gramHeader{United States Army} % {{{4
    {City of Washington, February}{17, 1862}
\gramTo{Maj. Gen.}{Karl Meyer}
    {Commanding, Dep't of the West}

\gramHi{General} Yours of this date just received. I will order the requisite
supply released from Gen'l Caldwell's command if and only if he is confident he
can continue supplying his command in the investment of Columbus. Barring that,
your final proposition to hold East Prairie while conducting judicious cavalry
reconnaissance is, to my eye, good.

\gramClosing{I remain vy rspy yr obt svt \&c}
    {C. N. Van Royne}
    {Major general cmdg USA}
\reportdinkus

\gramHeader{Headquarters, Army of the Tennessee} % {{{4
    {Field Headquarters, Kentucky City, Ky., February}{17, 1862}
\gramTo{Maj. Gen'ls.}{C. N. Van Royne \& Karl Meyer}
    {Commanding United States Army \& Army of Arkansas, resp'y}

\gramHi{Generals} Major General Smith had ordered a depot built at Charleston
before this command moved east. I believe it is still in place and if the Army
of the Arkansas were to deconstruct it, not only would they receive the supplies
needed, it would also return the balance of those invested to my command for
further use. In summary, I approve of the Army of the Arkansas cannibalizing the
depot and moving the supplies where needed.

\gramClosing{Yr. Obdt. Srvt.}
    {Thomas Caldwell}
    {Brigadier General, Commanding Army of the Tennessee}
\reportdinkus

\gramHeader{United States Army} % {{{4
    {City of Washington, February}{17, 1862}
\gramTo{Brig. Gen'ls.}{Howard \& Gates}
    {Army of the Kentucky}

\gramHi{Gen'ls} I am informed Gen'l Stoeffel is temporarily indisposed by fever.
Please communicate at earliest possible moment all developments in your
department over the last two weeks, and any progress made in the advance on
Bowling Green.

\gramClosing{I remain vy rspy yours \&c}
    {C. N. Van Royne}
    {Major general cmdg USA}
\reportdinkus

\gramHeader{Headquarters, Army of Kentucky} % {{{4
    {Leesville, Ky., Feb.}{17th, 1862}
\gramTo{Maj. Gen.}{Cornelius van Royne}
    {General in Chief, U.S. Army}

\gramHi{Sir} I am happy to report that Maj. Gen. Stoeffel seems to be recovering
and should soon be in full command once more.

In the interim, we have advanced to Leesville and Upton, leaving a strong
garrison of roughly a division to secure our rear areas and the rail line along
which we advance. While the enemy has not deployed in the main against us, he
continues to oppose us with a force of roughly 4,000 cavalry.  It is not capable
of truly stopping us or of destroying the whole of the rail line, but has been
able to burn rail bridges along our advance, and then establishing a new line
further south.  Despite constant skirmishing and some larger efforts, the enemy
cavalry has proven enormously---one might say preternaturally---resilient
against the sustained pressures of continuous combat and campaign and by
extension capable of establishing as if new each week to cause some further
mischief.

This has necessitated  frequent repairs and the aforementioned garrison efforts.
Our engineers are still very green and new to such matters---even small bridges
often cost a week of effort to restore even smaller spans with such limited
expertise. I sincerely encourage the incorporation of additional rail
specialists  in future endeavors so as to negate or limit the efficacy of this
method by the enemy. I believe the president was considering a consolidated
military railway department, and encourage you to advocate strongly to and for
that position with him.

As it were, we expect to reach Munfordville by the end of the week, hopefully
having prevented the enemy from destroying the bacon station rail bridge.
Munfordville is the last point where such mischief might work for the enemy,
after which he will have little choice but to meet us in battle, accept a siege
in Bowling Green or concede the state.

\gramClosing{Yr Obt Svt}
    {J. Howard}
    {Brig. Gen.}
\reportdinkus

\gramHeader{United States Army} % {{{4
    {City of Washington, February}{17, 1862}
\gramTo{Brig. Gen.}{James Howard}
    {Army of the Kentucky}

\gramHi{Gen'l} Yrs this date just received. Gen'l Steele, by my direction, has
transported the bulk of the Army of the Kanawha to Cairo and is to begin
immediate operations up the Tennessee River in support of Gen'l Blake. Gen'l
Caldwell is unable to take Columbus by direct action and must resort to a siege
of that place.

I believe firmly that if you can continue your advance to Bowling Green,
prosecuting it as aggressively as your good judgment permits and applying
maximal pressure on Gen'l Jackson there, then our winter campaign in the West
can be considered nothing other than a capital success. If you can pin whatever
forces he has and present a direct threat to Bowling Green, and attack it if the
opportunity presents, then it will prevent a junction of any of Jackson's troops
in the area, and those forces under Clarke \& Whisper at Fort Donelson.

I will restate my communication with Gen'l Stoeffel of the 3d inst., viz. that
Gen'l Jackson's force in totality is at minimum 17 thousand men and very likely
does not exceed 35 thousand. You enjoy a significant numerical superiority and,
supply and cavalry concerns having been noted, I implore you to exercise it to
the absolute best of your ability.

Should the enemy seek to destroy the rail \& bridges over the Big Barren River,
you may find it fruitful to dispatch a corps or any other body of men along the
Merry Oaks-Martinsville Road as an alternate route.

\gramClosing{I remain vy rspy yours \&c}
    {C. N. Van Royne}
    {Major general cmdg USA}
\reportdinkus

\gramHeader{Headquarters, Army of Kentucky} % {{{4
    {Leesville, Ky., February}{17, 1862}
\gramTo{Maj. Gen.}{Cornelius van Royne}
    {General in Chief, USA}

\gramHi{Sir} We intend to advance against Bowling Green for exactly those
reasons. For greater clarity, it is the movement of trains and not armies that
is our greatest impediment. We possess the supply for perhaps a single great
twenty mile extension of the army once we detach from the rail heads, forty all
told if the railhead aligns just so. With each lost rail bridge we have no great
difficulty moving the army across by other means, but find ourselves stymied  by
the practical range at which wagons can deliver for a force of our size. With a
bridge causing a week delay, this slows our campaign substantially more than the
minor deviations of roads or local crossings which suffice to get our main body
across. I assure you that once the conditions exist for aggressive action we
will take them.

On behalf of MG Stoeffel:

\gramClosing{Yr Obt Svt}
    {J. Howard}
    {Brig. Gen.}
\reportdinkus

\gramHeader{Headquarters, Army of the Cumberland} % {{{4
    {Happy Hollow, Tenn., February}{17, 1862}
\gramTo{Maj. Gen.}{Richard Steele}
    {Commanding, Army of the Kanawha}

\gramHi{General} Gen'l Blake is pleased to hear of your arrival in Cairo and
wishes to coordinate the actions of this army with your own. Brig. Gen.
Lawrence's division of cavalry is currently posted north of Fort Henry to
prevent the enemy from again approaching our northern flank unnoticed.

Once your army begins to land near Fort Henry, Gen'l Lawrence could move his
troopers southeast to delay any counterattack by the forces arrayed against us
at Fort Donelson. We believe the enemy cavalry currently outnumbers our own but,
with the addition of your own division of cavalry, our forces should match those
of the enemy. Gen'l Blake is prepared to place Gen'l Lawrence's cavalry under
your temporary command to support your operations around Fort Henry if you are
amenable.

What day do you anticipate the arrival of your forces near Fort Henry? We
anticipate the enemy will attack our positions shortly but, in the event he does
not, we are prepared to launch a diversion to limit the amount of force the
enemy could quickly send to resist your movements.

\gramClosing{Your obedient servant}
    {Walter Chekov}
    {Col., Adjutant General}
\reportdinkus

\gramHeader{Headquarters, Army of the Kanawha} % {{{4
    {Cairo, Ill., Feb.}{17, 1862}
\gramTo{Maj. Gen.}{James Blake}
    {Commanding, Army of the Cumberland}

\gramHi{General} As I am newly arrived in this theater I am not yet aware of the
supply infrastructure. Is there enough supply to provide for both of our armies?
If there is I can begin operations immediately.

\gramClosing{Your obedient servant}
    {Richard Steele}
    {Major General, Army of the Kanawha}
\reportdinkus

\gramHeader{Headquarters, Army of the Cumberland} % {{{4
    {Happy Hollow, Tenn., February}{17, 1862}
\gramTo{Maj. Gen.}{Richard Steele}
    {Commanding, Army of the Kanawha}

\gramHi{General} You can supply your army from Cairo. There are no spots along
the Tennessee River but there are enough supplies [152 LP] to support your army.

\gramClosing{Your obedient servant}
    {J. W. Blake}
    {Maj. Gen., Commanding}
\reportdinkus

\gramHeader{Headquarters, Army of the Arkansas} % {{{4
    {Field Headquarters, Black Bayou, Mo., Feb.}{17th, 1862}
\gramTo{Maj. Gen.}{Cornelius Van Royne}
    {Commanding General of the United States Army}
\gramTo{Brig. Gen.}{Thomas Caldwell}
    {Commanding, Army of the Tennessee}

\gramHi{Generals} I beg pardon for the abrupt change in strategy, but General
Ambrose has convinced me that since the Army of the Trans-Mississippi will most
likely be spending time reinforcing their lines, and since we do not have the
ability to hold New Madrid until Columbus is taken or until next campaigning
season, it would be wiser to fall back to Charleston and begin constructing
fieldworks of our own for when the next inevitable rebel attack begins.

I send this telegram to you to inform you of my new intentions. If General
Caldwell so desires, I will return his supplies back to him for better use.

\gramClosing{I am always, Generals, and shall ever remain, Your most humble and obedient of servants}
    {Karl Meyer}
    {Maj. Gen. cmdg Army of the Arkansas}
\reportdinkus

\gramHeader{Headquarters, Army of the Kanawha} % {{{4
    {Cairo, Ill., Feb.}{17, 1862}
\gramTo{Maj. Gen.}{Cornelius Van Royne}
    {Commanding, United States Army}

\gramHi{General} I am launching my expedition against Fort Henry, and I formally
request the use of all available gunboats to support my attack.

\gramClosing{Ob'd yours}
{Richard Steele}
{Major General, Commanding Army of the Kanawha}
\reportdinkus

\gramHeader{United States Army} % {{{4
    {City of Washington, February}{17, 1862}
\gramTo{Maj. Gen.}{Steele}
    {Army of the Kanawha cmdg.}

\gramHi{General} Yrs of this date just rcvd. Gen'l Blake will be instructed to
release to you all gunboats which are not critical for the immediate short-term
survival of his command.

\gramClosing{I remain vy rspy \&c.}
{C. N. Van Royne}
{Maj. Gen. cmdg USA}
\reportdinkus

\gramHeader{United States Army} % {{{4
    {City of Washington, February}{17, 1862}
\gramTo{Maj. Gen.}{James Blake}
    {Commanding Dep't of the Cumberland}

\gramHi{Sir} You may be aware that Gen'l Steele has arrived at Cairo with four
of his divisions. He is instructed to move against Fort Henry immediately, and
you are ordered to release to him all gunboats in your possession which are not
required for the immediate survival of your command in the short term. He will
require them to reduce Henry as quickly as possible, at which point he can move
to your support.

\gramClosing{I remain vy rspy yours \&c}
    {C. N. Van Royne}
    {Major general cmdg USA}
\reportdinkus

\gramHeader{United States Army} % {{{4
    {City of Washington, February}{17, 1862}
\gramTo{Maj. Gen.}{Karl Meyer}
    {Commanding, Dep't of the West}

\gramHi{General} Yours this date just rcvd. A more defensive posture for the
immediate future is amenable to us, particularly as supply becomes cause for
concern. I am told recruitment efforts these past several months have also been
excellent, and you may count on an increased contingent of cavalry to aid you in
future operations.

\gramClosing{I remain vy rspy yr obt svt \&c}
    {C. N. Van Royne}
    {Major general cmdg USA}
\reportdinkus

\gramHeader{Headquarters, Army of the Cumberland} % {{{4
    {Happy Hollow, Tenn., February}{17, 1862}
\gramTo{Maj.}{Andrew Mackay}
    {Chief Quartermaster}

\gramHi{Sir} Until further notice, you are hereby directed to release to Maj.
Gen. Steele, commanding the Army of the Kanawha, supplies sufficient for his
operations up the Tennessee River.

\gramClosingBy{Maj. Gen. Blake}
{George Campbell}
{Capt., Aide de Camp}
\reportdinkus

\gramHeader{Headquarters, Army of the Cumberland} % {{{4
    {Happy Hollow, Tenn., February}{17, 1862}
\gramTo{Cdre.}{Daniel Lewis}
    {U.S.N., Commanding Mississippi Squadron}

\gramHi{Sir} The support of your squadron has been most appreciated everywhere
it has been present. With the arrival of Gen'l Steele at Cairo, I request that
a portion of your force, especially one or more ironclads, be detailed to
support his anticipated landing near Fort Henry.

\gramClosing{Respectfully}
    {J. W. Blake}
    {Maj. Gen., Commanding}
\subsecdinkus

\subsection*{February 21, 1862}{} % {{{3

\gramHeader{Headquarters, Army of the Kanawha} % {{{4
{Fort Henry, 8.45 pm, February}{21, 1862}
\gramTo{Maj. Gen.}{Cornelius Van Royne}
{Commanding General, United States Army}

\gramHi{General} It is my honor to present to you, and the United States, Fort
Henry, on behalf of the Army of the Kanawha. Our brave soldiers began the attack
at 8.45~am this morning and at 3.15~pm we carried the enemy works. We then
repulsed Confederate attempts to relieve the garrison.

It was a hard fought victory, and it would not have been possible without the
gallant contributions of the U.S. Navy and the cavalry division of Brig. Gen.
Graham, provided to our forces by Maj. Gen. Blake.

Casualty figures will be forwarded in the coming days once we have taken stock
of the situation.

\gramClosing{Respectfully}
{Richard Steele}
{Maj. Gen., Commanding, Army of the Kanawha}
\reportdinkus

\gramHeader{Headquarters, United States Army} % {{{4
{City of Washington, February}{21, 1862}
\gramTo{Maj. Gen.}{Steele}
{Army of the Kanawha cmdg.}

\gramHi{General} Yrs of this date just rcvd. This news is worthy of celebration
across the whole Union. I do not feel it inappropriate to tell you that the
President is very pleased with this announcement.

How ensconced was the enemy? Was the balance of the garrison captured and taken
into captivity? How strong was the attempted enemy relief force, and to whose
command did it belong, if such information was able to be gleaned from any
prisoners taken?

\gramClosing{I remain vy rspy \&c.}
{C. N. Van Royne}
{Maj. Gen'l. cmdg USA}
\subsecdinkus

\subsection*{February 23, 1862}{} % {{{3

\gramHeader{Headquarters, Army of the Kanawha} % {{{4
{Fort Henry, February}{23, 1862}
\gramTo{Maj. Gen.}{Cornelius Van Royne}
{Commanding General, United States Army}

\gramHi{General} I have learned some additional facts about the enemies I
engaged. Cleburne's Division is a part of the Army of Mississippi but Clark's
Division is from the Army of Tennessee.

\gramClosing{Respectfully}
{Richard Steele}
{Maj. Gen., Commanding, Army of the Kanawha}
\reportdinkus

\gramHeader{Headquarters, Army of the Cumberland} % {{{4
{Happy Hollow, Tenn., February}{23, 1862}
\gramTo{Cdre.}{Daniel Lewis}
{U.S.N., Commanding Mississippi Squadron}

\gramHi{Commodore} My compliments to your boat captains and crews for their
support to Gen'l Steele's capture of Fort Henry. The gen'l has informed me that
your losses were not insignificant but, as soon as you are able, I urge you to
conduct a patrol up the Tennessee river to determine the location of the enemy's
next line of defense.

\gramClosing{Respectfully}
{J. W. Blake}
{Maj. Gen., Commanding}
\reportdinkus

\gramHeader{Headquarters, Army of the Kanawha} % {{{4
{Fort Henry, Tenn., February}{23d, 1862}
\gramTo{Maj. Gen.}{Cornelius Van Royne}
{Commanding General, United States Army}

\gramHi{General} It would be most helpful to our cause if I were allowed to
bring the rest of my army to Tennessee, or at least a force similar in strength.
As soon as possible. Blake's army is heavily fatigued and my army is smaller
than the enemy's.

\gramClosing{Respectfully}
{Richard Steele}
{Maj. Gen., Commanding, Army of the Kanawha}
\reportdinkus

\gramHeader{United States Army} %{{{4
{City of Washington, February}{23, 1862}
\gramTo{Maj. Gen.}{Richard Steele}
{Army of the Kanawha cmdg.}

\gramHi{General} Yours this date just received. You may draw on whatever forces
remain under your authority within the Department of the Kanawha so long as the
garrisons of the forts established there remain.

General Blake will be instructed to assume temporary total authority over both
your army and the Army of the Cumberland to ensure effective leadership in the
Donelson operations in the coming weeks.

\gramClosing{I remain vy rspy \&c.}
{C. N. Van Royne}
{Maj. Gen'l. cmdg, USA}
\reportdinkus

\gramHeader{United States Army} %{{{4
{City of Washington, February}{23, 1862}
\gramTo{Maj. Gen.}{James Blake}
{Commanding Dep't of the Cumberland}

\gramHi{General} You may assume temporary operational authority over the Army of
the Kanawha by virtue of seniority for the duration of the operations against
Donelson, to ensure effective leadership and the proper chain of command.

Has any progress been made in your discussions with Gen'l Steele regarding the
path forward? With the enemy withdrawing back over Hickman's Creek, can an
effort be made to invest Donelson?

\gramClosing{I remain vy rspy yours \&c}
{C. N. Van Royne}
{Major general cmdg, USA}
\reportdinkus

\gramHeader{Headquarters, Army of the Kanawha} % {{{4
{Fort Henry, Tenn., February}{23d, 1862}
\gramTo{Maj. Gen.}{Cornelius Van Royne}
{Commanding General, United States Army}

\gramHi{General} Your reply this date is also received. I acknowledge your
reminder of seniority of rank in this theater.   Maj. Gen. Blake and I had
already established his seniority in rank and the subordination of the Army of
the Kanawha to his orders. We continue to coordinate our efforts to defeat the
enemy. If you wish to know more specifics about our plans I will defer to
General Blake to brief you.

\gramClosing{Respectfully}
{Richard Steele}
{Maj. Gen., Commanding, Army of the Kanawha}
\subsecdinkus

\subsection*{February 24, 1862}{} % {{{3

\gramHeader{Headquarters, Army of the Kentucky} % {{{4
{Bacon Creek Station, Ky., February}{24th, 1862}
\gramTo{Maj. Gen.}{Cornelius Van Royne}
{Commanding General of the United States Army}

\gramHi{Sir} Our efforts so far have been most successful. We have been able to
push the enemy cavalry screen back all the way to the Green River at
Munfordville, where we have encountered enemy infantry. The bridges over the
Green River are burnt, and the river is swollen, making fords unusable.

The enemy seems intent to hold the river; we have so far observed the following
enemy forces: A division of infantry from the army of Tennessee is defending the
river and digging in south of Munfordville.  A division of infantry and cavalry
(potentially in two divisions), from the Army of East Tennessee is defending the
crossing points north of Mammoth Cave.  A division of cavalry from the Army of
Tennessee is defending the ferry points at Brownsville.

We suspect another division of cavalry is defending the river to our east, near
Port Royal, perhaps with infantry support.  More infantry is suspected in
reserve.

These sightings make the situation quite clear: The enemy intends to make a
stand at the river, using the steep banks of the river in an attempt to funnel
us over few potential crossing points. His need to guard the river eastward as
well gives us an opportunity however: If we are able to cross in the west, we
are very quickly between them and Bowling Green. We therefore present the
following plan:

The army will, in two wings, attempt two crossings across the Green River: At
the ferry point near Mammoth Cave, and somewhere suitable directly west of
Munfordville. This is to keep their forces divided, and either force them to
defend both points, or abandon the eastern one and let us cross for free. The
main effort will be the western crossing. From there, we will have to see how
the situation develops.

\gramClosing{I am your most obedient servant}
{Christopher Stoeffel}
{Maj. Gen., Commanding, Army of the Kentucky}
\reportdinkus

\gramHeader{United States Army} % {{{4
{City of Washington, February}{24, 1862}
\gramTo{Maj. Gen.}{Christopher Stoeffel}
{Commanding, Army of the Kentucky}

\gramHi{Sir} Yours of this date just received. Have you also considered a
possible crossing at Cramner, between Woodsonville and Port Royal? Otherwise if
good crossings can be made at your proposed point then I see no reason to deny
or delay your idea. Gen'l Steele has successfully captured Fort Henry but I grow
increasingly worried that the operations against Donelson will soon have to be
suspended. Vitality of movement must be exercised by your command so long as it
does not lead to your destruction.

\gramClosing{I remain vy rspy yours \&c}
{C. N. Van Royne}
{Major general cmdg, USA}
\reportdinkus

\gramHeader{Headquarters, Army of the Cumberland} % {{{4
{Trinity Church, Tenn., February}{24, 1862}
\gramTo{Cdre.}{Daniel Lewis, U.S.N.}
{Commanding Mississippi Squadron}

\gramHi{Commodore} In addition to the requested patrol up the Tennessee River it
would be most helpful if one or two of your gunboats could be stationed near the
rail bridge at Stewart, Ky. so as to prevent the enemy from sending supply and
troop trains across it. This would only be necessary until the rail road can be
cut by infantry or cavalry forces.

\gramClosing{Respectfully}
{J. W. Blake}
{Maj. Gen., Commanding}
\reportdinkus

\gramHeader{U.S. Naval Flotilla, Mississippi Squadron} % {{{4
    {Aboard Flagship Resolute, Off Fort Donelson, Tenn. January}{15th, 1862}
\gramTo{Major General}{Blake}
    {Commanding, Western Theater of Operations}

\gramHi{General} I regret to inform you that with recent losses and damage from
the Columbus and Fort Henry operations, all further taskings are on hold.  I
simply cannot risk opening Cairo to bombardment by stripping more gunboats from
patrol duties.

\gramClosing{I remain, sir, Your obedient servant}
    {Daniel Davyson Lewis}
    {Commodore, Commanding Officer, Mississippi Squadron, United States Navy}
\reportdinkus

\gramHeader{Headquarters, Army of the Cumberland} % {{{4
{Trinity Church, Tenn., February}{24, 1862}
\gramTo{Maj. Gen.}{Cornelius Van Royne}
{Commanding United States Army}

\gramHi{Sir} In response to your last, I will continue to exercise overall
command of the Donelson campaign until informed otherwise.

The plan developed between this headquarters and Gen'l Steele is intended to
bring this campaign to a close in no more than two weeks. At that point the Army
of the Cumberland will be forced to withdraw and refit.

This army will continue to maintain positions west of Fort Donelson to fix the
enemy opposite in place. Regiments will rotate off the front line as much as
possible to provide a much needed period of rest.

The cavalry of the Army of the Kanawha intends to cross the Cumberland River and
raid Clarksville, Tenn. intending to destroy the rail road there and preventing
the enemy from bringing supplies and troops into Fort Donelson from that
direction and also preventing troops there from easily moving to defend Bowling
Green.

The three infantry divisions of the Army of the Kanawha will maneuver south with
the two-fold objective of cutting the railroad across the Tennessee River and
placing the bulk of their force south of Fort Donelson.

The overall intent is to make further defense of Fort Donelson untenable through
maneuver and the cutting of supply routes rather than conducting another frontal
assault.

If this plan is not successful within two weeks, the Army of the Kanawha will be
left as the only field army left here as the Army of the Cumberland will have to
withdraw.

In any event, this army will likely not be capable of conducting another
campaign before mid-March.

\gramClosing{Respectfully}
{J. W. Blake}
{Maj. Gen., Commanding}
\reportdinkus

\gramHeader{United States Army} % {{{4
{City of Washington, February}{24, 1862}
\gramTo{Maj. Gen.}{James Blake}
{Commanding Dep't of the Cumberland}

\gramTo{General} Yours of this date just received. Your plan as proposed meets
with my complete approval. Gen'l Steele must be notified that the Army of
Tennessee under Jackson has placed itself at the numerous crossings up the Green
River protecting Bowling Green, and that even if held by a token force these
crossings would be formidable and could inflict serious loss on an opposing
army, to wit Gen'l Stoeffel's. If Jackson finds Steele in his rear, he may
decide to leave token forces at those crossings, then fall on the Army of the
Kanawha with the balance of his force. So long as he understands this risk, I
approve without hesitancy or delay.

\gramClosing{I remain vy rspy yours \&c}
{C. N. Van Royne}
{Major general cmdg, USA}
\reportdinkus

\gramHeader{Headquarters, Army of the Arkansas} % {{{4
{Charleston, Mo., February}{18th, 1862}
\gramTo{Maj. Gen.}{Cornelius Van Royne}
{Commanding General of the United States Army}

\gramHi{General} I send you this report to update you on the situation in
Missouri. Namely that nothing of note has occurred. For our part, we are halfway
through completing the first of our fieldworks with intentions to continue
constructing a second layer.

The cavalry scouts at Sikeston and Belmont report no enemy forces north of New
Madrid. In fact, there is no sign of enemy cavalry before New Madrid either,
despite the Cavalry Division attempting to skirmish with the Rebels in the town.

The enemy continues to construct defenses in front of the town, and I fear that
when the next season comes, the cost to take New Madrid will be significant for
there will be plenty of fieldworks before it. It is my belief that regardless of
if an attack is to happen this season or next, we will require the aid of
another army to carry the town.

\gramClosing{I am always, Generals, and shall ever remain, Your most humble and obedient of servants}
{Karl Meyer}
{Maj. Gen. cmdg Army of the Arkansas}
\reportdinkus

\gramHeader{Headquarters, Army of the Cumberland} % {{{4
{Trinity Church, Tenn., February}{24, 1862}
\gramTo{Maj. Gen.}{Cornelius Van Royne}
{Commanding General, United States Army}

\gramHi{Sir} I do not wish to overstep the bounds of my authority but the
construction of defenses around New Madrid by the enemy is quite concerning. If
the enemy is able to construct a river fort there it will be yet another
impediment on the movement of of our armies down the Mississippi.

I do not wish to impugn upon the abilities of Gen'l Meyer who understands the
situation in Missouri better than I but my conscience would not rest easy
without voicing my concerns.

\gramClosing{Respectfully}
{J. W. Blake}
{Maj. Gen., Commanding}
\reportdinkus

\gramHeader{United States Army} % {{{4
{City of Washington, February}{24, 1862}
\gramTo{Maj. Gen.}{James Blake}
{Commanding Dep't of the Cumberland}

\gramHi{General} Yours of this date just received. The Army of the Arkansas has
not the means to press upon New Madrid before April. Any offensive action must
be delayed until probably around the mid-point of that month. All hopes for
immediate success west of the Appalachians currently lies outside Nashville and
Columbus.

\gramClosing{I remain vy rspy yours \&c}
{C. N. Van Royne}
{Major general cmdg, USA}

\subsecdinkus

\subsection*{February 25, 1862}{} % {{{3

\gramHeader{Headquarters, Army of the Kanawha} % {{{4
{Near Dover, Tenn., Past noon, February}{25, 1862}
\gramTo{Maj. Gen.}{J. W. Blake}
{Commanding, Army of the Cumberland}

\gramHi{General} Yours of both 11 \& 12 o'clock just received. Congratulations
on your well-deserved victory. The Army of the Cumberland has cemented its
legacy in the annals of our young nation.

The Army of the Kanawha will attempt to interdict the enemy withdrawal, but like
yourself I fear it is too late. I have ordered Graham's Cavalry to attempt to
reach Lower Long Creek School ahead of the enemy, and if they find the enemy has
already passed they may halt there and rest.

I have ordered my 1st Cavalry Division to pursue the enemy army and determine
its location. I will follow with my army and see if we can do anything more
beneficial. I have an extra week of rations so I can move a little beyond supply
range, but also don't want to overextend myself. I intend to follow the enemy as
far as Yellow Creek to the southeast.

My 2d Division will send a detachment to secure Stewart Crossing and defend it.

4th Infantry Division and 2d Cavalry Division will have already received their
orders to join my army here but it is probably not too late to rescind that
directive if necessary. Or I can add their numbers to my own to give us a little
more weight in this theater. If we are able to advance on Nashville along the
left bank of the river we can make the enemy occupation of Bowling Green even
more untenable than Dover was.

\gramClosing{Respectfully}
{Richard Steele}
{Maj. Gen., Commanding}
\reportdinkus

\gramHeader{Headquarters, Army of the Cumberland} % {{{4
{Trinity Church, Tenn., Afternoon, February}{25, 1862}
\gramTo{Maj. Gen.}{Richard Steele}
{Commanding, Army of the Kanawha}

\gramHi{General} I see no reason to rescind the order for the remainder of your
army to come west but that is a matter best left between you and Washington as I
do not fully know the situation in western Virginia.

Continue to pursue the enemy as you see fit; your cavalry may be able to strike
at stragglers at the rear of the column. I trust your judgement in this.

If you can establish good positions on the route to Dover that would be ideal as
my army could join you there as soon as it is done refitting.

\gramClosing{Respectfully}
{J. W. Blake}
{Maj. Gen., Commanding}
\reportdinkus

\gramHeader{Headquarters, Army of the Kanawha} % {{{4
{South of Dover, Tenn., February}{25, 1862}
\gramTo{Maj. Gen.}{J. W. Blake}
{Commanding, Army of the Cumberland}

\gramHi{General} Your message this date is received. We will continue our
operations in pursuit of the enemy as far as Yellow Creek and establish a
forward position there as far as supply allows. I believe we should be in range
of Dover.

Once my second cavalry division arrives it will double my available cavalry
contingent and will enable us to provide a better screen.

\gramClosing{Respectfully}
{Richard Steele}
{Maj. Gen., Commanding}
\reportdinkus

\gramHeader{Headquarters, Army of the Kanawha} % {{{4
{South of Dover, Tenn., February}{25, 1862}
\gramTo{Maj. Gen.}{Cornelius Van Royne}
{Commanding, United States Army}

\gramHi{General} I have sent orders for my 4th Infantry and 2d Cavalry
Divisions to join my army in Tennessee. I deemed it necessary for a possible
assault on Dover. With the fall of Donelson and Dover to General Blake that may
be moot for that particular purpose, but I believe they will still be put to
better use with my army here. I am confident the garrisons of the various forts
and batteries in western Virginia will suffice for now to secure that region.

I believe we have a strategic opportunity here. Once the Army of the Cumberland
has refitted itself for offensive operations, combined with the Army of the
Kanawha we can move on Nashville. Any such move on our part would surely make an
enemy defense of Bowling Green impossible and force them to withdraw.

Similarly, if my army is instead sent west, we could position ourselves in such
a way as to threaten the supply and communications of any enemy forces along the
upper Mississippi and rid western Kentucky and southeastern Missouri of any
enemy presence. If such a course is considered, I believe I would require
substantial additions of men and materials to pull off such a campaign on my
own.

We stand ready to do our part, such as is required of us. In the meantime, we
will continue to work in coordination with General Blake to maximize our
immediate gains.

\gramClosing{Respectfully}
{Richard Steele}
{Maj. Gen., Commanding}
\reportdinkus

\gramHeader{United States Army} % {{{4
{City of Washington, February}{25, 1862}
\gramTo{Maj. Gen.}{James Blake}
{Commanding Dep't of the Cumberland}

\gramHi{General} Yrs this date just received. Your note is cause for celebration
in the entire capital. I will shortly be in discussion with the president
regarding your glorious exploits, and will contact General Steele shortly.
Ensure that you have the requisite supply remaining to nurture your army back to
health while also supporting General Steele's movements against Nashville.

If an effort can be made against Nashville before the end of March, it must be
done.

\gramClosing{I remain vy rspy yours \&c}
{C. N. Van Royne}
{Major general cmdg, USA}
\reportdinkus

\gramHeader{United States Army} % {{{4
{City of Washington, February}{25, 1862}
\gramTo{Maj. Gen.}{Richard Steele}
{Army of the Kanawha cmdg}

\gramHi{General} Yours this date just received. The original object of taking
Donelson was to take Nashville; and the liberation of Nashville was twofold, for
not only would it remove from the enemy a great population center full of
commerce and industry, but also compel the abandonment of Bowling Green and
Jackson's army there. Your intuition to strike and remove both pieces off the
board is astute.

Do what you can to mop up the remainder of the enemy, but await General Blake's
army before continuing in a general movement to Nashville, unless you deem the
city weakly held. Clarke and Whisper will no doubt fight hard once again to
prevent its fall. We received intelligence reports last month that the enemy is
fortifying the city, so it will likely be another difficult prospect to take it.
I believe even a serious threat to the city will compel Jackson's retreat.
General Stoeffel's army can then, even with some delay, be brought south to
bring the full weight of arms of the United States against the city.

\gramClosing{I remain vy rspy yours \&c}
{C. N. Van Royne}
{Major general cmdg, USA}
\reportdinkus

\gramHeader{Headquarters, Army of the Kanawha} % {{{4
{South of Dover, Tenn., February}{25, 1862}
\gramTo{Maj. Gen.}{Cornelius Van Royne}
{Commanding, United States Army}

\gramHi{General} Yours this date received. Once I have my full contingent of
cavalry from Virginia I will be better equipped to probe towards Nashville with
my scouts.

I have enough supplies remaining to allow my current army to operate beyond
supply range for an additional week, or my cavalry alone to operate well beyond
supply range on their own. If we were to receive additional supplies we could
establish adequate wagon depots to be ready for a full combined push to
Nashville. If such supplies are delivered I will consult with General Blake on
the optimum placement for our forces to operate according to his general
campaign strategy.

Any additional troops that could be sent our way would also be greatly welcomed.

\gramClosing{Respectfully}
{Richard Steele}
{Maj. Gen., Commanding}
\reportdinkus

\gramHeader{United States Army} % {{{4
{City of Washington, February}{25, 1862}
\gramTo{Maj. Gen.}{Richard Steele}
{Army of the Kanawha cmdg}

\gramHi{General}

Yours this date just rcvd. I regret to inform you that no such reserve supply
exists. For the next five weeks, given the small amount of supply left to you,
it is probably best to wrap up the pursuit in the most efficient way possible,
and prepare for more serious offensive movements in April, when I will be able
to supply the two armies with all of the supply they could need for a serious
campaign in Middle Tennessee.

Furthermore, all available troops in the rear areas nearby have been given over
to the armies of the Tennessee and the Arkansas to make up, at least in part,
for some of the losses suffered earlier this season in the Missouri panhandle.

\gramClosing{I remain vy rspy yours \&c}
{C. N. Van Royne}
{Major general cmdg, USA}
\reportdinkus

\gramHeader{Headquarters, Army of the Kanawha} % {{{4
{South of Dover, Tenn., February}{25, 1862}
\gramTo{Maj. Gen.}{Cornelius Van Royne}
{Commanding, United States Army}

\gramHi{General} The strategic situation is understood. We shall prioritize
preserving our gains and the strength of the army for future operations.

\gramClosing{Respectfully}
{Richard Steele}
{Maj. Gen., Commanding}
\reportdinkus

\gramHeader{Smithland, Ky.} % {{{4
{February}{25, 1862}
\gramTo{Maj. Gen.}{Richard Steele}
{Commanding, Army of the Kanawha}

\gramHi{Sir} I humbly request you direct any requests for supplies from this
headquarters to the undersigned.

\gramClosing{Very respectfully}
{Andrew Mackay}
{Maj., Chief Quartermaster}

\subsecdinkus

\subsection*{March 3, 1862}{} % {{{3

\gramHeader{Headquarters, Army of the Kanawha} % {{{4
{Cumberland, March}{3, 1862}
\gramTo{Maj.}{Andrew Mackay}
{Chief Quartermaster, Army of the Cumberland}

\gramHi{Major} Now that we have had a chance to pause after the recent campaign,
my quartermaster reports that the Army of the Kanawha expended the following:

Supplies to construct a river landing at Fort Henry. \\
One week's rations for 1st Cavalry Division in anticipation of their raid of
Clarksville, Tenn. \\
One week's rations for the remainder of my army for expected operations around
Dover, Tenn.

With the rapid capitulation of Ft. Donelson and Dover these supplies proved
unnecessary but they had already been issued to the men.

\gramClosing{Respectfully}
{Richard Steele}
{Commanding, Army of the Kanawha}
\reportdinkus

\gramHeader{Headquarters, Army of the Kanawha} % {{{4
{Cumberland, March}{3, 1862}
\gramTo{Maj. Gen.}{James W. Blake}
{Commanding, Army of the Cumberland}

\gramHi{General} The rest of my army has arrived at Ft. Henry so I had thought
to station them at Stewart's Crossing while the rest of my army holds here at
Cumberland.

\gramClosing{Respectfully}
{Richard Steele}
{Commanding, Army of the Kanawha}
\reportdinkus

\gramHeader{Headquarters, Army of the Cumberland} % {{{4
{Camp Carter, Tenn., March}{3, 1862}
\gramTo{Maj. Gen.}{Richard Steele}
{Commanding, Army of the Kanawha}

\gramHi{General} I trust your judgement as to the employment of your troops. As
you seem it prudent, attempt to force the enemy to continue to withdraw
southward. The previously discussed raid of Clarksville is an option as is
movement of some portion of your force past the enemy via river boat.

You have the initiative and the enemy likely does not yet know my army has
halted at Dover. Be cautious as you are potentially outnumbered but calculated
aggression may convince the enemy you have a stronger force than you actually
possess.

Keep this headquarters informed of your maneuvers and any intelligence gained. I
wish you good fortune and will rejoin you as this army recovers.

\gramClosing{Respectfully}
{J. W. Blake}
{Maj. Gen., Commanding}
\reportdinkus

\gramHeader{Headquarters, Army of the Kanawha} % {{{4
{Cumberland, March}{3, 1862}
\gramTo{Maj. Gen.}{James W. Blake}
{Commanding, Army of the Cumberland}

\gramHi{General} I do not believe I have the supply necessary to do much good in
the direction of Nashville. According to my quartermaster, the size of my army
would require 150 LP to create a wagon depot every 20 miles, with Nashville
lying about 70 miles from my current position. Alternatively, if we issued extra
rations instead of building permanent depots, the cost would drop to 63 LP per
week of maneuver. I only have 50LP remaining so this seems unfeasible at this
time.

It might be worth dispatching spies or scouts to determine what defenses exist
at Nashville and its environs before any campaign is launched, so we might
better anticipate the effort required. Perhaps a crossing of the river and a
move to sever the rail link to Bowling Green could also be scouted and
considered.

\gramClosing{Respectfully}
{Richard Steele}
{Commanding, Army of the Kanawha}
\reportdinkus

\gramHeader{Headquarters, Army of the Cumberland} % {{{4
{Camp Carter, Tenn., March}{3, 1862}
\gramTo{Maj. Gen.}{Richard Steele}
{Commanding, Army of the Kanawha}

\gramHi{General} I have received your last. Continue to do what you can to press
the enemy when practical. I trust you to conduct operations as you see fit. I
will give orders to release any remaining supplies for your use.

\gramClosing{Respectfully}
{J. W. Blake}
{Maj. Gen., Commanding}
\reportdinkus

\gramHeader{Headquarters, Army of the Kanawha} % {{{4
{Cumberland, March}{3, 1862}
\gramTo{Maj. Gen.}{James Blake}
{Commanding, Army of the Cumberland}

\gramHi{General} We have successfully secured Stewart's Crossing.

The rest of my army pursued the enemy to the east, but were checked at Sailor's
Rest by a cavalry force that exceeded my own. My army now currently sits at
Cumberland, opposite the enemy cavalry.

4th Infantry Division and 2d Cavalry Division have arrived at Paducah and are
ready to join my army for operations. My chief surgeon has informed me that
approximately 22\% of my army failed to report for roll call at least once in
the last week due to sickness or exhaustion. He does not believe it is cause for
alarm as of yet, but he expects this number to rise if we continue operations.

\gramClosing{Respectfully}
{Richard Steele}
{Commanding, Army of the Kanawha}
\reportdinkus

\gramHeader{Headquarters, Army of the Cumberland} % {{{4
{Camp Carter, Tenn., March}{3, 1862}
\gramTo{Maj. Gen.}{Cornelius Van Royne}
{Commanding General, United States Army}

\gramHi{Sir} This army is currently encamped at Camp Carter outside Dover, Ky.
My medical director reports that nearly half the men have missed muster at least
one day this past week due to illness or exhaustion. He estimates a period of rest
of at least three and possibly four weeks is required before the army is back to
full strength. I will continue to coordinate with Gen'l Steele to establish
positions from which we may begin the campaign to seize Nashville but that
operation will likely not begin until next month unless an opportunity presents
itself that warrants breaking camp early with less than a full complement of
effectives.

\gramClosing{Respectfully}
{J. W. Blake}
{Maj. Gen., Commanding}
\reportdinkus

\gramHeader{Headquarters, Army of the Cumberland} % {{{4
{Camp Carter, March}{3, 1862}
\gramTo{Col.}{Lorenzo Thomas}
{Adjutant General, United States Army}

\gramHi{Colonel} I have the honor to submit the following officers to be
considered for promotion:

{
    \centering
    \begin{dispatch}{
        colspec        = {l},
        cell{1,7,9}{1} = {c},
    }

        \textit{To Colonel of Volunteers:} \\
        Lieut. Col. Silas Adams \\
        Lieut. Col. John Alexander \\
        Lieut. Col. Horace Howland \\
        Lieut. Col. Thomas Ransom \\
        Lieut. Col. Richard Rowett \\
        \textit{To Lieutenant Colonel of Volunteers:} \\
        Maj. Joab Stafford \\
        \textit{To Brevet Colonel of Volunteers:} \\
        Maj. Joab Stafford \\
    \end{dispatch}
    \par
}

These officers have served faithfully in command positions commensurate with the
rank of their requested promotions. I request that they be promoted with a date
of rank of March 1st, 1862.

\gramClosing{Your obedient servant}
{Walter Chekov}
{Col., Adjutant General}
\reportdinkus

\gramHeader{United States Army} % {{{4
{City of Washington, March}{3, 1862}
\gramTo{Maj. Gen.}{James Blake}
{Commanding Dep't of the Cumberland}

\gramHi{General} Yrs this date, regarding recommendations for promotion in the
United States Volunteers, just received. Your request for consideration for
promotion of the listed officers is approved in full.

Yours also regarding the sickness infiltrating your army just received. Take any
and all precautions that tend toward the preservation of your command. If the
supply issue worsens, do not hesitate to recall Gen'l Steel back toward the
Dover area. The enemy remains nearby and in great strength, and together
outnumber the Army of the Kanawha. I hope for the rejuvenation of both armies
soon in preparation for an energetic spring campaign against Nashville, if such
an operation cannot be executed swiftly and efficiently this month.

\gramClosing{I remain vy rspy yours \&c}
{C. N. Van Royne}
{Major general cmdg, USA}
\reportdinkus

\gramHeader{Headquarters, Army of the Arkansas} % {{{4
{Charleston, Mo., February}{24, 1862}
\gramTo{Maj. Gen.}{Cornelius Van Royne}
{Commanding General of the United States Army}

\gramHi{General} The army has continued to encamp in the town of Charleston,
with no major movements from the enemy. The cavalry scouts have reported no
movements on the roads and no contact in front of New Madrid; the Confederates
appear to have not moved at all towards them.

My engineers have reported that Buffington Bridge should be repaired by the end
of the month, only then can work continue on extending the rail line towards
Poplar Bluff. I am currently looking into whether the construction of the rail
can be redirected towards New Madrid so as to ease the supply situation during
the spring campaign season.

In regards to the defenses of Charleston, the first set of breastworks have been
completed with the second set in progress, set to finish this week. With no
possibility of moving my forces, a third set of breastworks will also be ordered
constructed.

The Chief Surgeon reports that 11\% of the army missed muster at least once this
week due to illness, exhaustion, etc.  He reports this is a quite sustainable
level for he and his subordinates to manage.

\gramClosing{I am always, Generals, and shall ever remain, Your most humble and obedient of servants}
{Karl Meyer}
{Maj. Gen. cmdg Army of the Arkansas}
\reportdinkus

\gramHeader{Headquarters, Army of the Tennessee} % {{{4
{Field Headquarters, Kentucky City, Ky., March}{3, 1862}
\gramTo{Maj. Gen.}{Cornelius Van Royne}
{Commanding General, United States Army}

\gramHi{General} The siege continues, with the rebels denying my demands they
surrender. The XVIth Corps at Clinton has reported cavalry scouting up from the
south, so I anticipate either an attempt to relieve the siege or at the very
least more aggressive rebel movements in the coming weeks. With the Tennessee
and Cumberland now open, I am hesitant to waste men in an assault on a fort we
have completely isolated, but if the need arises my men will carry the works.

\gramClosing{Yr. Obdt. Srvt}
{Thomas Caldwell}
{Brigadier General, Commanding Army of the Tennessee}
\reportdinkus

\gramHeader{Headquarters, Army of the Cumberland} % {{{4
{Camp Carter, Tenn., March}{3, 1862}
\gramTo{Maj. Gen.}{Cornelius Van Royne}
{Commanding United States Army}

\gramHi{Sir} My intent for the next few weeks was to have Gen'l Steele attempt
to maneuver around the enemy at Cumberland either by land or river. The overall
intent would be to force the enemy to withdraw towards Nashville making it
easier to cut the rail lines leading north of there to Bowling Green.

However, based on Gen'l Caldwell's latest report, would you consider redirecting
the Army of the Kanawha west in an attempt to cut off Columbus completely and
possibly trap any forces already moving in relief between that army and the Army
of the Tennessee?

It is a shift in strategy but may be an unexpected move, gaining us an
advantage.

\gramClosing{Respectfully}
{J. W. Blake}
{Maj. Gen., Commanding}
\reportdinkus

\gramHeader{United States Army} % {{{4
{City of Washington, March}{3, 1862}
\gramTo{Brig. Gen.}{Thomas Caldwell}
{Army of the Tennessee cmdg}

\gramHi{Sir} Yours this date concerning siege of Columbus just received. Have
you the means to supply not only your force, but the addition of another four or
five divisions from the Army of the Kanawha, until the end of the month? I am
weighing the possibility of sending General Steele and his army to assist you in
repulsing the seemingly impending enemy advance on your circumvallation.

\gramClosing{I remain vy rspy yours \&c}
{C. N. Van Royne}
{Major general cmdg USA}
\reportdinkus

\gramHeader{United States Army} % {{{4
{City of Washington, March}{3, 1862}
\gramTo{Maj. Gen.}{James W. Blake}
{Commanding, Dep't of the Cumberland}

\gramHi{General} Yours of this date just rcvd. I will ask General Caldwell if he
believes he can support, in whole or in part, General Steele's army if sent to
his relief around Columbus.

\gramClosing{I remain vy rspy yours \&c}
{C. N. Van Royne}
{Major general cmdg USA}
\reportdinkus

\gramHeader{Headquarters, Army of the Tennessee} % {{{4
{Field Headquarters, Kentucky City, Ky., March}{3, 1862}
\gramTo{Maj. Gen.}{Cornelius Van Royne}
{Commanding United States Army}

\gramHi{General} Depending on the size of the force, I would be able to support
more troops in the vicinity of Viola or Paducah. However, my troops around
Columbus and Clinton are supplied via wagons from Milburn and I would be unable
to support many more forces from there.

\gramClosing{Yr. Obdt. Srvt}
{Thomas Caldwell}
{Brigadier General, Commanding Army of the Tennessee}
\reportdinkus

\gramHeader{United States Army} % {{{4
{City of Washington, March}{3, 1862}
\gramTo{Brig. Gen.}{Thomas Caldwell}
{Army of the Tennessee cmdg}

\gramHi{Sir} Yrs this date just received. Have you any indication at all as to
the composition or strength of the enemy force approaching from the south? Have
any prisoners been taken and plied for information to determine the identity of
the commands to which they belong?

\gramClosing{I remain vy rspy yours \&c}
{C. N. Van Royne}
{Major general cmdg USA}
\reportdinkus

\gramHeader{Headquarters, Army of the Tennessee} % {{{4
{Field Headquarters, Kentucky City, Ky., March}{3, 1862}
\gramTo{Maj. Gen.}{Cornelius Van Royne}
{Commanding United States Army}

\gramHi{General} I believe you misunderstood my report. As of now my pickets
have only encountered cavalry scouting, with the horsemen not even attempting to
skirmish. While I believe this may indicate a future advance in this sector,
that information is purely based on the value of Columbus to the enemy. Will
send a message with utmost urgency if an enemy force is discovered in my area,
if only to open aggression on another front.

\gramClosing{Yr. Obdt. Srvt}
{Thomas Caldwell}
{Brigadier General, Commanding Army of the Tennessee}
\reportdinkus

\gramHeader{United States Army} % {{{4
{City of Washington, March}{3, 1862}
\gramTo{Brig. Gen.}{Thomas Caldwell}
{Army of the Tennessee cmdg}

\gramHi{Sir} Yours of this date just received and understood. Put your troops in
the circumvallations and at Clinton on highest alert. Having ceded to us Dover
and Fort Donelson, the enemy is probably eager to move to the relief of Columbus
with at least some of the force dispatched to fight General Blake at Donelson. I
will see to it that a larger proportion of the riverine fleet is dispatched to
aid you in subduing the city and its defenses.

\gramClosing{I remain vy rspy yours \&c}
{C. N. Van Royne}
{Major general cmdg USA}
\reportdinkus

\gramHeader{United States Army} % {{{4
{City of Washington, March}{3, 1862}
\gramTo{Maj. Gen.}{James W. Blake}
{Commanding, Dep't of the Cumberland}

\gramHi{General} General Caldwell is not confident that the enemy is moving on
him in strength, but he has nevertheless been ordered to prepare for that
possibility to the best of his ability. Continue to rest your command in and
around Dover and Donelson. General Steele may continue his movements as you have
previously laid out.

If they can be spared, please dispatch a number of boats, including whatever
ironclads can be spared, to assist Caldwell in reducing the defenses of
Columbus.

\gramClosing{I remain vy rspy yours \&c}
{C. N. Van Royne}
{Major general cmdg USA}
\reportdinkus

\gramHeader{Headquarters, Army of the Arkansas} % {{{4
{Charleston, Mo., March}{3, 1862}
\gramTo{Maj. Gen.}{Cornelius Van Royne}
{Commanding General of the United States Army}

\gramHi{General} In regards to the construction of the rail towards New Madrid
from Charleston, my engineers have assured me that it can be done, albeit at the
loss of some amount of supplies  due to the change over.

The goal will be to extend the rail line to East Prairie ten miles away, from
which a depot will be able to supply an army to besiege New Madrid. It is
uncertain when this construction will be completed, but regardless will greatly
help with the delivery of supplies once finished.

\gramClosing{I am always, Generals, and shall ever remain, Your most humble and obedient of servants}
{Karl Meyer}
{Maj. Gen. cmdg Army of the Arkansas}
\reportdinkus

\gramHeader{United States Army} % {{{4
{City of Washington, March}{3, 1862}
\gramTo{Maj. Gen.}{Karl Meyer}
{Commanding, Dep't of the West}

\gramHi{General} Yours of 3d inst. just received. Your extension of the
rail line seems to my eye good, insofar as it will ease the difficulties of
operations against New Madrid in the spring and summer.

\gramClosing{I remain vy rspy yours \&c}
{C. N. Van Royne}
{Major general cmdg USA}
\reportdinkus

\gramHeader{Headquarters, Army of the Cumberland} % {{{4
{Camp Carter, Tenn., March}{3, 1862}
\gramTo{Maj. Gen.}{Cornelius Van Royne}
{Commanding, United States Army}

\gramHi{Sir} I will coordinate with Gen'l Steele to continue pressing the enemy
towards Nashville.

Gen'l Caldwell's request will be relayed to the Navy but Cdre. Lewis' squadron
has suffered not insignificant losses in the previous actions against enemy
forts and more of the same will surely continue to deplete his force. I will
urge him to exercise caution.

\gramClosing{Respectfully}
{J. W. Blake}
{Maj. Gen., Commanding}
\reportdinkus

\gramHeader{Headquarters, Army of the Kanawha} % {{{4
{Cumberland, March}{3, 1862}
\gramTo{Maj. Gen.}{James W. Blake}
{Commanding, Army of the Cumberland}

\gramHi{General}

I have issued the following orders to my army. I am forwarding a copy to you and
to Washington to keep you properly informed of my movements.

{\footnotesize
    Special Field Orders, No. 5.

    I. It is the task of this army to attempt to maneuver around the enemy
    blocking force and put pressure on Nashville.

    II. 1st Cavalry Division will continue to screen the enemy at Sailor's Rest.

    II. On March 4th, 4th Infantry Division and 2d Cavalry Division, will
    immediately transfer to Army of the Kanawha by steamer from Paducah and be
    dropped off at New Portland. It will then march inland to Mill Spring and
    join the army on the march.

    III. Once disembarked, 2d Cavalry Division will precede the army to White
    Oak, and from there northeast to the Yellow Creek east of Maysville and
    screen against possible enemy movement from the north.

    IV. On March 5th, 1st, 2d, \& 3d Divisions will march to Mill Spring, then
    head to Charlotte via Trousdale, White Oak, \& Batson's.

    V. On the evening of March 5th, 1st Cavalry Division will follow the rest of
    the army, and protect its rear.

    VI. A Wagon Depot will be built at White Oak.

    VII. Once the army reaches Charlotte, the two lead divisions will occupy
    Jones' Cross Road.

    VIII. If the enemy cavalry reacts in any way, the priority of our 1st \& 2d
    Cavalry Divisions will be to protect our infantry on the march and our
    supply lines. You are authorized to alter my given directions based on these
    priorities.
}

\gramClosing{Respectfully}
{Richard Steele}
{Commanding, Army of the Kanawha}
\reportdinkus

\gramHeader{United States Army} % {{{4
{City of Washington, March}{3, 1862}
\gramTo{Brig. Gen.}{James Howard}
{Army of the Kentucky}

\gramHi{Gen'l} Yrs this date just received. News of your success is joyously
received. As always, continue to press your advantage to whatever end as you are
able, and keep this headquarters apprised of all developments. Should you feel
the need to rest your command for five or seven days to tend to sick \& wounded
then it is likely for the best.

I have the pleasure to inform you that Gen'l Blake, with the assistance of
General Steele, has successfully captured Fort Donelson after a most lengthy and
bloody contest.

\gramClosing{I remain vy rspy yours \&c}
{C. N. Van Royne}
{Major general cmdg USA}
\reportdinkus

\gramHeader{Headquarters, Army of the Kentucky} % {{{4
{Munfordville, Ky., March}{3, 1862}
\gramTo{Maj. Gen.}{Cornelius Van Royne}
{General in Chief, United States Army}

I am pleased to report Maj. Gen. Stoeffel has made a full recovery and is
presently resuming unimpeded command of the Army of the Kentucky. I will
continue to provide administrative assistance to him as necessary, but should be
available for assignment in future seasons should our forces expand in size.

\gramClosing{Yr Obt Svt}
{James Howard}
{Brig. Gen.}

\subsecdinkus

\subsection*{March 10, 1862}{} % {{{3

\gramHeader{Headquarters, Army of the Cumberland} % {{{4
{Camp Carter, Tenn., March}{10, 1862}
\gramTo{Maj. Gen.}{Richard Steele}
{Commanding, Army of the Kanawha}

\gramHi{General} Gen'l Blake requests an update on the disposition of your army
soonest. He also wishes to report this army's condition is much improved. A full
campaign is not yet advisable but limited movements in support of your
operations should be possible.

\gramClosingBehalf{Maj. Gen. Blake}
{Geo. Campbell}
{Capt., Aide de camp}
\reportdinkus

\gramHeader{Headquarters, Army of the Kanawha}
{Charlotte, Tenn., March}{10, 1862}
\gramTo{Maj. Gen.}{James W. Blake}
{Commanding, Army of the Cumberland}

\gramHi{General} The Army of Kanawha has successfully maneuvered around the
enemy cavalry and reached Charlotte. We are now strengthened by the arrival of
4th Infantry and 2d Cavalry. We have established a Wagon Depot at White Oak and
it is enough to supply us here but not quite enough to supply a force at Jones'
Cross Road.

Our scouting indicated that the enemy cavalry located at Sailors Rest was caught
unawares of our maneuvering. I have cavalry posted at Yellow Creek and if I move
a few divisions up to Cloverdale the enemy will be completely cut off without an
avenue for escape.

My army can maintain its position threatening Nashville and serve as an anvil to
your attack on the cavalry, or we can double back behind them and drive them
towards Dover.

\gramClosing{Respectfully}
{Richard Steele}
{Commanding, Army of the Kanawha}
\reportdinkus

\gramHeader{Headquarters, Army of the Cumberland} % {{{4
{Camp Carter, Tenn., March}{10, 1862}
\gramTo{Maj. Gen.}{Richard Steele}
{Commanding, Army of the Kanawha}

\gramHi{General} Your last is well received. In addition to trapping the enemy
cavalry I would like to seize Clarksville as a threat to the enemy position at
Bowling Green.

I believe it is best if you propose a plan for the next week. This army should
be able to support you in any operations within range of the depot at Dover. Do
you believe we can both trap the enemy cavalry and seize Clarksville? What are
your recommendations? I would recommend waiting for a major movement until the
weather clears although, if you can set the trap that would be advisable.

\gramClosing{Respectfully}
{J. W. Blake}
{Maj. Gen., Commanding}
\reportdinkus

\gramHeader{Headquarters, Army of the Arkansas} % {{{4
{Charleston, Mo., March}{10, 1862}
\gramTo{Maj. Gen.}{Cornelius Van Royne}
{Commanding General of the United States Army}

\gramHi{General} The rail line to East Prairie has been established. My highest
praise to the Chief Engineer and his men for such quick work. It is my intention
to begin establishing the rail depot today with the understanding that it will
be operational on the 11th this instant.

The creation of this new depot will finally bring New Madrid within supply range
of our forces. With your permission, I will begin advancing upon the town
with the full might of the Army of the Arkansas.

As established previously, there was river traffic seen near New Madrid the 17th
ult. I am unaware of if this heralded the departure of Rebel forces or their
arrival, but feel obligated to make an attempt on the town in either case, given
that the town's capture will cut off supply to Columbus which General Caldwell
has placed under siege.

With your approval, it is most likely that a battle between my army and that of
the Army of the Trans-Mississippi will occur within the week.

\gramClosing{I am always, Generals, and shall ever remain, Your most humble and obedient of servants}
{Karl Meyer}
{Maj. Gen. cmdg Army of the Arkansas}
\reportdinkus

\gramHeader{United States Army} % {{{4
{City of Washington, March}{10, 1862}
\gramTo{Maj. Gen.}{Karl Meyer}
{Commanding, Dep't of the West}

\gramHi{General} Yours of this date just received. I think it probable this
enemy river traffic is a detachment of a part of the garrison of New Madrid to
reinforce any Confederate force south of Columbus, or even an attempt at
directly reinforcing the garrison at the riverine wharf. I will inform General
Caldwell of this possibility.

Judicious reconnaissance on your part must emphasized. If you believe in your
judgment that an attempt can be made on New Madrid with reasonable hopes of
success, I will support you in the attempt. Keep close watch on your supply
trains and your stocks; you must make them last for the next two and a half
weeks.

\gramClosing{I remain vy rspy yr obt svt \&c}
{C. N. Van Royne}
{Major General cmdg USA}
\reportdinkus

\gramHeader{United States Army} % {{{4
{City of Washington, March}{10, 1862}
\gramTo{Brig. Gen.}{Thomas Caldwell}
{Army of the Tennessee, cmdg}

\gramHi{Sir} General Meyer informs me that there have been increased river
traffic at New Madrid. There is no evidence to support this supposition, but I
believe it is possible the enemy has detached a part of the garrison at New
Madrid to reinforce, or otherwise establish, some body or force south of
Columbus, to lift your siege of that place. It may also be a bold attempt by the
enemy to reinforce the garrison of the city directly by water.

I again emphasize the necessity of you making all possible preparations to
establish a strong net of reconnaissance, covering all possibly enemy routes of
advance to your works at Columbus, and to make any necessary preparations, as
you see fit, for the survival of your command. The enemy will likely go to every
extremity to prevent the fall of Columbus. General Blake and General Steele is
tying down significant enemy forces in Middle Tennessee, and they are being
encouraged to cause as much distress for the enemy as possible, to distract from
your siege.

\gramClosing{I remain vy rspy yours \&c}
{C. N. Van Royne}
{Major General cmdg USA}
\reportdinkus

\gramHeader{Headquarters, Army of the Arkansas} % {{{4
{Charleston, Mo., March}{10, 1862}
\gramTo{Maj. Gen.}{Cornelius Van Royne}
{Commanding General of the United States Army}

\gramHi{General} Yrs of this date just received. I am unable to properly
reconnoiter the enemy due to the fact that they have at least fifteen regiments
of cavalry whereas my army only has ten. Similarly to General Smith's first
attempt on the town, this has prevented any proper scouting to occur.

Without additional cavalry, all that we are aware of is that the enemy has not
moved from New Madrid towards Charleston, and that more aggressive skirmishing
by their cavalry has occurred over the past week.

\gramClosing{I am always, Generals, and shall ever remain, Your most humble and obedient of servants}
{Karl Meyer}
{Maj. Gen. cmdg Army of the Arkansas}
\reportdinkus

\gramHeader{United States Army} % {{{4
{City of Washington, March}{10, 1862}
\gramTo{Maj. Gen.}{Karl Meyer}
{Commanding, Dep't of the West}

\gramHi{General} Yrs this date just rcvd. If the enemy still maintains a
significant numerical advantage in cavalry, then I am not sure an advance on New
Madrid is at present wise. I nevertheless ultimately leave it to your judgment.

\gramClosing{I remain vy rspy yours \&c}
{C. N. Van Royne}
{Major General cmdg USA}
\reportdinkus

\gramHeader{Headquarters, Army of the Arkansas} % {{{4
{Charleston, Mo., March}{10, 1862}
\gramTo{Maj. Gen.}{Cornelius Van Royne}
{Commanding General of the United States Army}

\gramHi{General} Yrs of this date just received. I am unable to resolve the
cavalry disparity for the duration of this campaigning season.

My staff estimates that the Army of the Trans-Mississippi, which was originally
estimated at 35,000 strong during the start of the year is now 30,600 to 31,100
strong, 4,500 to 4,800 (15--16 regiments) of which is assumed to be cavalry, with
the remaining 26,600 being either infantry or cavalry. It is estimated that the
enemy has somewhere around 42 regiments of infantry and 12 batteries of
artillery or 43 regiments of infantry and 8 batteries of artillery.

The current strength of the Army of the Arkansas is 48 regiments of infantry, 10
regiments of cavalry, and 16 batteries of artillery. If you deem an advance on
New Madrid unwise without at least parity in terms of cavalry, then I would
humbly request significant reinforcements in regards to cavalry during the start
of the spring season.

\gramClosing{I am always, Generals, and shall ever remain, Your most humble and obedient of servants}
{Karl Meyer}
{Maj. Gen. cmdg Army of the Arkansas}
\reportdinkus

\gramHeader{United States Army} % {{{4
{City of Washington, March}{10, 1862}
\gramTo{Maj. Gen.}{Karl Meyer}
{Commanding, Dep't of the West}

\gramHi{General} Yours of this date just received. I reiterate that the
conditions with the cavalry, which caused General Smith's defeat in his initial
defeat in his advance on New Madrid, having remain unchanged, I view any such
advance on that place as significantly risky. If the circumstances on the ground
lead you to believe that an advance on New Madrid promises good odds of success,
then I encourage you to take them. As with all armies, I will do what I can to
reinforce and resupply you at the earliest possible opportunity.

\gramClosing{I remain vy rspy yr obt svt \&c}
{C. N. Van Royne}
{Major General, cmdg USA}
\reportdinkus

\gramHeader{Headquarters, Army of the Tennessee} % {{{4
{Field Headquarters, Kentucky City, Ky., March}{10, 1862}
\gramTo{Maj. Gen.}{C. N. Van Royne}
{Commanding General, United States Army}

\gramHi{General} With the possible developments I plan to conduct an assault of
Fort Columbus on the 17th. I request if possible ironclad support. The enemy
garrison has been isolated for a number of weeks now, and I believe taking the
fort viable at this time.

\gramClosing{Yr. Obdt. Srvt}
{Thomas Caldwell}
{Brig. Gen., Army of the Tennessee, Commanding}
\reportdinkus

\gramHeader{Headquarters, Army of the Cumberland} % {{{4
{Camp Carter, Tenn., March}{10, 1862}
\gramTo{Brig. Gen.}{Thomas Caldwell}
{Commanding, Army of the Tennessee}

\gramHi{General} Your request for gunboat support at Fort Columbus will be
relayed to Cdre. Lewis.

\gramClosing{Respectfully}
{J. W. Blake}
{Maj. Gen., Commanding}
\reportdinkus

\gramHeader{Headquarters, Army of the Cumberland} % {{{4
{Camp Carter, Tenn., March}{10, 1862}
\gramTo{Cdre.}{Daniel Lewis}
{Commanding, Mississippi Squadron}

\gramHi{Cdre} If your squadron is again ready for action, its support is
requested at Fort Columbus as General Caldwell intends another assault.

\gramClosing{Respectfully}
{J. W. Blake}
{Maj. Gen., Commanding}
\reportdinkus

\gramHeader{United States Army} % {{{4
{City of Washington, March}{10, 1862}
\gramTo{Brig. Gen.}{Thomas Caldwell}
{Army of the Tennessee cmdg}

\gramHi{Sir} Yrs this date just received. Have you an estimate on the enemy
strength within the walls of Columbus? Has the garrison been reinforced by boat
since you invested the place? If you choose to go forward with the attack, which
is risky though I believe prudent, I urge you concentrate all your efforts at
whichever point you deem weakest, but remind you that should you be repulsed and
forced to quit the siege, I will be unable to reinforce or support you for
several more weeks.

\gramClosing{I remain vy rspy yr obt svt \&c}
{C. N. Van Royne}
{Major General, cmdg USA}
\reportdinkus

\gramHeader{Headquarters, XIth Corps, Army of the Kentucky} % {{{4
{Munfordville, Tenn., March}{10, 1862}
\gramTo{Maj. Gen'ls}{James Blake \& Richard Steele}
{Commanding Armies of the Cumberland \& Kanawha resp'y}

\gramHi{Gentlemen} I write you now to inform you of the situation in Kentucky.
Maj. Gen.  Stoeffel is soon to be, if not already, heavily engaged with the
rebel Army of Tennessee. I believe with Maj. Gen. Stoeffel's wing of the army
already having routed one division, the enemy having suffered heavily during my
own recent reverse, and now the enemy marching so shortly after an intensive
effort on the 8th that our forces may have some advantage, but not enough to be
sure of a decisive victory. The fortunes of war being fickle, the matter is
still in doubt. The rain clouds brewing do us no favors either.

However, your fine victory at Donelson and Henry may seal the opportunity either
way; what enemy do you have in front of you? I ask not just from interest in
operations nearby, but because should you sever the connection of Nashville to
the north with sufficient speed, the entirety of the rebel Army of TN will be
potentially cut off even as it is locked in struggle with us. I am eager for
your thoughts on the matter.

\gramClosing{Yr Obt Svt}
{James Howard}
{Brig. Gen.}
\reportdinkus

\gramHeader{Headquarters, Army of the Kanawha} % {{{4
{Charlotte, Tenn., March}{10th, 1862}
\gramTo{Maj. Gen.}{James W. Blake}
{Commander, Army of the Cumberland}
\gramTo{Brig. Gen.}{James Howard}
{XIth Corps, Army of the Kentucky}

\gramHi{Generals} There are no enemy forces presently before us at Charlotte. As
of yet I know of no impedance between my army and Nashville, but I have not had
time to do a proper reconnaissance. We also do not know where the enemy army at
Dover retreated to. The logical place is Nashville but this is speculation.
Frankly I find it remarkable that the enemy has chosen to make a stand at
Bowling Green with the two forts having fallen to us.

Unfortunately, the supply situation has hindered our ability to exploit the
victories gained. The Army of the Cumberland is convalescing from the campaign
at Dover, while my own army is at the maximum extent of our supply range at
Charlotte.

With no ability to increase my own supply range I can neither reach Nashville or
Clarksville. We have a wagon depot at White Oak that has allowed us to advance
this far, but no further.

At last report, the enemy cavalry detachment at Sailor's Rest still remains
there. We have an opportunity to entrap it but it currently still has an escape
route through Clarksville. It would be possible to scrap our depot at White Oak
to gain an extra week of supply, and thus maneuver against the enemy cavalry and
Clarksville.

But it would require successfully reopening the supply line to Dover within two
weeks or we would be cut off ourselves. Against this lone cavalry force I
believe it is possible.

It is my opinion that the most effective way of disposing of this cavalry and
set us up to move on both Clarksville and Nashville is to continue maneuvering
behind the cavalry toward Clarksville and then the Army of the Cumberland
advance against the cavalry. We would be the anvil to their hammer. There is
always the risk that the weather might dramatically hinder this endeavor.

With this enemy dispatched we could seize Clarksville, then establish river
depots far more economically that would allow us to cut the railroad north of
Nashville to Bowling Green, and then envelop Nashville from both sides of the
river.

But I will repeat that I do not have the resources to extend my supply lines any
further.

\gramClosing{Respectfully}
{Richard Steele}
{Commanding, Army of the Kanawha}
\reportdinkus

\gramHeader{Headquarters, Army of the Kanawha} % {{{4
{Charlotte, Tenn., March}{10th, 1862}
\gramTo{Maj. Gen.}{James W. Blake}
{Commanding, Army of the Cumberland}

\gramHi{General} After considering the map, I believe that opening up Bowling
Green is the single most important strategic objective in the entire western
theater. Doing so would allow us to supply field armies by rail all the way down
to Memphis and move up behind the enemy army at Columbus, Ky. and position a
force to cut off New Madrid. 

It is the logjam that would allow five of our armies to move anywhere from
Nashville to the Mississippi River. Then once Nashville falls it opens up
everything down to Corinth and Chattanooga. And all of that requires opening the
railroad through Bowling Green so we can get our rolling stock into Tennessee. 

Therefore, I believe Nashville can and should wait.

That being said, my army is unable to march on either objective. It exceeds our
supply range and with the threatening storm it is too risky to cut away from our
supply base to make a sprint for the river. 

I intend to maintain my cavalry screen at Yellow Creek and move two divisions of
infantry to Cloverdale to cut the road. The other two divisions will remain at
Charlotte. This would block the two main roads from Dover to Nashville.

Without another supply base this is the most we can accomplish this week.

\gramClosing{Respectfully}
{Richard Steele}
{Commanding, Army of the Kanawha}
\reportdinkus

\gramHeader{Headquarters, Army of the Cumberland} % {{{4
{Camp Carter, Tenn., March}{10, 1862}
\gramTo{Maj. Gen.}{Richard Steele}
{Commanding, Army of the Kanawha}

\gramHi{General} I intend to wait out the coming rain but once it ends I will
send General Graham's cavalry and Van Cleve's division towards the enemy
cavalry. If the enemy stands to fight we will simply tighten the noose; if they
run, it will be into your own forces.

If practicable, extend your position toward Clarksville. 

\gramClosing{Respectfully}
{J. W. Blake}
{Maj. Gen., Commanding}
\subsecdinkus

\subsection*{March 14, 1862}{} % {{{3

\gramHeader{United States Army} % {{{4
{City of Washington, March}{14, 1862}
\gramTo{Brig. Gen.}{James Howard}
{Army of the Kentucky, Commanding}

\gramHi{General} Yours of the 11th received. I have recalled General Stoeffel on
a provisional basis and ordered that you assume command of the department until
such time as the future of its leadership can be definitively decided. Take the
time afforded to you to concentrate your force and rail up all the necessary
supply to sustain you, now that a portion of the Army is on the left bank of the
Green. No doubt many of the men also need rest. General Steele is posing a
serious threat to Nashville, and I hope in early April to have him aided by
General Blake in a general advance on that place. With the Army of the Kentucky
already in close proximity to Bowling Green, the enemy must either choose to
fight for one place, thereby guaranteeing a loss of the other; or to forfeit
both entirely to us.

If you can safely remove XIth Corps from Munfordville once the enemy threat to
the rail line is passed, then you should do so, though you probably have already
decided that.

When the opportunity is given me, I will spare no expense in providing you with
everything you could realistically require, not just in food and powder but in
men and beast also, to ensure your success in the spring.

\gramClosing{I remain vy rspy yours \&c}
{C. N. Van Royne}
{Major general cmdg, USA}
\subsecdinkus

\subsection*{March 17, 1862}{} % {{{3

\gramHeader{Headquarters, Army of the Tennessee} % {{{4
{Field Headquarters, Kentucky City, Ky., March}{17, 1862}
\gramTo{Maj. Gen.}{C. N. Van Royne}
{Commanding General, United States Army}

\gramHi{General} I am pleased to finally report Fort Columbus has fallen, with
the garrison of 2,400 surrendering upon threat of assault this morning. The
timing was fortunate, as a mixed force confirmed to be of the rebel Army of the
Trans-Mississippi has advanced north and occupied Clinton, which I pulled the
XVIth Corps from to support the assault.

The current estimate for this force is a reinforced infantry division supported
by a brigade of cavalry, which if accurate should not be a dire risk to my
forces. If anything, I would encourage the Army of Arkansas to probe towards New
Madrid in case the deployment of this force has weakened that area.

\gramClosing{Yr. Obdt. Srvt}
{Thomas Caldwell}
{Brig. Gen., Army of the Tennessee, Commanding}
\reportdinkus

\gramHeader{Headquarters, Army of the Cumberland} % {{{4
{Camp Carter, Tenn., March}{17, 1862}
\gramTo{Maj. Gen.}{Richard Steele}
{Commanding, Army of the Kanawha}

\gramHi{General} I regret that the rain this past week prevented Gen'ls Graham
and Wool from marching on the enemy. However, Surg. Kenney reports no more than
the expected number of sick so, if the weather holds, this army is prepared to
begin marching south.

What is the supply situation? Have you expended all that  Maj. Mackay has made
available?

\gramClosing{Respectfully}
{J. W. Blake}
{Maj. Gen., Commanding}
\reportdinkus

\gramHeader{Headquarters, Army of the Kanawha} % {{{4
{Charlotte, Tenn., March}{17, 1862}
\gramTo{Maj. Gen.}{James W. Blake}
{Commanding, Army of the Cumberland}

\gramHi{General} My 3d and 4th Divisions were able to reach Cloverdale before
the storms turned all of the roads into mud. The Cavalry reports no movement of
Confederates around them, though as the storm raged and they were largely
confined to their camps.

No evidence has been seen of any other Confederate activity. Last reports before
the storm hit indicate the enemy cavalry was still located at Sailor's Rest.

My army has cut all direct roads to Nashville so if the enemy wishes to withdraw
now it would have to be to Clarksville. My army is well enough to make an
attempt on Clarksville, but would be out of supply range.

We have a small amount of supplies remaining so if we wished to be a little
audacious we could liquidate our wagon supply base and use the extra supplies to
push north to the river. In conjunction with a maneuver of your own from Dover
we might be able to entrap and destroy the cavalry, then establish a river port
at or near Clarksville.

With full control of the river past Clarksville we could push down to Nashville
and place forces north of the city which would also cut off rail supply to
Bowling Green.

\gramClosing{Respectfully}
{Richard Steele}
{Maj. Gen., Commanding}
\reportdinkus

\gramHeader{Headquarters, Army of the Kanawha} % {{{4
{Charlotte, Tenn., March}{17, 1862}
\gramTo{Maj. Gen.}{James W. Blake}
{Commanding, Army of the Cumberland}

\gramHi{General} As a follow up, I have two divisions of cavalry at Yellow
Creek. Two divisions of infantry at Cloverdale, and two divisions of infantry at
Charlotte.

If you wish, I can move the infantry to Yellow Creek and Mt Vernon Furnace as
the cavalry attempts to move by the flank to Carbondale. Then if you move out
from Dover we could have them.

Be advised, there may not be anyone covering the road from Mill Spring to
Sailor's Rest, so perhaps you could post somebody on that road just to ensure
every avenue is covered.

\gramClosing{Respectfully}
{Richard Steele}
{Maj. Gen., Commanding}
\reportdinkus

\gramHeader{Headquarters, Army of the Cumberland} % {{{4
{Camp Carter, Tenn., March}{17, 1862}
\gramTo{Maj. Gen.}{Richard Steele}
{Commanding, Army of the Kanawha}

\gramHi{General} As long as the weather holds, I will send Graham and Van Cleve
towards Sailor's Rest as previously discussed. I will also relieve your garrison
of the rail bridge at Crossing and order your troops to rejoin your army.

Do you believe you can establish a river landing in the Clarksville area
(preferably upriver towards Nashville) or, even better, near Jones' Cross Road
while also extending your army past Charlotte towards Nashville? If so, I
propose bringing a significant portion of my army to Charlotte where it should
be screened from enemy observation.

These moves would put us in a good position to make a move on Nashville and the
railroads leading north to Bowling Green.

\gramClosing{Respectfully}
{J. W. Blake}
{Maj. Gen., Commanding}
\reportdinkus

\gramHeader{Headquarters, Army of the Kanawha} % {{{4
{Charlotte, Tenn., March}{17, 1862}
\gramTo{Maj. Gen.}{James W. Blake}
{Commanding, Army of the Cumberland}

\gramHi{General} Yours this date just received. Yes, I should be able to reach
the river in one or two days in any direction I choose so long as no enemy
materializes to hinder me.

I can march through Jones' Cross Road and place a river port opposite Williams,
just on the edge of 20 miles upstream from Clarksville. If we wished to be a
little more daring we could instead march to the crossing halfway between
Williams and Nashville, which would place every railroad in and out of Nashville
on both sides of the river within supply range.

With sufficient investment in a river landing it should supply both of our
armies and allow us to fully invest the city, north and south, and sever any
rail link to Bowling Green.

\gramClosing{Respectfully}
{Richard Steele}
{Maj. Gen., Commanding}
\reportdinkus

\gramHeader{Headquarters, Army of the Kanawha} % {{{4
{Charlotte, Tenn., March}{17, 1862}
\gramTo{Maj. Gen.}{James W. Blake}
{Commanding, Army of the Cumberland}

\gramHi{General} As another follow up to last, I am willing to be so bold. I
propose I march my army to this crossing, establishing a base of operations, and
send one cavalry division to immediately interdict the railroad north of the
city, and the other division of cavalry toward the city itself to reconnoiter
its defenses.

\gramClosing{Respectfully}
{Richard Steele}
{Maj. Gen., Commanding}
\reportdinkus

\gramHeader{Headquarters, Army of the Kentucky} % {{{4
{Munfordville, Tenn., March}{17, 1862}
\gramTo{Maj. Gen.}{Cornelius Van Royne}
{General in Chief, United States Army}

\gramHi{Sir} Your telegram and letter just received this morning. Several days
of extraordinarily severe storms likely disrupted the lines and letter alike.
Likewise I have just regained contact with and accounting of the western wing of
the army. I acknowledge my provisional command of the Army of Kentucky. I will
provide an assessment of requirements for the spring campaign in following
correspondence, but my first unfortunate duty must be to present a clear eyed
representation of the current state of the Army of Kentucky.

The western wing is not fit for operations and will not be for a period of some
weeks. Maj. Gen. Stoeffel attempted to intercept the enemy during the
aforementioned storms, leading to multiple days exposed to the conditions
without camps. There was no engagement, however 600 men were lost to exposure,
nearly two divisions worth of powder and supply were ruined, the west wing now
has maybe one man in three left in ranks at roll call, and I am informed a
cholera epidemic has infected nearly one man in six. The men have taken to
burning powder to try to start fires, with predictable consequences. Morale
reflects the circumstances.  The one blessing is that the enemy does not appear
to be near by as it is my opinion that a single fresh division might be
sufficient to bring this entire wing to ruin in it's current state. It must be
withdrawn north of the river and reconstituted.  To be very clear: while this
wing has 21 regiments of infantry, 11 of cavalry, and 20 artillery batteries on
the books (two were seized from the Confederates), there is no offensive
potential, whatsoever, in these forces.

The eastern wing is in better shape, with 22 regiments of infantry and 18
batteries, plus 6 regiments in garrison. It is still recovering in some degree,
but was largely spared from the weather by the towns of Munfordville and
Woodson. However the roads have been turned to a mire, and it is said that a
movement of three miles is more taxing than a days march.

\gramClosingBehalf{Maj. Gen. Christopher Stoeffel}
{James Howard}
{Brig. Gen.}
\reportdinkus

\gramHeader{Headquarters, Army of the Arkansas} % {{{4
{Charleston, Mo., March}{17, 1862}
\gramTo{Maj. Gen.}{Cornelius Van Royne}
{Commanding General of the United States Army}
\gramTo{Brig. Gen.}{Thomas Caldwell}
{Commanding General of the Army of the Tennessee}

\gramHi{General} Firstly I must offer my congratulations to General Caldwell for
his taking of Columbus. I would also like to confirm with the good General that
the division from the Army of the Trans-Mississippi numbered more than 10,000 as
repeated by the papers, or if the number was inflated by the journalists.

In either case I believe that the presence of a portion of General Thomson's
army across the river means that it is absolutely necessary that New Madrid is
attacked this week while the enemy's infantry forces a substantially weakened.
Regardless of the cavalry disparity present, an opportunity such as this will
not come again lightly.

I am currently dispatching orders for the Army of the Arkansas to embark on the
train to East Prairie before marching from the town to New Madrid.

\gramClosing{I am always, Generals, and shall ever remain, Your most humble and obedient of servants}
{Karl Meyer}
{Maj. Gen. cmdg, Army of the Arkansas}
\reportdinkus

\gramHeader{Headquarters, Army of the Cumberland} % {{{4
{Camp Carter, Tenn., March}{17, 1862}
\gramTo{Maj. Gen.}{Cornelius Van Royne}
{Commanding General, United States Army}

\gramHi{Sir} I must impress upon you the need to expand the number of boats
assigned to the Mississippi Squadron. Although I welcome any strengthening of my
army, with three rivers to patrol, I believe an expanded navy to be of paramount
importance.

Cdre. Lewis' squadron is stretched quite thin; due to the need for gunboat
support at Fort Columbus, no foray up the Cumberland has made since the fall of
Fort Donelson on the 25th ult.

This lack of reconnaissance and inability to contest enemy movement across and
along the rivers will hamper our actions as we press south.

\gramClosing{Respectfully}
{J. W. Blake}
{Maj. Gen., Commanding}
\reportdinkus

\gramHeader{United States Army} % {{{4
{City of Washington, March}{17, 1862}
\gramTo{Brig. Gen.}{Thomas Caldwell}
{Army of the Tennessee cmdg}

\gramHi{Sir} Yours of this date just received. The capture of Columbus and its
prisoners is another in a string of great victories won by our armies this
winter. You and your men are deserving of the highest praises of the country.

Provide now for the best possible defense \& safety of your army. If you must
forfeit Columbus to that end, then do so, but ensure that you spike the guns and
demolish the defensive works to the best of your ability. Parole any prisoners
you cannot ship quickly upriver to Cairo or up the Ohio; do your best to
guarantee they do not immediately pass into the enemy lines, thereby providing
the enemy strong and recent intelligence on your army.

I will be in contact with General Meyer to urge that he press his advantage on
New Madrid. Unfortunately there are no other reinforcements that can be provided
to you, either from our reserves or from the other nearby armies.

\gramClosing{I remain vy rspy yours \&c}
{C. N. Van Royne}
{Major general cmdg, USA}
\reportdinkus

\gramHeader{United States Army} % {{{4
{City of Washington, March}{17, 1862}
\gramTo{Maj. Gen.}{Karl Meyer}
{Commanding, Dep't of the West}

\gramHi{General} Yrs this date just received. Your analysis is astute and
probably correct. Press your numerical advantage against New Madrid to the best
of your ability. Screening your advance from enemy detection is likely not of
terrible importance; support your cavalry with infantry and artillery wherever
possible and the enemy will be forced to give ground to you. The hopes of 20
millions lie with you.

\gramClosing{I remain vy rspy yours \&c}
{C. N. Van Royne}
{Major general cmdg, USA}
\reportdinkus

\gramHeader{United States Army} % {{{4
{City of Washington, March}{17, 1862}
\gramTo{Maj. Gen.}{James Blake}
{Commanding Dep't of the Cumberland}

\gramHi{General} Yours of this date just received. I am in full agreement with
your assessment. You may rest assured that I have flexed every muscle possible
in increasing the power of the riverine squadrons by way of more vessels,
particularly of ironclads, to aid in the advance of our armies into the interior
of the South.

If you believe that any aid whatsoever can be rendered to General Caldwell and
his men near Columbus, even by detaching some small portion of your force, then
I pre-emptively approve of any such move, but understand you may not be in
position to provide any assistance owing to the exhaustion and casualties
incurred at Donelson.

I will order General Caldwell to release any non-essential gunboats, including
all serviceable ironclads, back for service on the Cumberland to support your
army.

\gramClosing{I remain vy rspy yours \&c}
{C. N. Van Royne}
{Major general cmdg, USA}
\reportdinkus

\gramHeader{United States Army} % {{{4
{City of Washington, March}{17, 1862}
\gramTo{Brig. Gen.}{Thomas Caldwell}
{Army of the Tennessee cmdg}

\gramHi{Sir} Permit addendum to last. Please release to General Blake all
ironclads, and all non-essential gunboats of other wooden types, for service on
the Cumberland River. I believe greater pressure applied on both New Madrid and
Nashville may force General Thomson to abandon his advance on you.

\gramClosing{I remain vy rspy yours \&c}
{C. N. Van Royne}
{Major general cmdg, USA}
\reportdinkus

\gramHeader{Headquarters, Army or the Cumberland} % {{{4
{Camp Carter, Tenn., March}{17, 1862}
\gramTo{Maj. Gen.}{Cornelius Van Royne}
{Commanding General, United States Army}

\gramHi{Sir} I have just learned that the rebels have constructed a significant
fort at Clarksville. Although I believe this fort will be easier to reduce that
Henry and Donelson, it will slow our advance towards Nashville.

\gramClosing{Respectfully}
{J. W. Blake}
{Maj. Gen., Commanding}

Send copy to Gen'l Steele.

\reportdinkus

\gramHeader{Headquarters, Army of the Arkansas} % {{{4
{Charleston, Mo., March}{17, 1862}
\gramTo{Maj. Gen.}{J. W. Blake}
{Commanding General of the Army of the Cumberland}

\gramHi{General} With the fall of Columbus, I wish to request Commodore Lewis to
probe down the river and ascertain if there are any additional river batteries
between himself and New Madrid. If not, I would like to make an additional
request that he assists in my attack on New Madrid, which will most likely begin
this week on the 21st of this instant.

\gramClosing{I am always, Generals, and shall ever remain, Your most humble and obedient of servants}
{Karl Meyer}
{Maj. Gen. cmdg Army of the Arkansas}
\reportdinkus

\gramHeader{Headquarters, Army or the Cumberland} % {{{4
{Camp Carter, Tenn., March}{17, 1862}
\gramTo{Maj. Gen.}{Karl Meyer}
{Commanding, Army of the Arkansas}

\gramHi{General} I will request Cdre. Lewis scout the Mississippi past Fort
Columbus. As for the request to support your movement on New Madrid, I will
request that the Cdre. do so as long as it both does not interfere with other
pressing demands and that there is no fort or battery at New Madrid capable of
engaging his squadron. We cannot continue to sustain the losses experienced by
the Mississippi Squadron in front of rebel guns.

\gramClosing{Respectfully}
{J. W. Blake}
{Maj. Gen., Commanding}
\reportdinkus

\gramHeader{Headquarters, Army or the Cumberland} % {{{4
{Camp Carter, Tenn., March}{17, 1862}
\gramTo{Cdre.}{Daniel Lewis}
{Commanding, Mississippi Squadron}

\gramHi{Commodore} The assistance of your squadron in the taking of Fort
Columbus is greatly appreciated. I have heard from Washington that your
successes have led to a request to bolster your numbers as quickly as possible.
However, until that happens, we must do our best to help you preserve the small
force that you do have and use it wisely.

Until further notice your priority will remain to maintain control of the rivers
to the limit of friendly advance. Your secondary priority will be to patrol the
rivers to determine the next point at which the enemy has been able to deny you
further movement. The Cumberland River will take precedence as it is critical to
the planned advance on Nashville. The Mississippi and Tennessee Rivers will be
patrolled at your convenience.

Any requests for additional support that conflict these priorities and do not
come from this headquarters is to be denied. Please forward any such requests to
this headquarters as you feel necessary.

\gramClosing{Respectfully}
{Walter Chekov}
{Col., Adjutant General}

Send copies to Gen'ls Meyer, Caldwell \& Steele.

\reportdinkus

\gramHeader{Headquarters, Army or the Cumberland} % {{{4
{Camp Carter, Tenn., March}{17, 1862}
\gramTo{Cdre.}{Daniel Lewis}
{Commanding, Mississippi Squadron}

\gramHi{Cdre} Gen'l Blake requests you reconnoiter the enemy works at
Clarksville as you can without risking your squadron. A patrol up the
Mississippi River to determine the extent of navigability to your forces is also
requested.

Finally, and only if you deem it practical, Gen'l Meyer has requested support
for his attack on New Madrid on the 21st prox. This should only be undertaken if
you determine you can do so without risk to your squadron. Do not feel obligated
to engage enemy works defending the river.

\gramClosing{Respectfully}
{Geo. Campbell}
{Capt., Aide de Camp}
\reportdinkus

\gramHeader{Headquarters, Army of the Kanawha} % {{{4
{Charlotte, Tenn., March}{17, 1862}
\gramTo{Maj. Gen.}{James W. Blake}
{Commanding, Army of the Cumberland}

\gramHi{General} Your report on the fortifications of Clarksville are not
entirely unexpected, though the news is disappointing.

If this is the case, I propose my army moves to Steel Spring. Together we will
deal with the enemy cavalry and construct a river depot, providing supply for
our two armies.

A pontoon bridge would allow us to fully invest the fort on both sides of the
river while remaining in full supply. With naval support I believe we will have
a much easier time eliminating this fort.

\gramClosing{Respectfully}
{Richard Steele}
{Maj. Gen., Commanding}
\reportdinkus

\gramHeader{Headquarters, Army or the Cumberland} % {{{4
{Camp Carter, Tenn., March}{17, 1862}
\gramTo{Maj. Gen.}{Richard Steele}
{Commanding, Army of the Kanawha}

\gramHi{General} Your plan is just received and agreed upon. I will relieve the
garrison at the Crossing rail bridge and send P. Smith's corps against Sailor's
Rest. The remainder of my army will follow behind and plan to cross your pontoon
and maneuver to surround Clarksville and cut the rail line.

If my army is long delayed and you find yourself able to cross the river first,
then do so. If practical, continue your planned maneuver towards Jones' Cross
Road as you are able, without risking your army.

\gramClosing{Respectfully}
{J. W. Blake}
{Maj. Gen., Commanding}
\reportdinkus

\gramHeader{Headquarters, Army of the Kentucky} % {{{4
{Munfordville, Ky., March}{17, 1862}
\gramTo{Maj. Gen.}{Cornelius Van Royne}
{General in Chief, U.S. Army}

\gramHi{Sir} As promised, follow on correspondence concerning your questions for
a spring campaign. Over the coming two weeks I hope to assemble the Army of the
Kentucky near Munfordville and reverse it's current issues with readiness and
health. Should the opportunity present I will also attempt to advance out past
Munfordville, but I am hesitant to do so until I can return the cavalry that
was concentrated in the west. Marching a small force, likely outnumbered by the
enemy,  without cavalry in front of it,  for the sake of claiming a few more
miles a week or two early seems a poor proposition. I realize that the President
and Congress desire bold action, but perhaps you might remind them that bold
action carries bold risks . There are times and theaters where repulsed audacity
might be forgiven and even lauded by the nation, but I suspect not in eastern
Kentucky so soon after the Knob. 

In a full accounting of the current force, the Army of Kentucky has 54 regiments
of infantry, 11 of cavalry, and 29 artillery batteries. In addition, we retain
materiel in the summation of 75 LP and eight pontoon bridges, with a detailed
reckoning sent to the war office.

The enemy strength is not entirely known with recent events, but if I were to
venture an estimate it would 44 regiments of infantry, 11 of cavalry, and 11
batteries of artillery. My confidence in these numbers is moderate at best, but
we must plan off what we can. 

Given we still maintain a great superiority in artillery, I believe the army can
continue without further reinforcements of guns.  I would ask that we are
provided at least a handful of additional cavalry regiments this spring; as
operations so far have shown an advance along rail and road benefits greatly, or
is similarly hindered, by even a small difference in cavalry strengths.

In terms of infantry,  while I do not believe it would be practical to be made
whole to the level of before the Knob, I would urge you to understand that the
necessity of securing the rear of overland advances can consume 8-12 regiments.
When balancing the forces in an offensive it would be wise to deduct these from
the effective field force you envision. If the spring sees this army in mostly a
fixing role, advancing towards Bowling Green while other forces close around
Nashville, reinforcement can be limited. If you desire a general engagement with
the enemy, I would suggest we need 6 more regiments.

I do believe that this campaign's expense has been driven up very greatly by the
challenges it faced with bridges and rivers. I imagine a smaller provision of
materiel than form the winter will suffice, but would wish to see the overall
intent before specifying a number.

\gramClosing{Yr Obt Svt}
{James Howard}
{Brig. Gen., Commanding (Provisional)}
\reportdinkus

\gramHeader{Headquarters, Army of the Cumberland} % {{{4
{Camp Carter, Tenn., March}{17, 1862}
\gramTo{Maj. Gen.}{Richard Steele}
{Commanding, Army of the Kanawha}

\gramHi{General} After further deliberation, it will be faster to build a
pontoon bridge at Dover instead of Steel Spring. Once the bridge is laid XIIth
and XIVth Corps will cross the river at Dover and march to Clarksville. VIIIth
Corps will still move in Sailor's Rest.

Once the enemy there is dealt with I will likely have VIIIth Corps cross the
river as well, leaving your army to defend the approach to Dover.

\gramClosing{Respectfully}
{J.  W. Blake}
{Maj. Gen., Commanding}
\reportdinkus

\gramHeader{Headquarters, Army of the Tennessee} % {{{4
{Field Headquarters, Kentucky City, Ky., March}{17, 1862}
\gramTo{Maj. Gen.}{Karl Meyer}
{Army of Arkansas}

\gramHi{General} While my scouting has not confirmed the exact size, I have been
able to confirm there is at least an element of cavalry with the force at
Clinton, Ky. Unless they were reinforced, the New Madrid area may have donated
cavalry to the force opposing me.

\gramClosing{Yr. Obdt. Srvt}
{Thomas Caldwell}
{Brig. Gen., Army of the Tennessee, Commanding}
\reportdinkus

\gramHeader{Headquarters, Army of the Arkansas} % {{{4
{Charleston, Mo., March}{17th, 1862}
\gramTo{Brig. Gen.}{Thomas Caldwell}
{Commanding General of the Army of the Tennessee}

\gramHi{General} Yrs this date just received. Your news is most warmly welcomed,
as any force taken away from New Madrid would make the taking of the town
infinitely easier. I understand that you have not been able to confirm the size
of the cavalry contingent, but were you able to ascertain how many  infantrymen
opposed you?

I understand you are currently in a tenuous position, but any actions you could
take to prevent the elements of Thomson's army on the east side of the river
from crossing again, including but not limited to a demonstration,  would be
greatly appreciated.

\gramClosing{I am always, Generals, and shall ever remain, Your most humble and obedient of servants}
{Karl Meyer}
{Maj. Gen. cmdg Army of the Arkansas}
\reportdinkus

\gramHeader{Headquarters, Army of the Tennessee} % {{{4
{Field Headquarters, Kentucky City, Ky., March}{17, 1862}
\gramTo{Maj. Gen.}{Karl Meyer}
{Army of Arkansas}

\gramHi{General} I regret that I can provide no more detail than what my cavalry
reported as approximately a ``large enemy division supported by a brigade of
cavalry''. Due to their running battle they were unable to gather more
intelligence than that, including whether more rebel troops are behind those
encountered.

\gramClosing{Yr. Obdt. Srvt}
{Thomas Caldwell}
{Brig. Gen., Army of the Tennessee, Commanding}
\reportdinkus

\gramOrdersHeader{Headquarters, Army or the Cumberland} % {{{4
{Camp Carter, Tenn., March}{17, 1862}
{Special Field Orders}{16}

This army, in cooperation with Gen'l Steele's Army of the Kanawha will move to
invest the rebel fortifications at Clarksville, Ky. We will drive off and
ideally destroy the enemy cavalry at Sailor's Rest while moving to cross the
Cumberland River so as to surround Clarksville and cut the rail line leading
northeast to Bowling Green.

I. Gen'l Wool is to send the brigade [1/1/VIII] garrisoning Fort Henry to
Crossing, Tenn. and relieve the forces there to rejoin Gen'l Steele. One
regiment will temporarily remain behind to destroy the works at Fort Henry and
see that the remaining guns are moved to Camp Carter.

II. P. Smith's corps [VIII] will move towards the enemy cavalry reported near
Sailor's Rest. This movement is intended to trap the enemy force between our
forces and Gen'l Steele's.

III. The remainder of the army is to cross the Cumberland River at Dover and
march towards Clarksville. Graham's Division of Cavalry will scout ahead of the
army to identify the enemy positions and, if possible, cut the rail line leading
to Bowling Green.

V. Once the enemy lines are identified, the army will be positioned to cut off
Clarksville, being mindful of the possibility of a relief force arriving from
the direction of Bowling Green.

VI. P. Smith's corps [VIII] will move to rejoin the remainder of the army after
defeating the enemy force at Sailor's Rest.

\gramClosingBy{By order of Maj. Gen. Blake}
{Geo. Campbell}
{Capt., Aide de Camp}
\subsecdinkus

\subsection*{March 24, 1862}{} % {{{3

\gramHeader{Headquarters, Army of the Cumberland} % {{{4
{Clarksville, Tenn., March}{24, 1862}
\gramTo{Maj. Gen.}{Cornelius Van Royne}
{Commanding General, United States Army}

\gramHi{Sir} This army, in cooperation with Gen'l Steele's has seized
Clarksville, Tenn. and invested the nearby enemy fort.

On the 18th instant, XIIth and XIVth Corps, screened by this army's cavalry,
crossed the Cumberland River and marched on Clarksville, where an enemy depot,
burned to the ground, was found. These forces now occupy positions around the
town.

At the same time, VIIIth Corps advanced on the cavalry division of the Army of
Mississippi, positioned at Sailor's Rest, just east of Dover. The enemy was able
to slip through Gen'l Steele's positions although it is believed the enemy
suffered some loss to horses in order to escape the trap.

VIIIth Corps, along with the Army of the Kanawha, has now completely cut off the
enemy fort, located on the neck of land across the river from Clarksville. The
garrison is estimated at 12,000--20,000 men of Whisper's Army of Mississippi.
Our combined movements have trapped the garrison and leave it able to be
supplied only via riverboat from Nashville.

Gen'l Steele and I will discuss our options but I do not believe an assault on
the enemy works is at all advisable. Rather, I intend to keep the fort isolated
while maneuvering to render the fort useless and force the garrison to
surrender.

\gramClosing{Respectfully}
{J. W. Blake}
{Maj. Gen., Commanding}

Send copy to Gen'l Steele.

\reportdinkus

\gramHeader{United States Army} % {{{4
{City of Washington, March}{24, 1862}
\gramTo{Maj. Gen.}{Karl Meyer}
{Commanding, Dep't of the West}

\gramHi{General} Yours this date happily and just received. You may count on
further reinforcement in the spring, to help you better prosecute a campaign
down the Mississippi and in securing Government authority over Missouri. Island
No.~10 will therefore be your next objective, probably in conjunction with the
Army of the Tennessee. Take now all the time you require to rest your men and
heal them from the hospitals, but also prepare for the possibility of enemy
trickery, raiding of your lines of supply and communication, or even outright
counterattacks upon your position.

\gramClosing{I remain vy rspy yr obt svt \&c}
{C. N. Van Royne}
{Major General cmdg USA}
\reportdinkus

\gramHeader{United States Army} % {{{4
{City of Washington, March}{24, 1862}
\gramTo{Maj. Gen.}{James W. Blake}
{Commanding Dep't of the Cumberland}

\gramHi{General} Yours of this date just rcvd. Good tidings for the Nation. I
agree that an assault is inadvisable, but the capture of the entire army under
Whisper (has he replaced Clarke after his wounding?) would be a major victory
that cannot easily be passed up on. In the next several weeks I will forward you
and General Steele significant reinforcements. You must use these to strengthen
your positions, and prepare for the inevitable attempts by the enemy to lift the
siege and save Whisper and his army, with your armies split by the Cumberland
and isolated. Take all necessary precautions. Jackson, having won at Prewitt's
Knob, may likely turn against Clarksville from Bowling Green. Any movement to
strengthen your position, done prudently and skillfully, will post-hoc be
approved by me. Preserving the Memphis BR may be prudent fur near-future
operations, even though Bowling Green remains in enemy hands.

\gramClosing{I remain vy rspy yr obt svt \&c}
{C. N. Van Royne}
{Major General cmdg USA}
\reportdinkus

\gramHeader{Headquarters, Army of the Tennessee} % {{{4
{Field Headquarters, Clinton, Ky., March}{24, 1862}
\gramTo{Maj. Gen'ls}{C. N. Van Royne \& Karl Meyer}
{Commanding General, United States Army \& Commander, Army of Arkansas, resp'y}

\gramHi{Generals} The assumed rebel relief column has withdrawn southward. My
infantry is currently camped at Clinton and my cavalry has taken the town of
Hickman on the Mississippi, destroying an enemy battery and capturing an enemy
depot. Reports from my medical officers that 22\% of my force is on the sick
list. Plan to rest my force until that number lessens and weather permits
advance south.

\gramClosing{Yr. Obdt. Srvt.}
{Thomas Caldwell}
{Brig. Gen., Army of the Tennessee, Commanding}
\reportdinkus

\gramHeader{United States Army} % {{{4
{City of Washington, March}{24, 1862}
\gramTo{Maj. Gen.}{James W. Blake}
{Commanding Dep't of the Cumberland}

\gramHi{General} Gen. Caldwell reports that the enemy facing him at Clinton has
withdrawn southward. The strength of that force cannot be known; last Gen'l
Caldwell reported, it was a small detachment of cavalry, but nevertheless could
have been supporting some larger force. Gen. Meyer's capture of New Madrid means
that Thomson's army is also in retreat. It is distinctly possible either or both
force is moving to create a large army of relief for Whisper at Clarksville.
Double your guard and prepare for possible imminent attack by the enemy in the
next one or two weeks. I will do all within my power to support and reinforce
you.

\gramClosing{I remain vy rspy yr obt svt \&c}
{C. N. Van Royne}
{Major General cmdg USA}

Written in duplicate to Maj. Gen. Richard Steele.

\reportdinkus

\gramHeader{Headquarters, Army of the Arkansas} % {{{4
{Charleston, Mo., March}{24, 1862}
\gramTo{Brig. Gen.}{Thomas Caldwell}
{Commanding General of the Army of the Tennessee}

\gramHi{General} Yrs this date just received. I can inform you that the rebel
relief column that withdrew southwards has shown up this morning across the
river from New Madrid. Most likely they attempted to hard march the last few
days to arrive at New Madrid before my forces could engage them piecemeal. I am
unaware of where the cavalry may have gone as none was spotted.

I understand your current position and do not expect any movement from you this
coming week due to exhaustion of your army, indeed the Army of the Arkansas had
23\% of the soldiers missing muster.

But when the Army of the Tennessee has recovered, I would request that you
maneuver your army towards the town of Obionville, west of Hickman and interdict
the two roads leading from it towards the east. Once my own forces have extended
a rail line to New Madrid, I plan on establishing a force at Riddle's Point,
along with a battery of guns.

This will cut off river access to Island No.~10 as well as the ferry point
allowing them to move across the Mississippi. With control over the water lane
and all three roads, the Island will be placed under siege without supply, at
which point it would be possible for you to advance and attempt to carry the
works if practicable.

\gramClosing{I am always, Generals, and shall ever remain, Your most humble and obedient of servants}
{Karl Meyer}
{Maj. Gen. cmdg Army of the Arkansas}
\reportdinkus

\gramHeader{Headquarters, Army of the Cumberland} % {{{4
{Clarksville, Tenn., March}{24, 1862}
\gramTo{Maj. Gen.}{Cornelius Van Royne}
{Commanding General, United States Army}

\gramHi{Sir} With the possibility that enemy forces may attempt to relieve Fort
Clarksville, this army and Gen'l Steele's intends to spend the next week
strengthening our positions.

This army's XIIth and XIVth Corps will defend the approaches from Bowling Green
while VIIIth Corps isolates the fort. Significant artillery positions will be
established to limit the amount of supply that can reach the fort. The Army of
the Kanawha will defend against any approach from Nashville or from the south.

The armies to the east and west should be ready to exploit any opportunity
created by the withdrawal of enemy troops towards Clarksville.

As the Mississippi and Cumberland Rivers are again blocked by enemy forts, I
will request that Cdre. Lewis patrol the Tennessee River as far as possible.

\gramClosing{Respectfully}
{J. W. Blake}
{Maj. Gen., Commanding}
\subsecdinkus

\subsection*{March 25, 1862}{} % {{{3

\gramHeader{Office of the Secretary of War} % {{{4
{Washington City, March}{25, 1862}

\textit{All Commanding Major Generals of the Armies of the United States,
and the Chiefs of Staff, Heads of Departments, Inspector Generals,
and Senior Officers of Subsidiary Commands presently in the field:}\nopagebreak
\par\vspace{5pt}

\MakeUppercase{Strictly Confidential---Not to be Copied---Destroy After Reading}
\par\vspace{5pt}

\gramHi{Gentlemen} By direction of the President, and under my own authority as
Secretary of War, you are hereby ordered to report to Washington City for a
three-day closed council, to convene at the War Department at 0800 hours on the
1st of April, and to adjourn at no earlier than 10~p.m. on the 3d ult.

This meeting shall be held in absolute secrecy, the nature of which requires
your personal attendance without delay, without publicity, and without the
accompaniment of any unauthorized personnel.

Each officer will bring only his Chief of Staff and one assistant of his
choosing. No adjutants or political attachés are permitted.  Failure to comply
will be interpreted as dereliction of duty under wartime conditions.

In view of the military operations conducted since the commencement of winter,
and the recent allegation brought before the Joint Committee on the Conduct of
the War, the President and I have found it necessary to review, in full and
unsparing detail, the conduct, disposition, planning, and execution of all major
campaigns.

Further, this council shall determine the coordinated Spring Campaign Plan of
the Union armies.

All officers shall travel incognito where possible, using secure transport
arranged through the Chief Quartermaster.  Communications sent en route will
be sealed and routed through my office directly.

Any leakage of the nature, attendees, or proceedings of this council will be
prosecuted under military authority.

The President expects your punctual attendance, frank testimony, and unreserved
cooperation.  I expect your readiness to answer for your commands.

You are officers of the United States Army. Conduct yourselves accordingly.

\nopagebreak
\hspace{2em}By order of the Secretary of War, Edwin M. Stanton.\nopagebreak

\reportdinkus

\gramHeader{Headquarters, Army of the Kanawha} % {{{4
{Charlotte, Tenn., March}{25, 1862}
\gramTo{Maj. Gen.}{Cornelius Van Royne}
{Commanding General, United States Army}

\gramHi{Sir} through winter sickness, desertion, and other such attritional
losses, we have lost the equivalent of four regiments of infantry, one regiment
of cavalry, and one battery of artillery.

This leaves my current strength at 39 regiments of infantry, 11 regiments of
cavalry and 11 batteries of artillery.  I have a paper strength of 27,800 men
with 66 guns. The last report I received from our surgeon before departing for
Washington is about 22\% of my army unfit for duty, allowing me to field 21,684.

The only enemy contact we have is General Whisper's Army of Mississippi holed up
in Fort Defiance, and another detachment of cavalry to our southeast
in the direction of Nashville, supposedly comparable in strength to my own
cavalry.

My army currently is organized in four divisions of infantry and two light
divisions of cavalry.

\gramClosing{Respectfully}
{Richard Steele}
{Maj. Gen., Commanding}

\reportdinkus

\gramHeader{Headquarters, Army of the Kentucky} % {{{4
{Munfordville, Ky., March}{25, 1862}
\gramTo{Maj. Gen.}{Cornelius Van Royne}
{Commanding General, United States Army}

After accounting for seasonal attrition, this army has a strength of
approximately 35,000 men, arrayed as 49 regiments of infantry, 10 regiments of
cavalry, and 26 batteries of artillery.

Six of those infantry regiments are currently garrisoning the rail lines,
leaving a field force of 43 regiments of infantry, 10 of cavalry and 26
batteries of artillery.

At this point enemy information is two weeks out of date, however in our theater
we observed:

The Army of Tennessee, under Gen'l Jackson is believed to have 24,000 or more
men in the area arrayed as 32 regiments of infantry, 10 regiments of cavalry,
and 7 batteries of artillery. This force was last seen mid March near Prewitt's
Knob

The Army of East Tennessee, under Gen'l Whisper is believed to have at least a
division in the area with some cavalry. The strength of this force is unknown,
but is estimated at 12 regiments of infantry, 2 of cavalry and 2 batteries of
artillery. This force was last seen near Prewitt's Knob in mid March.

Both armies may have sent detachments to Fort Donelson at some point, the
relation of which remains unclear.

A short comparative analysis of losses in this army's winter campaign:

\vspace{5pt}
\begin{dispatch}[
]{
    colspec = {X[l]r},
}

Battle losses prior to Prewitt's Knob \\
\hspace{2em}Infantry\dotfill & 4,200 \\
\hspace{2em}Cavalry\dotfill & 0 \\
\hspace{2em}Artillery\dotfill & 4 \\

Battle losses at Prewitt's Knob \\
\hspace{2em}Infantry\dotfill & 9,000 \\
\hspace{2em}Cavalry\dotfill & 0 \\
\hspace{2em}Artillery\dotfill & 5 \\

Non-battle losses \\
\hspace{2em}Infantry\dotfill & 4,200 \\
\hspace{2em}Cavalry\dotfill & 1 \\
\hspace{2em}Artillery\dotfill & 3 \\

\textbf{Total losses}\dotfill \\
\hspace{2em}Infantry\dotfill & 17,400 \\
\hspace{2em}Cavalry\dotfill & 300 \\
\hspace{2em}Artillery\dotfill & (includes 2 captured batteries) 10 \\

\cmidrule{1-2}\SetRow{abovesep+=4pt}
Enemy battle losses \\
\hspace{2em}Infantry\dotfill & 4,800--5,400 \\
\hspace{2em}Cavalry\dotfill & 900 \\
\hspace{2em}Artillery\dotfill & 8--9 \\

Enemy non-battle losses (estimated) \\
\hspace{2em}Infantry\dotfill & 2,400 \\
\hspace{2em}Cavalry\dotfill & 300 \\
\hspace{2em}Artillery\dotfill & 1--2 \\

\cmidrule{1-2}\SetRow{abovesep+=4pt}
\textbf{Total enemy losses}\dotfill \\
\hspace{2em}Infantry\dotfill & 7,200--7,800 \\
\hspace{2em}Cavalry\dotfill & 1,200 \\
\hspace{2em}Artillery\dotfill & (includes 5 captured batteries) 4--6 \\
\end{dispatch}

\vspace{5pt}
\gramClosing{Yr Obt Svt}
{James Howard}
{Brig. Gen., Commanding (Provisional)}

\reportdinkus

\gramHeader{Headquarters, Army of the Cumberland} % {{{4
{Clarksville, Tenn., March}{25, 1862}
\gramTo{Maj. Gen.}{Cornelius Van Royne}
{Commanding General, United States Army}

\gramHi{Sir} It is my honor to submit the following report of the state of this
army at the end of the winter campaign.

This army was constituted at Evansville, Ind. in late December of last year with
a strength of 43,200 infantry, 3,600 cavalry and 174 pieces of artillery,
organized into 72 regiments of infantry, 12 of cavalry and 29 batteries of
artillery. After the losses detailed below, the current strength is 34,200
infantry (57 regiments), 2,700 cavalry (9 regiments) and 144 guns (24
batteries).  As of the 17th ult.  18\% of the army had missed muster during the
prior week. The army medical director has yet to provide a more recent report.

\vspace{5pt}
\begin{dispatch}[
]{
    colspec = {X[l]r},
}

Battle losses \\
\hspace{2em}Infantry\dotfill & 5,400 \\
\hspace{2em}Cavalry\dotfill & 600 \\
\hspace{2em}Artillery\dotfill & 18 \\

Other losses \\
\hspace{2em}Infantry\dotfill & 3,600 \\
\hspace{2em}Cavalry\dotfill & 300 \\
\hspace{2em}Artillery\dotfill & 18 \\

Total losses\dotfill \\
\hspace{2em}Infantry\dotfill & 9,000 \\
\hspace{2em}Cavalry\dotfill & 900 \\
\hspace{2em}Artillery\dotfill & 36 \\
\end{dispatch}

\vspace{5pt}

The Army of Mississippi, commanded by Gen'l Whisper is defending Fort Defiance
across the Cumberland River from Clarksville, Tenn. This force is reported as
between 12 and 20 thousand troops. That army's division of cavalry, reported to
be a match for Gen'l Steele's two divisions of cavalry, is to the southwest of
Fort Defiance, facing Gen'l Steele.

Ptolemy Smith's VIIIth Corps, less one brigade garrisoning the rail bridge
across the Tennessee River at Crossing, Tenn., is in position south of Fort
Defiance, preventing the enemy from moving in or out of his defensive works.

Charles Smith's XIIth Corps and McClernand's XIVth Corps along with Grahams'
Cavalry are stationed in and around Clarksville to prevent a relief of the fort
from the north as well as to present a threat to the enemy army around Bowling
Green.

\gramClosing{Very respectfully}
{Walter Chekov}
{Col., Adjutant General}

\reportdinkus

\gramHeader{Headquarters, Army of the Arkansas} % {{{4
{New Madrid, Mo., March}{25, 1862}
\gramTo{Maj. Gen.}{Cornelius Van Royne}
{Commanding General, United States Army}

\gramHi{General} With the conclusion of the winter campaign, it is my privilege
to present to you the following report on the state of the Army of the Arkansas.

The army was formed at St. Louis, Mo. just before the start of the campaign
season with a strength of 32,400 infantry, 3,600 troopers, and 84 pieces of
artillery, organized into 54 regiments of infantry, 12 regiments of cavalry,
and 14 batteries of artillery.

At the end of this campaign season, the army's current strength stands at 22,800
infantry (38 regiments), 2,700 troopers (9 regiments) and 84 pieces of artillery
(14 batteries). An additional 1,200 troopers (4 squadrons) are currently
attached to General Caldwell's Army of the Tennessee.  As of the 24th ult. 23\%
of the army had missed muster due to injuries, illness, or furlough.

In the three battles that this army has fought, the Army of the Arkansas has
lost 9,000 infantry (15 regiments), 600 troopers (2 regiments), and 12 guns (2
batteries). The last three months have also attrited my forces by 2,400 infantry
(4 regiments), 300 troopers (1 regiment), and another twelve pieces of
artillery. In total, the end of the winter campaign as seen an overall casualty
count of 11,400 infantry (19 regiments), 600 troopers (2 regiments), and 24 guns
(4 batteries) for a total of 12,400 casualties.

The Army of the Trans-Mississippi, commanded by General Thompson, has withdrawn
an unknown distance south of New Madrid, and the division that he sent to relief
the siege of Columbus was also detected on the east bank of the Mississippi
river across from New Madrid on this very morning.

Flag Officer Foote's ships also detected extensive fortifications at Island
Number~10, which in his words meant that, ``a naval advance beyond this point is
not practicable at present'' and that ``until a cooperating land force can be
brought to bear or the fort silenced by siege, the Squadron will confine its
efforts to securing and patrolling the river between Columbus and the upper
approaches to the island.''

\gramClosing{I am always, Generals, and shall ever remain, Your most humble and obedient of servants}
{Karl Meyer}
{Maj. Gen. cmdg Army of the Arkansas}
\reportdinkus

\gramHeader{Headquarters, Army of the Tennessee} % {{{4
\gramTo{Maj. Gen.}{Cornelius Van Royne}
{Commanding General, United States Army}

\gramHi{General} With the end of March my two corps, XVIth and XVIIth, currently
numbering approximately 16,200 with 10 batteries, along with approximately 2,700
cavalry, including those left over loaned from the Army of Arkansas, find
themselves with a firm position on the east bank of the Mississippi near 
Clinton. The cavalry division has advanced further and has occupied the town of
Hickman. Finally, there is a detached brigade of infantry with two batteries on
garrison duty at Fort Columbus.

The enemy relief column, which scouting determined to be a ``large enemy
division supported by a brigade of cavalry'' from the enemy Army of the
Trans-Mississippi, withdrew south beyond the range of even my cavalry scouts.
There are currently no other enemy forces on the immediate east bank of the
river, nor as far inland as the Paducah-Mayfield rail line.

The losses of the army during the season were heavy. Approximately 10,200
infantry (2,400 of which was lost in the opening engagement at New Madrid and
3,600 in the initial assault on Fort Columbus), 1,500 cavalry, and 3 batteries
of artillery were lost in order to liberate western Kentucky from the rebels.

In return for this, the army cleared the rail line south of Paducah to Mayfield,
captured the fort at Columbus intact and forced the enemy to destroy a depot
there, and captured an enemy depot at Hickman. Enemy battle losses from New
Madrid were not determined in the withdrawal, but 2,400 enemy infantry were
captured with Fort Columbus along with one battery being overrun by the cavalry
at Hickman.

With the spring coming, I believe the Army of the Tennessee is in an excellent
position to further threaten the east bank of the Mississippi provided it can be
reinforced and reorganized. I eagerly await discussion on the strategy for the
next season, and orders for my area of operations.

\gramClosing{Yr. Obdt. Srvt.}
{Thomas Caldwell}
{Brig. Gen., Army of the Tennessee, Commanding}
\subsecdinkus

\subsection*{March 30, 1862}{} % {{{3

\gramHeader{Office of the Chief Signal Officer} % {{{4
{St. Louis, Mo., March}{30, 1862}
\gramTo{Maj. Gen'ls}{James W. Blake \& Karl Meyer}
{Commanding, Departments of the Cumberland and West, resp'y}

\gramHi{Generals} In accordance with your request, I have the honor to submit a
seasonal weather outlook for the Western theaters, covering the months of April
through June, 1862, prepared in the practical, almanac-like manner long favored
by our farming population. Present indications remain steady and allow a
reasonable estimation of the season now approaching.

\textbf{April}---Mild, Open, and Favorable

April is expected to be mostly dry, with no extraordinary rains anticipated
between the Ohio and upper Mississippi. Temperatures should rise into the 70s,
with nights cooling to the 40s and low 50s. Roads will harden early, making the
month highly suitable for the movement of infantry, cavalry, and trains. River
levels should remain moderate and reliable.

\textbf{May}---Warm Breezes and Steady Skies

May continues this pattern of dry and temperate weather. Daytime highs will
reach the 70s and 80s, and evenings fall between the high 40s and mid-60s.
Ordinary showers may occur, but no severe downpours are foreseen. Prairie forage
will be plentiful. Campaigns may be pressed without hindrance from swollen
streams or sodden ground.

\textbf{June}---Heat Rising, Fair Weather Holding

June brings true summer. Temperatures will commonly fall in the 80s and low 90s,
with nights in the 60s to near 70. The outlook is again for mostly dry
conditions, though commanders should keep close account of water for men and
animals, especially on extended marches. Weather itself will pose little
obstacle to active operations.

\textbf{General Outlook for the West}

Taken together, the coming months promise a highly favorable campaigning season.
Dry roads, dependable skies, and manageable river conditions will support the
plans of both Departments. The hardships of the past winter are not expected to
recur in any degree of severity.

I trust this forecast will aid your planning for the operations now under
consideration. Should further observations be required, I remain at your
service.

\gramClosing{Respectfully submitted}
{Bartholomew Fairweather}
{Capt., Chief Signal Officer, Department of the West}

\secdinkus

\section[Correspondence, Etc.---April 1--June 30, 1862] % {{{2
{Correspondence, Orders, and Returns Relating to Operations in
    Missouri, Kentucky and Tennessee from April 1, 1862, to June 30, 1862}{}

\subsection*{April 1, 1862}{} % {{{3

\gramHeader{United States Army} % {{{4
    {City of Washington, April}{1, 1862}
\gramTo{Maj. Gen.}{James Blake}
{Commanding Department of the Cumberland}

\gramHi{Gen'l} Thank you for the disposition of your Department. We have taken
note and will proceed with the current numbers.

Please, as soon as possible, send us an accounting of required supply for the
season's campaign. Needs for depots, maneuvers, blockhouses, construction, et
cetera. The Commanding Gen'l requires time to review, return with questions and
give his approved budget.

\gramClosing{Yr obt svt}
{M. R. Turgon}
{Maj.}
\reportdinkus

\gramHeader{United States Army} % {{{4
    {Washington City, April}{1, 1862}
\gramTo{Maj. Gen.}{James Blake}
{Commanding Department of the Cumberland}

\gramHi{General} Spies in Nashville report an increasing concentration of
manpower and materiel just southwest of Nashville alongside large camps.
Increased frequency of locomotives to and from Bowling Green are reported. As
expected, we must assume the enemy is preparing for an imminent counterattack to
relieve Whisper's army.

\gramClosing{I remain very respectfully}
{C. N. Van Royne}
{Maj. Gen.,  commanding USA}
\reportdinkus

\subsection*{April 8, 1862}{} % {{{3

\gramHeader{Headquarters, Department of the Cumberland} % {{{4
{Clarksville, Tenn., April}{8, 1862}
\gramTo{Maj. Gen.}{Richard Steele}
{Commanding, Army of the Kanawha}

\gramHi{General} I have received reports of large concentrations of troops and
supplies assembling to the southwest of Nashville and of a number of trains
arriving there from Bowling Green. This is consistent with reports received at
the end of March.

Continue as planned but it is imperative that you establish forward positions
and especially your cavalry screen as soon as possible. Gen'l Fawcett remains
ready to support as needed.

\gramClosing{Respectfully}
{James W. Blake}
{Maj. Gen., Commanding}

Copy to Fawcett

\reportdinkus

\gramHeader{Headquarters, Department of the Cumberland} % {{{4
{Clarksville, Tenn., April}{8, 1862}
\gramTo{Maj.}{Andrew Mackay}
{Chief Quartermaster}

\gramHi{Major} Gen'l Blake wishes a report on the arrival of reinforcements from
Cairo as they will of necessity be moving through your post at Smithland. Have
such begun arriving and how long until the first troops reach their
destinations?

\gramClosing{Your servant}
{Walter Chekov}
{Col., Adjutant General}
\subsecdinkus

\subsection*{April 10, 1862}{} % {{{3

\gramHeader{Headquarters, Department of the Cumberland} % {{{4
{Clarksville, Tenn., April}{10, 1862}
\gramTo{Maj. Gen.}{Richard Steele}
{Commanding, Army of the Kanawha}

\gramHi{General} Gen'l Fawcett's cavalry was probed yesterday around the town of
Fredonia by a division of cavalry from the Army of Tennessee. The enemy may be
probing both sides of the river to determine our strength.

If you cannot reach your intended line along Jones' Creek, is it possible to
form a line along Louise and Yellow Creeks extending south towards Maysville? I
am concerned about your army being driven back against Fort Defiance and unable
to maneuver.

\gramClosing{Respectfully}
{James W. Blake}
{Maj. Gen., Commanding}
\reportdinkus

\gramHeader{Headquarters, Department of the Cumberland} % {{{4
{Clarksville, Tenn., April}{10, 1862}
\gramTo{Maj. Gen.}{James Howard}
{Commanding, Army of the Kentucky}

\gramHi{General} Enemy cavalry is present in strength on both banks of the
Cumberland River; on the 9th inst. possibly two divisions of the Army of
Tennessee probed Gen'l Fawcett around Fredonia and Gen'l Steele's cavalry
engaged another two divisions of the Army of the West south of McAlister's, Ky.

With reports of trains arriving in Nashville from Bowling Green, I expect you
may find the enemy force to your front to have thinned somewhat if not
considerably. If that is the case, it is imperative that you make the enemy pay
for ignoring your army. I urge you to be as aggressive as you feel prudent at
this stage.

\gramClosing{Respectfully}
{James W. Blake}
{Maj. Gen., Commanding}
\reportdinkus

\gramHeader{Headquarters, Department of the Cumberland} % {{{4
{Clarksville, Tenn., April}{10, 1862}
\gramTo{Brig. Gen.}{William Kellogg}
{Commanding, Cavalry Division}

\gramHi{General} You are to move your division to White Oak, Tenn. as soon as
possible and establish a screen against enemy movement towards Dover.

\gramClosing{Respectfully}
{James W. Blake}
{Maj. Gen., Commanding}
\subsecdinkus

\subsection*{April 12, 1862}{} % {{{3

\gramHeader{Headquarters, Department of the Cumberland} % {{{4
{Sailor's Rest Depot, Tenn., April}{12, 1862}
\gramTo{Maj. Gen.}{James Howard}
{Commanding, Army of the Kentucky}

\gramHi{General} Our forces here in the west are in retreat towards Dover,
having been defeated by a force that may be up to seven corps of infantry,
including Gen'l Whisper's Army of Mississippi.

I have reason to believe the enemy may have, as expected, withdrawn significant
force from the area around Bowling Green. If the condition of your army warrants
and your supply situation will support it, I urge you to press the advantage and
do what you can to seize or seriously threaten Bowling Green. I do not expect
any advantage you have to remain for more than a few weeks as the enemy will
likely shift forces away from here once content to have driven us away from Fort
Defiance.

\gramClosing{Respectfully}
{J. W. Blake}
{Maj. Gen., Commanding}
\reportdinkus

\gramHeader{Headquarters, Army of the Kentucky} % {{{4
{Munfordsville, Ky. April}{12, 1862}
\gramTo{Maj. Gen.}{James Blake}
{Commanding, Dep't of the Cumberland}

\gramHi{Sir} Yours received after some delays in the wires, no doubt aggravated
by the re-organization and refit of my command. As soon as we are able to
assemble any sort of report on if my units are where they are intended to be,
what their final strengths are, and so on, I will seek to act upon this
information. I feel I must bring up the unfortunate fact that the rail bridge
will not be repaired until mid-May. A full throated two corps advance on
Bowling Green prior to the repair with my two corps will consume the bulk of the
supplies, impedimenta, and funds available for the season if executed prior to
the rail repairs.

I intend to probe forward as soon as my army is made ready, but will refrain
from a full commitment of my resources until you feel comfortable with such a
move based on the overall positions of the enemy.

\gramClosing{Yr Obt Svt}
{James Howard}
{Brig. Gen., Commanding, Army of the Kentucky}
\reportdinkus

\gramHeader{Headquarters, Department of the Cumberland} % {{{4
{Sailor's Rest Depot, April}{12, 1862}
\gramTo{Maj. Gen.}{Cornelius Van Royne}
{Commanding General, United States Army}

\gramHi{Sir} I regret to inform you that the enemy has fallen upon Gen'l Steele
in strength and has placed him as well as two corps of the Army of the
Cumberland in a dangerous position. Gen'l Steele met the enemy south of Fort
Defiance yesterday with his army, joined by Benton's Corps and Wool's Division,
both of the Army of the Cumberland but, even with this significant force, was
greatly outnumbered by the enemy.

Gen'l Steele was able to hold the field but does not believe he can withstand
another attack. I have ordered him to withdraw before the enemy, with his usual
superiority in cavalry, threatens to cut off Gen'l Steele's line of retreat.

With these developments I am forced to concede the area around Fort Defiance to
the enemy and am beginning to withdraw the forces here to defensive positions
around Dover.

The situation is still developing and I will keep you informed if anything
changes. With the large enemy force facing me (estimated at five corps plus
Gen'l Whisper's army inside the fort) I believe that Gen'l Howard may find the
way to Bowling Green much less defended and I intend to order him to press the
advantage if possible.

\gramClosing{Respectfully}
{J. W. Blake}
{Maj. Gen., Commanding}
\reportdinkus

\gramHeader{United States Army} % {{{4
{Washington City, April}{12, 1862}
\gramTo{Maj. Gen.}{James Blake}
{Commanding Department of the Cumberland}

\gramHi{General} Yours of this date just received. Do all that you can to
extricate Gen'l Steele's army. Beware that the enemy, with internal lines, may
see your withdrawal, expect an offensive by Gen'l Howard, and move against him
in strength, while leaving some token force at Nashville to delay you until
their armies can return to Nashville's relief. We cannot risk a second Prewitt's
Knob.  Please provide report on enemy strength, dispositions, commanders et
cetera if known with any degree of confidence; as well as known \& suspected
casualties for our forces and those of the enemy, when available. How much of
the reinforcements have been received by your department \& are in the field thus
far?

\gramClosing{I remain very respectfully}
{C. N. Van Royne}
{Maj. Gen., Commanding USA}
\reportdinkus

\gramHeader{United States Army} % {{{4
{Washington City, April}{12, 1862}
\gramTo{Maj. Gen.}{Julius Adams}
{Commanding Department of the Potomac}

\gramHi{General} If you can determine the composition \& strength of the enemy
before you, \& have Gen. Walle conduct aggressive reconnaissance to determine the
strength \& dispositions of the enemy before him, you must do so. I do not
believe they would have completely emptied out the Valley. This is either some
attempt at subterfuge or a desperate gamble to unite all eastern armies against
you.

\gramClosing{I remain very respectfully}
{C. N. Van Royne}
{Maj. Gen., Commanding USA}
\reportdinkus

\gramHeader{Headquarters, Department of the Cumberland} % {{{4
{Sailor's Rest, Tenn., April}{12, 1862}
\gramTo{Maj. Gen.}{Cornelius Van Royne}
{Commanding General, United States Army}

\gramHi{Sir} I hesitate to speculate on the size and composition of the enemy
force until I receive more detailed reports but Gen'l Steele believes the enemy
numbers at least five corps of infantry and one of cavalry. He reports that his
troops fought as well as could be expected but that at times single divisions
were facing an entire enemy corps. During the fighting yesterday all three corps
commanders were grievously wounded although it seems not mortally.

Gen'l Jackson is believed to be in overall command of the enemy force with
``wings'' commanded by Gen'ls Hardee, Bragg, Johnston, Wesley and Harrison.
There are unconfirmed reports that Gen'l Anderson commands the rebel cavalry.
Fort Defiance is still occupied by the Army of the Mississippi and that garrison
may be an additional two corps.

All reinforcements assigned to this department were received and in position
prior to the enemy movements against us.

\gramClosing{Respectfully}
{J. W. Blake}
{Maj. Gen., Commanding}
\reportdinkus

\gramHeader{Headquarters, Department of the Cumberland} % {{{4
{Sailor's Rest, Tenn., April}{12, 1862}
\gramTo{Cdre.}{Daniel Lewis}
{Commander, Cumberland Squadron}

\gramHi{Sir} My department finds itself in need of your support to effect a
possible emergency river crossing near Hematite, Tenn. where Gen'l Steele may be
forced to evacuate his army to the north bank of the river.

Gunboat support may also be necessary between Sailor's Rest and Palmyra,
depending on how the events if the next day unfold. If at all possible, your
rapid assistance is requested.

\gramClosing{Respectfully}
{J. W. Blake}
{Maj. Gen., Commanding}
\reportdinkus

\gramHeader{Headquarters, Dep't of the Cumberland} % {{{4
{In the field, April}{12, 1862}
\gramTo{Maj. Gen.}{Harold Fawcett, III}
{Commanding, Army of the Cumberland}

\gramHi{General} Gen'l Steele is en route to Dover with what remains of his
command. Casualties were heavy but I believe he managed to extract a larger
portion than I expected.

I am leaving Gen'l Sherman here with his division, Wool's Division of P. Smith's
Corps and Kellogg's cavalry to establish a very light screen in the manner we
previously discussed prior to the breakout. Gen'l Sherman is under strict orders
to withdraw if threatened. If you deem it feasible, it may be wise to send Gen'l
Wood, reinforced by McHenry, Jr's brigade towards Mill Spring to prevent the
enemy moving on you from that direction undetected.

I intend to follow behind this message with Gen'l Steele.

\gramClosing{Respectfully}
{J. W. Blake}
{Maj. Gen., Commanding}
\subsecdinkus

\subsection*{April 14, 1862}{} % {{{3

\gramHeader{Headquarters, Dep't of the Cumberland} % {{{4
{Camp Carter, Tenn., April}{14, 1862}
\gramTo{Maj. Gen.}{Fawcett, III}
{Commanding the Army of the Cumberland}

\gramHi{General} The retreat to Dover is complete.  How goes your movements?

\gramClosing{Respectfully}
{J. W. Blake}
{Commanding}
\reportdinkus

\gramHeader{Headquarters, Army of the Cumberland} % {{{4
{Sugar Creek Depot, Tenn., April}{14, 1862}
\gramTo{Maj. Gen.}{James W. Blake}
{Commanding, Department of the Cumberland}

\gramHi{General} Riding south to support the action of the Army of the Kanawha,
I uncovered a secondary debacle in the assembly of the Department's pontoon
bridge against orders and the passage of two artillery battalions unescorted
into what proved to be a full Rebel cavalry division laying in ambush around
Palmyra.

Subsequent investigation indicated that Artillery Colonel A. Stevens was most
responsible for the decision to assemble the pontoon against most explicit
orders, and had he survived the ambush he would certainly be held up on charges.
Fortunately I was able to extricate the majority of the artillery back across
the bridge and see the bridge sections themselves detached from the shore before
the Rebel cavalry could cross and continue into the Federal Depot at Sugar Creek
Landing.

Our Cavalry has won another small cavalry action at Fredonia, and the
Clarksville rail bridge has been fired---we are en route westwards with our
heavy siege batteries opposite Palmyra at this time.

We have lost eight regiments infantry and a net two batteries of artillery,
largely from the action south of the river.

I intend continuing the movement west to Dover Ferry pursuant to your
directions, to prepare the defense there. The pontoon sections now being under
my sole and direct authority, I will use it to transport the heavy siege mortars
to Dover and improve our rate of march.

There being no prospect of the enemy interfering with our movement north of the
river, we will continue our march with cavalry screening our rear and take
charge of the preparations for our defense.

\gramClosing{My compliments}
{Fawcett}
{Maj. Gen.}
\reportdinkus

\gramHeader{Headquarters, Dep't of the Cumberland} % {{{4
{Camp Carter, Tenn., April}{14, 1862}
\gramTo{Maj. Gen.}{Harold Fawcett, III}
{Commanding, Army of the Cumberland}

\gramHi{General} As noted in my last, the retreat from Barton's Creek is now
complete. In total we have lost nearly three divisions of infantry, two brigades
of cavalry and a corps worth of artillery. I expect the enemy losses to not
quite equal our own as our losses were exacerbated by the unfortunate need to
leave many of the wounded behind.

If you believe a subordinate can continue to manage the retreat of your army to
Dover, your presence is requested here as soon as possible. In addition, any
cavalry you can spare should press forward so that they may be sent to join
Gen'l Kellogg in the field. Gen'l Sherman is currently at Sailor's Rest with two
divisions of infantry and Kellogg is at Yellow Creek. Their intent is to deceive
the enemy as to our strength and hopefully prevent a general advance on Dover.
As per our previous discussion they are to immediately withdraw if pressed by
anything other than light probes. Any additional cavalry in that area would be
of great value.

\gramClosing{Respectfully}
{J. W. Blake}
{Ma. Gen., Commanding}
\reportdinkus

\gramHeader{Headquarters, Army of the Kentucky} % {{{4
{Bells, Ky., April}{14, 1862}
\gramTo{Maj. Gen.}{Blake}
{Commanding, Dep't of the Cumberland}

\gramHi{Sir} The enemy has abandoned the state of Kentucky as far as Bowling Green,
though they took the time to burn the rail bridge in front of Bowling Green as
they went. I currently have my cavalry just opposite that city on the other side
of the river, with a road bridge still standing. I intend to occupy it at least
with the cavalry. The remainder of my force is concentrated near Bells which is
about as as I can extend them past Munfordsville without incurring undue
material needs. I am pleased to report Departmental forces sent by Van Royne
have occupied the garrisons, freeing my force for the field with 41,300 men. The
rail bridge at Bowling Green will take ten days to repair---I am contemplating
not beginning work until early May on this as without the Munfordsville Bridge
it serves little purpose, and if down limits the ability of any enemy to extend
past Bowling Green if they turn away from you.

\gramClosing{Yr Obdt Svt}
{James Howard}
{Maj. Gen., Army of Kentucky, Commanding}

Copies to Steele \& Fawcett.

\reportdinkus

\gramHeader{Headquarters, Dep't of the Cumberland} % {{{4
{Camp Carter, Tenn., April}{14, 1862}
\gramTo{Maj. Gen.}{James Howard}
{Commanding Army of the Kentucky}

\gramHi{General} Your report is well-received. If you can rapidly seize Bowling
Green that would be ideal.

With our armies pushed back from Fort Defiance with heavy loss I expect the
enemy to turn to face you so you must choose good ground to limit the gains the
enemy may make against you.

Gen'l Fawcett has three division that are as yet unengaged although they are
making the slow march back from Clarksville. Depending on how the situation
develops I may send some portion of his army to assist you.

Keep me updated on your progress.

\gramClosing{Respectfully}
{J. W. Blake}
{Maj. Gen., Commanding}
\reportdinkus

\gramHeader{Headquarters, Army of the Kentucky} % {{{4
{April}{14, 1862}
\gramTo{Maj. Gen.}{James W. Blake}
{Commanding, Dep't of the Cumberland}

Yrs recv'd. Will burn Franklin North and Russellville rail brdgs wt yr approval.

\gramClosing{}
{J.H.}
{}
\reportdinkus

\gramHeader{Headquarters, Dep't of the Cumberland} % {{{4
{Camp Carter, Tenn., April}{14, 1862}
\gramTo{Maj. Gen.}{Howard}
{Commanding, Army of the Kentucky}

\gramHi{General} If you believe destruction of the Franklin and Russellville
bridges is critical to enabling you to hold Bowling Green, then approval to
destroy them is granted. As I fully expect Gen'ls Jackson or Whisper to turn
against you, anything you can do to strengthen your positions is wise. I ask you
to hold Bowling Green as long as it does not risk destroying your army.

Please update me as to the positions of your forces as I did not understand you
to have advanced past Bowling Green from your previous message.

\gramClosing{Respectfully}
{J. W. Blake}
{Maj. Gen., Commanding}
\reportdinkus

\gramHeader{Headquarters, Army of the Kentucky} % {{{4
{Bells, Ky., April}{14, 1862}
\gramTo{Maj. Gen.}{James W. Blake}
{Commanding Dep't of the Cumberland}

\gramHi{Sir} I have not. My cavalry are on the east side of the river near
Bowling Green, but I intend to conduct a recon in force through Bowling Green,
burn the bridges to delay the enemy, and then return the cavalry to a more
sustainable position in Bowling Green. As I will not be in constant contact with
them during the raid, I am making the decision to burn or not before I dispatch
them.

\gramClosing{Yr Obt Svt}
{James Howard}
{Maj. Gen., Commanding, Army of the Kentucky}
\reportdinkus

\gramOrdersHeader{Headquarters, Dep't of the Cumberland} % {{{4
{Camp Carter, Tenn., April}{14, 1862}
{Special Field Orders}{3}

The forces of this Dep't between the Cumberland and Tennessee Rivers will
establish positions to defend Dover and Crossing, Tenn. to prevent the enemy
from opening the rail road across the Tennessee River.

I. Gen'l Sherman is in temporary command of ``Sherman's Corps'', consisting of
his own division, Wools' and Kellogg's Cavalry. This corps will maintain a light
screen of the Sailor's Rest depot to deceive the enemy of our strength in the
area and delay a general advance on Dover. Preservation of force is paramount,
however, and Gen'l Sherman will withdraw towards Dover if pressed by more than
light forces.

II. Gen'l Wood will move his to division to Mill Spring with the intent of
preventing the enemy from moving on our lines from the south. He is to withdraw
towards Crossing if pressed by more than light forces.

III. Upon arrival, Gen'l Fawcett will send Dickey's Cavalry Division to support
Kellogg and Bussey's Cavalry Division to screen Wood, positioning that division
around Trousdale. Upon arrival of said units, Gen'l Kellogg will detach his
division from Sherman's Corps and report to Brig. Gen. Graham. McClernand's
Corps and the bulk of C. Smith's Corps will establish positions south and east
of the rail road, exact positions to be determined on their arrival at Dover.

IV. Col. McHenry, Jr. is to maintain the garrison at Crossing, Tenn. and has
orders to destroy the bridge and withdraw to Dover if unable to hold the
position.

V. Until the arrival of the remainder of the Army of the Cumberland, Gen'l
Steele will remain in temporary command of units that army that are present
around Dover. Gen'l Steele will focus on the recovery of his command but will
also provide for the defense of Dover and Crossing until relieved by Gen'l
Fawcett.

\gramClosingBy{Gen'l Blake}
{Walter Chekov}
{Col., Adjutant General}

Copy to Fawcett

\reportdinkus

\gramHeader{United States Army} % {{{4
{Washington City, April}{14, 1862}
\gramTo{Maj. Gen.}{James Blake}
{Commanding Department of the Cumberland}

\gramHi{General} I am in receipt of your preliminary report of the engagement at
Barton's Creek. In conference with Gen'ls Johanson \& Benton, who have arrived
in Washington to recuperate from their wounds, this new information has been
most illuminating as to the odds your army faced south of Fort Defiance. I
intend to make the matter \& all circumstances clear to those civil authorities
who have asked of me clarification for recent events.

It is made clear to me that the enemy's casualties at Barton's Creek must have
been very heavy. It is unlikely he will pursue you in the near future.

I recommend strongly that your departmental reserve be dissolved and its
constituent units given out to the army of Gen. Steele to replenish, at least in
part, the losses he so lately suffered.

Make every effort to prepare your three armies for offensive operations and to
make another attempt on Nashville once they are ready. Your understanding of the
situation in the field is far better than mine, but I do not think I can
countenance an advance by General Howard now, as he will be mightily isolated
against a significant force of the enemy, should they turn against him. If the
circumstances do not in your mind make it completely foolish, Howard should
occupy Bowling Green with at least some of his cavalry, if the enemy has
abandoned his front.

\gramClosing{I remain very respectfully}
{C. N. Van Royne}
{Maj. Gen., Commanding USA}
\reportdinkus

\gramHeader{Headquarters, Dep't of the Cumberland} % {{{4
{Camp Carter, Tenn., April}{14, 1862}
\gramTo{Maj. Gen.}{Cornelius Van Royne}
{Commanding General, United States Army}

\gramHi{Sir} Your last is well-received here in Tennessee. I ask that you let
both Gen'ls Johanson \& Benton, as well as the civilian authorities, know that
their conduct is reported to have been exemplary under the circumstances.
Without such officers as these, our forces would most certainly have fared much
worse.

Once the armies here are again ready to take the field, I will indeed assign
reserves as needed to fill out the depleted formations.

Gen'l Howard intends to send his cavalry through Bowling Green en route to
destroying the rail bridges at Russellville and Franklin so as to make it more
difficult for the enemy to launch an offensive against him. He is unable to
advance the bulk of his army to that city until his supply situation improves
with the repair of previously destroyed bridges.

\gramClosing{Respectfully}
{J. W. Blake}
{Maj. Gen., Commanding}
\reportdinkus

\gramHeader{United States Army} % {{{4
{Washington City, April}{14, 1862}
\gramTo{Maj. Gen.}{Karl Meyer}
{Commanding Department of the West}

\gramHi{General} Do you believe it possible to establish a naval battery, or
several such batteries, at Riddles Point, once Thomson's army is guaranteed to
be invested on the Kentucky Bend? If Thomson is invested, the batteries at New
Madrid, as you said, could not cut off his support, but Riddles Point commands
the landings opposite on the Bend.

Owing to the events at Barton's Creek, the enemy may begin to turn their
attention to you \& the imminent siege of Is. 10 \& Thomson's army. Jackson's
losses near Nashville were likely very severe, \& his army probably quite
fatigued, but we cannot rule out the possibility that the enemy attempts a
combination against you. Do what you can to expedite the reinforcement of Gen.
Caldwell with forces from Gen. Ambrose.

\gramClosing{I remain very respectfully}
{C. N. Van Royne}
{Maj. Gen., commanding USA}
\reportdinkus

\gramHeader{Headquarters, Department of the West} % {{{4
{Crockett, Tenn., April}{14, 1862}
\gramTo{Maj. Gen.}{Cornelius Van Royne}
{Commanding General of the United States Army}

\gramHi{General} Yrs this date just received. If General Ambrose does not
encounter any enemies, then it may indeed be possible to establish such
batteries at Riddles' Point. However, we would need the cannons and equipment to
establish said batteries, due to the necessity of spending resources on the rail
line to New Madrid.

I am well aware of the issue that confronts me following Barton's Creek and have
urged General Ambrose to move with all due haste to discover whether Thomson has
stationed anything more than a small division of cavalry on the west bank. The
moment that he has established an understanding of the enemy to his front, I
will order him to transfer troops as needed.

\gramClosing{I am always, Generals, and shall ever remain, Your most humble and obedient of servants}
{Karl Meyer}
{Maj. Gen. cmdg, Dep't of the West}
\reportdinkus

\gramHeader{United States Army} % {{{4
{Washington City, April}{14, 1862}
\gramTo{Maj. Gen.}{James Blake}
{Commanding Department of the Cumberland}

\gramHi{General} I urgently require that a sworn statement from General Steele,
as well as from yourself \& other general officers should you \& the others be
willing, regarding known enemy strength at Barton's Creek, \& in general an
overall estimate of the strength of Jackson's army, be provided to my
headquarters immediately.

If any estimate with reasonable certainty can be made regarding enemy
casualties, I require them also. If they cannot be provided with reasonable
certainty, then do not provide anything at all.

\gramClosing{I remain very respectfully}
{C. N. Van Royne}
{Maj. Gen., commanding USA}
\subsecdinkus

\subsection*{April 21, 1862}{} % {{{3

\gramHeader{Headquarters, Dep't of the West} % {{{4
{Troy, Tenn., April}{21, 1862}
\gramTo{Maj. Gen.}{Cornelius Van Royne}
{Commanding General of the United States Army}

\gramHi{General} Dispatched with all due haste. The Cavalry division on the
south bank near Obionville has retired after a week of skirmishing revealing
only spiked cannons and an Island Fort still held by the Rebels. General Thomson
and his army are not present at Obionville and I have doubts that he was ever
there in the first place due to the strong cavalry screen against us.

This discovery has returned the Department's plans back to its original state
when we assumed that Thomson would not be present at Island Number 10. The
Division of Observation has moved to garrison the peninsula, and I will be
ordering General Ambrose to bring up the siege guns shortly. Though I have yet
to hear from him, I am certain that he will not find any trace of the army at
Gayoso nor Caruthersville.

My deepest apologies for failing to contain the Army of the Trans-Mississippi.
My greatest concern---though unsubstantiated---are that they may have been
shipped down to Memphis  the last week of the winter campaign, before being
railed to Clarksville where they faced off against General Steele. I will
endeavor to provide you further information once my army commanders write to me.

\gramClosing{I am always, General, and shall ever remain, Your most humble and obedient of servants}
{Karl Meyer}
{Maj. Gen. cmdg, Dep't of the West}
\reportdinkus

\gramHeader{United States Army} % {{{4
{Washington City, April}{21, 1862}
\gramTo{Maj. Gen.}{Karl Meyer}
{Commanding Department of the West}

\gramHi{General}

Yours of this date just received. We have no indications as yet that any
elements of Thomson's army were present at Barton's Creek, but we are continuing
to investigate reports and gather information on the matter. Continue to
prosecute a vigorous effort against Is. No. 10.

\gramClosing{I remain very respectfully}
{C. N. Van Royne}
{Maj. Gen., commanding USA}
\reportdinkus

\gramHeader{United States Army} % {{{4
{Washington City, April}{21, 1862}
\gramTo{Maj. Gen.}{James Blake}
{Commanding Department of the Cumberland}

\gramHi{General}

I am in receipt of a cable from Gen. Meyer informing me that he can no longer
account for the location of Thomson's army. You must expect that he will arrive
in your area of operation, if not already there. Every decision for the next two
or three weeks must be made toward the end of ensuring the survival and health
of the armies of your department.

\gramClosing{I remain very respectfully}
{C. N. Van Royne}
{Maj. Gen., commanding USA}
\reportdinkus

\gramHeader{Headquarters, Dep't of the West} % {{{4
{Troy, Tenn., April}{21, 1862}
\gramTo{Maj. Gen.}{Cornelius Van Royne}
{Commanding General of the United States Army}

\gramHi{General} General Ambrose has responded to my inquiries and has revealed
that  there had been major rebel activity on the river, with ships moving from
Obionville to an unknown location southward and back all week loaded with
troops.

It is most likely that Thomson is currently around the vicinity of Memphis and
intends either on moving to Corinth before up the Central Alabama R.R. to
Nashville or else using his new found freedom to maneuver around the Army of the
Tennessee.

General Ambrose has sent his Third Cavalry Division to the Army of the
Tennessee, bring up their total strength to 15 regiments. His cavalry force has
scouted as far south as Gayoso before encountering more rebel cavalry with no
accompanying infantry or artillery so far as we can tell in Caruthersville.

Furthermore, General Ambrose has informed me that he intends to detach said
cavalry of 15 regiments from his supply lines with a weeks worth of supply in
order to attempt to outmaneuver the enemy cavalry and scout closer to the enemy
army, which he believes is as far south as Cottonwood Point.

In light of these developments, General Caldwell intends on doing the same with
his own Cavalry Division of 15 regiments, pushing south with a weeks worth of
supply in order to establish a rail depot further south for the infantry to
travel down.

While these movements are occurring, the Division of Observation will begin the
siege of Island Number 10, bringing up 4 batteries of siege guns in order to
start reducing the fort. Unfortunately, the navy is still badly mauled from the
Winter Campaigns against Forts Henry, Donelson, and DuRussey, and would not be
able to join in the bombardment without risk of serious damage to their
ironclads. They have promised that a glut of ships will be in service come the
Summer when they will be much more active.

\gramClosing{I am always, General, and shall ever remain, Your most humble and obedient of servants}
{Karl Meyer}
{Maj. Gen. cmdg, Dep't of the West}
\reportdinkus

\gramHeader{Headquarters, Dep't of the Cumberland} % {{{4
{Camp Carter, Tenn., April}{21, 1862}
\gramTo{Col.}{John McHenry, Jr.}
{Crossing Garrison}

\gramHi{Colonel} You will maintain your position defending the bridge however
the crossing must not fall into enemy hands. If you are attacked in such force
that you do believe the enemy will seize the bridge, you are to destroy the
bridge immediately on your own authority.

Take measures, if you have not already done so, to ensure the bridge may be
destroyed without delay if necessary.

\gramClosingBy{Gen'l Blake}
{Geo. Campbell}
{Capt. Aide de Camp}
\reportdinkus

\gramHeader{Headquarters, Dep't of the Cumberland} % {{{4
{Camp Carter, Tenn., April}{21, 1862}
\gramTo{Col.}{James McMillan}
{Frankfort Garrison}

\gramHi{Colonel} Your report of the security of central Ky. is well received.
Upon receipt of this message you are to immediately decamp from Frankfort and
move by rail and river to Camp Carter, Tenn. for reassignment.

\gramClosingBy{Gen'l Blake}
{Geo. Campbell}
{Capt. Aide de Camp}
\reportdinkus

\gramHeader{Headquarters, Dep't of the Cumberland} % {{{4
{Camp Carter, Tenn., April}{21, 1862}
\gramTo{Cdre.}{Daniel Lewis}
{Commanding, Cumberland River Squadron}

\gramHi{Commodore} As the whereabouts of the enemy army under Gen'l Thomson are
currently unknown, I humbly request that you place extra emphasis on your
patrols of the Tennessee River so as to prevent the enemy crossing that river
without our knowledge.

In addition, any intelligence you may gain on the enemy fort at Pittsburg
Landing without undue risk to your squadron would be most appreciated.

\gramClosing{Respectfully}
{J. W. Blake}
{Maj. Gen., Commanding}
\subsecdinkus

\subsection*{April 28, 1862}{} % {{{3

\gramHeader{Headquarters, Department of the West} % {{{4
{Humboldt, Tenn., April}{28, 1862}
\gramTo{Maj. Gen.}{Cornelius Van Royne}
{Commanding General of the United States Army}
\gramTo{Maj. Gen.}{James W. Blake}
{Commanding General of the Department of the Cumberland}

\gramHi{Generals} I posit to you that the Army of the Tennessee may have
possibly regained contact with the Army of the Trans-Mississippi. During this
past week, General Caldwell aggressively moved his cavalry south to Trenton
without facing any resistance, establishing a depot at the town and bring up the
infantry before continuing onto Humboldt.

At Humboldt, the Cavalry Division came into contact with elements of cavalry from
the Army of the Trans-Mississippi, much smaller in numbers compared to Brigadier
General Daniel's 15 regiments. The rebels were defending a half finished fort
and a small depot at Humboldt but were driven southwest to Bell after a short and
sharp fight was had. We now occupy the sturdy walls of this fort, whose
construction has been finished but lacks the batteries to defend it.

General Ambrose's own cavalry corps under Brigadier General Stanley continued to
push south past Caruthersville and all the way to Cottonwood Landing,
encountering only the same 1,800--2,400 cavalry they had been meeting this
entire time. It is his suspicion as well as mine that Thomson is concentrating
entirely on the east bank of the river in terms of infantry and artillery
concentration.  With the Cavalry's retreat to Bell, it seems fairly certain they
are intending to make a stand at Memphis.

At Island Number 10, the Division of Observation has begun to set up positions
on the south bank of the river, with the siege guns begin to be emplaced. They
should be ready to commence their bombardment next week, at which point the
siege of the fort can finally commence.

With 6 to 8 regiments of cavalry on the east side of the river, and
significantly less than 15 on the right side, it seems that General Thomson has
either not been reinforced with cavalry this season, or sent at least a division
away to parts unknown.

With my chief surgeon reporting that 13\% of the Army of the Tennessee has
missed muster this week due to illness and other reasons, I am in the midst of
discussion with General Caldwell on whether to push further south towards
Memphis at this time.

\gramClosing{I am always, Generals, and shall ever remain, Your most humble and obedient of servants}
{Karl Meyer}
{Maj. Gen. cmdg, Dep't of the West}
\reportdinkus

\gramHeader{Headquarters, Dep't of the Cumberland} % {{{4
{Camp Carter, Tenn., April}{28, 1862}
\gramTo{Maj. Gen.}{Cornelius Van Royne}
{Commanding General, United States Army}

\gramHi{Sir} During the past week, my cavalry has aggressively probed the enemy,
attempting to receive him as to our weakened state, enabling our battered units
time to recover.

The enemy, cavalry refused to give battle, at times bringing up infantry to keep
my troopers at bay. The end result is that the enemy main body has fallen back
to Cloverdale, leaving a substantially reduced garrison to hold Fort Defiance.

I believe the enemy, having extricated Gen'l Whisper, has fallen back to his
supply line and possibly withdrawn troops to focus on threats elsewhere,
assuming we have been defeated for the time being.

However, our forces have recovered faster than anticipated, allowing us to take
advantage of the enemy retreat and go on the offensive.

Gen'l Fawcett, having sustained light casualties at Barton's Creek will advance
directly on Cloverdale along parallel roads while Gen'l Steele will approach
from the north as a supporting force.

I will, of course, be cautious not to fall into a trap, however, seizure of
Cloverdale would put us in the position we intended before Barton's Creek: able
to fend off the enemy field army while reducing the fort at our leisurely.

The news from Gen'l Howard leads me to believe the enemy may be moving to retake
Bowling Green or at least prevent him from advancing farther as his cavalry has
encountered enemy horse around the rail bridges he destroyed last week.

He intends to maintain his cavalry at Bowling Green as the rest of the army
cannot move up until the Munfordville bridge is repaired. His orders will be to
preserve his army even at the expense of losing Bowling Green.

\gramClosing{Respectfully}
{J. W. Blake}
{Maj. Gen., Commanding}
\reportdinkus

\gramHeader{United States Army} % {{{4
{Washington City, April}{28, 1862}
\gramTo{Maj. Gen'ls}{Blake \& Meyer}
{Commanding Dep'ts of the Cumberland and West, resp'y}

\gramHi{Gen'ls} I pray all is well. We have received intelligence out of
Nashville for your attention.

Between forty to sixty-thousand Confederate troops have moved from Clarksville
to Nashville (perhaps the Army of the West). This army is joined both by General
Whisper and Anderson. There is increased railroad activity on the Central
Alabama Line between Nashville and Decatur.

Though unsubstantiated, our current theory is thus: We believe at least one
corps of Anderson's force is bound either for Maj. Gen. Meyer in the west, Maj.
Gen. McCarroll in New Orleans, or to reinforce the east. We tell you this not to
cause hesitation or fear, but to be made aware. Though we cannot say for
certain, the most likely destination is near Memphis to attack Caldwell in
force.

Gen'l Commanding will return regarding your latest developments.

\gramClosing{Humbly and always, Yr obt svt}
{M. R. Turgon}
{Maj., Office of the Commanding Gen'l}
\reportdinkus

\gramHeader{Headquarters, Dep't of the Cumberland} % {{{4
{Camp Carter, Tenn., April}{28, 1862}
\gramTo{Maj.}{Turgon}
{}

\gramHi{Major} Can you clarify your last? The movement of significant rebel
troops to Nashville from the Fort Defiance area comports with the information
available to this headquarters, but they cannot have moved from Clarksville as
the bridge to that city from Fort Defiance was burned by Gen'l Fawcett. Perhaps
there is just a confusion in the particulars of the movement?

\gramClosing{Respectfully}
{Walter Chekov}
{Col., Adjutant General}
\reportdinkus

\gramHeader{United States Army} % {{{4
{Washington City, April}{28, 1862}
\gramTo{Col.}{Chekov}
{}

Apologies. Reports indicated they marched in from the northwest and west. Our
sources used Clarksville as the identifier within said report. In reality, these
troops almost certainly came from the fighting at Defiance. Apologies again,
thank you for catching that.

\gramClosing{Yr obt svt}
{M. R. Turgon}
{Maj., Office of the Commanding Gen'l}
\reportdinkus

\gramHeader{Headquarters, Army of the Kentucky} % {{{4
{Dripping Springs, Ky., April}{28, 1862}
\gramTo{Maj. Gen.}{James W. Blake}
{Commanding Department of the Cumberland}

\gramHi{Sir} An overall quiet week, but long patrols by our cavalry have found
enemy cavalry in unknown strength---truly unknown, could be patrols or brigades
---near the Russellville and North Franklin bridges that we burnt last week. 

Our own cavalry is encamped in Bowling Green, patrolling out to surveil the
approaches before returning to the limit of our reach. My main body remains east
of Dripping Springs, held back by the still unfinished rail bridge at
Munfordsville. 

I intend to establish a second cavalry depot in Bowling Green to allow a more
permanent forward presence in light of enemy activity. This measure will
effectively prohibit any future effort to move the entire army by wagon, but I
suspect the adjusted schedule after Barton's Creek means that is no longer a
necessary option.

\gramClosing{Yr Obt Svt}
{James Howard}
{Maj. Gen., Army of the Kentucky, Commanding}

Copies to Maj. Gen. Harold K. Fawcett, III \& Maj. Gen. Richard Steele 

\reportdinkus

\gramHeader{Headquarters, Dep't of the Cumberland} % {{{4
{Camp Carter, Tenn., April}{28, 1862}
\gramTo{Maj. Gen.}{James Howard}
{Commanding Army of the Kentucky}

\gramHi{General} Reports from Washington are that 40 to 60 thousand rebel troops
have moved from the Fort Defiance area to Nashville. Although Washington does
not believe this force will move in your direction, I cannot rule out that
possibility.

As previously ordered, preservation of your army is of more importance than
holding Bowling Green. I trust your judgement in this matter

Here between the rivers, the enemy has fallen back to Cloverdale, again exposing
Fort Defiance, with a much reduced garrison, to siege. I intend to take the
offensive here as I believe we have an advantage, the enemy assuming we are not
yet recovered from Barton's Creek.

\gramClosing{Respectfully}
{J. W. Blake}
{Maj. Gen., Commanding}
\reportdinkus

\gramHeader{Headquarters, Dep't of the Cumberland} % {{{4
{Camp Carter, Tenn., April}{28, 1862}
\gramTo{Brig. Gen'ls}{Thomas Wood, \& William Kellogg}
{}
\gramTo{Col's}{John McHenry, Jr. \& Nathan Kimball}
{Commanding resp. divisions and brigades}

\gramHi{Sirs} It is my honor to relay to you the orders for your respective
commands during this headquarters movement against the enemy at Cloverdale,
Tenn.

Gen'l Kellogg will continue to attach his division to Gen'l Graham's corps of
cavalry in support of the Army of the Cumberland.

Gen'l Wood is to move his division to White Oak and prevent the enemy from
falling on the right flank of the Army of the Cumberland moving east from
Maysville and prevent the enemy from pushing through White Oak toward Mill
Spring.

Col. McHenry, Jr.'s brigade will continue to garrison the bridge at Crossing,
immediately destroying said bridge if it is in danger of being seized by the
enemy.

Col. Kimball's brigade will establish positions to defend against a cavalry raid
towards Dover and act as the Department reserve for the current operation.

\gramClosingBy{Gen'l Blake}
{Walter Chekov}
{Col., Adjutant General}
\reportdinkus

\gramHeader{Headquarters, Dep't of the Cumberland} % {{{4
{Camp Carter, Tenn., April}{28, 1862}
\gramTo{Maj. Gen.}{Karl Meyer}
{Commanding,, Dep't of the West}

\gramHi{General} With your forces occupying Humboldt, if we clear the line
between there and my garrison at Crossing, Tenn., we will have a direct rail
line between our departments, enabling rapid transfer of forces.

I am not sure I can spare any troops this week to clear the path but it is
something to consider

\gramClosing{Respectfully}
{J. W. Blake}
{Maj. Gen., Commanding}
\reportdinkus

\gramHeader{Headquarters, Department of the West} % {{{4
{Humboldt, Tenn. April}{28, 1862}
\gramTo{Maj. Gen.}{James Blake}
{Commanding General of the Department of the Cumberland}

\gramHi{General} Yrs of this date just received. I am transferring the 3d
Cavalry Division from General Ambrose to General Caldwell and sending them from
Hickman in the direction of Dresden in order to clear the rail line as far as
McKenzie.

\gramClosing{I am always, Generals, and shall ever remain, Your most humble and obedient of servants}
{Karl Meyer}
{Maj. Gen. cmdg, Dep't of the West}
\reportdinkus

\gramHeader{United States Army} % {{{4
{Washington City, April}{28, 1862}
\gramTo{Maj. Gen.}{James Blake}
{Commanding Department of the Cumberland}

\gramHi{General} I had thought to issue these orders peremptorily, but having
now received word of your advance again on Defiance \& Nashville, as of yours of
this date which is just received, I must ask in a discretionary manner instead
for you to consider preparations for an expedition of a small force, perhaps
division-sized, up the Tennessee, to land at Coffee Landing, and demolish the
enemy fort being built at Pittsburg Landing \& destroy any force posted there to
defend it. Upon completion of this task, the force may be returned downriver to
its parent unit. Reconnaissance by river vessel may be best to determine if the
fort is still being built \& is held lightly, as determined by the last
reconnaissances conducted late last month. If it is held in great strength, or
if Coffee Landing is now also held in strength, then these orders may be
disregarded.

If such an operation is deemed by your judgment impossible, owing to your
dispositions around Fort Defiance \& the threat posed by the enemy nearby,
consider the order, again, merely discretionary, but still possibly to be
carried out in the future if time \& resources permit. The purpose of such an
operation is to pre-empt any enemy attempt to fall back on that fort, should the
rest of West Tennessee fall to our forces, and to render a defense of Corinth
difficult or impossible.

I wish to know any developments regarding Gen. Howard and his command over the
last week, as soon as they are known to you.

Feel the enemy out gingerly, as you thus far have done, and as always keep
Nashville as your primary goal.

Excuse the length of this message.

\gramClosing{I remain very respectfully}
{C. N. Van Royne}
{Maj. gen. commanding USA}
\reportdinkus

\gramHeader{United States Army} % {{{4
{Washington City, April}{28, 1862}
\gramTo{Maj. Gen.}{Karl Meyer}
{Commanding Department of the West}

\gramHi{General} As you were previously informed, I believe it possible, though
I cannot verify this suspicion, that the enemy seeks to reinforce Thomson with
some portion of his strength; perhaps a corps or more. I had thought to have
Gen. Blake detach a corps on a temporary basis to further strengthen Gen.
Caldwell under these circumstances, but his renewed advance makes this an
unfriendly suggestion.

I know that you have detached a division of cavalry from Gen. Ambrose and
attached it to the Army of the Tennessee. Is it possible for him to also detach
a division of his infantry to Gen. Caldwell? Gen. Ambrose can then command his
remaining division near New Madrid, the Division of Observation engaged in the
siege of Is. No. 10, and all the cavalry of the Army of the Arkansas; with his
primary objective being the reduction \& capture of 10. If this arrangement has
already been made without my knowledge, then you may of course ignore this.

\gramClosing{I remain very respectfully}
{C. N. Van Royne}
{Maj. gen. commanding USA}
\reportdinkus

\gramHeader{Headquarters, Dep't of the Cumberland} % {{{4
{Camp Carter, Tenn., April}{28, 1862}
\gramTo{Maj. Gen.}{Cornelius Van Royne}
{Commanding General, United States Army}

\gramHi{Sir} I apologize that my last was not clear regarding the disposition of
the Army of the Kentucky. Gen'l Howard's cavalry has made contact with enemy
cavalry near the Franklin and Russellville bridges he destroyed last week. The
strength of this force entirely is unknown.

General Howard intends to establish a depot at Bowling Green to enable his
cavalry to maneuver more freely but his main body still awaits the repair of the
Munfordville bridge.

He has been given orders to fall back from Bowling Green is necessary to
preserve his army.

As for your suggested maneuver against Pittsburg Landing, Cdre. Lewis sent an
ironclad against the fort last week but it withdrew when fired upon by enemy
guns. It is my understanding that the fort there is indeed complete and likely
requires more than a division to siege or reduce.

However, with General Meyer's seizure of Humboldt, it may be possible for me to
send a corps in that direction although I am not certain I can spare that amount
of force until the Cloverdale issue is decided. However, I am conferring with my
generals at this time.

\gramClosing{Respectfully}
{J. W. Blake}
{Maj. Gen., Commanding}
\reportdinkus

\gramHeader{Headquarters, Department of the West} % {{{4
{Humboldt, Tenn., April}{28, 1862}
\gramTo{Maj. Gen.}{Cornelius Van Royne}
{Commanding General of the United States Army}

\gramHi{General} Yrs of this date just received. General Ambrose has offered to
dispatch both his 2d Cavalry Division of 7 regiments of cavalry, as well as his
2d Division of 16 regiments of infantry and two batteries of artillery to join
General Caldwell's Army of the Tennessee. I will of course return the Division
of Observation to his command as we are moving further south.

The addition of these two Divisions would bring the strength of this army to 61
regiments of infantry, 22 regiments of cavalry, and 25 batteries of artillery
for a total of 45,700 men. Given that General Thomson's army is estimated to be
anywhere from 22,000--25,000 strong with no more than 30,000 soldiers, I would
guess that the appearance of another enemy corps would give them a slight
advantage in numbers.

It is for this reason, that I would respectfully make several requests of you.
The first of which would be the garrisoning of the blockhouses along the rail
line that is currently supplying this army. This job is currently being done by
our own men, but freeing up even two extra regiments may prove to be worthwhile
should we come into contact with the enemy.

My second request would be for the garrisoning and completion of this unfinished
fort at Humboldt to once again free up the Army of the Tennessee to maneuver as
it sees fit. Holding the rail junction at Humboldt is crucial, and the Rebels
seemed to have realized this given their efforts at building a fort at this
location.

The current works can fit 2 regiments behind its walls and though it would take
a significant investment, I feel that holding Humboldt junction would secure the
rail lines leading north to Hickman, Columbus, and most importantly, towards
Crossing.

Please inform me if you would prefer that I wait at Humboldt to receive the
enemy, advance towards Corinth to intercept him, or if you would prefer for me
to continue my advance towards Memphis in light of the eventual arrival of this
new enemy corps.

\gramClosing{I am always, Generals, and shall ever remain, Your most humble and obedient of servants}
{Karl Meyer}
{Maj. Gen. cmdg, Dep't of the West}
\reportdinkus

\gramHeader{Headquarters, Dep't of the Cumberland} % {{{4
{Camp Carter, Tenn., April}{28, 1862}
\gramTo{Maj. Gen.}{Cornelius Van Royne}
{Commanding General, United States Army}

\gramHi{Sir} I believe Washington not to be in possession of the most up to date
reports on the rebel fort at Pittsburg Landing. Cdre. Lewis's gunboats have been
fired upon during both patrols of that area, indicating the fort is indeed
completed.

As such, sending a division against the fort would do no more than risk our
forces needlessly. I would prefer to wait until we have driven back the enemy
from Cloverdale and are able to send a stronger force. Ideally, General Howard
will break through soon and the bulk of his army could be directed south.

\gramClosing{Respectfully}
{J. W. Blake}
{Maj. Gen., Commanding}
\reportdinkus

\gramHeader{Headquarters, Department of the West} % {{{4
{Humboldt, Tenn. April}{28, 1862}
\gramTo{Maj. Gen.}{Cornelius Van Royne}
{Commanding General of the United States Army}

\gramHi{General} I fear we are running short of time. Every moment we delay, the
forces of General Anderson may be drawing closer toward us.

I humbly and respectfully request a response regarding my last message
regarding, additional garrisoning regiments for the blockhouses and the fort,
the completion of the half finished fort at Humboldt, additional supplies if they
can be spared, and your thoughts upon our future movements in the face of enemy
reinforcements.

\gramClosing{I am always, Generals, and shall ever remain, Your most humble and obedient of servants}
{Karl Meyer}
{Maj. Gen. cmdg, Dep't of the West}
\reportdinkus

\gramHeader{United States Army} % {{{4
{Washington City, April}{28, 1862}
\gramTo{Maj. Gen.}{Karl Meyer}
{Commanding Department of the West}

\gramHi{General} Yrs of this date rcvd. The supplies you requested released to
your department shall be delivered without delay. I will order three
regiments from Fort Hassendeubel at Columbus to garrison newly-established
blockhouses along the rail-lines. I cannot spare other regiments from Cairo,
Miss., Ky., or Indiana due to the concerns of the Govt. Additionally, requisite
supply will be released to complete the fort at Humboldt, the naming of which I
leave to your judgment. The new deployment will leave but one regiment and one
battery at Fort Hassendeubel. The garrison at Hickman will remain unmoved.

Any forces which can be immediately spared from Gen. Ambrose and sent to Gen.
Caldwell must be dispatched immediately, particularly those of infantry, and all
shifts of strength currently underway must be expedited with haste.

Because a significant expenditure is being made for the fort at Humboldt, I
require that you make a stand to defend it, if you are attacked, unless
intelligence and reason suggest your army be at risk of destruction. You must
not under any circumstance allow any part of your force to be invested at
Humboldt. If the place is surrounded by the enemy, relief is not likely. General
Blake is focused solely on the capture of Nashville.

In any event, if you are forced to abandon Humboldt, you must do everything
within your power to demolish and destroy the fortifications there, and destroy
the guns in as thorough a manner as possible.

You must be cognizant of a possible enemy advance not only from the southwest,
but also due south.

How much supply remains in your reserve? I wish to know so that I can expect,
broadly speaking, what your requests will be in the future. I have already
issued some supply to Gen. Adams and Gen. Garland.

\gramClosing{I remain very respectfully}
{C. N. Van Royne}
{Maj. gen. commanding USA}
\reportdinkus

\gramHeader{Headquarters, Department of the West} % {{{4
{Humboldt, Tenn. April}{28, 1862}
\gramTo{Maj. Gen.}{Cornelius Van Royne}
{Commanding General of the United States Army}

\gramHi{General} Yrs of this date rcvd. In regards to Gen. Ambrose, he has sent
the 2d Cavalry Division with 7 regiments and the 2d Infantry Division with 16
regiments east to the Army of the Tennessee.

The 2d Cavalry will be used to clear the rail line up to Crossing in order to
establish a line of transportation between myself and Gen. Blake. This is why I
have requested additional regiments, so as to preserve this line once cleared
from partisans. The 2d Infantry Division will be joining the rest of the army
at Humboldt.

The XVIII Corps will be moving towards Gadsen to preserve the rail bridge
crossing across the South Fork. The 3d Cavalry Division will be moving towards
the ford north of Quincy to secure that route of crossing, while the 1st Cavalry
Division will construct a depot at Jackson, before attempting to raid Corinth.

The Army of the Tennessee currently has a moderate amount of supply in its
reserve before your resupply. Due to a miscalculation by Gen. Caldwell regarding
how much supply it would take to sustain each regiment, the cost of sustaining
our advance was perceived to be much higher than in truth. I have since
corrected him of this notion, but the main emergency reserve for a wagon depot
is currently still being held should we be forced off the railroads which we are
currently sustaining ourselves by.

Gen. Ambrose reports that he has a small amount of supply remaining, though he
is quite certain that neither he nor the enemy will be making any further
movements west of the Mississippi, and that his single division in Missouri is
now stretched thin defending New Madrid as well as the crossing at Riddle's
point.

\gramClosing{I am always, Generals, and shall ever remain, Your most humble and obedient of servants}
{Karl Meyer}
{Maj. Gen. cmdg, Dep't of the West}
\reportdinkus

\gramHeader{United States Army} % {{{4
{Washington City, April}{28, 1862}
\gramTo{Maj. Gen.}{Karl Meyer}
{Commanding Department of the West}

\gramHi{General} Yrs of this date rcvd.

The planned raid to Corinth is enterprising, but I fear divesting yourself of
cavalry on the eve of possible battle in the vicinity of Humboldt. We know that
the fort at Pittsburg Landing is manned \& its guns are controlling the upper
Tennessee. Corinth itself may also have a small garrison. As we are unsure
whether or not the enemy will come at Humboldt from the south, \& we know Gen.
Thomson has a rather significant force of cavalry, I advise against committing
to that raid, unless it can be guaranteed they will return within just several
days, \& that Thomson \& the forces potentially coming up from the Tennessee \&
Ohio Railroad, i.e direction of Bolivar and Holly Springs, can be kept at bay
until the cavalry return. The distance to Corinth is some 75 miles by rail. You
may be without an entire division of cavalry for a full week, and they will
require recuperation for several days, or longer, thereafter.

Apprise me of the siege of Island Number 10 as soon as you are made aware of any
developments there.

\gramClosing{I remain very respectfully}
{C. N. Van Royne}
{Maj. gen. commanding USA}
\subsecdinkus

\subsection*{May 2, 1862}{} % {{{3

\gramOrdersHeader{Headquarters, Dep't of the Cumberland} % {{{4
{Cloverdale, Tenn., May}{2, 1862}
{Special Field Orders}{3}

The forces between the Cumberland and Tennessee Rivers will move to defeat the
enemy Army of Tennessee and capture the town of Charlotte, Tenn.

I. The cavalry corps of the Armies of the Cumberland and Kanawha, under the
overall command of Brig. Gen. Graham, will move through Jones' Cross Road and
cut the roads from Nashville to Jones' Cross Road and Charlotte. Gen'l Graham is
recommended to place his own corps astride the road to Charlotte but the exact
employment of the four divisions under his command is left to his own
initiative. If a sizable force approaches from Nashville, the cavalry is to
screen their movement to delay them reaching Charlotte. If the enemy retreats
from Charlotte, the cavalry is to pursue retreating forces.

II. Gen'l Steele moves his army through Jones' Cross Road and across Jones'
Creek to approach Charlotte from the flank. Unless presented with an opportunity
requiring his own initiative he will not attack unless ordered.

III. Gen'l Blake assumes temporary command of the Army of the Cumberland and
moves against Charlotte to fix the enemy in place. Once the enemy positions are
scouted an attack may be launched only on orders from this headquarters.

IV. Gen'l Wood moves his division from White Oak to Fort Defiance to begin a
siege of the enemy works. Once in position, he is to order the siege batteries
currently located at Camp Carter to join him via river and road movement. If it
is practicable to control or at least enemy the enemy use of the southern end of
the Clarksville bridge, Gen'l Wood is to do so. Mapping the occupied enemy
positions and determining the strength of the enemy garrison is of the highest
priority.

V. Gen'l Kellogg will move his cavalry division to White Oak to prevent enemy
movement toward Sailor's Rest, Mill Spring or Dover.

VI. Col. McMillan moves his brigade from Dover to Sailor's Rest to defend the
river landing there.

VII. Col. McHenry, Jr.'s brigade continues to hold the bridge at Crossing, ready
to destroy said bridge if the enemy threatens to capture it.

VIII. Maj. Mackay is to send forward supplies [114 LP] to establish a depot at
Cloverdale for the purpose of supplying the advancing cavalry. He will also send
forward a week's rations [75 LP] for the remainder of the Armies of the
Cumberland and Kanawha.

\gramClosingBy{Maj. Gen. James Blake}
{Walter Chekov}
{Col., Adjutant General}
\reportdinkus

\gramHeader{Headquarters, Dep't of the Cumberland} % {{{4
{Cloverdale, Tenn., May}{2, 1862}
\gramTo{Maj. Gen.}{Cornelius Van Royne}
{Commanding General, United States Army}

\gramHi{Sir} It is my honor to report that this headquarters has avenged the
losses of Barton's Creek. We have engaged and soundly defeated the Army of
Tennessee.

Our losses were 7,200 infantry, 900 cavalry and one battery of artillery, almost
all from the Army of the Cumberland. To that loss I must add Gen'l Fawcett whose
wounds will keep him out of action for a few months.

The enemy certainly suffered more loss, including 1,800 men and 54 guns
captured. The guns are being recovered to be sent rearward for repair.

The enemy has fallen back to Charlotte allowing Fort Defiance and its small
garrison to again be invested. I intend to do so while maneuvering to force the
enemy to withdraw all the way to Nashville.

\gramClosing{Respectfully}
{J. W. Blake}
{Maj. Gen., Commanding}
\reportdinkus

\subsecdinkus

\subsection*{May 5, 1862}{} % {{{3

\gramHeader{Headquarters, Dep't of the Cumberland} % {{{4
{Charlotte, Tenn., May}{5, 1862}
\gramTo{Maj. Gen.}{James Howard}
{Commanding Army of the Kentucky}

\gramHi{General} I am anxious to hear from you.

\gramClosing{Respectfully}
{J. W. Blake}
{Maj. Gen., Commanding}
\reportdinkus

\gramHeader{Headquarters, Army of the Kentucky} % {{{4
{Dripping Springs, Ky. May}{5, 1862}
\gramTo{Maj. Gen.}{James Blake}
{Commanding, Department of the Cumberland}

\gramHi{Sir} All remains quiet here. My quartermaster informed me that the
Bowling Green depot would cost twice my initial estimate as wagons would be
needed to feed other wagons and on. I remained merely patrolling out of Bowling
Green rather than reconnoitering in force.

We observed enemy cavalry near the Russellville and north Franklin burnt bridges
on the 29th of April, but by the 30th the enemy had withdrawn and we did not
regain them.

Work on the Bowling Green and Munfordsville rail bridges is nearing completion,
expected to complete on the 13th and 15th of May respectively.

\gramClosing{Yr Obt Svt}
{James Howard}
{Maj. Gen., Commanding, Army of the Kentucky}
\reportdinkus

\gramHeader{Headquarters, Dep't of the Cumberland} % {{{4
{Charlotte, Tenn., May}{5, 1862}
\gramTo{Maj. Gen.}{Cornelius Van Royne}
{Commanding General, United States Army}

\gramHi{Sir} The town of Charlotte has been seized, the enemy having withdrawn
before he could be trapped here. My cavalry is currently facing two divisions of
cavalry of the Army of Tennessee some distance to the southeast.

I am at the limits of my supply and can advance no further until Fort Defiance
is taken, barring a significant expenditure.

\gramClosing{Respectfully}
{J. W. Blake}
{Maj. Gen., Commanding}
\reportdinkus

\gramHeader{United States Army} % {{{4
{Washington City, May}{5, 1862}
\gramTo{Maj. Gen.}{James Blake}
{Commanding Department of the Cumberland}

\gramHi{General} Yours of the 2d and this date, the 5th, inst., just received.
Your victory is cause for great celebration. We are sad to hear that Gen'l
Fawcett has been wounded but entrust his command now to your able \& reliable
hands. Who, by seniority, now commands the Army of the Cumberland?

I believe that unless Defiance is held in great strength, it is imperative for
your armies and this nation that the works be stormed by grand assault, unless
you believe enemy resistance so great as to render any attempt, even if
successful, liable to wreck your commands. If a siege be necessary, you must
naturally make every effort to expedite its satisfactory conclusion.

With Jackson---as we must assume he still leads that army, with Anderson still
leading the Dept of the West---fleeing before you and probably falling on
Nashville, I think it feasible for you to bring Howard to you, overland with all
haste, to receive supply from the Cumberland, and vastly increase your own
strength. A small detachment can be left to hold Bowling Green, if necessary. I
will see if some portions of the garrisons of Louisville and Lexington can be
detached to hold Munfordville or bolster the defenses of Bowling Green. Let me
know your thoughts on this proposal, and whether you believe present
circumstances would allow it.

\gramClosing{I remain very respectfully}
{C. N. Van Royne}
{Maj. Gen., commanding USA}
\reportdinkus

\gramHeader{Headquarters, Dep't of the Cumberland} % {{{4
{Charlotte, Tenn., May}{5, 1862}
\gramTo{Maj. Gen.}{Cornelius Van Royne}
{Commanding General, United States Army}

\gramHi{Sir} In response to your last, Maj. Gen. Ptolemy Smith is the senior
corps commander in the Army of the Cumberland, but for the time being I intend
to command the army myself. With the possibility of Howard joining the armies
here and the heavy losses to Steele's army a reorganization may soon be in
order.

Gen'l Howard reports the enemy cavalry spotted south of Bowling Green had
departed by the 30th ult., prior to the battle at Cloverdale. As the enemy
cavalry strength around Charlotte has not increased, I expect they have been
sent elsewhere. It is possible the enemy, having long delayed Gen'l Howard is
intent to cede Bowling Green now that we are approaching Nashville. Howard
reports the necessary rail bridges will not be repaired until the middle of this
month and the bridges south of Bowling Green, destroyed as a defensive measure,
will not be repaired until after that.

The Army of the Kentucky can only join us here if Gen'l Howard abandons his
supply line and marches on what his army can carry. While doable, this would
leave Bowling Green defended by only what garrison you can provide or Gen'l
Howard leaves behind. How exactly to precede over the next week is still being
debated.

Col. McHenry, Jr., commanding the garrison at Crossing on the Tennessee River
reports his scouts have cleared the railroad through to McKenzie and there
contacted scouts from the Army of the Tennessee. The rail lines between
Columbus, Humboldt and Dover are now in our control. Although this headquarters
is not responsible for the portion of the rail to McKenzie, funds and units are
available to construct blockhouses from Crossing to McKenzie and garrison them
with up to two regiments of infantry.

\gramClosing{Respectfully}
{J. W. Blake}
{Maj. Gen., Commanding}
\reportdinkus

\gramHeader{Headquarters, Dep't of the Cumberland} % {{{4
{Charlotte, Tenn., May}{5, 1862}
\gramTo{Maj. Gen.}{James Howard}
{Commanding, Army of the Kentucky}

\gramHi{General} This headquarters intends, on Washington's orders, to shortly
launch an assault on Fort Defiance. Final orders for the coming week will be
forwarded shortly but will entail your army marching on what it can carry from
Dripping Springs to Clarksville. Do you have enough supply available to furnish
rations for this movement or do you require additional?

During your movement you will detach such force as necessary to rebuild the
Russellville bridge, leaving the Franklin bridge destroyed. Include necessary
supply for the repairs into your total request.

Washington intends to move forces forward to Dripping Springs to occupy Bowling
Green once the railroad supply line is established. I will send details on this
once I have them.

Please respond soonest.

\gramClosing{Respectfully}
{J. W. Blake}
{Maj. Gen., Commanding}
\reportdinkus

\gramHeader{Headquarters, Dep't of the Cumberland} % {{{4
{Charlotte, Tenn., May}{5, 1862}
\gramTo{Maj. Gen.}{Cornelius Van Royne}
{Commanding General, United States Army}

\gramHi{Sir} Per your orders I intend to fall back to Cloverdale, maintaining a
cavalry screen in front of Charlotte. Brig. Gen. Garfield will move to Fort
Defiance and join the forces already there to conduct an assault of the fort. He
will have the equivalent of two reinforced divisions plus the siege guns at his
disposal. As a full reconnaissance of the current defenses has not yet occurred,
Gen'l Garfield will be under orders to delay the assault if he feels he does not
have enough force available.

Gen'l Howard will move along the Memphis Branch railroad to Clarksville,
beginning repair of the Russellville bridge along his way. The current supply
line reaches only as far as Dripping Spring until all bridges east of Bowling
Green are repaired on about the 15th inst. Do you require Gen'l Howard to leave
any troops behind as temporary or permanent garrisons while reserves in eastern
Kentucky are moved into position?

\gramClosing{Respectfully}
{J. W. Blake}
{Maj. Gen., Commanding}
\reportdinkus

\gramHeader{Headquarters, Dep't of the Cumberland} % {{{4
{Charlotte, Tenn., May}{5, 1862}
\gramTo{Maj. Gen.}{Cornelius Van Royne}
{Commander, United States Army}

\gramHi{General} I have the honor to submit the following officers to be
considered for promotion:

{
    \centering
    \begin{dispatch}{
        colspec        = {l|l},
        cell{1}{1}     = {c},
        cell{1,3,5}{2} = {c},
    }

        \textit{To Major General of Volunteers} & \textit{To Brigadier General of Volunteers} \\
        Brig. Gen. Lawrence Graham              & Col. (Bvt. Brig. Gen.) John De Courcey \\
        Brig. Gen. Charles Smith \\
        Brig. Gen. John McClernand \\


        \textit{To Colonel of Volunteers:} & \textit{To Colonel of Regulars:} \\
        Lieut. Col. William Carroll        & Lieut. Col. John Kung \\
    \end{dispatch}
    \par
}

These officers have served faithfully in command positions commensurate with the
rank of their requested promotions. The officers requested for promotion to
Major General are senior corps commanders while Col. De Courcey commands a
division. I request that they be promoted with a date of rank of May 15, 1862.

\gramClosing{Respectfully}
{J. W. Blake}
{Maj. Gen., Commanding}
\reportdinkus

\gramHeader{Headquarters, Army of the Kentucky} % {{{4
{Dripping Springs Ky.May}{5, 1862}
\gramTo{Maj. Gen.}{James W. Blake}
{Commanding, Department of the Cumberland}

\gramHi{Sir} I must stress that this would be a maneuver of the extensive
difficulty and risk, and one unlikely to gain the advantage you are envisioning.
I might be able to push a cavalry corps into Clarksville in the manner
described, but not the army.

My main body is 85 miles or more from Clarksville near Dripping Springs and
Bells, Ky. and already at the extent of the range to which supplies offloaded from
rail in Munfordsville can be delivered effectively. I currently require a
moderate amount of additional supply to cover that required by my army.

If there were no opposition whatsoever, I might by dint of a forced march over
the entire week reach Clarksville. Or I might not. While my men are eager, it
would be foolish to treat recent volunteers as veterans of the Grande Armée. Even
18 mile days might prove unachievable often, let alone for five straight
marches.

This would bear a significant materiel cost to cover the foodstuffs and feed of
the army that week, or in short the whole of what I have plus additional
supplies. I cannot guess at the cost in men as precisely, but would not be
surprised to leave a third or more of the army straggling behind if I were to
force the march in the manner needed. All to likely arrive quite late to the
assault if it is this week.

Needless to say, at this point I would have a ragged, unsupplied and potentially
unsuppliable, unless the river is open, force at Clarksville or soon to arrive.
Or, should we have spent a monstrous amount of materiel, they would have one
week's sustenance on hand.

If you absolutely require a force there this week, I would recommend sending my
cavalry, whom I can afford to supply with two weeks provisions and might reach the
town this week unless the enemy interferes.

I should point out that the Munfordsville rail bridge will be repaired on the
15th, and Bowling Green the 13th, meaning that in just one more weeks time this
all might (other than the speed of the march), be accomplished fairly easily and
at little cost.

I would urge you to consider the full necessity of my occupying Clarksville
sometime in the next ten days and if it is worth the risk and expenditure.

\gramClosing{Yr obt svt}
{James Howard}
{Maj. Gen., Army of Kentucky, Commanding}

Copies to: Maj. Gen. Harold K. Fawcett, III \& Maj. Gen. Richard Steele

\reportdinkus

\gramHeader{Headquarters, Dep't of the Cumberland} % {{{4
{Charlotte, Tenn., May}{5, 1862}
\gramTo{Maj. Gen.}{James Howard}
{Commanding, Army of the Kentucky}

\gramHi{General} Your response is appreciated, however, I must belatedly update
you on the situation here. On the 2d inst., the Armies of the Cumberland and
Kanawha drove the enemy from Cloverdale and Charlotte. Although our losses were
significant the enemy's were certainly larger, including 1,800 men and 54 guns
captured. The enemy withdrew east of Chestnut Grove, towards Nashville, screened
by our cavalry. I intend to withdraw to my supply line at Cloverdale while the
cavalry remains east of Charlotte.

The garrison of Fort Defiance is reported to be significantly reduced and I
intend to assault it within the week, before the enemy can gather for a
counterattack. I regret to inform you that Gen'l Fawcett has been seriously
wounded and will spend several months recuperating; for the time being I will
assume control of both his army and the department.

The reason for the suggested movement to Clarksville is to consolidate our
forces for a final push to Nashville, although your army would be greatly needed
if the fort does not fall by the time you arrive.

The objections relayed in your last are of concern although on one point I am
able to reassure you. The depot at Sugar Creek, just across the Cumberland River
from Palmyra, Tenn., could be rebuilt prior to your arrival at Clarksville so
there should be no concern about your army being resupplied at that point. This
is how Gen'l Fawcett's army was supplied previously.

The amount of supply necessary is acceptable as a smaller amount was used here
than expected. More concerning is your expectation that your army will suffer
significant loss during the march. Do you believe this would be reduced if the
maneuver is not attempted until after the bridges are repaired? Waiting an
additional 10 days may by acceptable in that case.

\gramClosing{Respectfully}
{J. W. Blake}
{Maj. Gen., Commanding}
\reportdinkus

\gramHeader{United States Army} % {{{4
{Washington City, May}{5, 1862}
\gramTo{Maj. Gen.}{James Blake}
{Commanding Department of the Cumberland}

\gramHi{General} Yours this date just rcvd. I have forwarded the listed general
officers to the Congress for promotion in the United States Volunteers.

I have begun communications with the civil authorities regarding the possible
shifting of some troops assigned to Lexington and Louisville southward to
Munfordville. If released for that purpose, they can then be moved to Bowling
Green once the rail link to that city is restored.

If you believe a shift in the base of supply of the Army of the Kentucky is
possible, I urge you to begin preparations for it immediately with Gen. Howard,
but to only execute it once I can confirm the arrival of troops from northern
Kentucky. This force may not amount to more than two brigades of infantry,
several regiments of cavalry, and two or three batteries of artillery. Under
such circumstances, you should detach another small force of relatively equal
strength to bolster that force and place it in Munfordville; and then, once the
bridge over the Green is restored to working order, in Bowling Green. If this
movement is carried out, the combined force at Munfordville/Bowling Green should
be instructed to destroy all bridges in their path \& withdraw slowly up the
rail line toward Louisville if they are pressed by a vastly superior force. In
the future, this force can be given a corps designation within the Department of
the Cumberland if circumstances permit.

I am happy to report to you that Gen'ls Benton and Johanson, on my
recommendation, have been promoted to Major Generals of Volunteers, to rank from
April 11th, for their heroic services rendered at Barton's Creek. They, and
Gen'l Fawcett, will be returned to your department as soon as they have
recovered, to be dispensed with as you please.

\gramClosing{I remain very respectfully}
{C. N. Van Royne}
{Maj. Gen., commanding USA}
\reportdinkus

\gramHeader{Headquarters, Army of the Kentucky} % {{{4
{Dripping Springs, Ky., May}{5, 1862}
\gramTo{Maj. Gen.}{James W. Blake}
{Commanding, Department of the Cumberland}

\gramHi{Sir} First, congratulations on your victory. The nation sorely needs
one, and even the numerous combined forces of the enemy can scarce afford to
bleed in defeat often.

To practical matters, reaching Clarksville this week will require forcing the
march.  Right now the basic planning factor for a week is five days at ten miles
a day. Eighty five miles is a not inconsiderable amount beyond that; it can be
done with six fifteen mile days if the enemy elects not to interfere, though
the state of the army at that point will be in some question. I am not sure
anyone has attempted such a march yet this war, so we have few references to
compare to.

If you need me there at the most immediate opportunity and are certain there is
no chance of my being cut off, I can attempt the march and will expend some all
remaining supply, requiring additional from your stocks and then further support
by riverine.

But if the intent is simply a larger consolidation of forces, I would recommend
I start the march on either the 9th or the 12th. The latter option would include
as much rail movement as could be practically executed for preservation of the
force and expeditious movement and as such would likely arrive in Clarksville at
near the same time on the 19th. The question is whether the seven days a forced
march might gain would be worth the expense in materiel and readiness.

I await your decision on this matter.

\gramClosing{Yr Obt Svt}
{James Howard}
{Maj. Gen., Commanding, Army of Kentucky}
\reportdinkus


\gramHeader{Headquarters, Department of the West} % {{{4
{Humboldt, Tenn., May}{5, 1862}
\gramTo{Maj. Gen.}{Cornelius Van Royne}
{Commanding General of the United States Army}

\gramHi{General} The Army of the Tennessee has encamped at Humboldt with the
XVIIIth Corps at Gadsen, the 1st Cavalry Division at the crossing north of
Quincy, and the 2d Cavalry Division at Jackson. The 3d Cavalry Division newly
arrived from Gen'l Ambrose has taken McKenzie and the furthest scouts have met
Gen'l Blake's scouts at Paris, ensuring that we now have a secure route of
travel between our two departments.

The Army of the Arkansas has nothing new to report on their side of the
Mississippi. The siege batteries are up and firing, however they have yet to
discern any effect on the fortifications.

In regards to the enemy, more rebel cavalry from the Army of the
Trans-Mississippi has been spotted along the Southern Fork of the Deer River
from Cherryville to Jackson, estimated at two or three divisions in strength but
no firm numbers exactly. It may be possible that the cavalry force that Gen'l
Ambrose pushed from Cottonwood Point has now rejoined the Army of the
Trans-Mississippi.

I have yet to come into contact with Anderson's Corps. If they do not engage us
by this week, I would posit that Anderson's Corps has either joined Thomson at
Memphis, made their way to New Orleans to strengthen the defenses of that city,
or else moved east to reinforce the enemy in Virginia.

I will be continuing to press forwards toward Memphis down the Memphis and Ohio
R.R. General Caldwell reports that he cannot press an advance against Grand
Junction or Corinth without at least another corps, though we are discussing the
possibility of destroying bridges and track from Corinth to Jackson in order to
protect our flank as we move towards Memphis.

\gramClosing{I am always, Generals, and shall ever remain, Your most humble and obedient of servants} 
{Karl Meyer}
{Maj. Gen. cmdg, Dep't of the West}
\reportdinkus

\gramHeader{United States Army} % {{{4
{Washington City, May}{5, 1862}
\gramTo{Maj. Gen.}{Karl Meyer}
{Commanding Department of the West}

\gramHi{General} Yours this date just received.

Anderson is the commander of the enemy Dept of the West, encompassing what
appears to be all enemy forces in Kentucky and Tennessee. Gen. Blake reports the
enemy has completely abandoned all of Kentucky \& the Army of Tennessee has been
put to flight back to Nashville. Fort Defiance is to be invested and possibly
taken by assault.

Having struck a major blow in Middle Tennessee, it is indeed very likely the
enemy seeks to either strongly reinforce New Orleans, as they have likely
received word of the imminent attack there, or to reinforce Thomson, to envelop
you and relieve Island No. 10. General Garland's attempt to capture New Bern
ended disastrously, and it must be assumed that the enemy was somehow forewarned
of the attack there with less than week's advance notice.

The enemy will either turn the forces departing Nashville back around to
strengthen the city's defenses, or they will continue on to Thomson in an
attempt to strike a major blow against you.

I must ask you to report the remainder of the corps of your department for our
records, XVIIIth Corps exclusive. Which corps is at Humboldt with Gen. Caldwell?
Which is the corps with Gen. Ambrose?

\gramClosing{I remain very respectfully}
{C. N. Van Royne}
{Maj. Gen. Commanding USA}
\reportdinkus

\gramHeader{Headquarters, Department of the West} % {{{4
{Humboldt, Tenn., May}{5, 1862}
\gramTo{Maj. Gen.}{Cornelius Van Royne}
{Commanding General of the United States Army}

\gramHi{General} Yrs this date just rcvd. The XVIIIth Corps is currently at
Gadsen as part of Gen'l Caldwell's Army of the Tennessee. The XIXth Corps is
currently at Humboldt as part of Gen'l Caldwell's Army of the Tennessee. The
XVth Corps' 2d Divisions is at Trenton, moving to link up with Gen'l Caldwell as
part of his army.

Of the Cavalry Corps of the Army of the Tennessee, the 1st Division is at the
crossing north of Quincy, the 2d Division is at Jackson, the 3d is at McKenzie.

The XVth Corps' 1st Division is at New Madrid under the command of Gen'l
Ambrose's Army of the Arkansas. The Division of Observation is at Obionville,
besieging Is. 10 as part of Gen'l Ambrose's Army of the Arkansas.

Gen'l Ambrose's Cavalry Corps has been reduced to a single division of 6
regiments and a horse battery and is present at Riddle's Point.

Gen'l Caldwell and I have planned to move the Cavalry Corps of the Army down the
Tenn. \& Ohio R.R. in order to destroy the tracks south and east of Grand
Junction to prevent Anderson's link up with Thomson if that has yet to happen.
The Division of the Arkansas at Trenton will be following to occupy Grand
Junction the week after if they should prove successful.

Given the enemy has already built one fort at the crossing at Humboldt, it seems
possible that they are currently or have finished building forts at Grand
Junction and Corinth as well.

\gramClosing{I am always, Generals, and shall ever remain, Your most humble and obedient of servants} 
{Karl Meyer}
{Maj. Gen. cmdg, Dep't of the West}
\reportdinkus

\gramHeader{Headquarters, Dep't of the Cumberland} % {{{4
{Charlotte, Tenn., May}{5, 1862}
\gramTo{Maj. Gen.}{Cornelius Van Royne}
{Commanding General, United States Army}

\gramHi{Sir} In regards to your last concerning the movement of the Army of
Kentucky to the Fort Defiance area I have consulted with Gen'l Howard and my
Chief Quartermaster and propose the following:

Gen'l Howard immediately sends two divisions of VIIth Corps and one division of
XIth Corps north to Munfordville where they will move via rail and river to
Sailor's Rest. Upon arrival of troops from northern Kentucky, Gen'l Howard moves
with the remaining division of IXth Corps, cavalry and artillery reserve via the
same route to Sailor's Rest. The 3d Division of VIIth Corps is detached and
remains in the Bowling Green area.

The rail and river movement may take slightly longer than the direct overland
route to Clarksville but should limit the number of men lost to a forced march.
This also requires no additional expenditure of supplies.

\gramClosing{Respectfully}
{J. W. Blake}
{Maj. Gen., Commanding}
\reportdinkus

\gramHeader{Headquarters, Dep't of the Cumberland} % {{{4
{Charlotte, Tenn., May}{5, 1862}
\gramTo{Maj. Gen.}{Thomas Howard}
{Commanding, Army of the Kentucky}

\gramHi{General} I have consulted with Chief Quartermaster Maj. Mackay and
propose the following, to be confirmed in orders once the particulars are
established:

Instead of marching overland to Clarksville, your army will move to Sailor's
Rest by marching to the railhead at Munfordville and then moving via rail and
river. This movement may be slightly slower than a forced march overland but
should limit the number of men lost to a forced march. This also requires no
additional expenditure of supplies.

As you cannot entirely abandon your position prior to the arrival of additional
forces from northern Kentucky, I propose a staggered movement. 1st and 2d
Divisions of VIIth Corps and 1st Division of IXth Corps will being movement
immediately while you and the remainder of your army will remain in position
until additional forces arrive.

Brig. Gen. Thomas' division will remain behind and assume command of the
additional forces, responsible for eventually occupying and defending Bowling
Green. For the time being, Thomas' division will nominally remain part of VIIth
Corps although it will likely be split into a separate command in the near
future.

I do wish your opinion on sending your cavalry on the overland route so as to
ascertain the state of the bridge at Clarksville. If this were done, either a
depot would be constructed at Sugar Creek, or the cavalry would cross over to
Sailor's Rest from Sugar Creek.

\gramClosing{Respectfully}
{J. W. Blake}
{Maj. Gen., Commanding}
\reportdinkus

\gramHeader{Headquarters, Army of the Kentucky} % {{{4
{Dripping Springs, Ky., May}{5, 1862}
\gramTo{Maj. Gen.}{James W. Blake}
{Commanding, Dep't of the Cumberland}

\gramHi{Sir} I Acknowledge you wish me to march three divisions Munfordsville
(taking about two days), rail the remainder the roughly 65 miles to
Louisville---the locks along the Green are all destroyed---the take riverine
transport from Louisville down to the Ohio, to the Cumberland, and the onwards
to Sailor's Rest. For simplicity sake I will simply send VIIth Corps, which has
three divisions, while I retain IXth Corps with its two divisions.

The latter part of the journey will take 7--11 days based on the estimates of my
engineers, quartermasters, and commanders. In all I would expect to arrive at
Sailor's Rest somewhere between the 14th and the 18th of May. I will send VIIth
corps promptly unless ordered otherwise. I will order Brig. Gen. Thomas Clancy, the
commander, to report to you upon arrival.

I will order one division of my cavalry to attempt to advance to Clarksville if
relatively unopposed while the other continues to screen the eastern rail
approach from Nashville. I suspect that by the time VIIth Corps arrives to
reinforce you, I will be able to begin moving IXth Corps more directly into
position via rail, either to threaten Nashville if the enemy has been drawn west,
or to reinforce your final efforts.

\gramClosing{Yr Obt Svt}
{James Howard}
{Maj. Gen., Commanding, Army of the Kentucky}
\reportdinkus

\gramHeader{Headquarters, Dep't of the Cumberland} % {{{4
{Charlotte, Tenn., May}{5, 1862}
\gramTo{Maj. Gen.}{James Howard}
{Commanding, Army of the Kentucky}

\gramHi{General} Your adjustments to the proposal are approved. When the time
comes to move the remainder of your army, take whatever route is most
appropriate at the time, even if that requires expending additional supply. I
believe that you have enough available to you now that at least three divisions
are moving via rail and river.

\gramClosing{Respectfully}
{J. W. Blake}
{Maj. Gen., Commanding}

\gramHeader{United States Army} % {{{4
{Washington City, May}{6, 1862}
\gramTo{Maj. Gen.}{James Blake}
{Cmdg Dept of the Cumberland}

\gramHi{General} Yrs of the 5th inst. just received. To the matter of Gen'l
Howard \& the Army of the Kentucky, I have secured the redeployment of infantry
\& cavalry from the interior of Kentucky, plus batteries from Ohio, to wit eight
infantry regiments, two cavalry regiments, and two batteries of artillery. This
is sufficient for a small multi-arms division. This headquarters will provide
for its proper supply and all costs of maintenance. They have been approved to
garrison Munfordville but may not pass beyond that point. Any force Gen'l Howard
leaves behind should be instructed to fall on Munfordville \& combine with that
force, but they thereafter must maintain a purely defensive character \& posture
only; by which I mean only the garrison, redeployed to Munfordville, cannot be
moved without the permission of the War Dep't Gen'l Howard must leave behind at
least two reg'ts of cavalry with his division to provide some warning of an
enemy advance \& screening capability should they be forced to withdraw to the
Green. We do not need another repeat of Strasburg.

You must therefore begin Gen'l Howard's movement without delay, in the
speediest, most efficient, \& most prudent manner you believe possible. If this
is by the Louisville \& Nashville Railroad then a descent of the Ohio \& an
ascent of the Cumberland, then execute it at once. The Munfordville garrison
should arrive by about May 9th.

If the enemy is not pressing you immediately, \& you believe Defiance could
possibly be taken by storm, I encourage the commitment of as much of your force
as practical to ensure an immediate success \& to decrease the likelihood that
any single division of Gen'l Garfield's force suffers a disproportionate amount
of the enemy fire. If another corps, or even two, from either Gen'l Steele or
Gen'l Smith, can be spared for just several days, then the matter can be over
quickly, \& the path to Nashville will then be wide open. I even encourage you
to commit an entire army of yours if it is not impractical. Those troops can
then immediately be shifted back to your front. The enemy is almost assuredly in
no condition to test you in the field, at least over the next few days, after
having suffered so great a loss in strength at Barton's Creek and Cloverdale. In
any event, maintain a strong \& active screen of cavalry.

Should Defiance fall by assault, I should like to know if it can still be
garrisoned \& maintained for our own purposes. If it must be repaired \&
replenished of pieces, inform me also, and additional supply can be diverted to
ensure its completion without exhausting your own stocks. Be cautious that the
enemy, if they believe the fort lost \& not capable of being relieved, may
attempt an escape over or up the Cumberland, either by raft or makeshift bridge.

If you hope to reassign commands within your department once your three armies
are united, I must stipulate that all three armies persist, though their
composition \& assignment of corps may of course be altered as you see fit. Do
not displace Gen'l Steele, Smith, or Howard from command unless approval is
requested \& given. Inform this headquarters of any changes for our records.

Gen'l Garland's bloody repulse at New Bern is supplemented with news from Gen'l
McCarroll that he is making progress, however slow, in reducing the various
forts defending New Orleans. I have informed him that we believe at least some
portion of the force at Barton's Creek was redirected to his front. We believe
Gen'l Clarke is in overall command of the enemy forces in the region. The rest
of the forces who fought under Anderson at Barton's Creek, save the Army of
Tennessee, is still unknown, and could conceivably be in any corner of the
South.

\gramClosing{I remain very respectfully}
{C. N. Van Royne}
{Maj. Gen., commanding USA}
\subsecdinkus

\subsection*{May 6, 1862}{} % {{{3

\gramHeader{Headquarters, Dep't of the Cumberland} % {{{4
{Charlotte, Tenn., May}{6, 1862}
\gramTo{Maj. Gen.}{Cornelius Van Royne}
{Commanding General, United States Army}

\gramHi{Sir} In regards to your last regarding the movements of Gen'l Howard's
army, I will order the cavalry detachment to remain in Bowling Green in order to
set them far enough forward to give adequate warning as well as to maintain a
presence in the town. As soon as Fort Defiance falls, I will redirect forces to
open the rail road to Bowling Green.

The exact amount of force to be sent against Fort Defiance is still to be
determined but this headquarters will make every effort to bring that portion of
the campaign to a rapid close.

\gramClosing{Respectfully}
{J. W. Blake}
{Maj. Gen., Commanding}
\reportdinkus

\gramOrdersHeader{Headquarters, Dep't of the Cumberland} % {{{4
{Charlotte, Tenn., May}{6, 1862}
{Special Field Orders}{4}

This Department will consolidate and endeavor to capture Fort Defiance in
preparation for a move on Nashville in late Spring or early Summer.

I. Gen'l Howard moves his army, less a garrison at Bowling Green and
Munfordville, as follows: VIIth Corps immediately moves to Sailor's Rest via
rail and river transport. On arrival, Gen'l Clancy reports to this headquarters
for orders. IXth Corps and the artillery reserve withdraws to Munfordville and
awaits the arrival of garrison forces from northern Kentucky. Once said forces
arrive, Gen'l Howard moves with 1st Division, IXth Corps and the artillery
reserve to Sailor's Rest or Clarksville following the most expeditious route
available at the time. 2d Division, IXth Corps is detached and  remains at
Munfordville to bolster the garrison. The 2d Cavalry Division is also detached
and remains at Bowling Green to screen the approaches to Munfordville. 1st
Cavalry Division moves along the railroad to Clarksville and, on arrival,
requests new orders.

II. Brig. Gen. Graham retains temporary command of all cavalry east of Charlotte
and maintains the cavalry screen in front of the enemy. Brig. Gen. Kellogg's
division remains at White Oak.

III. Brig. Gen. Wood conducts aggressive reconnaissance of Fort Defiance to
determine the strength and disposition of the enemy defense.

IV. Gen'l Steele moves his army to Fort Defiance and, after consultation with
Gen'l Wood, conducts, if practical, an immediate assault to seize the fort.
McMillan's and McHenry, Jr.'s brigades from the Division of Observation have
been ordered in separate correspondence to move to Fort Defiance and join the
attack. Gen'l Steele need not wait for their arrival if he deems the additional
force unnecessary.

V. The Army of the Cumberland falls back to Cloverdale, establishing defensive
positions and reorganizing.

VI. In the event the enemy launches an attack on Cloverdale before Fort Defiance
falls, the general line of retreat for all forces will be through Yellow Creek
and Sailor's Rest towards Dover. Cloverdale must hold long enough to allow the
forces around Fort Defiance to withdraw from the neck.

VII. Orders for the Division of Observation are detailed separately.

\gramClosingBy{Maj. Gen. Blake}
{Walter Chekov}
{Col., Adjutant General}
\reportdinkus

\gramHeader{Headquarters, Dep't of the Cumberland} % {{{4
{Charlotte, Tenn., May}{6, 1862}
\gramTo{Maj. Gen.}{James Howard}
{Commanding, Army of the Kentucky}

\gramHi{Sir} To clarify the orders sent previously, your forces left behind at
Munfordville and Bowling Green after your departure are to fall under this
headquarters for the time being. Please inform this headquarters soonest of the
senior officer in command of that force.

\gramClosingBehalf{Man. Gen. Blake}
{Walter Chekov}
{Col., Adjutant General}
\reportdinkus

\gramOrdersHeader{Headquarters, Dep't of the Cumberland} % {{{4
{Charlotte, Tenn., May}{6, 1862}
{General Orders}{1}

The Division of Observation is reorganized as follows:

I. Col. McHenry, Jr. at Crossing is to detach the 58th Indiana and 17th Kentucky
Regiments to garrison the rail line from Crossing to McKenzie, Tenn.  The
remainder of the brigade will move to Fort Defiance and report to Brig. Gen.
Garfield.

II. Col. Kimball at Smithland is to detach the 11th Kentucky, 30th Ohio and 14th
Indiana Regiments and Battery F, 2d Illinois Light Artillery to form a new 4th
Brigade.

III. Col. McMillan at Sailor's Rest will move his brigade to Fort Defiance and
report to Brig. Gen. Garfield.

IV. Col. Pierce Hawkins is to command the new 4th Brigade. Once formed, move by
river to Dover and establish defensive positions.

\gramClosingBy{Maj. Gen. Blake}
{Walter Chekov}
{Col., Adjutant General}
\reportdinkus

\gramHeader{Headquarters, Dep't of the Cumberland} % {{{4
{Charlotte, Tenn., May}{6, 1862}
\gramTo{Maj.}{Andrew Mackay}
{Chief Quartermaster}

\gramHi{Maj} You are to send supplies forward for the construction of
blockhouses defending the rail line from Crossing to McKenzie, Tenn. 

\gramClosing{Respectfully}
{Walter Chekov}
{Col., Adjutant General}
\reportdinkus

\gramHeader{Headquarters, Dep't of the Cumberland} % {{{4
{Charlotte, Tenn., May}{6, 1862}
\gramTo{Cdre.}{Daniel Lewis}
{Commanding, Cumberland River Squadron}

\gramHi{Commodore} This headquarters intends to assault Fort Defiance within the
week.  The presence of one or two of your ironclads is requested. They are, of
course, not expected to engage the fort with its heavy guns but their mere
presence out of range may keep the enemy's attention. Secondarily, your
ironclads would be able to immediately conduct a patrol up river as soon as the
fort were taken, taking care to avoid an engagement with the enemy bastion known
to be present at Nashville.

\gramClosing{Respectfully}
{J. W. Blake}
{Maj. Gen., Commanding}

\subsecdinkus

\subsection*{May 9, 1862}{} % {{{3

\gramHeader{Headquarters, Dep't of the Cumberland} % {{{4
{Cloverdale, Tenn., May}{9, 1862}
\gramTo{Maj. Gen.}{Richard Steele}
{Commanding, Army of the Kanawha}

\gramHi{General} Your news of the fall of Fort Defiance is cause for
celebration. You, Gen'ls Milroy and Ryan and your men are to be congratulated.

I ask that, save troops needed to garrison the fort and the bridge, you send any
fresh troops, including any Dep't troops, to Cloverdale immediately while the
rest of your army rests.

I will order the Chief Quartermaster to send forward engineers and supplies to
repair the Clarksville bridge but see to what repairs you can in the meantime.

\gramClosing{Respectfully}
{J. W. Blake}
{Maj. Gen., Commanding}
\reportdinkus

\gramHeader{Headquarters, Dep't of the Cumberland} % {{{4
{Cloverdale, Tenn., May}{9, 1862}
\gramTo{Maj.}{Andrew Mackay}
{Chief Quartermaster}

\gramHi{Major} Gen'l Steele has this day captured Fort Defiance. You are
requested to immediately send forward engineers and supplies to begin repairing
the bridge at Clarksville.

\gramClosingBehalf{Maj. Gen. Blake}
{Geo. Campbell}
{Capt., Aide de Camp}
\subsecdinkus

\subsection*{May 12, 1862}{} % {{{3

\gramHeader{Headquarters, Dep't of the Cumberland} % {{{4
{Cloverdale, Tenn., May}{12, 1862}
\gramTo{Maj. Gen.}{James Howard}
{Commanding, Army of the Kentucky}

\gramHi{General} I am able to report that on the 9th inst., Gen'l Steele's Army
of the Kanawha captured Fort Defiance. The Cumberland River is now open to
Nashville. Your VIIth Corps has arrived at Sailor's Rest and your cavalry has
occupied the town of Clarksville.

If nothing has changed in your area, proceed with your previous orders, moving
the remainder of your forces here via whatever route is most advantageous.

Once our forces are concentrated here around Cloverdale, we can begin the
campaign to seize Nashville in earnest.

\gramClosing{Respectfully}
{J. W. Blake}
{Maj. Gen., Commanding}
\reportdinkus

\gramHeader{Headquarters, Army of the Kentucky} % {{{4
{Dripping Springs, Ky., May}{12, 1862}
\gramTo{Maj. Gen.}{James W. Blake}
{Commanding, Department of the Cumberland}

\gramHi{Sir} It gladdens me greatly to hear of our success at Fort Defiance. I
have confirmation from both 1st Cavalry Division and VIIth Corps of their
arrival and transfer under your command.

In the interim, my army has completed the blockhouse line out as far as bowling
green to secure it against partisans and raiders, and the Munfordsville and
Bowling Green rail bridges will be complete by the 15th. The Russellville rail
bridge will be complete by the 22d.

I will bring IXth Corps by rail to Russellville starting on the 15th, then march
to Clarksville. I expect to be assembled there by the 19th or 20th. I will
require 2d Cavalry, currently at Bowling Green, to continue to screen my
eastern flank and secure bridge repair. While not critical now, at some point we
will need to relieve pressure on river transport along the Cumberland and be
able to extend south of Nashville, so I intend to continue the blockhouse line
along the rails after we approach Nashville.

More immediately, upon arrival at Clarksville I can either advance along the
north bank, or cross to join you in the south. In the north I would require 1st
Cavalry Division to be returned for reconnoitering my advance. I also am in
possession of eight bridging trains; I intend to rail them to the department
once the Russellville bridge is functional.  The engineers are experienced after
last season, and this should allow us to rapidly shift multiple corps from one
bank or another during operations near Nashville.

\gramClosing{Yr Obt Svt}
{James Howard}
{Maj. Gen., Commanding, Army of the Kentucky}
\reportdinkus

\gramHeader{Headquarters, Dep't of the Cumberland} % {{{4
{Cloverdale, Tenn., May}{12, 1862}
\gramTo{Maj. Gen.}{Cornelius Van Royne}
{Commanding General, United States Army}

\gramHi{Sir} I have the honor to report that Fort Defiance fell to Gen'l
Steele's Army of the Kanawha on the 9th inst. A full report will follow but his
casualties in numbered 1,800 against 1,200 for the enemy, 900 of which were
captured.

Gen'l Howard's 1st Cavalry Division has occupied the town of Clarksville while
his VIIth Corps has arrived at Sailor's Rest. With both the Armies of the
Cumberland and Kanawha now concentrated at Cloverdale and the enemy withdrawn
closer to Nashville I intend to begin the initial movements against Nashville
this week while Gen'l Steele's forces rest.

\gramClosing{Respectfully}
{J. W. Blake}
{Maj. Gen., Commanding}
\reportdinkus

\gramHeader{Headquarters, Dep't of the West} % {{{4
{Bell, Tenn., May}{12, 1862}
\gramTo{Maj. Gen'ls}{Cornelius Van Royne \& James Blake}
{Commanding United States Army \& Dep't of the Cumberland, resp'y}

\gramHi{Generals} I write to you with urgent news. While setting out to raid
Corinth, my cavalry encountered advancing Confederate infantry from the Rebel
Army of the West moving towards Anderson's Store on May 8th. The troopers
spotted at least a Corps, most likely more marching up the roads before pulling
back to Mount Pine and burning the bridge there to slow down the enemy advance
which they estimate is about a day from Mt. Pine.

A second cavalry raid on Grand Junction succeeded in cutting every line and even
set fire to the bridge there and then withdrew to Hickory Valley and set up a
depot there.

As of now, the Army of the Tennessee has currently made contact and engaged the
Army of the Trans-Mississippi at the Battle of Holly Grove near the town of
Jones on the 7th of this instant. The army was able to seize the bridgehead
across the South Fork of the Forked Deer River and pushed the enemy force back
some distance. We were unable to break through the enemy though and fell back to
the bridge with the majority of the army currently encamped at Bell.

While we have taken 2,400 casualties during the battle and lost 6 Guns,  reports
from the men believe they inflicted more upon the enemy than what we took.
During the battle, we were able to identify elements of of at least three corps,
several divisions worth of cavalry, estimates 4--6,000, and roughly 15 batteries
though only seven were seen at any one time. In addition there was reported
enemy infantry presence at the crossing at Cherryville though they did not make
any movements throughout the week.

I will be ordering General Ambrose to bring his division at New Madrid down to
reinforce the Army of the Tennessee with all due haste, but I am certain that it
will not be enough. I am therefore requesting reinforcements from General
Blake's Department of the Cumberland to be sent down the Memphis, Clarksville,
and Louisville Railroad that was cleared last week.

I intend to make a stand at Mt. Pine with three of the divisions at Bell in
order to buy time for any possible reinforcements to arrive to the field. 

\gramClosing{I am always, General, and shall ever remain, Your most humble and obedient of servants}
{Karl Meyer}
{Maj. Gen. cmdg Dept. of the West}

P. S.---

General Ambrose has informed me that Island No. 10 has begun to take damage from
the siege guns with several fires spotted throughout the bombardment, however
they are still firing back. The New Madrid batteries will be completed June 1st,
which will officially cut them off.

However, the departure of the Division at New Madrid may leave the batteries
vulnerable, and so I would request that garrisons in Missouri be redeployed to
protect the town and batteries. I hope that this will not be an issue as the
garrisons will not be deployed outside of the state that they have already been
assigned to.

\reportdinkus

\gramHeader{United States Army} % {{{4
{Washington City, May}{12, 1862}
\gramTo{Maj. Gen.}{Karl Meyer}
{Commanding Department of the West}

\gramHi{General} Yrs this date just received.

I have received continual reports from our commanders in the field, perhaps at
every major engagement to this point, that the enemy has suffered greater loss
than us. Yet I am mystified that the enemy attacks us at nearly every point and
in many places---not yours---we are unable to advance more than a few miles at
most.

Several of my commanders, furthermore, seem to have taken a liking to giving
over to the enemy as many of their pieces as they deem physically possible, such
that one of the great strengths of our army, namely our artillery, which is
superior both in quality and quantity, is fast being eroded in the face of the
enemy. I say these things only to caution you against rashness, and to only
ensure you are relaying the most accurate information to me as possible.

Take care that the force at the crossing of the South Fork, whatever its
composition or strength, may find its path of retreat cut should the enemy corps
at Anderson's Store advance and push you out of Humboldt, along with the enemy
infantry at Cherryville advancing up to Quincy \& beyond. I am confident in your
ability to offer the stoutest imaginable resistance to the enemy, but you must
make provision to withdraw your whole command northward and forfeit Humboldt to
siege and storming if necessary. In that case the fortress guns, if they have
arrived, must be spiked, \& the powder stores destroyed if they cannot
comfortably \& speedily be taken with you.

Ensure that the detachment from Gen. Ambrose transits the Mississippi far enough
upriver from Is. 10 that the enemy cannot detect its movement. Ensure Ambrose's
cavalry conduct a vigorous demonstration to maintain the appearance of great
strength at New Madrid. I will investigate the possibility of deploying troops
to defend New Madrid, but cannot make any promises as to making the decision
itself.

I will communicate your request to General Blake, and see if he cannot release a
corps to you on a temporary basis. I hope that he will have sufficient strength
to assail Nashville's defenses while you beat off the advance of Thomson and his
colleagues.

\gramClosing{I remain very respectfully}
{C. N. Van Royne}
{Maj. Gen., Commanding USA}
\reportdinkus

\gramHeader{United States Army} % {{{4
{Washington City, May}{12, 1862}
\gramTo{Maj. Gen.}{James Blake}
{Commanding Department of the Cumberland}

\gramHi{General} We are happy to hear that Defiance is now ours.

It seems that after the battle at Barton's Creek, the enemy saw fit to weaken
himself at Nashville, and put some portion of his strength in West Tennessee.
General Meyer reports that at least a corps has been detached moving up from
Corinth to his position at Humboldt \& along the South Fork of the Forked Deer
River. The enemy seeks to dislodge him and throw him back upon the Mississippi,
thus relieving pressure on Memphis.

I now require of you two things:

You must move with celerity against Nashville. I realize your men must be tired,
and logistical constraints may tie your hands in some regard. But we must
continue the application of great \& even pressure against the enemy. With Gen.
Howard now joining you in force, I hope this may be achieved.

You must also inform me immediately if you believe that a corps-sized force can
be dispatched, without any delay whatsoever, down the Memphis, Clarksville, \&
Louisville Railroad to Humboldt, to be transferred to Gen. Meyer's command for
the duration of the present crisis in his department, and to be returned to you
at the first convenience for continued service in the Dep't of the Cumberland.
If such a movement is possible, you must execute such a movement as soon as you
make that determination, and thereafter inform me.

\gramClosing{I remain very respectfully}
{C. N. Van Royne}
{Maj. Gen., Commanding USA}
\reportdinkus

\gramHeader{Headquarters, Dep't of the West} % {{{4
{Bell, Tenn., May}{12, 1862}
\gramTo{Maj. Gen.}{Cornelius Van Royne}
{Commanding General, United States Army}

\gramHi{General} Yrs this date just rcvd. I would be pleased to report to you
that we have yet to have turned over any pieces to the enemy. Indeed I too am
somewhat skeptical that we as an attacking force were able to inflict more
casualties on the enemy while driving them off a hill. But yet these are the
reports my officers and the soldiers on the ground are giving me and I have
nothing to dispute these claims aside from my disbelief.

In regards to the crossing at South Fork, though it will set back progress
greatly, it may be prudent to destroy the bridge and fall back behind the river.
With the rail line at Grand Junction cut, Thomson will be forced to repair the
bridge at South Fork in order to join with the enemy army at Anderson's Store.
We will destroy the bridge at Jackson as well, destroying the depot and moving
behind the cover of the Middle Fork south of Humboldt in order to await
reinforcements from Gen'ls Ambrose and Blake.

I wish to also clarify that there is more than a corps approaching me.

I shall do my best to prevent the destruction of this army and give up land if
necessary to accomplish this goal. I would posit that the rebels have decided
that Gen'l Blake's concentration of forces are to significant to handle and have
decided to turn their attention upon me, as I have a much smaller force and am
dealing with Thomson.

I will relay your orders to Gen'l Ambrose, but I believe we cannot forfeit
Humboldt. This is not because of the supplies we have put into the fort, but
because of the fact that Gen'l Blake's forces would not be able to join mine
with any ease, if Humboldt is taken. Indeed the enemy would split our forces
apart.

\gramClosing{I am always, General, and shall ever remain, Your most humble and obedient of servants}
{Karl Meyer}
{Maj. Gen. cmdg Dept. of the West}
\reportdinkus

\gramHeader{United States Army} % {{{4
{Washington City, May}{12, 1862}
\gramTo{Maj. Gen.}{Karl Meyer}
{Commanding Department of the West}

\gramHi{General} Yours this date just received. The order to make preparations
to evacuate Humboldt was made with an eye toward ensuring you are ready to
extricate your force on the Fork should you have the earliest indications that
its line of retreat is under serious threat. The force at Cherryville indicates
to me that the enemy may intend to execute a double envelopment against that
force, from the northwest and southeast.

Delaying the enemy's advance from the southwest---Thomson's main body---is
prudent, and destroying the bridge may be necessary. It will also permit a
general concentration of your strength in the general vicinity of Humboldt. If
General Blake can speedily release a corps to you, do what you can to mask its
arrival by train at Humboldt with judicious screening by the main body of
Caldwell's army.

It is perhaps not necessary to emphasize this, but should it come down to it,
the loss of Humboldt would pale in comparison to the loss of your army.

\gramClosing{I remain very respectfully}
{C. N. Van Royne}
{Maj. Gen., Commanding USA}
\reportdinkus

\gramHeader{United States Army} % {{{4
{Washington City, May}{12, 1862}
\gramTo{Maj. Gen.}{Karl Meyer}
{Commanding Department of the West}

\gramHi{General} On the matter of the ``Army of the West'' under Anderson, I
wish to know if you can identify any of the units or commanders within this
force.  Are Hardee, Johnston, Bragg, etc among them? Is Anderson confirmed to be
with them? You claim it is more than one corps coming from Corinth. Is it two,
or can the number not be verified?

It is possible Thomson has been subordinated to Anderson. I cannot yet confirm
if Thomson's department remains independent of Anderson's.

\gramClosing{Very respectfully}
{C. N. Van Royne}
{Maj. Gen., Commanding USA}
\reportdinkus

\gramHeader{Headquarters, Dep't of the Cumberland} % {{{4
{Cloverdale, Tenn., May}{12, 1862}
\gramTo{Maj. Gen.}{Cornelius Van Royne}
{Commanding General, United States Army}

\gramHi{Sir} Gen'l Bragg was present at Cloverdale on the 2d inst. I do not know
if he has since moved his corps westward, although doing so would comport with
my belief that the forces defending Nashville are weaker than expected.

\gramClosing{Respectfully}
{J. W. Blake}
{Maj. Gen., Commanding}
\reportdinkus

\gramHeader{Headquarters, Dep't of the Cumberland} % {{{4
{Cloverdale, Tenn., May}{12, 1862}
\gramTo{Maj. Gen'ls}{Cornelius Van Royne \& Karl Meyer}
{Commanding United States Army \& Dep't of the West, resp'y}

\gramHi{Generals} In response to the request for additional forces for Gen'l
Meyer, I intend to immediately send two divisions Clancy's Corps, of the Army of
the Kentucky to Humboldt.

Gen'l Meyer, is your cavalry at Hickory Valley able to ride west to sever the
supply line of the enemy facing you at the South Fork? If possible this could
give you time to establish your positions while awaiting Howard's force.

I intend to immediately move the Army of the Cumberland against Nashville and
push the enemy back against the bastion there as far as possible. Gen'l Steele's
army will remain in camp for a week, resting after the battle at Fort Defiance
but then, if thought useful, could move up the Tennessee River to approach the
fort at Pittsburg Landing or to threaten the rear of the enemy should he have
advanced past Anderson's Store.

I await your response to these proposals.

\gramClosing{Respectfully}
{J. W. Blake}
{Maj. Gen., Commanding}
\reportdinkus

\gramHeader{Headquarters, Dep't of the West} % {{{4
{Bell, Tenn., May}{12, 1862}
\gramTo{Maj. Gen.}{Cornelius Van Royne}
{Commanding General of the United States Army}

\gramHi{General} Yrs this date just rcvd. The troopers of the 1st Division were
only able to establish that the force moving towards them looked like it was
larger than a corps. They were unable to identify any units or commanders save
for the fact that the force is at least in part the Army of the West. I am
equally unable to identify if Thomson is no longer independent of Anderson.

Regarding your earlier orders concerning the fort at Humboldt, I have been told
by my Chief of engineers that there is no way to demolish the fort that would
not take weeks of effort. Indeed even if I had started the moment that I had
captured the fort it would still be in the process of being demolished. The
heavy guns and ammunition have not arrived as of yet and thus do not need to be
removed or evacuated.

Gen'l Blake has informed me that he will be dispatching two divisions of the
Army of the Kentucky while also raising the possibility of landing a force
behind the Army of the West to threaten their rear.

While the possibility is tantalizing, I am unsure of if it would be possible to
coordinate such an attack with Gen'l Steele. Such a landing would threaten the
enemy, though it could also leave Gen'l Steele pinned against Thomson and this
new Army if I am unable to support him. A more cautious measure would simply be
to have him come down the rail line and reinforce the existing forces on the
Middle River.

\gramClosing{I am always, General, and shall ever remain, Your most humble and obedient of servants}
{Karl Meyer}
{Maj. Gen. cmdg Dept. of the West}
\reportdinkus

\gramHeader{Headquarters, Dep't of the West} % {{{4
{Bell, Tenn., May}{12, 1862}
\gramTo{Maj. Gen.}{James Blake}
{Commanding Department of the Cumberland}

\gramHi{General} Yrs this date just rcvd. I am most grateful for the support and
would ask if you are aware how long it will take for these forces to arrive at
Humboldt, as well as how many regiments and of what type these divisions entail.

As for severing the supply lines of Gen'l Thompson, there is a possibility,
though not a sure one of accomplishing just that. The 2d Cavalry Division of
the Army of the Tennessee is currently stationed at Hickory Valley, having just
cut the communication and rail lines at Grand Junction and then establishing a
small depot.

It may be possible for them ride to Loosahatchie river and destroy the bridge
there, cutting off supply by rail to the Army of the Trans-Mississippi. Though
this is assuming that there are no significant forces patrolling the rail line.

As for the possibility of Gen'l Steele landing at Crump's Landing or Coffee
Lodge, I am unsure of if it would be possible to coordinate such an attack with
Gen'l Steele with this maneuver, especially since the bridge at Mt. Pine has
been destroyed to delay the enemy and we are uncertain of how long it would take
to fix it.

I can assure you that by next week the enemy will most definitely be past
Anderson's Store and may even be on the south bank of the Middle River. I offer
the more cautious approach of having Gen'l Steele arrive to Humboldt in the same
fashion as Gen'l Clancy's Corps, which would allow us to concentrate in force
and avoid defeat in detail.

It is clear that the route to Memphis has been barred. I can only hope that the
movement of enemy forces from your department to mine will allow you to take
Nashville with the same vigor by which you have taken Fort Defiance.

\gramClosing{I am always, General, and shall ever remain, Your most humble and obedient of servants}
{Karl Meyer}
{Maj. Gen. cmdg Dept. of the West}
\reportdinkus

\gramHeader{Headquarters, Dep't of the Cumberland} % {{{4
{Cloverdale, Tenn., May}{12, 1862}
\gramTo{Maj. Gen.}{Karl Meyer}
{Commanding Dep't of the West}

\gramHi{General} The two divisions of VIIth Corps number 24 regiments and 6
batteries in total. They are presently on the rail line so should arrive within
a few days

Any movement by General Steele will not take place for another week so we have
time to determine the best plan.

\gramClosing{Respectfully}
{J. W. Blake}
{Maj. Gen., Commanding}
\reportdinkus

\gramHeader{Headquarters, Dep't of the West} % {{{4
{Bell, Tenn., May}{12, 1862}
\gramTo{Maj. Gen.}{Cornelius Van Royne}
{Commanding General of the United States Army}

\gramHi{General} I humbly wish to request if any progress has been made on the
possibility of deploying troops to defend New Madrid as Gen'l Ambrose intends to
bring the Cavalry Division stationed on the west bank across to join our forces
at Humboldt.

I have told the Gen'l to keep his cavalry around New Madrid for the time being
as we cannot afford to risk the river batteries being constructed there. Nor can
we afford to risk the siege guns, as he intended to pull the Division of
Observation away from Obionville as well.

I have ordered Gen'l Ambrose that both these elements are to remain to guard
their respective heavy guns. The former until a replacement garrison arrives at
New Madrid. The later indefinitely until Is. 10 surrenders.

\gramClosing{I am always, General, and shall ever remain, Your most humble and obedient of servants}
{Karl Meyer}
{Maj. Gen. cmdg Dept. of the West}
\reportdinkus

\gramHeader{United States Army} % {{{4
{Washington City, May}{12, 1862}
\gramTo{Maj. Gen.}{Karl Meyer}
{Commanding Department of the West}

\gramHi{General} I have received yours of this date. I have made a proposal that
a small division be detached for garrison duty at New Madrid, but in the
mean-time you must in any case make all movements therefrom to ensure the
survival of your forces at Humboldt. I will continue working to establish some
defensive force at New Madrid. You must otherwise maintain the siege of Island
No. 10 with the Division of Observation.

\gramClosing{Very respectfully}
{C. N. Van Royne}
{Maj. Gen., commanding USA}
\reportdinkus

\gramOrdersHeader{Headquarters, Dep't of the Cumberland} % {{{4
{Cloverdale, Tenn., May}{12, 1862}
{General Order}{1}

Effective immediately, corps and army commanders will no longer directly lead
troops in combat except in dire emergencies. Division commanders are to limit
their presence near the front line as much as possible.

Casualties among senior officers have been far too common over the past few
months and are not sustainable.

\gramClosingBy{Maj. Gen. Blake}
{Walter Chekov}
{Col., Adjutant General}
\subsecdinkus

\subsection*{May 15, 1862}{} % {{{3

\gramHeader{Headquarters, Dep't of the Cumberland} % {{{4
{In the field east of Chestnut Grove, Tenn., May}{15, 1862}
\gramTo{Brig. Gen.}{James Garfield}
{Commanding, XVIth Corps}

\gramHi{General} Move your corps towards Nashville to identify and, if possible,
invest any fortifications there. Do not engage any forts without orders.

\gramClosing{Respectfully}
{J. W. Blake}
{Maj. Gen., Commanding}
\reportdinkus

\gramHeader{Headquarters, Dep't of the Cumberland} % {{{4
{In the field east of Chestnut Grove, Tenn., May}{15, 1862}
\gramTo{Maj. Gen.}{Lawrence Graham}
{Commanding, Army of the Cumberland Cavalry}

\gramHi{General} Continue driving in the enemy cavalry screen but avoid a
pitched battle.

\gramClosing{Respectfully}
{J. W. Blake}
{Maj. Gen., Commanding}
\reportdinkus

\gramHeader{Headquarters, Dep't of the Cumberland} % {{{4
{In the field east of Chestnut Grove, Tenn., May}{15, 1862}
\gramTo{Brig. Gen.}{William Kellogg}
{Commanding, Dep't of the Cumberland Cavalry}

\gramHi{General} The enemy cavalry has withdrawn south from Chestnut Grove and
may be in your line of approach. If this is the case, you are to scout the enemy
positions and ascertain if you can cut the rail line farther south. Preservation
of your force is more important than cutting the rail line at this time.

\gramClosing{Respectfully}
{J. W. Blake}
{Maj. Gen., Commanding}
\reportdinkus

\gramOrdersHeader{Headquarters, Dep't of the Cumberland} % {{{4
{In the field near Brentwood, Tenn., May}{17, 1862}
{Special Field Orders}{6}

The Army of the Cumberland will establish defensive positions along the Harpeth
River to prevent relief of Nashville.

I. Brig. Gen. Kellogg will push south of the enemy if able and attempt to cut
the rail line south of Franklin. Once complete, return to screen the army's
right flank.

II. Maj. Gen. Graham's cavalry will scout the enemy lines as much as possible.

III. If Brig. Gen. Garfield does not require his Second Division to continue
Fort Nashville, said division will rejoin the army.

IV. The remainder of the Army of the Cumberland will defend north of the Harpeth
River.

\gramClosingBy{Maj. Gen. Blake}
{Walter Chekov}
{Col., Adjutant General}
\reportdinkus

\gramHeader{Headquarters, Dep't of the Cumberland} % {{{4
{In the field near Brentwood, Tenn., May}{17, 1862}
\gramTo{Maj. Gen.}{Richard Steele}
{Commanding, Army of Kanawha}

\gramHi{General} Your presence is requested at Gen'l Blake's headquarters. Leave
your army and attached forces in the command of Gen'l Ryan and have him prepare
to march here on orders

\gramClosingBehalf{Maj. Gen. Blake}
{Geo. Campbell}
{Capt., Aide de Camp}
\reportdinkus

\subsection*{May 19, 1862}{} % {{{3

\gramHeader{Headquarters, Dep't of the West} % {{{4
{North Bank of Middle River Fork, Deer River, Tenn., May}{19, 1862}
\gramTo{Maj. Gen'ls}{Cornelius Van Royne \& James Blake}
{Commanding, United States Army \& Dep't of the Cumberland, resp'y}

\gramHi{Generals} I write to you with tentative optimism. The Army of the
Tennessee was able to withdraw in good order back to Humboldt, destroying the
rail bridge at Bell, and the road bridges north of Quincy, north of Carroll, and
south of South Carroll. Scouts reported no movement from Gen’l Thomson and his
army on the south bank of the South Fork. By the 15th, the VIIth Corps had
arrived at Humboldt in its entirety and was stationed as a reserve along with
the Deutsch Division, the XVIIIth Corps was stationed at Gadsden, and the XIXth
Corps was stationed at the rail bridge north of Carroll. 

That evening, the Army of the West, estimated at a couple corps, made camp on
the south bank of the Middle Fork. Orders were given to the XVIIIth and XIXth
Corps to construct fieldworks and to ready themselves for an attack across
either rail bridge. The VIIth Corps and the Deutsch Division were to act as a
reserve, with priority placed against the bridge north of Carroll. 

On the morning of the 19th, the Army of the West began a two hour long
bombardment against the XIXth Corps and their positions at the bridge north of
Carroll, after which they launched a full attack. The men of the XIXth beat back
two assaults with only a couple hundred killed or wounded. During this time, a
second battle took place at the crossing at Gadsden which was also beaten back
with little issues. Estimated place enemy’s casualties from the two battles at
1,200. Word from prisoners are that Vexley’s Wing attacked at Gadsden while
Johnson’s attacked XIXth Corps. 

Our scouts reported that after the battle the enemy was not seen across from
either position and has fallen back to Jackson. They were unable to give anymore
information as a heavy enemy cavalry presence was encountered. This evening,
Gen’l Ambrose and the 1st Division of the Army of the Arkansas arrived at
Humboldt, giving the Department two fresh corps at hand.

With no sign of Gen’l Thomson during the attack, Gen’l Ambrose has suggested
that the Army of the Tennessee and the Army of the Arkansas---that being the 1st
Division, the Deutsch Division now returned to status as the 2d Division, and
the VIIth Corps---should press an attack against the Army of the West while their
backs are to the South Fork and they are suspected to remain separated from the
Army of the Trans-Mississippi. However since only two corps were engaged this
morning, I suspect there is at least another fresh corps back at Jackson in
reserve.

I await any orders or recommendations that you may have.

\gramClosing{I am always, General, and shall ever remain, Your most humble and obedient of servants}
{Karl Meyer}
{Maj. Gen. cmdg Dept. of the West}
\reportdinkus

\gramHeader{Headquarters, Dep't of the Cumberland} % {{{4
{In the field north of Franklin, Tenn., May}{19, 1862}
\gramTo{Maj. Gen.}{Karl Meyer}
{Commanding, Dep't of the West}

\gramHi{General} You are to be congratulated on your victory at the Deer River.
Is the presence of VIIth Corps still required or can they be returned here to
Gen'l Howard's command?

\gramClosing{Respectfully}
{J. W. Blake}
{Maj. Gen., Commanding}
\reportdinkus

\gramHeader{Headquarters, Dep't of the Cumberland} % {{{4
{In the field north of Franklin, Tenn., May}{19, 1862}
\gramTo{Maj. Gen.}{Cornelius Van Royne}
{Commanding General, United States Army}

\gramHi{Sir} It is my honor to present to you the city of Nashville. While the
enemy bastion remains in rebel hands, I have credible reports that it is
garrisoned by less than a brigade of troops the enemy having abandoned the city
entirely when we advanced towards it.

Governor Harris has also abandoned the city, contributing greatly to the poor
morale of its inhabitants who are neither happy with our occupation or the rapid
retreat of their own army.

Gen'l Howard has arrived at Clarksville with the balance of his army although,
with the detachment of forces to remain behind and of VIIth Corps to aid Gen'l
Meyer, he commands merely a single corps of infantry and a small cavalry
division.

My focus for now is to consolidate our hold on Nashville to prevent its relief
by the enemy and the capture by force or surrender of the rebel bastion.

\gramClosing{Respectfully}
{J. W. Blake}
{Maj. Gen., Commanding}
\reportdinkus

\gramHeader{Headquarters, Dep't of the West} % {{{4
{North Bank of Middle River Fork, Deer River, Tenn., May}{19, 1862}
\gramTo{Maj. Gen'ls}{James Blake}
{Commanding, Dep't of the Cumberland}

\gramHi{General} Yrs this date just rcvd. I will humbly convey your
congratulations to the men of the Army of the Tennessee, the Arkansas, and the
Kentucky who made this victory possible. If it would not prevent the securing of
Nashville and its bastion, I would keep the VIIth Corps present in the
Department of the West.

This is for reasons twofold. The first being the fact that the VIIth and XVth
Corps are the only fresh units in the Department as of this moment, with the
Army of the Tennessee exhausted from two months of marching and now two battles,
and at least six different enemy corps present in the department. Three under
Thomson and three under the Army of the West.

The second reason is the one mentioned in the last telegram, that of the
possibility that the Armies of the Tennessee and Arkansas will fall upon what we
hope is the isolated three corps of the Army of the West and inflict serious
casualties before they are reinforced by Gen’l Thomson. A defanged enemy would
allow for the present reinforcements to be transferred back to their prior
department without fear of an enemy counter attack.

\gramClosing{I am always, General, and shall ever remain, Your most humble and obedient of servants}
{Karl Meyer}
{Maj. Gen. cmdg Dept. of the West}
\reportdinkus

\gramHeader{United States Army} % {{{4
{Washington City, May}{19, 1862}
\gramTo{Maj. Gen.}{Karl Meyer}
{Commanding Department of the West}

\gramHi{General} Yrs of this date just rcvd. Gladdened to hear of your success
at Carroll \& the Middle Fork. Offensive operations under such conditions viz.
your comparative lack of cavalry are necessarily hazardous. Have there been no
reports whatever on your right regarding the presence of Thomson, who was last
seen at Brownsville \& that approximate area? What is the approximate strength
\& organization, if any can be provided at all, of Anderson's force?

\gramClosing{Very respectfully \&c \&c}
{C. N. Van Royne}
{Maj. Gen., Commanding USA}
\reportdinkus

\gramHeader{Headquarters, Dep't of the Cumberland} % {{{4
{In the field north of Franklin, Tenn., May}{19, 1862}
\gramTo{Maj. Gen.}{Cornelius Van Royne}
{Commanding General, United States Army}

\gramHi{Sir} Unfortunately I must report that, contrary to your orders, Gen'l
Howard left only a cavalry division in Bowling Green; an infantry division was
not left behind in Munfordville as intended. This oversight is entirely my own
fault as my orders to Gen'l Howard were not clear enough.

If necessary, I can order a division be sent at once from Clarksville to
Munfordville. However, I am not sure this is necessary as with our capture of
Nashville, the enemy should find it difficult to mount more than a cavalry raid
into central Kentucky. As it happens, a force of enemy cavalry (perhaps a
division) was reported to have crossed into Nashville from the north on its way
south of the city. I expect this to be the last of the enemy troops north of the
Cumberland River.

\gramClosing{Respectfully}
{J. W. Blake}
{Maj. Gen., Commanding}
\reportdinkus

\gramHeader{Headquarters, Dep't of the Cumberland} % {{{4
{In the field near Franklin, Tenn., May}{19, 1862}
\gramTo{Maj. Gen.}{Cornelius Van Royne}
{Commanding General, United States Army}

\gramHi{Sir} As an addendum to my last, I am pleased to report that the hulks of
six scuttled rebel gunboats (not ironclads however) were found at Nashville.
Outside the range of the fort's guns I believe we control the entirety of the
Cumberland River

\gramClosing{Respectfully}
{J. W. Blake}
{Maj. Gen., Commanding}
\reportdinkus

\gramHeader{United States Army} % {{{4
{Washington City, May}{19, 1862}
\gramTo{Maj. Gen.}{James Blake}
{Commanding Department of the Cumberland}

\gramHi{General} I am in receipt of yours several of this date. The double-blow
of the capture of Defiance and of Nashville are great blows made in the name of
Liberty and Union. I have the privilege to inform you that you have been
recommended to the Congress for the rank of brigadier general of Regulars, to
date from the 16th inst.

The mistake in the orders to Gen'l Howard are, I believe, of little concern.
With Nashville now ours, the enemy cannot support any force in Kentucky except
by plunder and pillage. I will see if I can detach part of the Munfordville
force to cover Bowling Green. You may otherwise prepare and, when ready, issue
orders to have Gen'l Howard's detached cavalry force to your main body.

I again say that you should consider whether storming the fort at Nashville
could yield the profit of a quick and relatively bloodless victory. It will save
us precious time and permit you, once your men have rested, to commence
operations against the enemy again wholesale. If such a thing is not possible,
an investment \& active siege will have to suffice.

I furthermore have two questions, and wish to know:

Whether Defiance's guns are operable, and whether you believe the fort, even
with some minor repairs, could be used for our own purposes; the identity of the
force you fought at Cloverdale, and its commander.

See to it that the men of your command know how happy \& fortunate I am that
they serve this, the country \& government of their forefathers.

To ease problems of communication \& coordination, you may feel free to dissolve
the Army of the Kentucky should you desire, and move Gen'l Howard to the command
of the Army of the Cumberland. Do not view this as an order.

\gramClosing{I remain very respectfully}
{C. N. Van Royne}
{Maj. Gen. Commanding USA}
\reportdinkus

\gramHeader{Headquarters, Dep't of the Cumberland} % {{{4
{Nashville, Tenn., May}{19, 1862}
\gramTo{}{Senior officer}
{Commanding the garrison of the fort at Nashville}

\gramHi{Sir} You and your men are to be commended for remaining at your posts
while Nashville was abandoned by the armies of the Confederacy and the civilian
government of the State of Tennessee. However, the city is entirely in the hands
of my armies and, with no relief forthcoming, your forces cannot hold. In order
to avoid unnecessary bloodshed, the immediate surrender of the fort and its
garrison is demanded.

Do not allow your men to suffer the same fate as the garrisons of Forts Henry
and Defiance which were also abandoned and left to defend themselves against
hopeless odds.

I offer no terms; I expect nothing less than unconditional surrender. You have
until dawn tomorrow to accept this offer.

\gramClosing{Respectfully}
{J. W. Blake}
{Maj. Gen., Commanding}
\reportdinkus

\gramHeader{Headquarters, Fort Nashbrough} % {{{4
{Fort Nashbrough, Tenn., May}{19, 1862}
\gramTo{Maj. Gen.}{J. W. Blake}
{Commanding Dep't of the Cumberland}

\gramHi{General} I have received your communication of this date demanding the
immediate and unconditional surrender of this fort and its garrison. I thank you
for the courtesy with which you acknowledge the conduct of my officers and men
in remaining at their posts after the evacuation of Nashville, and I assure you
that such conduct was guided by duty rather than expectation of relief.

I am fully sensible of the condition of affairs surrounding this position. The
city of Nashville is in your possession, no Confederate force remains in
supporting distance, and further resistance here can serve no military purpose
commensurate with the loss of life it would entail. I am not disposed to
sacrifice my command in a hopeless defense, nor to subject the men under me to
destruction when no advantage to the Confederate cause could result.

I therefore propose the following, in the interest of humanity and order. If
allowed two days' time to place matters in proper condition, secure public and
private property within the fort, and prepare the command for surrender, and if
my officers and men are granted parole to return within the Confederate lines, I
will then deliver Fort Nashbrough and its armament to your forces without
resistance.

This request is made not to delay the inevitable, but to ensure that the
surrender is conducted with discipline and without confusion, and that the
garrison may be spared the needless hardship of imprisonment when its further
service in this position is plainly at an end.

Should these terms be unacceptable, I will take such course as honor and duty
require of me under the circumstances.

\gramClosing{I am, General, Very respectfully, Your obedient servant}
{Thomas Hill}
{Colonel, C.S.A., Commanding Fort Nashbrough}
\reportdinkus

\gramHeader{Headquarters, Dep't of the Cumberland} % {{{4
{Nashville, Tenn., May}{19, 1862}
\gramTo{Col.}{Thomas Hill}
{Commanding Fort Nashbrough}

\gramHi{Sir} Your response is received. You have until tomorrow morning to
surrender the fort. No other terms will be offered.

\gramClosingBehalf{Maj. Gen. Blake}
{Geo. Campbell}
{Capt., Aide de Camp}
\reportdinkus

\gramHeader{Headquarters, Fort Nashbrough} % {{{4
{Fort Nashbrough, Tenn., May}{19, 1862}
\gramTo{Capt.}{George Campbell}
{Dep't of the Cumberland}

\gramHi{Sir} Your communication is received. If my officers and men are granted
parole, I will surrender Fort Nashbrough and its garrison tomorrow morning,
without resistance.

\gramClosing{Very respectfully, Your obedient servant}
{Thomas Hill}
{Colonel, C.S.A., Commanding Fort Nashbrough}
\reportdinkus

\gramHeader{Headquarters, Dep't of the Cumberland} % {{{4
{Nashville, Tenn., May}{19, 1862}
\gramTo{Col.}{Thomas Hill}
{Commanding Fort Nashbrough}

\gramHi{Sir} If the fort is surrendered without resistance by 7~o'clock
tomorrow, parole will be granted.

\gramClosingBehalf{Maj. Gen. Blake}
{Geo. Campbell}
{Capt., Aide de Camp}
\reportdinkus

\gramHeader{Headquarters, Dep't of the Cumberland} % {{{4
{In the field near Franklin, Tenn., May}{20, 1862}
\gramTo{Maj. Gen.}{Cornelius Van Royne}
{Commanding General, United States Army}

\gramHi{Sir} It is my privilege to report to you that, as of 7 o'clock this
morning, Brig. Gen. Garfield accepted the surrender of Fort Nashbrough. The
commander, Col. Thomas Hill, and the garrison of 1,200 men will be paroled
shortly. The fort was captured intact along with eight batteries of guns.

With Nashville, the fort and the bridges across the Cumberland River secure, I
see no reason to deliberately attack the enemy position at Franklin. With your
approval, I intend to consolidate my Department into the Armies of the Kanawha
and Cumberland (commanded as suggested by Gen'l Howard). The Army of the
Cumberland will defend Nashville and attempt to drive the enemy farther south
while the Army of the Kanawha will prepare for operations up the Tennessee River
against Pittsburg Landing and Corinth. I anticipate it will take about a week
until I am prepared to make any movements other than directly south from
Nashville.

\gramClosing{Respectfully}
{J. W. Blake}
{Maj. Gen., Commanding}
\subsecdinkus

\subsection*{May 20, 1862}{} % {{{3

\gramHeader{Headquarters, Dep't of the Cumberland} % {{{4
{In the field near Franklin, Tenn., May}{20, 1862}
\gramTo{Maj. Gen.}{Cornelius Van Royne}
{Commanding General, United States Army}

\gramHi{Sir} It is my privilege to report to you that, as of 7 o'clock this
morning, Brig. Gen. Garfield accepted the surrender of Fort Nashbrough. The
commander, Col. Thomas Hill, and the garrison of 1,200 men will be paroled
shortly. The fort was captured intact along with eight batteries of guns.

With Nashville, the fort and the bridges across the Cumberland River secure, I
see no reason to deliberately attack the enemy position at Franklin. With your
approval, I intend to consolidate my Department into the Armies of the Kanawha
and Cumberland (commanded as suggested by Gen'l Howard). The Army of the
Cumberland will defend Nashville and attempt to drive the enemy farther south
while the Army of the Kanawha will prepare for operations up the Tennessee River
against Pittsburg Landing and Corinth. I anticipate it will take about a week
until I am prepared to make any movements other than directly south from
Nashville.

\gramClosing{Respectfully}
{J. W. Blake}
{Maj. Gen., Commanding}
\reportdinkus

\gramHeader{Headquarters, Dep't of the Cumberland} % {{{4
{In the field near Franklin, Tenn., May}{20, 1862}
\gramTo{Maj. Gen.}{Cornelius Van Royne}
{Commanding General, United States Army}

\gramHi{Sir} The enemy spiked the guns at Fort Defiance but the walls are still
in good shape although the fort, overall, is a bit battered.

\gramClosing{Respectfully}
{J. W. Blake}
{Maj. Gen., Commanding}
\reportdinkus

\gramHeader{United States Army} % {{{4
{Washington City, May}{20, 1862}
\gramTo{Maj. Gen.}{James Blake}
{Commanding Department of the Cumberland}

\gramHi{General} Received yours of this date; it is a most joyous birthday gift.
I will begin examining whether we can afford to immediately restore Defiance to
its former strength.

General Meyer reports that he has repulsed Anderson in a minor engagement to his
south, but that Thomson at this moment is completely unaccounted for. I fear
that if Gen. Steele is sent up the Tennessee to strike Pittsburg Landing \&
threaten Corinth, Anderson may very easily fall on him and leave some detachment
to his rear--perhaps Thomson--to hold him at bay long enough for him to attack
Steele \& attempt to throw him into the river. I have requested that Gen. Meyer
provide me an estimate, if at all possible, on Anderson's strength. Thomson was
last estimated in excess of 25 thousand men. The enemy rail down the length of
the Tennessee \& Ohio, in the direction of Holly Springs Miss., has been damaged
to some extent by our cavalry. The line to Corinth is therefore Anderson's only
route of supply. I have no doubt he will act aggressively to protect it. Steele
must therefore be sufficiently strengthened such that he will be able to repel
an assault in excess of 60 thousand soldiers of the enemy. This assumes that
Steele is also able to debark his entire army before the enemy is able to move
against him.

All of this is to say that I first wish to hear how you or Gen. Steele plan to
react to a violent movement by Anderson against him.

Should you wish to rename Defiance I should like to know what you propose. I
should also like to know what you similarly suggest for Fort Nashborough.

Your request to consolidate your dept as proposed is approved forthwith. As soon
as you are able, provide abstract returns on the assignments \& strengths of the
various corps of your two armies.

\gramClosing{I remain very respectfully}
{C. N. Van Royne}
{Maj. Gen., Commanding USA}
\reportdinkus

\gramHeader{Headquarters, Dep't of the West} % {{{4
{North Bank of Middle Fork Deer River, Tenn., May}{19, 1862}
\gramTo{Maj. Gen.}{Cornelius Van Royne}
{Commanding General of the United States Army}

\gramHi{General} Yrs this date just rcvd. I can confirm that there have been no
reports on my right flank regarding Thomson. In fact it was the Army of the West
which stuck my right flank with Vexley's Wing at Gadsden.  As for their
strength, our forces were only able to make out that the enemy have a couple
corps, two of which attacked this morning.

Gen'l Caldwell has advised that we rest the army behind the Middle Fork River
for this week and instead send out a probing raid of seven regiments of cavalry
down the backroads to Corinth with the intent of raiding what is presumably
their depot or else destroying rail to interdict the supply of the Army of the
West.

From their current position at Spain's, the round trip journey would be 150
miles and an estimated 10 days, and expend a small amount of supplies. The raid
if unobstructed would arrive at Corinth roughly on the 25th or 26th instant.

Regarding Gen'l Blake's offer of having Gen'l Steele attacking Corinth, I
believe it is risky, but given the fall of Nashville, Corinth and Memphis are
the next obvious targets. Should this attack be carried out, I will order both
armies under my command to advance in order to pin down the Army of the West,
hopefully only leaving Thomson's Army of the Trans-Mississippi as a threat.

The combined forces under this department's command excluding the raiding forces
would amount to 55,800 infantry, 33 batteries of artillery, and 6,900 troopers.
Such a force moving towards Jackson where the enemy was last spotted, and
further if they do not destroy the bridge, would hopefully prevent them from
attacking Gen'l Steele though this is not certain.

\gramClosing{I am always, General, and shall ever remain, Your most humble and obedient of servants}
{Karl Meyer}
{Maj. Gen. cmdg Dep't of the West}
\reportdinkus

\gramHeader{Headquarters, Dep't of the Cumberland} % {{{4
{Nashville, Tenn., May}{20, 1862}
\gramTo{Maj. Gen.}{Cornelius Van Royne}
{Commanding General, United States Army}

\gramHi{Sir} After reorganizing, the Army of the Kanawha will number more than
45,000 and the Army of the Cumberland will number over 50,000. (More detailed
returns will be sent separately.)

I do not believe either army alone could withstand an attack by 60,000 of the
enemy. A corps of about 13,000 could be transferred from Gen'l Howard to Gen'l
Steele if a move Corinth is your priority although that may limit Howard's
ability to push south from Nashville.

Any move up the Tennessee will likely take more than a week to develop so there
is time to consider our options.

\gramClosing{Respectfully}
{J. W. Blake}
{Maj. Gen., Commanding}
\reportdinkus

\gramHeader{United States Army} % {{{4
{Washington City, May}{20, 1862}
\gramTo{Maj. Gen'ls}{James Blake \& Karl Meyer}
{Commanding Dep'ts of the Cumberland \& West, resp'y}

\gramHi{Generals} A crossroads in the Western strategy is before us, and I wish
to hear from you on how it should be met.

On the one hand, the enemy has endured significant losses of late at Barton's
Creek and Cloverdale. He has furthermore been repulsed at Carroll with some
loss, although light. Nashville has fallen, as have the forts \& garrisons of
Defiance and Nashborough. Island No. 10 is put to siege and likely soon to fall.

Yet on the other, the enemy still possesses great strength. He will be under
great pressure by the political authorities of the rebellion to act immediately,
restore faith in the army, reform their line in the West, and score a victory.
He may tend to boldness, and seek to execute a series of aggressive combinations
which we will not expect.

Their commanders Jackson and Thomson have shown themselves to be at least of
some ability. Anderson, leading all forces west of the Appalachians, has shown
himself to be aggressive. He marshaled great strength to attack us at Barton's
Creek and lift the siege of Defiance \& Whisper's army therein. He then swung
west with most of the Nashville army, leaving Jackson at Nashville, and
evidently wished to combine with Thomson against Gen. Caldwell at Humboldt.

The enemy has several courses which they may attempt, both aggressive and
defensive. He may unite further before Meyer with Jackson's army and attempt to
drive him out of Humboldt, thus securing the path to Memphis. He may stay in his
present position on all fronts, awaiting reinforcement and resting his men. He
may seek again to combine in Middle Tennessee in a drive on Nashville. He may
count on the exhaustion of our armies, and shift some of his strength eastward
to bolster McMullen in the fight against Gen. Adams in Virginia. He may even
seek to strike north into Kentucky, following the course of the Tennessee or
moving through the Cumberland Gap.

I wish to know the answer to several questions:

1. Whether the armies of the Kanawha \& the Cumberland are fit for immediate
   service, or could profit greatly with one week's rest;

2. Whether Gen. Meyer believes he could apply significant pressure toward
   Corinth in cooperation with an ascent of the Tennessee by General Steele,
   whereby he can attack Anderson's rear if Anderson turns to attack Steele's
   landing;

3. An estimate, if possible, of the time left to reduce Island No. 10 to
   surrender or destruction.

I am willing also to hear further suggestions for the next several weeks. I did
not anticipate that the enemy would forfeit Nashville without a battle at the
gates of the city itself. If that is the case, Jackson's army may be weaker than
first surmised.

\gramClosing{I remain very respectfully}
{C. N. Van Royne}
{Maj. Gen., Commanding USA}
\reportdinkus

\gramHeader{Headquarters, Dep't of the Cumberland} % {{{4
{Nashville, Tenn., May}{20, 1862}
\gramTo{Maj. Gen'ls}{Cornelius Van Royne \& Karl Keyer}
{Commanding United States Army \& Dep't of the West, resp'y}

\gramHi{Generals} The Army of the Cumberland and half the Army of the Kanawha
are ready for immediate action. The remainder of the Army of the Kanawha could
begin operations immediately but would benefit from an additional week to
reorganize.

\gramClosing{Respectfully}
{J. W. Blake}
{Maj. Gen., Commanding}
\reportdinkus

\gramHeader{Headquarters, Dep't of the West} % {{{4
{North Bank of Middle Fork Deer River, Tenn., May}{20, 1862}
\gramTo{Maj. Gen.}{Cornelius Van Royne}
{Commanding General of the United States Army}

\gramHi{General} Yrs this date just rcvd. I believe the department is able to
apply pressure towards Corinth, though the bridge at Mt. Pine would have to be
repaired and further bridges might be dropped by the enemy in order to delay our
advance.

In regards to Island No. 10, the bastion has taken noticeable damage, but is
currently holding. We are currently at the start of the fourth week of the
bombardment, with the river batteries at New Madrid being completed on the first
prox. By my conservative estimates, the bastion will not capitulate until the
end of this campaign season at the very soonest.

Gen'l Caldwell also raises concerns about Gen'l Steele landing near the bastion
at Pittsburg Landing so far from support and if his own line of supply might be
interdicted by the garrison sallying out. He has inquired as to the possibility
of the Department of the Cumberland moving south down the rail line to Decatur
before approaching Corinth from the east. This would also put the department in
the position of moving west and striking at Chattanooga if so desired.

\gramClosing{I am always, General, and shall ever remain,,Your most humble and obedient of servants}
{Karl Meyer}
{Maj. Gen. cmdg Dep't of the West}
\reportdinkus

\gramHeader{Headquarters, Dep't of the Cumberland} % {{{4
{Nashville, Tenn., May}{20, 1862}
\gramTo{Maj. Gen.}{Cornelius Van Royne}
{Commanding General, United States Army}

\gramHi{Sir} I present three options for the operations of this Department over
the coming weeks for your consideration.

I. The Army of the Kanawha with 54,00 infantry, 6,000 cavalry troopers and 38
batteries of artillery organized into five corps of infantry and one corps of
cavalry moves up the Tennessee River within the next two weeks and lands down
river of the enemy fort at Pittsburg Landing. Gen'l Steele's orders will be to
make moves towards Corinth of the enemy fort as he deems appropriate in order to
draw the enemy away from Gen'l Meyer.

II. The Army of the Kanawha with 55,200 infantry, 6,000 cavalry troopers and 39
batteries of artillery organized into four corps of infantry and one corps of
cavalry will move to join Gen'l Meyer within the next two weeks. The two
divisions numbering 14,400 infantry and six batteries currently detached to
Gen'l Meyer will rejoin Gen'l Steele. His orders will be to assist Gen'l Meyer
in driving on Memphis and Corinth.

III. The Armies of the Kanawha and Cumberland will move south from Nashville
intending towards Athens and Decatur, Ga. This movement would indirectly assist
Gen'l Meyer as the enemy could only halt the advance by assigning significant
force to the defense of northern Georgia.

The first option of moving on Corinth carries the greatest risk of defeat in
detail but success would significantly hamper the enemy's ability to resist
Gen'l Meyer's advance.

The option to directly reinforce Gen'l Meyer is the safest and does the most to
guarantee his success, although with only a single route of advance the enemy
will be able to concentrate.

The third option carries less risk as well but has the advantage of possibly
being unexpected, allowing an advance that is lightly opposed

I await your guidance.

\gramClosing{Respectfully}
{J. W. Blake}
{Maj. Gen., Commanding}
\reportdinkus

\gramHeader{Headquarters, Dep't of the Cumberland} % {{{4
{Nashville, Tenn., May}{20, 1862}
\gramTo{Maj.}{Andrew Mackay}
{Chief Quartermaster}

\gramHi{Major} Please assign engineers and necessary equipment to repair the
Nashville R.R. bridge at Franklin, Ky. and to begin the construction of
blockhouses along the Memphis, Clarksville \& Louisville R.R. between Crossing
and Clarksville, Tenn.

Updates are requested on the estimated dates that the R.R.. bridge at
Russellville, Ky. and the bridge and train depot at Clarksville, Tenn. will be
completed.

Lastly, ensure that a proper river depot is constructed at Nashville.

\gramClosing{Your servant}
{Walter Chekov}
{Col., Adjutant General}
\reportdinkus

\gramHeader{Headquarters, Quartermaster’s Department} % {{{4
{Smithland, Ky., May}{20, 1862}
\gramTo{Col.}{Walter Chekov}
{Adjutant General, Department of the Cumberland}

\gramHi{Colonel} Your communication of the 20th instant is received. Engineers
and the necessary tools and materials have been assigned to the repair of the
Nashville Railroad bridge at Franklin, Kentucky, and work will proceed without
delay. Completion of that bridge is now estimated for May 24th.

Engineer details have also been directed to commence construction of blockhouses
along the Memphis, Clarksville \& Louisville Railroad between Crossing and
Clarksville, Tennessee, in accordance with your instructions.

With respect to the works you are tracking: the bridge at Russellville,
Kentucky, is expected to be completed by May 22d. The bridge and train depot at
Clarksville, Tennessee, are now scheduled for completion on June 30th.

Further progress reports will be forwarded as these works advance.

\gramClosing{Very respectfully, Your obedient servant}
{Andrew Mackay}
{Major and Chief Quartermaster, Department of the Cumberland}
\reportdinkus

\gramHeader{United States Army} % {{{4
{Washington City, May}{20, 1862}
\gramTo{Maj. Gen.}{Karl Meyer}
{Commanding, Department of the West}

\gramHi{General} If unmolested, how quickly do you believe that you could effect
repairs to the bridges at Carroll, and what would it cost your dept in supplies?
Respond immediately.

\gramClosing{I remain very respectfully \&c}
{C. N. Van Royne}
{Maj. Gen., commanding USA}
\reportdinkus

\gramHeader{Headquarters, Dep't of the West} % {{{4
{North Bank of Middle Fork Deer River, Tenn., May}{20, 1862}
\gramTo{Maj. Gen.}{Cornelius Van Royne}
{Commanding General of the United States Army}

\gramHi{General} Yrs this date just rcvd. If unmolested it would take one day to
repair the road bridge at Carroll and cost one supply, which we are intending to
do. The rail bridge at Carroll was not destroyed. We are also repairing the road
bridge south of Spain's which will take four days and cost three supplies. The
Gadsen rail bridge is not destroyed and there are no reports of any enemies on
the other side. The Quincy road bridge will take four days to repair and cost
three supplies.

\gramClosing{I am always, General, and shall ever remain, Your most humble and obedient of servants}
{Karl Meyer}
{Maj. Gen., cmdg Dep't of the West}

P.S.---

I wish to add that the rail bridges at Bell and Mt. Pine are also destroyed,
though I do not have any idea how long it will take to rebuild them, nor the
supplies necessary in their construction.

\reportdinkus

\gramHeader{United States Army} % {{{4
{Washington City, May}{20, 1862}
\gramTo{Maj. Gen.}{James Blake \& Karl Meyer}
{Commanding, Departments of the Cumberland \& West, resp'y}

\gramHi{Generals} I am in receipt of yours both of this date. You must forgive
the length of the following message, but the both of you must read it in its
entirety.

A movement of sufficient strength down the rail to Decatur would be a
significant expense. We must use the Tennessee whenever and wherever we can.

Furthermore, Fort Defiance is hereby named Fort Benton, and Fort Nashborough
is hereby named Fort Fawcett. Garfield will remain inoperative for the time
being.

General Blake: I wish for you to dispatch another division by rail immediately
for Humboldt, to bring VIIIth Corps up to a strength of three divisions. Detach
it from whichever corps or body of your department you deem fit.

General Howard should probably demonstrate strongly southward toward Franklin,
primarily with his cavalry, while he rests his men \& recuperates. Gen'l Jackson
must not be permitted to give you the slip, even if with just a portion of his
force. If Jackson is found to have weakened himself or relocated any part of his
command, General Meyer must be notified immediately.

General Steele must embark with his army and make for Coffee Landing or the
landing just north of Pittsburg, debark with all haste, reduce \& capture the
fort at Pittsburg, and take Corinth. If he believes the fort can be stormed,
such a decision must naturally be made. He must also be prepared to embark his
army once more and descend the Tennessee if the situation proves unworkable or
if he believes he cannot hold the landing in the face of an assault. He should
expect to no longer be able to communicate with dept headquarters by cable until
he can establish communications with Gen'l Meyer. It must be stressed to Steele
that he moves without any delay whatsoever. The objective is the destruction of
the Pittsburg fort and the seizure of Corinth. Gen'l Steele must be informed that
he is to aid in the destruction of Anderson's army if the realistic opportunity
presents, but is also informed that the likeliest enemy course of action is to
immediately fall on his army while it is still on the banks of the Tennessee.

Gen'l Meyer: you are to receive an extra division from Gen'l Blake, which is to be
attached to VIIth Corps. This will bring its strength to three divisions. VIIth
Corps and some small detachment of cavalry is to be posted at Gadsden, with
cavalry ranging ahead and screening, in the event Thomson returns from Memphis,
if indeed that is where he is. Once Gen'l Steele executes his landing north of
Pittsburg, you must be prepared to have Gen'l Caldwell immediately give chase to
Anderson, should he weaken or altogether abandon your front while racing to
protect Corinth or to attack Steele.

The Carroll bridges should be repaired so that you can quickly move across the
Middle Fork and supply yourself by rail. Should this happen, your moves must be
prudent yet rapid. You must press Anderson all the way, and not allow him to
attack Steele's landing in the manner he desires. If he quits entirely and
instead chooses to defend Corinth, that is the preferable outcome. If Steele is
assailed, you are obligated to provide him whatever assistance you can to
prevent his destruction.

For the duration of the operation, as Gen'l Steele will be operating within the
boundaries of the Dept of the West, he must answer to Gen'l Meyer. I expect that
this transfer of authority will not last more than several weeks.

If you must, during the pursuit, Caldwell's army can shift more onto his left,
and thereby draw supply from the Tennessee, which will lessen the strain on your
dept supply by rail via Humboldt.

Caldwell must begin to press Anderson, and aggressively so, on Steele's
scheduled landing date. Anderson must be put in an uncomfortable position to
make the decision of withdrawing back to Corinth and possibly attacking Steele.
If Anderson chooses to deploy to hold Caldwell at bay, Caldwell need not attack,
but cavalry scouting must be judicious to determine if Anderson is still there
in strength or if he has left a rear-guard to delay him.

Furthermore: are you certain Is. No. 10 will not sue prior to July? If so it
must be mightily well-provisioned for the siege, or otherwise its garrison is
terribly weak \& can last for quite a while without much in the way of rations.
If recent events have provided any information, I believe it the enemy policy to
erect many forts, and to then garrison them minimally. This was the case with
forts Henry, Donelson, Defiance, and Nashborough, excepting those times when
Whisper's army was holed up inside. I encourage you to tell Gen'l Ambrose to make
a determination, with that knowledge in hand, of whether or not he believes he
can storm the fort. If he believes he can, it must be done without delay.

If it can be established definitively that Thomson is with Anderson and not
actually shielding Memphis---I do not believe it likely but it is a
possibility---VIIth Corps may rejoin Gen'l Caldwell's army to provide him
additional strength in pressing Anderson.

When we prove successful, the Army of the Kanawha will be returned to the
jurisdiction of the Department of the Cumberland for continued operations in
Tennessee and against Alabama and Georgia.

I will release an additional shipment of supply to the Department of the West
immediately.

\gramClosing{I remain very respectfully \&c}
{C. N. Van Royne}
{Maj. Gen., commanding USA}
\reportdinkus

\gramHeader{Headquarters, Dep't of the West} % {{{4
{North Bank of Middle Fork Deer River, Tenn., May}{20, 1862}
\gramTo{Maj. Gen.}{Cornelius Van Royne}
{Commanding General of the United States Army}

\gramHi{General} Yrs this date just rcvd. I understand that you wish for Gen'l
Caldwell to press Anderson just before or on the day of Gen'l Steele's landing.
I would like to inquire as to the date of this landing or if it is yet to be
confirmed.

In regards to the VIIth Corps, I would ask if it is merely to serve as a
blocking force at Gadsden, with the infantry not ranging further down the
Memphis and Ohio Railroad. If this is the case, I believe it would be more
prudent to leave one of Gen'l Caldwell's Corps behind, given that the Army of
the Tennessee is the only portion of the Department of the West to have seen
combat while the VIIth Corps and the XV Corps are if not fresh, then rested.

Finally, in regards to Is. No. 10, the main obstruction with the taking of the
bastion is that it is an island on the Mississippi River. Unlike Forts Henry,
Donelson, DeRussy, Defiance, and Nashborough I cannot storm the fort by land.
If such an action were to be taken, my men would need to row to the island which
is fully encircled by breastworks, under fire the entire time, while fighting
the river current.

Needless to say, with these kind of obstacles, a garrison of merely 1,200 might
repulse an attacking force of a much larger size with ease.

That river is the source of the garrison's provisions, allowing rebel ships to
travel up to the fort and then downriver without fear of interdiction. As of
yet, they have not been fully cut off from supply.

For now, the siege batteries and the Mississippi River Squadron continue to
pound away at the defenses, even as gunboats are damaged and sailed up river to
repair at St. Louis. I expect that once the New Madrid river batteries are
finished, the garrison will begin to starve as their last line of supply is cut.
From there, it should only take a month at most to see them run out of food at
the very least, if not ammunition.

\gramClosing{I am always, General, and shall ever remain, Your most humble and obedient of servants}
{Karl Meyer}
{Maj. Gen., cmdg Dep't of the West}
\reportdinkus

\gramHeader{United States Army} % {{{4
{Washington City, May}{20, 1862}
\gramTo{Maj. Gen.}{Karl Meyer}
{Commanding, Department of the West}

\gramHi{General} Yours this date just rcvd. VIIth Corps was involved in the
combats involving the Army of the Kentucky in February and March. They have seen
more action than some units of your own department. Their purpose should indeed
be as a blocking force. Once Corinth is taken--and the seizure of that place
should take precedence over the fort at Pittsburg Landing--you may secure lines
of supply and communication from Corinth and the Tennessee, and then repair back
to Humboldt with the balance of your command, then press on to Memphis. I do not
believe Thomson has abandoned the defense of Memphis. I think he has drawn in
closer to the city in order to better protect his line of supply, and to lessen
his logistical burden.

I suggest an assault on Island No. 10 only insofar as it may be possible to
construct a causeway, a la Alexander at Tyre, or otherwise initiate a night raid
to gain a point on the parapet. If the suggestion leads to a reply in the
affirmative by Gen. Ambrose then it should be done. Otherwise I continue to
entrust the siege to his care.

On the note of Gen. Steele's landing, should it be misconstrued in the future, I
wish for Gen. Caldwell to press and not lose contact with Anderson. Gen. Steele
will likely be subjected to great pressure. It will be your job to assist him.

As for the expected arrival date of Steele on the banks of the Tennessee, such
an estimate must be provided by General Blake.

\gramClosing{I remain very respectfully \&c}
{C. N. Van Royne}
{Maj. Gen., commanding USA}
\reportdinkus

\gramHeader{Headquarters, Dep't of the Cumberland} % {{{4
{Nashville, Tenn., May}{20, 1862}
\gramTo{Maj. Gen'ls}{Cornelius Van Royne \& Karl Meyer}
{Commanding United States Army \& Dep't of the West, resp'y}

\gramHi{Generals} Orders will be given immediately for 3d Division to
immediately move to rejoin VIIth Corps

Gen'ls Steele and Howard will consolidate their armies into a single command as
quickly as able while I will retain temporary command of the Army of the
Cumberland and endeavor to drive Jackson south from Nashville.

I will shortly report the date Gen'l Steele expects to land but I expect it will
be at least a week as his army needs a few days to consolidate and a week's
rest would ensure his command enters the field fresh. It is currently still
recovering from the fighting at Fort Benton.

\gramClosing{Respectfully}
{J. W. Blake}
{Maj. Gen., Commanding}
\reportdinkus

\gramHeader{Headquarters, Dep't of the West} % {{{4
{North Bank of Middle Fork Deer River, Tenn., May}{20, 1862}
\gramTo{Maj. Gen.}{Cornelius Van Royne}
{Commanding General of the United States Army}

\gramHi{General} Yrs this date just rcvd. I have passed on your suggestions to
my Chief of Engineers, Col. Flad. Unfortunately the Col. has informed me that
creating a land bridge is not feasible in that it would most likely take two
years to accomplish the task.

Likewise I have held council with my various division and brigade commanders who
have come to the consensus that a night raid would be infeasible as the men
simply do not have the amphibious training needed to accomplish such a task and
the river is once again very strong.

If there is a desire to increase the speed of the reduction of the fort, then it
may be possible to take the supplies you have given me and use them to transport
a further ten batteries of siege artillery to bombard the island over the next
month. This would almost triple the amount of columbiads present at the siege to
fourteen batteries, and would increase the damage being done to the fort. The
river batteries at New Madrid will be complete in a week and a half, at which
point all supply will be cut off to the fort.

I have forwarded your other orders to my army commanders and have told them that
next week they are to advance until making contact with Anderson’s main body and
to take hold of the enemy and not to let him escape us once we have him in our
grasp.

\gramClosing{I am always, General, and shall ever remain, Your most humble and obedient of servants}
{Karl Meyer}
{Maj. Gen., cmdg Dep't of the West}
\reportdinkus

\gramOrdersHeader{Headquarters, Dep't of the Cumberland} % {{{4
{Nashville, Tenn., May}{20, 18620}
{General Orders}{8}

The forces of this Department are immediately reorganized as summarized below:

I. Maj. Gen. James Howard assumes command of the Army of the Cumberland
organized into VIIth, XIVth and XVIth Corps, commanded by Maj. Gen'ls P. Smith
\& McClernand and Brig. Gen. Garfield, resp'y, supported by Maj. Gen. Graham's
Corps of Cavalry and Col. Cotter's Artillery Reserve.

II. The Army of the Kanawha, commanded by Maj. Gen. Richard Steele, is organized
into VIIth (detached), IXth, XIth, XIIth and XVIIth Corps, commanded by Brig.
Gen'ls Clancy, Gates, Ryan, C. Smith and Milroy, resp'y, supported by Brig. Gen.
Hall's Corps of Cavalry and the Artillery Reserve formerly assigned to the Army
of the Kentucky.

III. Brig. Gen. Hall's Corps of Cavalry will absorb the 1st \& 2d Divisions of
Cavalry formerly of the Army of Kentucky to be reorganized as ordered by Gen'l
Steele.

IV. Brig. Gen. Thomas Wood's Division, Brig. Gen. William Kellogg's Division of
Cavalry and Brig. Gen. Curran Pope's Division of Observation remain assigned to
this Department.

\gramClosingBy{Maj. Gen. Blake}
{Walter Chekov}
{Col., Adjutant General}
\reportdinkus

\gramOrdersHeader{Headquarters, Dep't of the Cumberland} % {{{4
{Nashville, Tenn., May}{20, 1862}
{Special Field Orders}{7}

This Department intends to maintain contact with the enemy Army of Tennessee
while assisting the Dep't of the West by moving up the Tennessee River so as to
threaten or take Corinth and the enemy fort at Pittsburg Landing.

I. The Army of the Kanawha will, within two weeks, move by boat to Coffee
Landing on the Tennessee River and make efforts to seize the fort at Pittsburg
Landing and capture the town of Corinth. Gen'l Steele will withdraw immediately
if so pressed by the enemy that destruction of his army is threatened.

II. The Army of the Cumberland immediately advances to drive in the enemy
cavalry screen south of the Harpeth River and maneuvers so as to drive the enemy
farther south. Any loss of contact with enemy forces must be reported to this
headquarters immediately. C. Smith's Corps is to withdraw and join the Army of
the Kanawha in the Clarksville area.

III. Brig. Gen. Wood will immediately move his division to Nashville and
advances towards Antioch to prevent the enemy from falling on Gen'l Howard's
flank.

IV. Col. McHenry, Jr.'s Brigade of the Division of Observation will occupy the
blockhouses defending the rail road from McKenzie in the west to Clarksville in
the east.

V. Col. Hawkins' Brigade of the Division of Observation moves to occupy Fort
Defiance.

VI. Kellogg's Division of Cavalry and McMillan's Brigade of the Division of
Observation will move to the Clarksville area and attach to the Army of the
Kanawha.

VII. 3d Division, VIIth Corps is to immediately move to rejoin VIIth Corps near
Humboldt, Tenn.

\gramClosingBy{Maj. Gen. Blake}
{Walter Chekov}
{Col., Adjutant General}
\reportdinkus

\gramHeader{Headquarters, Dep't of the Cumberland} % {{{4
{Nashville, Tenn., May}{20, 1862}
\gramTo{Cdre.}{Daniel Lewis}
{Commanding, Cumberland River Squadron}

\gramHi{Commodore} Gen'l Steele intends to land his army near Crump's Landing on
the [???] inst. The support of your squadron is requested both to support his
landing and to secure his initial movements inland. Your squadron is not
expected to approach the enemy fort at Pittsburg Landing.

\gramClosing{Respectfully}
{J. W. Blake}
{Maj. Gen., Commanding}
\reportdinkus

\subsection*{May 26, 1862}{} % {{{3

\gramHeader{Headquarters, Dep't of the West} % {{{4
{North Bank of Middle Fork Deer River, Tenn., May}{26, 1862}
\gramTo{Maj. Gen'ls}{Cornelius Van Royne \& James W. Blake}
{Commanding United States Army \& Department of the Cumberland, resp'y}

\gramHi{Generals} The corps are in position and the road bridges at Spains and
Carroll are up. The Army of the Arkansas 2d Cavalry Division has been sent out
towards Purdy to cut communication and rail lines in preparation for Gen'l
Steele's landing.

The Army of the Arkansas 1st Cavalry Division was ordered to cross at Gadsden
and scout the area around Bells. They quickly returned with reports that they
spotted a very large body of cavalry and infantry from the Army of the West
moving towards Bell's and Jones. Most likely they have repaired the damage we
caused to the bridge.

Across from the bridge near Carroll, enemy cavalry has pushed up to the bank
flying the flags of the Army of the Trans-Mississippi. Enemy cavalry is now
screening both Carroll and Gadsden, with Thomson's cavalry screening the Carroll
bridge and a division of Anderson's cavalry reported to be across from Gadsden.

The siege of Island No. 10 continues with no real damages seen. The naval
batteries at New Madrid will be finished at the start of next week.

With the presence of Thomson's cavalry, I am unsure if the Army of the West
intends on coordinating with the Army of the Trans-Mississippi in striking
across the Gadsden and Carroll bridges again. Indeed I have still yet to spot
Thomson's infantry.

If you deem it wise to continue ahead with our original plans to press towards
Jackson and to find Anderson, then I shall do so. Though it seems he has moved
across to Gadsden instead where the VIIth Corps is located.

\gramClosing{I am always, General, and shall ever remain, Your most humble and obedient of servants}
{Karl Meyer}
{Maj. Gen. cmdg Dep't of the West}
\reportdinkus

\gramHeader{Headquarters, Dep't of the Cumberland} % {{{4
{Franklin, Tenn., May}{26, 1862}
\gramTo{Maj.}{Andrew Mackay}
{Chief Quartermaster}

\gramHi{Major} Your are to move immediately to Nashville and establish yourself
there.

Please forward supplies and engineers to repair the bridge at Smyrna, Tenn. and
to replace the supplies expended repairing the bridges at Antioch and Franklin,
Tenn. over the past week.

\gramClosingBy{Maj. Gen. Blake}
{Walter Chekov}
{Col., Adjutant General}
\reportdinkus

\gramHeader{Headquarters, Dep't of the Cumberland} % {{{4
{Franklin, Tenn., May}{26, 1862}
\gramTo{Brig. Gen.}{Curran Pope}
{Commanding, Division of Observation}

\gramHi{General} Detach Dennis' Battery from McHenry, Jr.'s Brigade and send it,
along with Hawkins' Brigade to garrison Fort Fawcett at Nashville.

\gramClosingBy{Maj. Gen. Blake}
{Walter Chekov}
{Col., Adjutant General}
\reportdinkus

\gramHeader{United States Army} % {{{4
{Washington City, May}{26, 1862}
\gramTo{Maj. Gen.}{J. W. Blake}
{Commanding Department of the Cumberland}

\gramHi{General} I apologize in advance for the length of this message.

Understood on making contact with the Army of Tennessee, their withdrawal, and
that preparations are nearly ready for Army of the Kanawha to make their move.
Understood, waiting to hear his anticipated landing date. Has there been any
sign of General Whisper's command---the Army of Mississippi?

Please, so this office can convey accurate numbers if asked by the White House,
Cabinet, or Congress, provide your updated order of battle and troop counts
following this consolidation. Is Gen'l Howard remaining with the Army of the
Cumberland, or is he travelling with Gen'l Steele?

We've received a few extra officers here---are you in need of more Department
officers (I pray Gen'l Benton arrived safely), or more Corps commanders or staff
officers in your various armies?

As we are rapidly nearing the end of the spring, so too are we, unfortunately,
approaching the period of the audit.

Please, if they have not already been able during the course of the campaign,
ask your staff to begin compiling documentation of expenditures, both of
supplies and currency  over the course of the spring for your Department,
including any remaining balance and plans for using these leftovers, if any such
plan exists.

I know this may seem early, but I want to ensure you have the necessary time to
get everything in order, as these reports should be sent to this office on June
23, 1862.

I will send another reminder of this one week before.

\gramClosing{Always and humbly, Yr. obt. svt.}
{M. R. Turgon}
{Col., Office of the Commanding Gen'l}
\reportdinkus

\gramHeader{Headquarters, Dep't of the Cumberland} % {{{4
{Franklin, Tenn., May}{26, 1862}
\gramTo{Col.}{M. R. Turgon}
{Office of the Commanding General}

\gramHi{Colonel} I will begin compiling accurate returns of the Dep't strength
and the supply expenditures and balance remaining and hope to forward that to
your office within the week.

Gen'l Steele and the Army of the Kanawha are currently in the area of Fort
Defiance making final preparations for the movement up the Tennessee. I will
inform you of his expected landing date as soon as I know it.

Gen'l Howard is now in full command of the Army of the Cumberland, having
arrived here yesterday. Gen'l Benton has also arrived and will most likely
resume command of his corps shortly. Additional officers would be most welcome.
In particular, Gen'l Blake is in need of a diligent Chief of Staff.

\gramClosing{Your servant}
{Walter Chekov}
{Col., Adjutant General}
\reportdinkus

\gramHeader{Headquarters, Dep't of the West} % {{{4
{North Bank of Middle Fork Deer River, Tenn., May}{20, 1862}
\gramTo{Maj. Gen.}{Cornelius Van Royne}
{Commanding General of the United States Army}

\gramHi{General} I am unsure of whether an attack across the Middle Fork at this
point in time would be wise. Both of my subordinates have relayed to me that
they fear an attack resulting in the same result as did happen to the
Confederates last week when they attempted to cross.

If you deem it so, I shall order the Department forwards regardless, but am
prepared to suffer dearly for such an attack. I have attempted to look for other
methods of circumventing the bottleneck, but as of yet have not found such a
solution.

\gramClosing{I am always, General, and shall ever remain, Your most humble and obedient of servants}
{Karl Meyer}
{Maj. Gen. cmdg Dep't of the West}
\reportdinkus

\gramHeader{United States Army} % {{{4
{Washington City, May}{26, 1862}
\gramTo{Maj. Gen.}{K. Meyer}
{Commanding Department of the West}

\gramHi{Gen'l} At time of this message, an attack across and toward the
Confederates is not required. Gen'l Steele is scheduled to land on the 31st---
that is the date of action. Certainly the enemy will catch word of his landing
quickly and attempt to withdraw to assault him. If the entire enemy army deploys
against you when you give pressure, that buys Steele precious time. If they
leave a rear guard, you must dispense of it quickly with the full weight of your
army and give pursuit, so Steele is not attacked in isolation. When he lands, do
your best to maintain contact with him.

Until such a time, my prayers are with you, Gen'ls Ambrose and Caldwell, and the
brave men in blue around Humboldt. Hold firm; Steele is on the way.

Also, we are sending you an additional officer to use however you see fit. His
name is Brig. Gen. C. G. T. Wooldridge. A respectable fellow, a former judge,
from Maine.

\gramClosing{Humbly and always, Yr obt svt.}
{M. R. Turgon}
{Brig. Gen., Office of the Commanding Gen'l, Quartermaster Gen'l}
\reportdinkus

\gramHeader{United States Army} % {{{4
{Washington City, May}{26, 1862}
\gramTo{Col.}{Walter Chekov}
{Adjutant General, Department of the Cumberland}

\gramHi{Colonel} Received, thank you on all accounts, looking forward to that
information.

I've made the Gen'l Commanding aware of Gen'l Blake's request for a Chief of
Staff. To confirm, the information this office has in regards to staff within
your department:

\begin{center}
    \begin{dispatch}[
    ]{
        colspec = {l|l|l},
    }

    \MakeUppercase{Department of the Cumberland}: & \MakeUppercase{Army of the Cumberland}: & \MakeUppercase{Army of the Kanawha}: \\
    H. K. Fawcett (wounded \& out of action) & P. Smith & A. F. P. Murder \\
                                             & Z. Smith & J. C. Ryan \\

    \end{dispatch}
\end{center}

Please confirm their status, and inform if we are missing anyone.

\gramClosing{Yr obt. svt.}
{M. R. Turgon}
{Col., Office of the Commanding Gen'l}
\reportdinkus

\gramHeader{Headquarters, Dep't of the Cumberland} % {{{4
{Franklin, Tenn., May}{26, 1862}
\gramTo{Col.}{M. R. Turgon}
{Office of the Commanding General}

\gramHi{Colonel} Maj. Gen. Benton has filled the position of Chief of Staff to
Gen'l Blake as lingering effects of his wounds preclude, for now, his taking an
active command. Col. Murder returned home earlier this spring after resigning
his commission. Gen'l P. Smith has led his corps ably but the mud and cold in
front of Donelson was not kind to his health and I fear I will need a
replacement soon. Brig. Gen. Ryan continues to perform admirably as a corps
commander and de facto second in command to Gen'l Steele.

\gramClosing{Your servant}
{Walter Chekov}
{Col., Adjutant General}
\reportdinkus

\gramHeader{United States Army} % {{{4
{Washington City, May}{26, 1862}
\gramTo{Col.}{Walter Chekov}
{Adjutant General, Department of the Cumberland}

\gramHi{Colonel} It is unfortunate to hear regarding Colonel Murder, as well as
the ill health of Gen'l Smith. I am gladdened to hear of Gen'l Ryan's admirable
service; please send him and Gen'l Steele my best when you are able.

We are sending you some officers to replace those you have lost. They will be
leaving Washington City shortly, and should take about a week or so to reach
you. I tell you now, so you may have some time to think about their arrival and
where they may be placed.

The fellows in question are Brig. Gen'ls M. H. Williams \& O. Gerritsen. Maj.
Gen. H. Johanson shall return to your department as well, good sir.

We have not assigned them specific armies or duties within the department, we
leave that to Gen'ls Blake, Benton, and yourself. I ask once they get settled,
please inform this office where they have been placed.

\gramClosing{Yr obt svt.}
{M. R. Turgon}
{Col., Office of the Commanding Gen'l}
\reportdinkus

\gramHeader{Headquarters, Army of the Kanawha} % {{{4
{Clarksville, Tenn., May}{26, 1862}
\gramTo{Maj. Gen.}{James W. Blake}
{Commanding Dep't of the Cumberland}

\gramHi{Sir} I write on behalf of the Commanding General of the Army of the
Kanawha upon his request. At this time, the Army has assembled and the date for
landing has been set with the Navy for May 31st. The Army will land at Crump's
Landing across from Savannah before advancing on the Pittsburg Landing Fort,
Purdy, and finally Corinth.

To achieve this goal, General Steele requests an assessment of the LP available
to supply Kanawha's forces. Supply projections indicate that a river depot will
be needed at Crump's Landing alongside a Wagon Depot at Purdy in order to supply
the push to Corinth. Supply up to this stage has been managed by the Department.
Upon review, this Headquarters requests on what may be available to enable this
march at Kanawha's current strength (337 Force total). Conceivably, the Depot at
Purdy will not be required to supply the entirety of Kanawha's forces but at
least one or two corps alongside two Divisions of scouting cavalry will be
needed for the Corinth push.

Please advise what may be available when you are able.

\gramClosing{Respectfully}
{J. C. Ryan}
{Brig. Gen., XI Corps}

Copy to Meyer

\reportdinkus

\gramHeader{Headquarters, Dep't of the Cumberland} % {{{4
{Franklin, Tenn., May}{26, 1862}
\gramTo{Maj. Gen.}{Karl Meyer}
{Commanding, Dep't of the West}

\gramHi{General} Gen'l Steele intends to land his army at Crump's Landing on the
31st inst.

\gramClosing{Respectfully}
{J. W. Blake}
{Maj. Gen., Commanding}
\reportdinkus

\gramHeader{Headquarters, Army of the Kanawha} % {{{4
{Clarksville, Tenn., May}{26, 1862}
\gramTo{Brig. Gen.}{Jacob Ryan}
{Commanding, XIth Corps}

\gramHi{General} Enough supplies will be allocated to the Army of the Kanawha to
construct a river landing able to support your entire force and establish a
wagon depot capable of supporting two corps of infantry and two small cavalry
divisions. [I am allocating 172 LP to your army.]

\gramClosing{Respectfully}
{J. W. Blake}
{Maj. Gen., Commanding}

Copy to Maj. Mackay

\reportdinkus

\gramHeader{Smithland, Ky.} % {{{4
{May}{26, 1862}
\gramTo{Col.}{M. R. Turgon}
{Office of the Commanding General}

\gramHi{Colonel} The requirements of the Corinth expedition have stretched the
supplies available to this Dep't quite thin. If a moderate amount is available
to be sent post haste to Nashville this should allow the Army of the Cumberland
to continue to push southwards if the enemy gives ground.

\gramClosing{Your obedient servant}
{Andrew Mackay}
{Maj., Chief Quartermaster}
\reportdinkus

\gramHeader{United States Army} % {{{4
{Washington City, May}{26, 1862}
\gramTo{Maj.}{Andrew Mackay}
{Chief Quartermaster, Dep't of the Cumberland}

\gramHi{Major} The situation is understood. At this time we are releasing the
supply that can be spared currently to the Department of the Cumberland,
immediately.

\gramClosing{Yr obt svt.}
{M. R. Turgon}
{Brig. Gen., Chief Quartermaster}
\reportdinkus

\gramHeader{Headquarters, Dep't of the Cumberland} % {{{4
{Franklin, Tenn., May}{26, 1862}
\gramTo{Maj.}{Andrew Mackay}
{Chief Quartermaster}

\gramHi{Major} Your are to move immediately to Nashville and establish yourself
there.

Please forward supplies and engineers to repair the bridge at Smyrna, Tenn. and
to replace the supplies expended repairing the bridges at Antioch and Franklin,
Tenn. over the past week.

Finally allocate the necessary supplies to Gen'l Steele for his campaign against
Corinth [172 LP].

\gramClosingBy{Maj. Gen. Blake}
{Walter Chekov}
{Col., Adjutant General}
\reportdinkus

\gramHeader{Headquarters, Dep't of the Cumberland} % {{{4
{Franklin, Tenn., May}{26, 1862}
\gramTo{Brig. Gen.}{Curran Pope}
{Commanding, Division of Observation}

\gramHi{General} Detach Dennis' Battery from McHenry, Jr.'s Brigade and send it,
along with Hawkins' Brigade, via the river to garrison Fort Fawcett at
Nashville.

\gramClosingBy{Maj. Gen. Blake}
{Walter Chekov}
{Col., Adjutant General}
\reportdinkus

\gramHeader{Headquarters, Dep't of the Cumberland} % {{{4
{Franklin, Tenn., May}{26, 1862}
\gramTo{Brig. Gen.}{Thomas Wood}
{Commanding, Reserve Division}

\gramHi{General} You are to advance to Murfreesborough once your supply line to
Nashville is secure and establish defensive positions against enemy movement
north along the Nashville R.R.

\gramClosing{Respectfully}
{J. W. Blake}
{Maj. Gen., Commanding}
\reportdinkus

\gramHeader{Headquarters, Army of the Kanawha} % {{{4
{Clarksville, Tenn., May}{26, 1862}
\gramTo{Maj. Gen.}{James Blake}
{Commanding Department of the Cumberland}

\gramHi{General} I have at my disposal 63,000 men assembled for the expedition.

I have at my disposal the supplies allocated by Maj. Mackay and a lesser amount
allocated from the stores of the Dep't of the West.  When we land we will
establish a river port and an additional wagon depot at Purdy.

My main force at Purdy will be sustained by the river port and the forces I
dispatch to Corinth will subsist off of the wagon depot.

Corinth will be at the limit of our range from Purdy, so if and when that city
is secured we will not have the ability to do much else. I am expecting to leave
the bulk of my army around Purdy in anticipation of a relief force from that
direction. I also do not intend to expend too many of my men on assaults on the
fort at Pittsburg Landing. It does not hinder my movements toward Corinth so I
would rather save my exertions for that city. To that end I plan on detaching a
reinforced brigade to deal with the fort as I move upon Corinth.

\gramClosing{Your obedient servant}
{Richard Steele}
{Maj. Gen., Commanding Army of the Kanawha}
\reportdinkus


\gramHeader{United States Army} % {{{4
{Washington City, May}{20, 1862}
\gramTo{Maj. Gen.}{K. Meyer}
{Commanding Department of the West}

\gramHi{Gen'l} The construction of pontoons is permitted, of course, but moving
on the third will be too late to properly support General Steele. In conference
with the Commanding General, it is clear that the morning of 1st June (one day
after his scheduled landing) is the absolute latest a move against the enemy can
be made.

Again, if you can force the enemy to deploy against you entirely and delay any
action toward Steele, an attack is not necessarily required. Your scouts must do
good work to judge whether the enemy facing you is a small delaying force or the
entirety of the command.

\gramClosing{Humbly and always, Yr. obt. svt.}
{M. R. Turgon}
{Brig. Gen., Chief Quartermaster}
\reportdinkus

\gramHeader{Headquarters, Dep't of the Cumberland} % {{{4
{Franklin, Tenn., May}{26, 1862}
\gramTo{Maj. Gen.}{Richard Steele}
{Commanding, Army of the Kanawha}

\gramHi{General} I am forwarding you a portion of a message from Gen'l Meyer:

% TODO: How to quote this?
\small{
I wish to inform you that in order to support Gen’l Steele yet also avoid
marching straight into the enemy guns, I have ordered the construction of
pontoons to be used to ford the river north of South Carroll. Chief Engineer
Col. Flad has informed me that it will take the entire week to construct the
bridges, and then a further two days to move them to the crossing.

I hope to either fall upon a weakened enemy who is in the process of reacting to
Gen’l Steele’s landing or otherwise unhinge their defensive position as the
crossing is not guarded. This movement will most likely occur on the 3d prox.
and involve the entire XVth Corps and a portion of the Army of the Tennessee so
as to avoid alerting the rebels to our movements.
}

\gramClosing{Respectfully}
{J. W. Blake}
{Maj. Gen., Commanding}
\reportdinkus

\gramHeader{Headquarters, Army of the Kanawha} % {{{4
{Clarksville, Tenn., May}{26, 1862}
\gramTo{Maj. Gen.}{James W. Blake}
{Commanding, Dep't of the Cumberland}

\gramHi{Sir} I write on behalf of the Commanding General of the Army of the
Kanawha upon his request. Our command is tied into the Department of the West's
telegram network and we can report receiving the full text of the message this
excerpt comes from. Landing orders are set and the men are ready. We do not fear
that which is known. It is the unknown that concerns us. All is in the Lord's
hands now.

Divine Providence be with us.

\gramClosing{Respectfully}
{J. C. Ryan}
{Brig. Gen., Commanding XIth Corps}
\subsecdinkus

\subsection*{June 2, 1862}{} % {{{3

\gramHeader{Headquarters, Dep't of the West} % {{{4
{North Bank of Middle Fork Deer River, Tenn., June}{2, 1862}
\gramTo{Maj. Gen.}{Cornelius Van Royne}
{Commanding General of the United States Army}

\gramHi{Sir} My apologies for the late dispatch, it seems that partisans had
sabotaged our telegraph wires. It is my privilege to report to you that, as of
9~ o'clock the 2d inst., Brig. Gen. Julius White accepted the surrender of the
fort at Island Number 10, along with the commander and his garrison of 950 men.

The fort was taken intact, though the batteries were all spiked, and the Navy
now has command of the Mississippi as far south as the Chickasaw Bluffs, where a
battery at Plum Point fired on them. They also sighted an unfinished fort on the
other side of the river at Osceola. It is my intention to send the Division of
Observation to land near this fort, establish a depot, and take it before it can
be completed or reinforced. The cavalry of the 3rd Division will be railed to
the river in order to reinforce them at a latter point.

Given the various attacks by partisans on Gen’l’s Walle and Blake’s rear lines,
and given that you intended on the VIIth Corps to act in a defensive manner due
to their exhaustion in the previous campaign, it is my intention to garrison Is.
10 and the depots at Union City, Trenton, McKenzie, and Humboldt with the 3d
Division of the VIIth Corps. I would be grateful if Secretary Stanton could
send garrison troops to these locations but understand that it may not be
feasible militarily nor politically.

In regard to our operations further south, fortune has indeed smiled upon us.
Both the Army of the West and the Army of the Trans-Mississippi spent the week
building pontoons before attempting to launch another assault with five corps
across the Middle Fork River on the 31st ult. I say attempting because our
artillery batteries opened up on the enemy’s engineers and within 30 minutes
destroyed all four bridges.

This development is due in no small part to Gen’ls Ambrose and Caldwell, who
urged against any crossing this week. Indeed if we had crossed, I imaging we
would have been dealing with the full might of Thomson and Anderson’s armies
with our backs to the Middle Fork.

By the morning of the 1st of this instant, the enemy was seen withdrawing from
both Carroll and Gadsden and towards Jackson. By the 2d, all sight of the enemy
was lost. There were three corps sighted belonging to the Army of the West, and
we are already aware that Gen’l Thomson has three corps in his Army of the
Trans-Mississippi. Thus we conclude there are at least six corps in action
against us, of varying sizes.

On the Tennessee River, Gen’l Steele has made a successful landing on the 31st
ult. before establishing a depot at Purdy and sending cavalry as far Anderson’s
Store. Per his report before the wires were cut, his army is five miles away
from Corinth and has not seen any evidence of it being defended, and indeed
possibly sighting smoke on the horizon. In addition, the fort at Pittsburg
landing has been invested for the time being so as to prevent any mischief the
rebels may attempt.

It is my belief that the Army of the Kanawha will reach Corinth before the enemy
does, given they started moving during the 1st and 2d inst.  Furthermore, having
realized that Gen’l Steele is behind them, the enemy is most likely moving south
in an effort to throw them back into the Tennessee River.

Gen’ls Ambrose, Caldwell, and I have determined that with the enemy withdrawing
from the Middle Fork, we shall pursue him and do our best to hamper his attempts
to reach the Army of the Kanawha, moving towards Gen’l Steele’s forces at
Anderson’s Store with all due haste.

\gramClosing{I am always, General, and shall ever remain, Your most humble and obedient of servants}
{Karl Meyer}
{Maj. Gen. cmdg Dep't of the West}

Copy to Turgon.

\reportdinkus

\gramHeader{Headquarters, Dep't of the Cumberland} % {{{4
{Franklin, Tenn., June}{2, 1862}
\gramTo{Maj. Gen.}{C. N. Van Royne}
{Commanding General, United States Army}

\gramHi{Sir} I regret that lack of garrisons allowed enemy irregulars to raid
Bowling Green and Cloverdale although the loss at Cloverdale does not materially
affect our operations here. I intend to create a District of Middle Tennessee
with Maj. Gen. Benton in command, to manage garrisons in this area. I must ask
if I am expected to also manage the garrisons north of the Cumberland River?

Gen'l Howard's cavalry and XVIth Corps skirmished with the enemy southeast of
Franklin during the week. Both sides appear to have lost about 300 cavalrymen. I
can confirm there are 7--9,000 infantry under Breckenridge and 6,500--8,000
cavalry under Forrest. Additional cavalry (likely another division or so) has
been spotted but I am unsure about any additional troops. I expect Gen'l Jackson
has between four and six divisions in total.

As the enemy has burnt the bridge at Columbia, I do not expect and advance from
that direction. Gen'l Howard intends to consolidate his army at Franklin and
then, if the enemy has not withdrawn, strike southeast.

\gramClosing{Respectfully}
{J. W. Blake}
{Maj. Gen., Commanding}

Copy to Meyer.

\reportdinkus

\gramHeader{United States Army} % {{{4
{Washington City, June}{2, 1862}
\gramTo{Maj. Gen.}{K. Meyer}
{Commanding Department of the West}

\gramHi{Gen'l} Yrs received as of this date. Your message is very well received,
sir!

The withdrawal of the enemy across from you is noted---please clarify: All six
enemy corps withdrew toward Jackson?

Understood on Steele's landing. It goes without saying, please warn him of the
approaching enemy in his rear. You must give chase with all haste.

The garrison requests you have made are not feasible at this time,
unfortunately. It may be prudent, if you have not done already, to build
blockhouses along the T. \& O. north of Humboldt. I recall you've already done
so on the M. \& C. line toward Nashville. A general order regarding garrisons in
hostile territory will be coming in the next few days.

\gramClosing{Yr. obt. svt.}
{M. R. Turgon}
{Brig. Gen., Chief Quartermaster}
\reportdinkus

\gramHeader{United States Army} % {{{4
{Washington City, June}{2, 1862}
\gramTo{Maj. Gen.}{J. Blake}
{Commanding Department of the Cumberland}

\gramHi{Gen'l} Received yrs as of this date. Good work with Steele's landing.
Please confirm what Corps have departed with Kanawha and the overall troop count
of the army, and who remains with yourself and General Howard.

Understood Gen'l Benton will be managing garrisons for you. Please garrison as
far north as Bowling Green and Munfordsville. It would be prudent to build
blockhouses along vulnerable stretches of the L. \& N. We are in the drafting
phase of a new General Order, which should be going to all Department Commanders
in the next few days.

Ensure both rail lines into Nashville are protected from Confederate
counterattack.

\gramClosing{Yr. obt. svt.}
{M. R. Turgon}
{Brig. Gen., Chief Quartermaster}
\reportdinkus

\gramHeader{Headquarters, Dep't of the West} % {{{4
{North Bank of Middle Fork Deer River, Tenn., June}{2, 1862}
\gramTo{Brig. Gen.}{M. R. Turgon}
{Chief Quartermaster of the United States Army}

\gramHi{General} Yrs this date just rcvd. In regards to the withdrawal of the
enemy, we spotted five corps that participated in the attack. Two from Thomson
and three from Anderson. These five corps were spotted pulling back in the
direction of Jackson. From our battle at Holly Grove, we are aware that Gen'l
Thomson has a third corps, though it did not participate in the attack and we
are unsure of its location. I believe it is most likely that if the other five
corps moved towards Jackson, then the last corps must have done so as well, or
opened itself up to defeat in detail.

In regards to Gen'l Steele, a telegram was dispatched to him minutes after the
one that you received. It goes without saying that we will be moving down the
Mobile and Ohio RR with all practicable haste.

The infeasibility to provide additional garrisons is understood. In this case
the 3d Division of VIIth Corps will be detached for garrison duties at Is. 10,
Union City, McKenzie, Trenton, and Humboldt. In regards to blockhouses, the
entire railroad from Columbus to Humboldt is garrisoned.

Of the three regiments that Gen'l Van Royne released from Columbus, the first
protects the rail from Columbus to Union City and from Hickman to the Obion
River near Tottens’s Well, the second guards the rail from Union City to
Humboldt, and the third is stationed from the Obion River near Tottens’s Well to
McKenzie then south to the South Fork of the Obion River.

The 26th Missouri Regiment of the Army of the Arkansas is protecting the line
between McKenzie to Crossing at the Tennessee River, and the 11th Iowa of the
Army of the Tennessee is stationed from the South Fork of the  Obion River to
Humboldt, with additional troops ready to protect the lines going to the Bell
and Jackson crossings when north bank of the South Fork Deer Fork River is
retaken.

I wish to reiterate that Gen'l Ambrose's Division of Observation which took part
in the siege of Is. 10, is to be shipped across the river to take the unfinished
fort at Osceola. The 3d Cavalry Division will be railed up to the nearest river
port in order to provide follow up reinforcements as well as scouting.

The Bureau of Military Information has been sifting through the reports compiled
for the ultimate. During this scouring, and after interrogating several
prisoners, it was determined that the commander of the Army of the West is in
fact Gen'l Whisper.

\gramClosing{I am always, General, and shall ever remain, Your most humble and obedient of servants}
{Karl Meyer}
{Maj. Gen. cmdg Dep't of the West}
\reportdinkus

\gramHeader{Headquarters, Dep't of the Cumberland} % {{{4
{Franklin, Tenn., June}{2, 1862}
\gramTo{Maj. Gen.}{Cornelius Van Royne}
{Commanding General, United States Army}

\gramHi{Sir} I need clarification on a few points so that Gen'l Benton may
appropriately assign garrisons. Will Brig. Gen. McKean's troops be responsible
for Bowling Green and points north along the Louisville \& Nashville R.R.? As
for the railroads leading south from Bowling Green, with the limited force
available to Gen'l Benton, I am inclined to focus on the Memphis Branch leading
to Clarksville and leave the main line to Nashville unprotected for the time
being. Finally, am I to continue to garrison Smithland, Ky. and Battery Truxtun?

The bridge at Clarksville should be repaired by the end of the month and keeping
that branch open will allow rapid travel between eastern Kentucky and western
Tennessee. The rail leading into Nashville is of less importance as long as the
Cumberland River is secure.

\gramClosing{Respectfully}
{J. W. Blake}
{Maj. Gen., Commanding}
\reportdinkus

\gramHeader{Headquarters, Dep't of the Cumberland} % {{{4
{Franklin, Tenn., June}{2, 1862}
\gramTo{Maj. Gen.}{Richard Steele}
{Commanding, Army of the Kanawha}

\gramHi{General} Gen'l Meyer informs me that your landing went well and that you
are approaching Corinth. I trust your judgement entirely in this operation and
wish you the best of luck.

Gen'l Johansen has returned and I will be sending him to you within a few days.

\gramClosing{Respectfully}
{J. W. Blake}
{Maj. Gen., Commanding}
\reportdinkus
% Index {{{1
\newpage
\backmatter
\pagestyle{backmatter}

\let\chapter\Oldchapter % Stop using chapter subtitles

\indexprologue{
    \begin{centering}\begin{small}
            \textit{%
                Brigades, Divisions, Corps, Armies, and improvised organizations
                are ``Mentioned'' under name of commanding officer; State and
                other organizations under their official designation.
            }
            \par
        \myrule
    \end{small}\end{centering}
}
\printindex

\end{document} % {{{1

